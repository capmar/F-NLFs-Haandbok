\part{Hoppfeltlederinstruks}
\setcounter{section}{506}

\section{Hoppfeltlederinstruks}
\subsection{Hoppfeltleder (HFL)}
HFL utpekes av HL, er underlagt denne, og har til oppgave å administrere, organisere og overvåke hoppfeltet, og påse at hoppingen gjennomføres etter gjeldende bestemmelser. Det kan etter HLs vurdering utpekes flere HFLer.

\subsection{Kvalifikasjoner}
Som tilstrekkelig kvalifikasjon for HFL regnes: Ved hopping med elever:
\begin{itemize}
	\item Fallskjermhoppere med B-sertifikat eller høyere, eller personell som etter HLs vurdering antas å være spesielt kvalifisert gjennom inngående praktisk erfaring under elevhopping.
\end{itemize}

Ved hopping med hoppere med A-sertifikat eller høyere:
\begin{itemize}
	\item Personell som på forhånd er instruert av HL om de oppgaver som påligger HFL.
\end{itemize}

Ved natthopp, vannhopp og oksygenhopp:
\begin{itemize}
	\item Hopper med Instruktørlisens 3 eller høyere
\end{itemize}

\subsection{Ansvar}
HFL har ansvar for:
\begin{itemize}
	\item At HLs anvisninger for kontroll av hoppernes sertifikater, lisens og utstyr blir fulgt ved manifestering.
	\item At hoppingen hele tiden gjennomføres overensstemmende med F/NLFs bestemmelser.
	\item At hoppingen straks avbrytes dersom forholdene tilsier det.
	\item At all innlasting, flyging og hopping skjer overensstemmende med manifestene.
	\item At HLs anvisninger blir fulgt.
\end{itemize}

\subsection{Myndighet}
HFL har myndighet til å sammensette løftene, beordre hopperne til innlasting, disponere flyene etter retningslinjer gitt av HL, lede alt bakkepersonell og rekvirere nødvendig assistanse fra deltakende hoppere.

\subsection{Plikter}
HFL har følgende spesielle plikter:

\subsubsection{Signalsystem}
Påse at jordtegn er på plass, og ved hjelp av disse gi nødvendige signaler til hoppmester i flyet, eventuelt kommunisere med hoppmester og/eller Flyger på andre måter iht. Del 100.

\subsubsection{Vind}
Holde seg, og HL informert om vindstyrke og -retning og sørge for at endringer i vindforholdene blir tatt hensyn til av Hoppmester. Stanse hoppingen dersom vindstyrken overstiger de fastsatte grenser.

\subsubsection{Syketransport}
Sørge for at transportmiddel egnet til transport av skadede til enhver tid er disponibelt med fører, og at førstehjelpsutstyr er tilgjengelig og klart til bruk.

\subsubsection{Hoppmester}
Utpeke HM til hvert løft, og i rimelig tid informere denne om alle forhold som har betydning for dennes gjennomføring av hoppmestringen.

\subsubsection{Manifest}
Føre manifest for alle hopp, og kontrollere at hoppene blir gjennomført overensstemmende med manifestet, eventuelt ajourføre manifestet.

\subsubsection{Innlasting}
Sørge for at hopperne i tide gjør seg klar til innlasting og kontrollere at HM for hvert løft er fastsatt.

\subsubsection{Kontroll med hoppene}
Følge hoppenes gjennomføring, notere hoppenes utførelse, holde rede på hvor den enkelte lander, evt. beordre assistanse til hoppere som trenger dette. Sammen med HM gi tilbakemelding til elever.

\subsubsection{Materiell}
Anvise plass for oppbevaring og pakking av skjermer. Påse at eventuelt skadet materiell behørig merkes og legges på spesielt sted. Etter avsluttet hopping besørger HFL alt materiell brakt i hus og hoppfeltet ryddet, samt at manifestene blir arkivert iht. klubbens retningslinjer.

\subsection{Andre forhold}
\subsubsection{Funksjonstid}
HFLs funksjonstid skal være forhåndsfastsatt. Dersom han skal avløses av en annen HFL, skal han ikke forlate hoppfeltet eller opphøre å virke som HFL før avløseren har tatt over.

\subsubsection{Deltagelse i hopping}
HFL skal være på bakken, og tillates ikke å delta i hoppingen.

\subsubsection{Handling ved ulykke}
Ved ulykke straks stanse all hopping, beordre nødvendig syketransport, ta vare på bevis, informere HL og meddele ham alle tilgjengelige opplysninger om ulykken.
