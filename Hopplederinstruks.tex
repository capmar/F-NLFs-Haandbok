\part{Hopplederinstruks}
\setcounter{section}{505}

\section{Hopplederinstruks}

\subsection{Hoppleder (HL)}
HL er den person som er overlatt ansvaret for gjennomføring av praktisk hoppvirksomhet på ETT STED i et BESTEMT TIDSROM. HL skal være utpekt før hopping startes.

\subsection{Kvalifikasjoner}
Som tilstrekkelig kvalifikasjon for HL regnes:
\begin{itemize}
	\item For hopping med hoppere med A-sertifikat og høyere:

	C-sertifikat
	\item Ved hopping med elever:

	D-sertifikat eller Instruktør 2.

	HI kan unntaksvis og etter særskilt vurdering for elevhopping godkjenne HL med C sert og I-3 når vedkommende minimum har 2 års erfaring som I-3, har deltatt jevnlig i klubbens utdanningsvirksomhet og operative drift i tillegg til å ha inngående lokalkunnskaper. Dette gjelder kun ved hopping fra fly med maks 6 hoppere og kun ved dropp fra et fly. Godkjente HL’er ihht dette, skal framkomme i HI’s plan.

	\item Ved natt-, vann- og oksygenhopping:

	Instruktør 1
\end{itemize}

\subsection{Ansvar}
\subsubsection{Varighet}
Den som er utpekt til HL har fra det tidspunkt hoppingen påbegynnes og inntil hoppingen er avsluttet, eller han er avløst, det hele ansvar for organisering og drift av hoppfeltet, uansett om hoppere med høyere status er til stede.

\subsubsection{Offentlige bestemmelser}
Foruten de fallskjermtekniske bestemmelser og instrukser, skal HL også påse at den virksomhet som foregår, ikke strider mot offentlige lover og regler.

\subsubsection{Omfang}
Begrepet hoppfelt innbefatter alt som berører hoppingen direkte eller indirekte, således også materiell, transport, fly, myndigheter, og alminnelig ro og orden.

\subsection{Myndighet}
\subsubsection{Disposisjonsrett}
HL er hoppfeltets øverste myndighet, og alt personell har plikt til å etterfølge hans anvisninger. HL disponerer flyene, samt alt klubbmateriell på feltet. HL kan bortvise fra feltet personer som ikke retter seg etter hans anvisninger.

\subsubsection{Overprøving av beslutninger}
HLs beslutninger er definitive på hoppfeltet, uansett om hopper med høyere status er tilstede. Dersom HL treffer beslutninger som andre er uenige i, kan beslutningene ankes til egen HI, evt. gjennom denne og videre til SU.

\subsection{Plikter}
\subsubsection{Tillatelser}
Kontrollere at NOTAM er utstedt og at grunneiers (og evt. politiets) tillatelse til hopping på det aktuelle tidspunktet er innhentet.

Forsikre seg om at hoppfeltet er godkjent for aktuell type hopping.

\subsubsection{Forberedelser}
Være på hoppfeltet i god tid før hopping er tillyst, og vurdere om forholdene tillater hopping, påse at tilfredsstillende bakkemannskap er tilstede, samt sikre seg at klubbens materiell som nyttes er i luftdyktig stand og ikke pakkeforeldet. Sørge for at kvalifisert pakker overvåker transport og behandling av materiellet og pakketjenesten på hoppfeltet.

HL skal sikre seg at transportmiddel for syketransport er tilgjengelig og at nødvendige opplysninger for varsling ved ulykke er kjent.

\subsubsection{Kontroll av fly}
HL skal forsikre seg om at flyene som blir brukt er godkjent av Luftfartstilsynet for fallskjermhopping.

\subsubsection{Pakking av elevutstyr}
Eventuelt ta ut pakkere for pakking av elevenes fallskjermer.

\subsubsection{Hoppfeltledelse}
Ta ut det nødvendige antall HFLer, kontrollere at disse har tilstrekkelige kvalifikasjoner for den aktuelle hopping, og evt. ta ut materiellassistent. HL skal fordele oppgaver ved bruk av flere HFLer. Kontrollere at HFL utfører de oppgaver han er pålagt iht. vedlegg til Del 500 ”Hoppfeltlederinstruks”. HL kan evt. selv være HFL og følger de gjeldende bestemmelser for denne i tillegg til HL-instruksen.

\subsubsection{Kontroll}
Påse at virksomheten hele tiden er i henhold til gjeldende bestemmelser, herunder opprettholde et kontrollsystem for gyldighet av de deltakende hopperes sertifikater og/eller beviser og luftdyktighet for det fallskjermmateriell som benyttes. Dersom HFL eller noen annen person med operativ oppgave på hoppfeltet er i ferd med å miste kontrollen med situasjonen, eller foretar noe i strid med gjeldende bestemmelser, skal HL omgående gripe inn. Dersom forholdene gjør det nødvendig å stanse hoppingen, skal HL overvåke situasjonen og sørge for at det ikke kan gjenopptas hopping utenfor hans kontroll.

\subsection{Andre forhold}
\subsubsection{Funksjonstid}
HLs funksjonstid skal være forhåndsfastsatt. Dersom han skal avløses av en annen HL, skal han ikke forlate hoppfeltet eller opphøre å virke som HL før avløseren har overtatt.

\subsubsection{Deltagelse i hopping}
HL kan under sin tjeneste være på bakken eller i luften og delta aktivt i hoppingen, dog på en slik måte at han hele tiden har kontroll med virksomheten.

\subsection{Handling ved ulykke}
\subsubsection{Varsling}
Dersom ulykke inntreffer på hoppfeltet, skal HL besørge tilkalling av legeassistanse, og deretter gjennomføre varsling iht. ”Handlingsinstruks ved ulykker”.

HUSK: Det er politiets ansvar å varsle forulykket persons pårørende.

HL skal samarbeide med politiet, og sørge for at alle opplysninger som ønskes bringes til veie så langt det er mulig.

\subsubsection{Rapportering}
Uavhengig av politiets etterforskning skal HL uten unødig opphold utarbeide detaljert rapport om hendelsesforløpet, inkludert sine egne vurderinger. Rapporten skal spesielt inneholde navn og adresse på, og egne skriftlige uttalelser fra:

HFL, HM, Flyger evt. andre impliserte parter, samt vitner til hendelsen.

Særtrykk av ``Handlingsinstruks ved ulykker'' skal, med utfylte data, bestandig være tilgjengelig på hoppfeltet.

\subsubsection{Rapporteringspliktige hendelser}
\paragraph{Næruhell:}
Utilsiktet skjermåpning, utløsning av reserveskjerm, feilfunksjoner, utilsiktet vannlanding eller andre forhold som nær hadde forårsaket uhell/ulykke. Som næruhell regnes også brudd på F/NLFs bestemmelser.

\paragraph{Uhell:}
Hendelse hvor hopper er påført skade som medfører behandling av lege og/eller av sykehus og/eller påfølgende sykefravær i kortere eller lengre tid. Skaden må ha inntruffet mellom avgang med fly og fullført landing. Som uhell regnes også større skader på materiell.

\paragraph{Ulykke:}
Hendelse som har medført død for en eller flere personer mellom avgang med fly og landing i fallskjerm. Som ulykke regnes også store skader der situasjonen er uavklart eller at skaden er så alvorlig at det er risiko for store varige mén.

\paragraph{Farlig/ukontrollert hopp (FU)}
Hendelse i samsvar med kriterier for FU iht. Del 600. FU kan av HI også gis til selvstendige hoppere.

\paragraph{Permanent hoppforbud}
Hendelse eller opptreden av en slik karakter at egen eller andres sikkerhet settes i fare, og gjentagelse vurderes som sannsynlig. Midlertidig hoppforbud gis av HI. Dersom dette bør være permanent fremmes det for avgjørelse i SU og registreres i sertifikatregisteret av F/NLFs Sekretariat.
