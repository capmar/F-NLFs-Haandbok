\part{Nødåpnere og Instrumenter}

\section{Generelt}
Det finnes tre typer godkjente nødåpnere for reserveskjerm for bruk i Norge. De er laget for å åpne reserve pakksekken, styrt av barometrisk målt eller beregnet høyde.
\begin{description}
\item[FXC - 12000] Mekanisk virkemåte, produsert av Parachute de France / FXC Corporation, USA (Poynter vol 2 pkt. 9.4.7.side 503).
\item[Cypres] Elektronisk virkemåte, produsert av Airtec i Tyskland
\item[Astra] Elektronisk, produsert av FXC Corporation siden 1995
\end{description}

Disse vil bli omhandlet videre i egne kapitler.

\subsection{Historikk}
Nødåpnere har siden 1993 fått en økt bruk blant erfarne hoppere, og det kan forventes at flere typer vil komme på markedet i tiden etter dette.

Det kan argumenteres både for og imot mekaniske og elektroniske åpnere. Dette forblir et tema som baserer seg på produsentenes argumenter. Der den ene påstår at elektronikk er bedre en mekanikk på grunn av slitasje etc, vil den andre påstå at dette tvert imot er en fordel der en ikke må basere seg på elektronikk som er avhengig av strømkilder som kan svikte. Valget av nødåpner må forbli en sak der erfaringer i bruk over tid må veie tungt.

Alle typer nødåpnere virker på en relativ lik måte. De er konstruert for å åpne reservepakksekken i en gitt høyde dersom den vertikale fallhastigheten overstiger en viss hastighet. Aktiviseringshøyden kan til en viss grad justeres av brukeren selv.

Det benyttes i grove trekk to måter å åpne reservepakksekken på. Den ene er å benytte det eksisterende trekksystem der åpneren trekker ut reservepinnen fra låseløkka ved hjelp av en kraftig spennfjær. Den andre er designet for å skjære av låseløkka inne i pakksekken. Til dette kan det benyttes en kniv aktivert av en liten sprengladning, eller også fjæropptrukkede roterende kniver.

Det er ikke vanskelig å avgjøre hva som er best, nødåpner eller ikke. Det er et faktum at mange menneskeliv kunne vært reddet med nødåpnere, og de siste års utvikling viser at mange av argumentene mot nødåpner allikevel ikke hadde gyldighet.

Nødåpner betyr et større økonomisk løft ved kjøp av utstyr, men tar allikevel ikke bort ansvaret fra hver enkelt hopper til å ta vare på seg selv.

F/NLFs bestemmelser er at det til elevhopping og tandemhopping skal benyttes nødåpnere mens den erfarne hopper bør avgjøre selv om han/hun vil bruke nødåpner.

\subsection{Godkjenning av nødåpnere}
\begin{itemize}
\item SU godkjenner nødåpnere for bruk
\item For elevutstyr er
\begin{enumerate}
\item FXC 12000 som er oppdatert til ``J'' standard, eller
\item Airtec Cypres og Cypres 2 elevversjon godkjent
\end{enumerate}

\item For tandemutstyr er kun Cypres og Cypres 2 Tandemversjon godkjent, og er påbudt for all hopping med dette utstyret.
\item Når nødåpner er installert skal fabrikantens og/eller F/NLFs syklus for vedlikehold følges, ellers er ikke utstyret luftdyktig.
\end{itemize}

\subsection{Nødåpnerens virkemåte}
\textbf{FXC} (og andre mekaniske nødåpnere) er basert på en såkalt aneroide. Dette er i prinsipp en lufttett boks med elastiske sider. Denne boksen inneholder en viss luftmengde. Ved minkende atmosfæretrykk (stigende høyde) vil luftmengden i boksen utvide seg i volum og boksen blir større. Det motsatte vil skje ved stigende atmosfæretrykk (minkende høyde).

Tar vi en slik aneroide og monterer på en vektarm, viser og en gradert skala kan vi direkte lese av høyden over havet. Det er på dette prinsippet høydemåleren er bygget opp. Ved hjelp av enkle mekanismer kan vi få denne aneroiden til å starte en ønsket prosess i en viss høyde. (Se videre pkt 8.8.2.2 - Virkemåte FXC 12000).

\textbf{Elektroniske åpnere} baserer seg på elektronikk og et avansert beregningssystem for å beregne den sanne høyde og fallhastighet på hopperen. Ved hjelp av elektronikk og batteri blir det gitt et fyringssignal til en kutter, som igjen skjærer over loopen som holder reservepakksekken lukket. (Se videre pkt. 8.8.3 – CYPRES).

\section{FXC 12000}
FXC 12000 er en rent mekanisk åpner. Den er opprinnelig utviklet av HI-TEK Corporation, senere videreutviklet og produsert av FXC Corporation. (Begge fra California, USA.)

Etterhvert begynte det franske firmaet Parachutes de France å produsere FXC 12000 på lisens. Den amerikanske modellen hadde noen svakheter som førte til at nødåpneren fyrte i utide. En av årsakene til dette, var lekkasjer i luftslangen. Parachutes de France har videreutviklet modellen og blant annet forbedret kvaliteten på luftslangens tilkoblinger.

Siden første utvikling har FXC gjennomgått en rekke oppdateringer og forbedringer. Deler av innmaten har stadig blitt forbedret og byttet. I 1994 anbefalte FXC Corporation, USA, at alle FXC 12000 skulle oppdateres til ``J'' standard – (alle oppdateringer har sin egen bokstavkombinasjon) – der den ble vesentlig endret med flere luftfiltre for sikrere og jevnere innslipp av luft. Høsten 1994 ble det bestemt at samtlige av Norges FXC 12000 skulle overhales og oppdateres til ``J'' standard, og etter dette er denne den eneste som er tillatt for elevbruk i Norge.

Siden Norges FXC opprinnelig ble innkjøpt av Parachutes de France, og har hatt en rekke vedlikehold på PdF fabrikken, og senere er blitt oppdatert på fabrikken i USA, består i dag disse åpnere av en blanding av franske og amerikanske deler. Enkelte av de franske deler var bedre enn de tilsvarende amerikanske, og de er oppdatert slik at de minst tilsvarer en ``J'' standard fra FXC Corporation. Med en ``J'' standard oppdatering vil det være sjeldnere behov for fabrikkvedlikehold, forutsatt at åpneren er funksjonstestet med jevne mellomrom i mellomtiden.

FXC 12000 er et presisjonsinstrument konstruert spesielt for sportshopping. Konstruksjonen er både robust og kompakt slik at den skal kunne tåle vanlig behandling på feltet, men har en innvendig finmekanisk konstruksjon som ikke tåler hva som helst.

Det følgende beskriver bruk og virkemåte for FXC 12000.

\subsection{Beskrivelse}
FXC 12000 kan brukes både på hoved- og reserveskjerm, men blir hovedsaklig brukt på reserveskjermen. Dens funksjon er å trekke ut pinne automatisk i det øyeblikk hopperen passerer eller befinner seg under den høyden som er innstilt og fallhastigheten overstiger 40-65 fot/sek. (dvs. 20-30\% av terminalhastigheten).

Åpneren består av to deler forbundet med en luftslange. Den største delen inneholder kraftkilden som er en dobbeltspunnet spiralfjær, et variometer samt mekaniske utløser- og låsemekanismer. Den minste delen, betjeningsdelen, er forsynt med en gradert høydeskala av/på bryter innstillingsskrue og aneroide. Åpneren er vist i fig 1.

\subsubsection{Betjeningsdelen}
\begin{description}
\item[Høydeskalaen] er gradert fra 0 til 4. Dette innebærer at vi har muligheter for å innstille den til å fyre i området mellom 0 og 4000 fot over bakkenivå. Høydeskalaen er vist på fig 2.
\item[JUMP/OFF – av/på bryter]. Denne sitter på siden av betjeningsdelen se fig 1.
\item[Høydestillskruen] sitter på betjeningsdelen på motsatt side av luftslangen. Denne brukes til å stille inn høyden åpneren skal fyre på. Innstilling til bakkenivå skjer automatisk når høyden stilles inn.
\item[Luftfiltre]: Etter oppdatering til ``J'' standard er det innmontert tre stk gullfargete filtre som slipper luft inn i betjeningsdelen. Kun åpnere oppdatert til ``J'' standard har disse filtrene. Betjeningsdelen er den eneste synlige og utsatte delen på åpneren, og bør beskyttes med velcro eller lignende.
\end{description}

\begin{figure}
	%\includegraphics[width=60mm]{Strekktesting av kalottduk.pdf}
	\caption{Betjeningsdelen}
\end{figure}

\subsubsection{Hoveddelen}
Den største delen blir kalt for hoveddelen. Det er en sort firkantet boks som inneholder fjærmekanismen til åpneren. Det er også et filter for luftinnslipp montert i kassen.

Her finner vi en dobbeltspent spiralfjær av stål. Det er denne fjæren som er kraftkilden som trekker pinnen ut av låseløkka.

Ved siden av fjæren inneholder den også låsemekanismen som holder fjæren spent, og et variometer som står i direkte forbindelse med låsemekanismen.

\subsubsection{Tekniske data}
\begin{description}
\item[Mål:] 152*82*35 mm (hoveddelen)
\item[Vekt:] 1 kg
\item[Nøyaktighet:] Mellom 0 og 1800 fot: -100 / +300 fot

Mellom 1800 og 3500 fot: -200 / +400 fot

For hoppfelt som ligger så høyt at nødåpnerens fyringshøyde overstiger 3500 fot over havet er nøyaktigheten ikke lenger akseptabel. Man kan i slike tilfeller få kalibrert nødåpneren hos produsenten, spesielt for bestemte hoppfelt høyder.
\item[Åpning:] Ingen ved fallhastigheter under 40 fot/sek

Mulig ved fallhastigheter over 40 fot/sek

Effektiv ved fallhastigheter over 65 fot/sek
\item[Temperatur:] -60° C til +93° C
\item[Fuktighet:] 95\% Relativ Fuktighet
\item[Trekkraft:] Ved start av kabelbevegelsen: 36,0 kg

Ved slutt av kabelbevegelsen: 13,5 kg
\item[Kabelbevegelse:] 50 mm
\end{description}

\subsection{Virkemåte FXC 12000}
\subsubsection{Funksjon og konstruksjon}
Hoveddelen består av to kamre som begge har tilgang til det utvendige lufttrykk gjennom hvert sitt tynne kapillærrør. Høyre kammer har gjennom luftslangen forbindelse med betjeningsdelen. Mellom kamrene finnes en membran som er koblet til en hake, som igjen holder den oppspente fjæren spent. Hvis variometeret utsettes for raske trykkøkninger vil ikke luft komme tilstrekkelig raskt inn i begge kamre i boksen, og dette påvirker membranen. Når membranen bukter seg utløses sperren og fjæren trekker seg sammen og trekker wire og reservepinnen ut av låseløkka.

Lufttrykket varierer med høyden over bakken og med meteorologiske forhold. For å få aneroiden (8.8.2.1, Beskrivelse) til å starte den ønskede prosessen i rett høyde over bakken, må vi kompensere for det varierende lufttrykket. Vi må kalibrere den.

I en nødåpner kombineres variometeret med aneroiden på en slik måte at variometeret hindrer aneroiden å utløse skjermen dersom fallhastigheten er under en viss grense. Variometeret tillater bare aneroiden å gi impulser til å åpne skjermen dersom fallhastigheten overstiger en viss grense. Aneroiden kan derfor bare aktivisere skjermåpning når den innstilte høyden passeres og fallhastigheten er over grensen.

Hvis fallhastigheten er under grensen (vi henger i en bærende hovedskjerm), vil variometeret blokkere aneroiden og åpneren vil ikke fyre. Skulle plutselig fallhastigheten øke (cut-away) og vi befinner oss under den innstilte høyden vil variometeret frigjøre aneroiden som igjen vil aktivisere skjermåpning.

Før hopperen setter seg i flymaskinen settes ``JUMP/OFF'' bryteren på ``JUMP'' og åpneren innstilles/kalibreres.

Åpneren innstilles ved at aneroidens størrelse justeres slik at den tetter til luftslangen ved den bestemte høyden (1000 fot som avleses på skalaen på fronten). Dette betyr også at aneroiden tillater ventilen å åpne når den innstilte høyden passeres på vei nedover.

\begin{figure}
	%\includegraphics[width=60mm]{Strekktesting av kalottduk.pdf}
	\caption{FXC Skjematisk oppbygging - på bakken}
\end{figure}

Når flyet klatrer vil lufttrykket minke og luft vil sige ut av begge kammere, både gjennom kapillærrørene og luftslangen. Aneroiden ekspanderer, og luftslangen stenges på den innstilte høyden (1000 fot). Luft strømmer nå kun ut gjennom kapillærrørene fra begge kamrene. Trykksenkningen skjer imidlertid såvidt langsomt at begge kamrenes trykk er utlignet med det utvendige trykket, og ingenting vil skje. Samtidig vil nålen bevege seg ut av vinduet og forsvinne på veien oppover med flyet. Nålen vil først komme til syne igjen når en nærmer seg 1000 fot på nedtur. Kamrene er i balanse før utsprang.

\begin{figure}
	%\includegraphics[width=60mm]{Strekktesting av kalottduk.pdf}
	\caption{Under klatring med flyet}
\end{figure}

Ved utsprang begynner det utvendige trykket å stige raskt, i og med at hopperen faller nedover. Luft vil igjen strømme inn i kamrene, gjennom de respektive kapillærrør, men trykkutjevningen skjer ikke like fort som det utvendige trykket. Når hopperen utløser fallskjermen på 3000 fot kommer kamrenes trykk til å være innbyrdes like, men allikevel mindre enn det utvendige trykket. Når hopperen flyr videre nedover i åpen skjerm fortsetter trykkutjevningen inn i begge kamrene, og ``tar igjen'' det utvendige trykket. Denne trykkutjevningen skjer høyere enn 1000 fot.

\begin{figure}
	%\includegraphics[width=60mm]{Strekktesting av kalottduk.pdf}
	\caption{Ved utsprang}
\end{figure}

Ved nedstigning øker trykket og aneroiden blir presset sammen. Idet en passerer 1000 fot vil aneroiden være så sammenpresset at den har åpnet for ventilen til slangen som går til det høyre kammeret. Luft vil da kunne passere gjennom luftslangen og til ene siden av membranen i variometeret.

Er fallhastigheten mindre enn 40 fot/sek, vil ikke trykkforskjellen mellom kamrene være stor nok til at membranen beveger seg og åpneren vil derfor ikke fyre.

\begin{figure}
	%\includegraphics[width=60mm]{Strekktesting av kalottduk.pdf}
	\caption{Ved normal skjermåpning}
\end{figure}

Ved en hastighet over 40-65 fot/sek vil dette utgjøre en så stor trykkendring at luft vil strømme hurtigere inn i det høyre kammeret enn det venstre, og det vil oppstå en vesentlig trykkforskjell mellom disse. Dette vil føre til at membranen beveger seg med den følge at mothakene slipper og åpneren vil fyre.

\begin{figure}
	%\includegraphics[width=60mm]{Strekktesting av kalottduk.pdf}
	\caption{FXC Ved høy hastighet under 1000 fot}
\end{figure}

\subsubsection{Lavtrekk og harde svinger}
Fabrikanten anbefaler at hovedfallskjermen skal utløses minst 1500 fot over den innstilte utløsningshøyden for FXC’en. Meningen er at hopperen skal henge i hovedskjerm på høyere høyde enn 1000 fot slik at trykket i kamrene rekker å utjevne seg med det utvendige trykket. Om ikke kamrene rekker å få utlignet trykket før 1000 fot får høyre kammer et høyere trykk enn det venstre da luftslangen åpner, og membranen kan bevege seg enten så mye at utløsning skjer, eller såvidt lite at mothakene bare beveger seg noe. Dette kan igjen gi feilutløsning på senere hopp.

Det samme kan til en viss grad skje ved harde spiralsvinger.

FXC skal ved mistanke om ``nesten-fyring'' alltid spennes opp igjen. (Se 8.8.2.4.1, Oppspenning).

\subsection{Brukerinstruks}
Åpneren skal \textbf{alltid} stå i ``JUMP'' posisjon. Dette gjelder også ved forsendelse.

Den skal kun stå i ``OFF'' posisjon i kortere tidsrom,
\begin{itemize}
\item ved nedstigning med flyet
\item ved vannlanding
\item landing i dyp snø
\item før lokket smelles igjen når åpneren blir lagt i ett tett bagasjerom på bil for transport.
\end{itemize}

Årsaken til dette er at pakningen som stenger for ventilen til luftslangen kan bli deformert ved at den er tett tilsluttet ventilåpningen. Denne deformasjonen kan påvirke utløsningshøyden ved at den ikke tetter tilstrekkelig ved alminnelig bruk.

Åpneren skal innstilles før hver hoppdag. Det er ikke nødvendig å gjøre noe i løpet av hoppdagen hvis ikke lufttrykket har endret seg eller åpneren har vært skrudd av. Innstillingshøyden kontrolleres før hvert hopp.

\subsubsection{Kalibrering (HL/HFL)}
Det anbefales at hoppleder starter dagen med kalibrering av nødåpnerne samtidig med hans kontroll av utstyret på feltet.

Før åpneren skal kalibreres bør den ligge minst 15 minutter i utetemperatur slik at alle komponenter har fått samme temperatur (utetemperatur). Hvis ikke, vil vi kunne få en feil innstilt åpner på grunn av temperaturbevegelser i materialene. Åpneren innstilles/kalibreres ved at man gjør følgende:

Åpneren skal stå i ``JUMP'' posisjon.

Juster høydestillingsskruen. Til dette formål passer ett gammelt kronestykke bra:

Plasser kronestykket i sporet i skruen og vri sakte slik at nålen beveger seg i ønsket retning. Synes ikke nålen betyr det at trykket har endret seg såvidt mye at den er under åpningen i plata. Skru med klokken for at pila går oppover. Pass på at ikke du presser for hardt i sporet med kronestykket. Inne i sporet finnes en festeskrue for aneroiden, og denne kan ødelegges ved for stort trykk i sporet, eller ved bruk av for skarp eller lite justeringsverktøy.

Juster åpneren til 1000 fot. Vær oppmerksom på skyggevirkning av nåla. Nålas tykkelse tilsvarer cirka 100 fot. Nåla skal stå midt på streken for 1000 fot.

Justering med hensyn til bakkenivå og lufttrykk skjer automatisk når åpneren innstilles på denne måten. Samme prosedyren brukes hvis åpneren skal innstilles til å fyre ved andre høyder.

\subsubsection{Instruks for Hoppmester (HM)}
Før innlasting i flyet sjekkes det at nødåpneren står på ``JUMP'' og er innstilt på riktig høyde. I tillegg kontrolleres at luftslangen og betjenings-delen ikke har fått synlige ytre skader.

På baksiden under reserveskjermens dekklaff kontrolleres det at kabelføringen sitter godt festet og ikke har fått skader.

Likedan kontrolleres at øyet på kabelen er riktig festet til utløserhåndtakets pinne. På Campus-riggen skal øyet sitte innerst på pinnen. Se Figur 8 - Oversikt over reserveskjermens dekklaff Campus 1

Elevene skal drilles i den nødprosedyre de har lært på kurset UTEN å tenke på at de benytter nødåpner. Hoppmester må sørge for at utsprangs/trekkhøyde er minimum 1500 fot over den høyden åpneren er innstilt på.

Siden åpneren er innstilt på 1000 fot, skal åpningshøyden være min 2500 fot. For elever er åpningshøyden 3000 fot.

Hvis flyet av en eller annen grunn må gå ned og lande, må alle åpnere skrues av (``OFF''-stilling). Ved hurtig nedstigning med flyet kan åpneren fyre ved passering av den innstilte høyden. På grunn av turbulent vind inn gjennom døren kan det bli brå trykkforandringer og åpneren kan fyre hvis vi er på den innstilte høyden eller lavere.

\subsection{Montering og pakking}
\subsubsection{Oppspenning}
\textbf{NB} – FXC skal alltid demonteres fra utstyret for oppspenning, det er ikke tillatt å spenne opp eller trekke i wiren mens kabelføringen er festet på reserveklaffen. Dette medfører slitasje på indre ``usynlige'' deler.

Kabelføring, luftslange og betjenings-del med tilkoblinger inspiseres for synlige skader.

Nødåpneren spennes opp i JUMP posisjon.

En egnet bøyle for FXC er det beste å bruke; en rund skrutrekker gjennom øyet i enden av kabelen kan også benyttes. Plassér bøylen rundt foten, (eventuelt stå på skrutrekker), og trekk i selve hoveddelen av åpneren, rett oppover. IKKE trekk i kabelføring eller luftslange.

Trekk rolig \underline{HELT UT} til den stopper og hold et par sekunder. Trekk ikke for hardt, eller videre etter at wiren er trukket i ytre stilling, dette kan skade triangelfjæren som styrer utløsermekanismen.

Ikke bruk mer kraft enn nødvendig. Slipp rolig tilbake til det stopper opp.

\begin{figure}
	%\includegraphics[width=60mm]{Strekktesting av kalottduk.pdf}
	\caption{Korrekt oppspenning}
\end{figure}

\subsubsection{Montering i Campus-riggen}
Dette er en montering alle ompakkere av reserveskjerm må kjenne til.
\begin{enumerate}
\item Inne i reservepakksekken er det en lomme på venstre side. Nødåpnerens hoveddel plasseres i denne lommen med dataplaten opp og luftslangen til venstre.
\item Velcroen som skal holde åpneren på plass i lommen, festes. Deretter tres kabelføringen for utløserkabelen gjennom hullet i tøystrømpen til kabelføringen for reservehåndtaket og under denne.
\item Kabelføringen festes til reserveskjermens øvre dekklaff med en festebøyle og 2 skruer med muttere. Mutterne skrues godt fast NB! Mutterne skal ikke brukes mer enn en gang. Hvis de brukes om igjen, kan låseringen av plast være ødelagt og mutterne vil kunne løsne etterhvert. Skruene skal være tilpasset i lengde slik at de stikker ut og kan hekte fast LOR lina.
\item Åpne velcroklaffen over venstre reserveløftestropp og legg luftslangen helt til venstre og langs kanten av denne.
\item Åpne velcroklaffen på luftslangens strømpe som sitter ved reservehåndtaket. Slangen legges i strømpen og denne låses med velcro.
\item Fest betjeningsdelen: På baksiden av betjeningsdelen er det en festeplate med to skruer. Skru ut disse skruene slik at platen løsner.
\item På seletøyet like under håndtaklommen til reserveskjermen er det påsydd en bit webbing. Før festeplaten under denne og skru den fast til betjeningsdelen igjen.
\end{enumerate}

Når dette er gjort, skal en gå over og kontrollere at alt er riktig utført og ingenting er glemt. Sjekk at luftslange og kabelføring er riktig tredd og at kabel-føringen er godt festet til reserveskjermens dekklaff. Nødåpneren er nå montert og reserveskjermen kan pakkes.

\subsubsection{Montering i andre systemer:}
Andre systemer er ikke like godt tilrettelagt for FXC - 12000 som Campusriggen.

Før montering i disse systemene henvises til fallskjermreparatør eller fabrikant som sitter inne med monteringsanvisning for de fleste systemer. De fleste av disse systemene krever også syarbeid ved montasjen.

For en materiellkontrollør som får en slik rigg med nødåpner for hovedkontroll, anbefales å ta kontakt med produsent eller materiellreparatør. Det kan være spesielle ting ved hvert enkelt system som en skal være spesielt på vakt overfor.

\subsection{Kontroll og vedlikehold}
FXC 12000 krever svært lite vedlikehold. Det eneste vi kan gjøre selv er å børste av støv og skitt.

Ved bytte av glass, som er en vanlig slitasje skade, må åpneren inn til reparatør, da lukkingen av glasset har innvirkning på justeringsskruen.

En nødåpner er et presisjonsinstrument og skal selvfølgelig behandles deretter. På Campus-riggen er det meste av åpneren godt beskyttet inne i reservepakksekken når riggen er montert på ryggen. Når skjermen skal pakkes derimot, er åpneren svært utsatt for slag gjennom bakveggen i reservecontineren. Dette gjelder også ved forflytning av utstyret. Betjeningsdelen og deler av luftslangen sitter også relativt ubeskyttet.

Tillat aldri at noen kaster fra seg sekk/seletøyet på bakken. Betjeningsdelen kan f.eks treffe en stein og bli skadet. Like selvfølgelig er det at riggen aldri skal slepes langs bakken eller ligge på fuktig underlag.

Det anbefales å montere på en velcrobeksyttelser rundt Betjeningsdelen. Denne tar av for skrubbskader og lignende.

Skulle åpneren få ytre skader skal den tas ut av aktiv hopping og sendes fabrikken for kontroll og eventuell reparasjon. Selv ved små skader utvendig, kan en ikke vite hva som har skjedd inne i åpneren.

Vær spesielt oppmerksom på at FXC’ens fyringskarakteristikk kan påvirkes av at den blir utsatt for fuktighet. Når en tar med kalde nødåpnere inn i varme rom vil det utfelles fukt fra den varme og fuktigere lufta på nødåpneren. Det er alltid en del partikler oppløst i denne fuktigheten. Når fuktigheten tørker fra nødåpneren, vil mange av disse partiklene bli igjen i kamrene i hoveddelen. Partiklene vil kunne påvirke luftgjennomstrømningen i de små kapilærrørene. Vi har tidligere opplevd i F/NLFs klubber at FXC’er har fyrt ved lavere hastigheter enn spesifikasjonen, fordi kapilærrør var delvis blokkert av partikler utfelt fra fukt. Unngå å utsette FXC’er for unødige temperatursvingninger. Unngå koking og andre aktiviteter som øker luftfuktigheten i de rom hvor FXC skal oppbevares.

Hvis åpneren har kommet i kontakt med snø, børste forsiktig av snøen. Unngå å blåse, da fuktig luft fra lungene vil kunne få deler til å fryse. På Campus-riggen er det kun betjeningsdelen som vil være utsatt for snø.

Ved landing i vann må åpneren sendes reparatør så snart som mulig for inspeksjon og reparasjon.

Det vi kan gjøre for å begrense korrosjonsskader er å legge den våte åpneren i en tett plastpose. Klem eller sug ut luften, og tett igjen posen så godt som mulig. Det er bedre at åpneren ankommer gjennomvåt til reparatør – enn at det forsøkes med egen tørking før forsendelse. Har åpneren vært i kontakt med saltvann, skal den legges i rent vann før den sendes reparatør.

\subsubsection{Funksjonstest:}
F/NLF v/SU har bestemt at alle FXC 12000 nødåpnere som er oppdatert til ``J'' standard, skal sendes til funksjonstest i trykkammer med minst 7 måneders mellomrom.

Dette innebærer en visuell inspeksjon, samt testing av fyringshøyde og -hastighet, samt test at FXC ikke fyrer i ``OFF'' posisjon. Ved godkjent test returneres åpneren tilbake til aktivt bruk, ellers sendes den til godkjent reparatør for nødvendig reparasjon og utskifting av deler, samt kalibrering til godkjente spesifikasjoner.

Tidligere ble det benyttet en ``plastposetest'' for kontroll av funksjon. Erfaringer med dette har vist at denne testen gir såvidt store trykkendringer i aneroiden slik at disse har blitt skadet. ``Plastposetesten'' utgår derfor, og må ikke brukes.

Åpneren må spennes igjen etter funksjonstest før den monteres og kobles til reserveskjermens pinne (se side 13, Oppspenning). I tillegg skal synlige deler av FXC ettersees ved hver brukskontroll.

Hvis skruene som fester kabelføringen til reserveskjermens dekklaff er skrudd ut skal mutterne byttes i nye for at plastringen i dem skal gi skikkelig låseeffekt.

Reserveskjermen pakkes slik manualen viser og låses med pinnen fra utløserhåndtaket. Når man setter på øyet fra nødåpneren skal dette sitte innerst på pinnen. Så kommer ringen fra LOR systemet og deretter reservens låseløkke.

\begin{figure}
	%\includegraphics[width=60mm]{Strekktesting av kalottduk.pdf}
	\caption{Oversikt over reserveskjermens dekklaff Campus 1}
\end{figure}

\begin{enumerate}
\item Reservekabel
\item Styrering for reservekabel
\item Øye for FXC kabel
\item Reservepinne
\item Ring (eller løkke) for LOR line
\item Låseløkke
\item LOR bånd
\item Kabelføring for reservekabel
\item Kabelføring for FXC kabel
\item FXC kabel
\end{enumerate}

\begin{figure}
	%\includegraphics[width=60mm]{Strekktesting av kalottduk.pdf}
	\caption{Oversikt over reserveskjermens dekklaff Campus2}
\end{figure}

\subsubsection{Fabrikkvedlikehold}
FXC 12000J i F/NLFs klubber ble vedlikeholdt av FXC Corporation i USA i januar- februar 2002 og januar-februar 2004.

Avhengig av resultater fra funksjonstesting av åpnerne i perioden 2004-2006, og den generelle utvikling fra fabrikanten vil det senere bli avgjort om åpnerne skal inn til fabrikanten for vedlikehold.

\section{CYPRES}
Det følgende beskriver bruk og virkemåte for Cypres elektronisk nødåpner.

Cypres leveres i tre forskjellige utgaver:
\begin{description}
\item[Ekspert versjon:] Fyringshøyde cirka 750 fot over bakkenivå – fyringshastighet over 125 km/t. Har rød AV/PÅ-knapp.
\item[Elev versjon:] To fyringshøyder: \\
Fyringshøyde cirka 1000 fot over bakkenivå – fyringshastighet mellom 47 km/t og 125 km/t \\
Fyringshøyde cirka 750 fot over bakkenivå – fyringshastighet over 125 km/t. \\
Har gul AV/PÅ-knapp merket ``Student''
\item[Tandem versjon:] Fyringshøyde cirka 1 900 fot over bakkenivå – fyringshastighet over 125 km/t. \\
Har blå AV/PÅ-knapp merket ``Tandem''
\end{description}

Cypres finnes for både én og to pins reservecontainere i alle utgaver, og leveres i både fot og meterversjon.

\subsection{Beskrivelse og virkemåte}
Cypres baserer seg på en mikroprosessor som kalkulerer den ``sanne'' fallhastighet og høyde over bakken ved hjelp av målinger av trykket. Den ble utviklet av Helmut Cloth, fra firma Airtec i Wünnenberg i Tyskland i 1990. Målet for utvikling av en Cypres var å utvikle en nødåpner som i stor grad også ble akseptert og brukt av andre enn elever. Den måtte blant annet være absolutt til å stole på, være lett under kalibrering og bruk, den måtte være liten og lett, og grei å installere inne i et hvilket som helst fallskjermsett.

Fram til 2003 er det solgt cirka 75 000 enheter. Det er i IPC/FAIs årlige Safety Survey anslått at mer enn 50 hoppere redder livet på verdensbasis årlig fordi nødåpner brukes. Den er forlengst blitt akseptert av erfarne hoppere, og har i stor grad bidratt til at flere og flere hopper med en nødåpner på reserven.

Cypres består av tre deler, samt to kabler som forbinder disse:
\begin{description}
\item[Prosessorenheten (hoveddelen)] som inneholder mikroprosessoren. Denne monteres på veggen mellom reserve- og hovedcontainer.
\item[Kontrollenheten] som virker som av/på bryter og kontroll for virkemåte og innstilling. Denne monteres vanligvis under reserveklaffen, eller et annen lett tilgjengelig sted.
\item[Fyringsenheten] som monteres rundt reserveloopen og skjærer av denne ved aktivisering.
\end{description}

Både prosessenheten og fyringsdelen er montert inne i reservepakksekken.

I motsetning til mekaniske åpnere er Cypres konstruert for å virke uavhengig av det eksisterende åpningssystem på reserven. Bortsett fra kontrollenheten er Cypres montert inne i reservecontaineren både for funksjon og beskyttelse. Dersom hopperen passerer eller befinner seg under den innstilte høyden over en angitt hastighet (pkt 2), blir et fyringssignal sendt til frigjøringsenheten som ved hjelp av en fyringsmekanisme skjærer av reserveloopen. Resten av reserveåpningen vil foregå som normalt, ved at reservepiloten ikke har noe som holder den igjen inne i reservecontaineren.

Cypres er konstruert for å virke i 14 timer av gangen. Når den slås på foretas en omfattende selvtest for å forsikre om at det er nok batteristyrke, samt at det er full funksjonalitet mellom alle deler.

Cypres har en egen innebygget kontrollfunksjon, der den kontinuerlig overvåker trykket i hele perioden den er påslått. Endrer trykket seg i løpet av dagen, vil Cypres selv endre sitt eget nullpunkt. Når flyet tar av vil trykkendringen være såvidt stor over kort tid at Cypres låser seg på det sist avleste bakketrykk, og bruke dette som utgangspunkt for beregnet fyringshøyde.

Etter at Cypres er påslått, og ferdig med egenkontroll er det således ikke mer hopperen trenger å foreta seg under bruk. Cypres slår seg av automatisk etter 14 timer.

\subsubsection{Prosessorenheten}
Hoveddelen inneholder som nevnt en avansert mikroprosessor, en kontrollenhet for egenkontroll, samt batterier.

Alle data om lufttrykk blir samlet og kalkulert til en beregnet kurve for fallhastighet og høyde som er tilnærmet det riktige, uavhengig om kroppsstilling etc hos hopperen. Slik beregnes høyden for om Cypres sender fyringssignal eller ikke. I motsetning til en mekanisk åpner kan ikke Cypres ``nesten-fyre'', og være i en ustabil stilling ved senere hopp. Elektronikken sørger for at det enten sendes fyringssignal eller ikke til fyringsenheten.

Vanligvis trengs kun to parametere (høyde og hastighet) for å initiere fyring. For i størst mulig grad å forsikre seg mot feilfyring er det innebygget et avansert regneprogram i Cypres som foretar ytterligere 5 forskjellige egentester. Kun når alle disse 7 parametrene gir godkjent resultat vil Cypres fyre.

Batteriene er spesialutviklet for Cypres sitt virkeområde. Disse skal byttes annethvert år eller etter cirka 500 hopp. Under selvtesten ved start sjekker Cypres batteristyrken, og at det er nok strømstyrke til å kunne fyre av ladningen i løpet av de neste 14 timer etter at den er påslått. I tillegg gir også enheten et signal om gjenværende batteristyrke på displayet under oppstart. Dette viser seg som et tall mellom 5700 og 6900 som blir stående et par sekunder, som indikerer en batteristyrke på mellom 5,7 og 6,9 volt.

Dersom det etter oppstartsprosedyre vises et tall på 8996, 8998 eller 8999 er dette et signal om at batterikapasiteten er såvidt lav at det ikke kan påregnes fyring i 14 timer fremover. Deretter slår enheten seg av. Det henvises videre til pkt. 1.4 i Cypres brukermanual, samt instruks for batteribytte, se punkt for vedlikehold.

\subsubsection{Kontrollenheten}
Kontrollenheten er den eneste synlige delen av Cypres som vanligvis monteres under reserveklaffen, eller i en egen lomme konstruert for formålet. Etterhvert er det blitt vanlig at alle seletøyfabrikanter leverer seletøy med ferdig oppsett for Cypres.

Kontrollenheten fungerer som PÅ/AV knapp, og en kontroll for bruker at Cypres gjennomgår og godkjenner sin egen selvtest.

For å forhindre uønsket på– og avslåing er prosedyren ved på– og avslåing laget på en interaktiv måte, der bruker må trykke på PÅ/AV knappen når Cypres gir fra seg et lysblink som sier fra at den er klar til å ta imot signal.

Prosedyre for på– og avslåing er like, men i enkelte tilfelle kan det virke som om den reagerer forskjellig på tastetrykket på disse to. Det sliter mindre på batteriet om Cypres slår av seg selv, enn ved å gjøre dette manuelt.

Kontrollenheten viser oppstartsprosedyren i et digitalt display, ved hjelp av en nedtelling fra 9999 mot 0. Når displayet viser ``0'' er enheten aktivert og klart til bruk, og gjennomgått egentest. Dersom det forekommer feil i oppstartsprosedyren viser den et firesifret nummer. Disse tall henviser til feilkoder som forklart i kapittel 5 i Cypres brukermanualen.

\textbf{NB: Dersom displayet viser 8996, 8998 eller 8999 må batteriet byttes før Cypres kan tas i bruk igjen.}

\begin{itemize}
\item Kontrollenheten kan også benyttes til å innstille annen fyringshøyde ved hopping på annet elevasjon i forhold til stedet der flyet tar av. Dette er utførlig beskrevet i Cypres manualens pkt. 4.4, se forøvrig 8.8.3.2.2 Hopping på andre hoppfelt
\end{itemize}

\begin{figure}
	%\includegraphics[width=60mm]{Strekktesting av kalottduk.pdf}
	\caption{Kontrollenheten}
\end{figure}

\subsubsection{Fyringsdelen / kutter}
Fra prosessorenheten går det en kabel ut til fyringsdelen. Denne består av en liten hylse med rustfri stålhylse, der det er et liten eksplosiv ladning, et stempel og en skarp kniv i. Det er et hull i hylsa der reserveloopen skal tres i ved pakking.

Ved aktivisering gis det et strømsignal til en tenner, hvorpå ladningen eksploderer og skyver kniven frem, og tvers forbi hullet, og skjærer over låseløkka.

Inne i hullet for loopen er det en plastikk eller silikonbeskyttelse for beskyttelse av indre deler av fyringsdelen, og skal ettersees ved montering og pakking.

\begin{figure}
	%\includegraphics[width=60mm]{Strekktesting av kalottduk.pdf}
	\caption{Fyringsdelen}
\end{figure}

\subsubsection{Tekniske data Cypres:}
\begin{description}
\item[Mål:] 89*57*28 mm (hoveddelen)
\item[Vekt:] 260 gr.
\item[Åpning:] Se versjonsforskjeller pkt. 2.0
\item[Arbeidstemperatur:] -20° C til +63° C (temperatur på selve enheten, skal ikke forveksles med utetemperatur).
\item[Lagringstemperatur] -25° C til +71° C
\item[Fuktighet:] opp til 98\% relativ fuktighet
\item[Høyderegulering:] ± 1500 fot
\item[Virkeområde:] minus 1500 fot til pluss 24 000 fot MSL
\item[Arbeidstid:] 14 timer
\item[Batteri levetid:] cirka 500 hopp eller 2 år
\item[Vedlikehold:] 4 og 8 år etter produksjonsdato hos produsenten
\item[Total levetid:] 12 år +3 mnd
\end{description}

\subsection{Brukerinstruks}
\subsubsection{Innstilling}
Cypres er konstruert for en meget enkel brukerterskel, der bruker kun slår Cypres PÅ og eventuelt AV etter hopping.

Det er ikke nødvendig å slå Cypres av etter endt hopping, den vil automatisk slå seg av 14 timer etter den ble slått på. Det tar noe mer strøm å slå den av enn å la den slå seg av selv.

Se Cypres manualen for korrekt PÅ- og AV-slåingsprosedyre.

Alle feilbeskjeder gis under oppstartsprosedyren via displayet på kontrollenheten. Når displayet viser ``0$\downarrow$'' er Cypres aktivert og klar til bruk i 14 timer.

\subsubsection{Hopping på andre hoppfelt}
Kontrollenheten brukes også til å stille inn eventuell annen høyde ved landing på andre elevasjoner. Fremgangsmåten for dette er følgende:
\begin{enumerate}
\item Ved siste (fjerde) trykk i oppstartsprosedyren, holdes på knappen nede helt til Cypres har gått gjennom sin oppstartsprosedyre.
\item Cypres gå inn i en justeringsmodus, der tall for ny elevasjon foreslås, sammen med en pil som peker oppover eller nedover. Cypres ``spør'' med dette brukeren om høyden på det nye landingsområdet i forhold til ekisterende høyde.
\item Numrene blinker med 30 fots (10 meter på meterversjon) økning fra gang til gang.
\item Når den riktige høydeforskjellen vises, sammen med angitt pil om hoppfeltet er høyere eller lavere enn der den kalibreres, slippes knappen. Den nye høyden vil vises i panelet.
\item Ved landing på det nye hoppfeltet vil Cypres etter en kort stund justere seg til denne elevasjonen, og vise ``0$\downarrow$'' i displayet.
\item Dersom en for eksempel kjører bil tilbake til flyplassen for å hoppe igjen, må Cypres slås av, og kalibreres en gang til på samme måte.
\end{enumerate}

Merk at pilene som peker opp og ned indikerer hvor det nye landingsområdet er \textbf{i forhold til det stedet Cypres innstilles på} og ikke motsatt.
\begin{itemize}
\item Når pilen peker oppover - $\uparrow$ - betyr det at det tiltenkte landingsområde er x fot \textbf{høyere} enn der Cypres blir påslått.
\item Når pilen peker nedover - $\downarrow$ - betyr det at det tiltenkte landingsområde er x fot \textbf{lavere} enn der Cypres blir påslått.
\end{itemize}

\textbf{Husk at Cypres alltid skal kalibreres der en tar av med flyet, og ikke på det tiltenkte landingsområde.}

\subsubsection{Rekalibrering}
For vanlig hopping er Cypres kalibrert når den er påslått og kalibrerer seg selv under hele tiden den er påslått (14 timer). I enkelte tilfelle kan det være ønskelig å rekalibrere Cypres. Dette skjer ved at Cypres slås av manuelt og deretter slås på igjen.

Cypres kalibrere seg iht. det stedet en lander etter kort tid.

Grunner for å rekalibrere Cypres kan være:
\begin{itemize}
\item Utelanding der landingsstedet er høyere eller lavere enn det stedet Cypres ble opprinnelig kalibrert. Dette gjelder også hvis transport tilbake til landingsområde er i et område med større høydeforskjeller enn 10 meter.
\item Utstyret blir fraktet vekk fra kalibreringsstedet og så tilbake igjen (f.eks. blir med på en biltur).
\item Hvis den totale flygetid for løftet er over halvannen time må Cypres rekalibreres ved landing. (Den fungerer allikevel normalt under hoppet).
\item Når flyplass og landingsområde er på forskjellige steder må Cypres kalibreres på det stedet en tar av fra. Dette gjelder både om det er eller ikke er elevasjonsforskjell mellom flyplass og landingsområde.
\item Ved landing og eventuell videre hopping på nytt landingsområde må Cypres rekalibreres.
\item Dersom en er i tvil om Cypres er riktig kalibrert på grunn av en eller kombinasjoner av ovenstående er det å anbefale å rekalibrere.
\end{itemize}

\subsubsection{Vannhopp}
Ved vannhopp må Cypres demonteres, og ikke monteres tilbake før utstyret er fullstendig tørket. Ved kort kontakt med vann, trenger det ikke å ha skjedd noe med Cypres. Dette kan kontrolleres ved en fuktighetsinspeksjon i lomma.

Ved tvil om Cypres er utsatt for fuktighet må den sendes til produsent for kontroll.

\subsubsection{Feilmeldinger}
Dersom Cypres avgir feilmelding under oppstartsprosedyre må feilen utbedres før den igjen er luftdyktig. Ved melding om lav batterispenning må batteri skiftes omgående, selv om det er mulig å få startet opp Cypres igjen ved flere forsøk.

\begin{table}
	\caption{Feilmeldinger Cypres}
	\begin{tabular}{ | p{2cm} | p{5cm} | p{3cm} | }
		\hline
		Feilmelding & Årsak & Feilretting \\
		\hline
		8999 8998 & Lav batterispenning & Bytt batteri før neste hopp. \\
		\hline
		8997 & Ikke kontakt med fyringsenhet & Mulig kabelbrudd - kobling \\
		\hline
		100 eller 4000 & Større variasjoner i den omgivende luft under selvtest & Prøv ny oppstart \\
		\hline
		9999 9998 9997 9996 & Cypres klarer ikke å måle jevnt lufttrykk. Kan være forsøkt påslått i heis, eller i en bil under høydeeendring & Vent, og prøv ny oppstart \\
		\hline
		5000 8995 8994 8993 8992 8990 &Kan stoppe opp selvstartprosedyre med ett av disse nummer & Start opp Cypres 6 ganger etter hverandre for normal oppstart. Dersom feil gjentar seg, må enhet til fabrikant for kontroll.\\
		\hline
	\end{tabular}
\end{table}

\subsection{Montering og pakking}
Cypreslomme og føringer må kun installeres av Materiellreparatør som har utstyr autorisert av Airtec.

Materiellkontrollør skal ha inngående kjennskap til Cypres og installasjon ved pakking og montering av Cypres. Fabrikanten har utarbeidet ``Cypres Rigger’s Kit'' som skal forefinnes ved montering som beskriver korrekt montering i de forskjellige seletøy.

For pakking skal det brukes originaldeler som finnes i ``Cypres Packer’s Kit''. Merk at sjekkliste skal brukes under pakking/montering av Cypres.

Cypres kan installeres i alle kjente seletøy. Fabrikanten har utarbeidet en egen installasjonsmanual med video, der installasjon i forskjellige seletøy blir gjennomgått i detalj. Denne skal følges ved installasjon.

Dersom utstyret allerede er satt opp for Cypres må det vises stor forsiktighet med følgende (ref pkt 3 i Cypres brukermanual):

\begin{itemize}
\item Løft alltid Cypres i hoveddelen, aldri i kablene. Trekk heller aldri i kablene ved demontering.
\item Vri aldri kablene.
\item Legg alltid den tynne kabelen innerst dersom kablene overlapper hverandre.
\item Vis forsiktighet med fyringsenheten slik at det innvendige belegget ikke skades.
\item Bruk \textbf{kun} original Cypres låseløkke fra fabrikanten.
\item Bytt låseløkke ved hver ompakk og ved slitasje
\item De ytterste fire cm av låseløkke skal settes inn med silikon. Benytt Airtec's ``CYPRES Loop Silicon''.
\item All slakk i kabler skal legges inn i velcroklaffen på monteringslomma.
\item Ved montering i strikklomme må kablene som går ut av prosessorenheten ligge inn mot veggen mellom reserve og hovedcontainer.
\end{itemize}

\begin{figure}
	%\includegraphics[width=60mm]{Strekktesting av kalottduk.pdf}
	\caption{Plassering av prosessorenhet}
\end{figure}

Vanligvis skal fyringsdelen monteres så nært toppen av pilotskjermen som mulig. På enkelte rigger (bl.a. Vector 2 og 3) kan Cypres monteres på klaff 1, dvs nærmest innerbagen for firkantreserve.

\textbf{NB: På alle utgaver av Vector 1 skal fyringsenheten kun monteres på undersiden av klaff 3 – dvs ligge an mot toppen av pilotskjermen. Dette bør spesielt sjekkes ved pakking av Vector 1.}

\textbf{På Vector 2 \& 3 er fyringsenheten plassert på klaff 1.}

Materiellkontrollør skal forøvrig ha kjennskap til hver enkelt seletøys monteringsan- visning for installasjon for de utstyr han/hun pakker og hovedkontrollerer.

\subsection{Kontroll og vedlikehold}
Kontroll av Cypres begrenser seg stort sett til visuell kontroll av kabler, fyringsdel og kontroll under oppstartsprosedyre.

Cypres er vedlikeholdsfri, bortsett fra batteribytte etter 2 år eller 500 hopp, og en funksjonskontroll hos produsenten hvert fjerde år. Disse kontroller er obligatorisk for at Cypres skal fungere tilfredsstillende under alle forhold.

\textbf{Vær spesielt påpasselig med disse intervallene når det gjelder tandem– og elevutstyr.}

\textbf{Der Cypres erstatter LOR skal fabrikantens vedlikeholdsintervaller overholdes for at utstyret er luftdyktig.}

\subsubsection{Batteribytte}
Batteribytte er relativt enkelt, og kan foretas av Materiellkontrollør. Prosedyren er utførlig beskrevet i Cypres brukermanualen, pkt 6.2. Det presiseres at det kun skal nyttes originale batterier levert av produsenten.

Ved batteribytte skal det også anføres på oransje merkelapp utenpå prosessorenheten når batteribytte har skjedd, samt på kontrollkort inne i batterihuset, og i tillegg på hovedkontrollkortet for utstyret.

Vis stor forsiktighet med kabler og koblinger ved batteribytte.

Etter at batterier er byttet skal Cypres slås på og av minimum fire ganger etter hverandre, og kontrolleres for vanlig oppstart. Dette gjøres for at batteriene igangsettes fra en ``dvale'' (lagrings) tilstand.

Cypres batterier byttes annethvert år eller etter 500 hopp.

\subsubsection{Cypres 4 års service}
Cypres skal inn til test hos produsenten hvert fjerde år, og Airtec godtar et avvik på +/- tre måneder slik at bruker skal kunne tilpasse dette til ompakksyklus. Airtec opplyser spesielt at fireårs kontrollen er hvert fjerde år etter produksjonsdatoen, uavhengig av tidligere kontroller.

Kommer åpnerne tidligere eller senere enn 4 år +/- 3 måneder, får ikke Airtec tatt en komplett 4-års sjekk. Grunnen til dette er at Airtec sammenligner data fra én åpner med data fra de andre åpnerne i samme produksjonsperiode. Dermed kan én åpner vise ingen avvik i forhold til basiskravene, men den kan ha avvik i forhold til de andre åpnerne som den produsert sammen med. Airtec må da foreta endel utvidede tester for å tilfredsstille de statistiske prosedyrer som sikrer høy standard på åpnerne. Kommer åpneren for sent, kan ikke Airtec bruke dataene å sammenligne med før det er gått fire nye år. Åpnerne må derfor inn når de er 4 og 8 år gamle.

4-års kontrollen består av blant annet følgende:
\begin{itemize}
\item Alle kabler blir visuelt og elektronisk testet.
\item Huset åpnes og metallbeskyttelsen mot elektromagnetisk påvirkning fjernes.
\item Enheten kulde- og varmetestes i ekstreme temperaturområder.
\item Tekniske data sammenlignes med data som ble registrert når enheten ble produsert. Avvik eller feil analyseres og rettes. Data sammenlignes også med statistiske data fra andre åpnere fra samme produksjonsperiode
\item Enheten utsettes for en rekke simulerte forhold som får den til å aktivere for å måle nøyaktighet.
\item Strømforbruk under forskjellige funksjonsfaser testes.
\item Selve kutteren testes elektronisk.
\item Eventuelle tekniske forbedringer og oppdateringer utføres.
\end{itemize}

Benytt originalemballasje ved innsendelse av åpner.

\subsubsection{Bytte av fyringsdelen / kutter}
Fyringsdelen delbar slik at kutter kan byttes ved fyring, uten å måtte sendes inn til produsenten. Dette er en relativt grei prosess og er vel dokumentert i manualen.

Tidligere måtte enheten inn til produsenten for bytte av kutter og fyringsdel. Ny delbar kutter blir montert på eldre typer etter fyring.

\textbf{Ved bytte av kutter er det viktig å koble fra batteriene, slik at ikke snikstrøm kan forårsake fyring under byttet.}

\section{CYPRES 2}
Våren 2003 avsluttet Airtec GmbH produksjonen av Cypres. Erstatteren Cypres 2 ble da introdusert. Det følgende beskriver bruk og virkemåte for Cypres 2 elektronisk nødåpner.

Cypres 2 leveres i tre forskjellige modeller:
\begin{description}
\item[Ekspert versjon:] Fyringshøyde cirka 750 fot over bakkenivå – fyringshastighet over 125 km/t. Har rød AV/PÅ-knapp.
\item[Elev versjon:] To fyringshøyder: \\
Fyringshøyde cirka 1000 fot over bakkenivå – fyringshastighet mellom 47 km/t og 125 km/t \\
Fyringshøyde cirka 750 fot over bakkenivå – fyringshastighet over 125 km/t. \\
Har gul AV/PÅ-knapp merket ``Student''

\item[Tandem versjon:] Fyringshøyde cirka 1 900 fot over bakkenivå – fyringshastighet over 125 km/t. \\
Har blå AV/PÅ-knapp merket ``Tandem''
\end{description}

Cypres 2 finnes for både én og to pins reservecontainere i alle utgaver, og leveres i både fot og meterversjon.

\subsection{Beskrivelse og virkemåte}
Cypres 2 består av prosessorenhet (hoveddel), kontrollenhet og fyringsenhet/kutter. Cypres 2 er konstruert for å fungere etter samme prinsipp som Cypres, men det er gjort følgende forbedringer:
\begin{itemize}
\item Cypres 2 er vanntett i 15 minutter på dyp inntil 5 meter i både salt- og ferskvann. Men filter må byttes hvis prosessorenheten har vært i direkte kontakt med vann.
\item Vedlikeholdsfri strømforsyning for bruker fordi batteri byttes ifb. med fabrikkvedlikehold. Det er derfor ikke behov for å følge opp dato for batteribytte eller telle antall hopp mellom hvert batteribytte.
\item Serienummer kan leses fra display i kontrollenheten.
\item Dato for neste fabrikkvedlikehold kan leses fra display i kontrollenheten.
\item Automatisk påminnelse om neste fabrikkvedlikehold under selvtest 6 mnd. før forfall.
\item Utvidet intervall for fabrikkvedlikehold til +/- 6 mnd.
\item Selvtest tar 10 sekunder
\end{itemize}

\subsubsection{Prosessorenheten}
Hoveddelen inneholder som nevnt en avansert mikroprosessor, en kontrollenhet for egenkontroll, samt batterier. Prossesorenheten på Cypres 2 skiller seg fra Cypres ved at den er noe mindre, har rundede hjørner og et filter. Filteret sikrer korrekt måling av endring i lufttrykk, samtidig som det hindrer vann å trenge inn i enheten. Åpninger til kabler til fyrings- og kontrollenhet er også forseglet for å unngå inntrengning av vann.

Filteret skal byttes hvis det har vært i kontakt med vann.

\subsubsection{Kontrollenheten}
Kontrollenheten er den eneste synlige delen av Cypres 2 som vanligvis monteres under reserveklaffen, eller i en egen lomme konstruert for formålet. Cypres 2 er beskyttet mot elektromagnetiske bølger. Kontrollenheten er forseglet for å hindre inntrengning av vann.

Kontrollenheten fungerer som PÅ/AV knapp, og en kontroll for bruker at Cypres gjennomgår og godkjenner selvtest.

Prosedyre for på– og avslåing er identisk for Cypres 2 og Cypres Det sliter mindre på batteriet om Cypres 2 slår av seg selv, enn ved å gjøre dette manuelt.

Kontrollenheten viser oppstartsprosedyren i et digitalt display, ved hjelp av en nedtelling fra 10 mot 0. Når displayet viser ``0?'' er enheten aktivert og klart til bruk, og gjennomgått egentest. Dersom det forekommer feil i oppstartsprosedyren viser den et firesifret nummer i ca. to sekunder. Deretter slår enheten seg av, og displayet blir blankt. Feilkoder er forklart i kapittel 5 i Cypres 2 brukermanualen.

\begin{itemize}
\item Kontrollenheten kan også benyttes til å innstille annen fyringshøyde ved hopping på annen elevasjon i forhold til stedet der flyet tar av. Rekkevidde er +/- 1500 ft. Dette er utførlig beskrevet i Cypres 2 manualens pkt. 4.4, se forøvrig 8.3.2.2 Hopping på andre hoppfelt
\end{itemize}

\begin{figure}
	%\includegraphics[width=60mm]{Strekktesting av kalottduk.pdf}
	\caption{Kontrollenheten}
\end{figure}

\subsubsection{Fyringsdelen / kutter}
Fyringsdel/kutter fra Cypres og Cyres 2 kan brukes om hverandre, men vær oppmerksom på at kutter fra Cypres ikke er vanntett.

Fyringsdel/kutter på både Cypres og Cypres 2 kan byttes uten å måtte sendes til fabrikk. ``Jackpluggen'' på Cypres 2 har en aluminiumshylse, og er vanntett.

\subsubsection{Tekniske data Cypres 2:}
\begin{description}
\item[Mål:] 83*43*32 mm (hovedenhet)

65*18*6,5 mm (kontrollenhet)

43*8 mm (fyringsenhet/kutter)

\item[Vekt:] 182 og 199g (hhv. Ekspert-/Tandem- og Elevversjon)
\item[Åpning:] Se versjonsforskjeller pkt. 2.0 i Cypres 2 manualen
\item[Arbeidstemperatur:] -20° C til +63° C (temperatur på selve Cypres 2 åpneren, skal ikke forveksles med utetemperatur).
\item[Lagringstemperatur] -25° C til +71° C
\item[Fuktighet:] opp til 99,9 \% relativ fuktighet
\item[Høyderegulering:] ± 1500 fot
\item[Virkeområde:] minus 1500 fot til pluss 26 000 fot MSL
\item[Arbeidstid:] 14 timer
\item[Batteri levetid:] Garantert innenfor vedlikeholdsintervall, dvs. 4 år.
\item[Vedlikehold:] 4 og 8 år etter produksjonsdato
\item[Total levetid:] 12 år etter produksjonsdato (anslått av produsent i 2003)
\end{description}

\subsection{Brukerinstruks}
\subsubsection{Innstilling}
Cypres 2 slåes AV og PÅ ved samme prosedyre som Cypres.

Trykk og slipp AV/PÅ-knapp. Vent til LED-lys lyser rødt. Trykk og slipp AV/PÅ knapp hver gang LED-lys lyser. LED-lys vil lyse i alt tre ganger. Etter siste trykk vil Cypres 2 starte selvtest og kalibrering. Enheten teller da ned fra ``10'' til ``0''. Selvtest og kalibrering tar 10 sekunder.

Etter korrekt selvtest og kalibrering vil display vise ``0?''.

For å slå av nødåpneren benyttes samme prosedyre som ved påslåing. Etter fjerde trykk blir displayet blankt, og enheten er avslått.

\subsubsection{Hopping på andre hoppfelt}
Ved hopping på hoppfelt med annen elevasjon benyttes samme prosedyre som beskrevet for Cypres, se 8.3.2.2 Hopping på andre hoppfelt.

\subsubsection{Avlesing av serienummer og neste vedlikehold}
Ved å holde AV/PÅ-knapp nede kontinuerlig etter fjerde trykk i oppstartsprosedyren, vil displayet i kontrollenheten gå gjennom alle mulige elevasjoner fra +/-30 ft til +/- 1500 ft. Om AV/PÅ-knapp holdes kontinuerlig nede etter displayet viser ”1500?” vil displayet bli blankt i 1⁄2 sekund, deretter vil enhetens serienummer vises i 5 sekunder. Hvis knappen fremdeles holdes nede vil displayet bli blankt i 1⁄2 sekund før neste vedlikehold vises. Vedlikeholdstidspunkt blir angitt som mm/åå.

\subsubsection{Rekalibrering}
Cypres 2 skal rekalibreres manuelt ved å foreta gjennomføre AV- og PÅ prosedyrer i de samme situasjoner som dette er nødvendig for Cypres.

Det viktigste prinsippet i denne sammenheng er at dersom en er i tvil om Cypres 2 er riktig kalibrert, skal rekalibrering foretas.

\subsubsection{Vannhopp}
Etter vannhopp eller andre situasjoner hvor filteret i hovedenheten av Cypres 2 har vært i direkte kontakt med vann, skal filteret byttes. Dette kan gjøres av Materiellkontrollør. Bruk Airtec’s filterverktøy til å skru filtrene ut og inn i hovedenheten.
\begin{enumerate}
\item Sett verktøyet rett på det gamle filteret med den sporede siden av verktøyet mot filteret.
\item Skru det gamle filteret ut ved å vri verktøyet mot klokka.
\item Om det er vann i filterhuset, skal dette tørkes bort med en ren klut.
\item Sett det nye filteret inn i filterverktøyets sporede side.
\item Skru det nye filteret med klokka inn i filterhuset. Du vil til å begynne med kjenne motstand.
\item Skru det nye filteret med klokka helt til filterverktøyet glipper filteret.
\item Trekk filterverktøyet rett av.
\end{enumerate}

Om vann har trengt inn i ``Jackplug'' på fyringsdel/kutter, må pluggen åpnes og ”hunkjønn” delen av pluggen bankes lett mot en plan og ren overflate. Sørg for at mest mulig vann kommer ut. Tørk ``hunkjønn''-delen i minst 24 timer før kutter settes på plass.

\subsubsection{Feilmeldinger}
Cypres 2 avgir feilmelding under oppstartsprosedyre. Ved feil på enheten, vil en firesifret feilmelding vise i displayet i to sekunder etter fullført selvtest. Deretter vil enheten slå seg av. Displayet blir blankt.

\begin{table}
	\caption{Feilmeldinger Cypres 2}
	\begin{tabular}{ | p{2cm} | p{5cm} | p{3cm} | }
		\hline
		Feilmelding & Årsak & Feilretting \\
		\hline
		1111 & Det oppnås ikke kontakt med fyringsenhet/kutter. & Sjekk ”Jackplugg” Bytt kutter Se etter kabelbrudd \\
		\hline
		2222 & Det oppnås ikke kontakt med fyringsenhet/kutter, 2-pins versjon. & Sjekk ”Jackplugger” Bytt kutter(e) Se etter kabelbrudd\\
		\hline
		3333 & Større variasjoner i lufttrykk under selvtest/oppstart & Prøv ny oppstart \\
		\hline
		Andre siffer & Andre feil & Noter feilkode og kontakt leverandør. \\
		\hline
	\end{tabular}
\end{table}

\subsection{Montering og pakking}
Prosedyrer beskrevet for Cypres skal benyttes.

\subsection{Kontroll og vedlikehold}
Kontroll av Cypres 2 begrenser seg til visuell kontroll av kabler, fyringsdel og kontroll under oppstartsprosedyre.

Cypres 2 er vedlikeholdsfri, bortsett fra funksjonskontroll hos produsenten hvert fjerde år. Disse kontroller er obligatorisk for at Cypres 2 skal fungere tilfredsstillende under alle forhold.

\textbf{Vær spesielt påpasselig med disse intervallene når det gjelder tandem– og elevutstyr.}

\textbf{Der Cypres erstatter LOR skal fabrikantens vedlikeholdsintervaller overholdes for at utstyret er luftdyktig.}

\subsubsection{Batteribytte}
Batteribytte foretas under fabrikkvedlikehold, 4 og 8 år etter produksjonsdato. Det er ingen grense for maksimalt antall hopp som kan gjennomføres i batteriets levetid.

\subsubsection{Fabrikkvedlikehold}
Cypres 2 skal inn til test hos produsenten 4 og 8 år etter produksjonsdato. Airtec godtar et avvik på +/- seks måneder.

Benytt originalemballasje ved innsendelse av åpner.

\subsubsection{Bytte av fyringsdelen / kutter}
Fyringsenheten er gjort delbar slik at kutter kan byttes etter fyring, uten å måtte sendes inn til produsenten. Prosedyren er dokumentert i manualen, pkt 6.

Airtec fraråder å vri ``Jackpluggen'' ved demontering og montering av kutter.

Batteri på Cypres 2 kan ikke koples fra, og er dermed ikke nødvendig ved bytte av kutter på Cypres 2.

Kutter fra Cypres og Cypres 2 kan benyttes om hverandre. Bare kutter spesielt produsert for Cypres 2 montert på en Cypres 2 hovedenhet er vanntett.

\section{Astra}
Astra automatåpner er godkjent for bruk på privat utstyr fra 1.5.1997.

Da denne automatåpner i store trekk er lik Cypres i både funksjon og virkemåte vil den ikke bli behandlet like detaljert som Cypres.

\subsection{Beskrivelse og virkemåte}
Astra er designet for å kutte reservecontainerens lukkeloop dersom den vertikale hastigheten er høyere enn 100 fot i sekundet synk under en høyde på 1 000 fot (Ekspert versjon). Minste hastighet den er innstilt til å fyre på er 80 fot/sek i synk.

Den består av tre deler, en kontrollenhet, en kutter, og en batterienhet.

Når den slås på med PÅ/AV bryter kalibrerer den seg automatisk, foretar en selvkontroll, og overvåker batteristatus kontinuerlig.

\subsection{Brukerinstruks}
Astra må slås på før hvert hopp, og av etter landing. Dette er en vesentlig forskjell fra Cypres der denne overvåker kontinuerlig trykk og batterikapasitet så lenge den er på. Dersom man glemmer å slå den av vil batteriet tappes fort og får en vesentlig lavere levetid.

Når den slås på blinker en lampe hurtig i fem sekunder. Deretter vil den blinke sakte minst 10 ganger, under selvtesten.

Når denne er gjennomført blinker den omtrent hvert sekund som viser at den er kalibrert til 1000 fot over hoppfeltet. Dette indikerer også at Astra er slått på og er kalibrert.

Dersom det grønne lyset går av, eller står kontinuerlig på, er enten batterispenningen for lav, eller kutteren eller koblingen er defekt. Dersom dette fortsetter ved flere forsøk må instrumentet returneres fabrikanten for kontroll.

Astra aktiverer seg selv etter at flyet har passert 1400 fot over hoppfeltet. Enheten skal slås av dersom et planlagt hopp avbrytes, og flyet går ned.

\subsection{Montering og pakking}
Astra monteres på samme måte som en Cypres. Innvendig lomme og føringer for Cypres kan også benyttes til Astra.

Ved montering bør Kontrollenheten monteres mer tilgjenglig enn de innelukkede typer for Cypres, for å kunne betjene PÅ/AV knappen. Dette er en liten ``flipbryter'', og bør kunne nås enkelt.

Forøvrig gjelder stort sett samme begrensninger for montering som for Cypres.

\subsection{Kontroll og vedlikehold}
Det er kun batteribytte som nødvendig vedlikehold på Astra. Batterilevetid er beregnet til cirka 150 timer.

Ved hver ompakk anbefaler fabrikanten at enheten testes i et trykkammer, med en kutterplugg (erstatningsplugg for kutteren). Kutteren har en beregnet levetid på 10 år.

For ytterligere informasjon henvises til Astras brukermanual

\section{Instrumenter}
I kategorien instrumenter har vi høydemålere og akustiske høydevarslere.

Det er lite vi kan gjøre med dem, annet enn å være påpasselig mot skader og støt, samt generelt ytre renhold.

\subsection{Høydemålere}
Er man i tvil om at instrumentet er kalibrert riktig, bør det testes.

Test utføres i et lavtrykkskammer med riktig kalibrert referansehøydemåler, hvor man simulerer opp- og nedstigning til 13000 fot.

Man bør kontrollere korrekt visning ved opp og nedstigning med terminalhastighet: ca 200 fot pr. sek..

Alle høydemålere og akustiske høydevarslere har et lite hull som luften må kunne passere fritt gjennom. Tettes dette hullet av skitt e.l., vil høydemåleren kunne vise feil høyde.

Vær også klar over at kroppstilling og høydemålerens plassering i fritt fall kan endre så mye som ± 700 fot på avlesning av korrekt høyde.

Noen kjente mekaniske høydemålere:
\begin{itemize}
\item Altimaster I, II, III, IV
\item Parachutes de France Altimeter
\item Barigo
\item Eureka FT-50
\end{itemize}

Digitale høydemålere
\begin{itemize}
\item Altimaster Neptun (kombinasjonsinstrument, både elektronisk høydemåler og akustisk høydevarsler)
\item Digitude
\end{itemize}

\subsection{Akustiske høydevarslere:}
Disse instrumentene har i tillegg til høydemålerne ett batteri som varsler med lyd på en forhåndsinnstilt høyde når denne passeres. Alle kjente høydevarslere vil verifisere innstilt høyde og kalibrering ved å gi lydsignal under oppstigning med fly.

Se ellers fabrikantens anvisninger.

Noen kjente typer:
\begin{itemize}
\item Dytter, Pro Track, Pro Dytter (Larsen \& Brusgaard ApS, Danmark)
\item Time Out (Cool \& Groovy Fridge Ltd, England)
\item SkyTronic (Parasport Italia)
\item Altimaster Neptun (kombinasjonsinstrument, både elektronisk høydemåler og akustisk høydevarsler)
\end{itemize}

\section{Vedlegg fra produsenter}
\subsection{Adresser}

\begin{table}
	%\caption{Feilmeldinger Cypres 2}
	\begin{tabular}{ | p{5cm} | p{2cm} | p{3cm} | }
		\hline
		Navn & & Referanse \\
		\hline
		\textbf{FXC Corporation}
		David Aguilar
		3412 S. Susan St. Santa Ana, CA92704, USA
		Tel: +1 714 557 8032
		Fax: +1 714 641 5093 &
		Produsent FXC
		Produsent Astra &
		Poynter Vol. 2, 9.5.2 \\
		\hline
		\textbf{Airtec}
		Helmut Cloth
		Mittelstrasse 69
		D-4798 Wünnenberg
		Tel: +49 2953 8010
		Fax: +49 2953 1293 &
		Produsent Cypres &
		Poynter Vol. 2, 9.5.4 \\
		\hline
		\textbf{Paramecanic AB}
		Göran Lilja
		Mandelblomvägen 1
		S-746 51 Bålsta, Sverige
		Tel: +46 171 555 25
		Fax: +46 171 580 95 &
		Reparasjon FXC & \\
		\hline
		\textbf{Sky Design}
		Pål Bergan
		Dronningens gt 13
		0152 Oslo
		Tel: 22 33 68 01
		Fax: 22 33 68 02 &
		Funksjonstesting FXC
		Reparasjon/Kalibrering høydemålere & \\
		\hline
	\end{tabular}
\end{table}
