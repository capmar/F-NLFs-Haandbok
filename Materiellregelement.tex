\part{Materiellregelement}
\setcounter{section}{200}

\section{Generelt}
Del 200 av F/NLF Håndboka gir de overordnede retningslinjer for materielltjenesten innen F/NLF.

\subsection{Hjemmel}
Disse bestemmelser er utarbeidet og blir revidert av Fallskjermseksjonen/Norges Luftsportsforbund (F/NLF) v/ Sikkerhets- og utdanningskomiteen (SU), med hjemmel i Bestemmelser for Sivil Luftfart (BSL) D 4-2 § 4 (2) b) (se vedlegg).

\subsection{Gydlighetsområde}
Disse bestemmelser gjelder for all materielltjeneste innen F/NLF med tilsluttede klubber.

\subsubsection{Unntak}
Bestemmelsene kan fravikes dersom det er nødvendig som følge av at virksomheten utføres som ledd i eksperiment- eller forsøksvirksomhet etter eksperimenttillatelse (XT) iht Del 500.

\subsection{Revisjon}
Bestemmelsene i Materiellreglementet revideres av SU, som også er ansvarlig for utfyllende bestemmelser.

\subsubsection{Utfyllende bestemmelser}
Utfyllende bestemmelser angående godkjente materielltyper og godkjente og påbudte modifikasjoner angående fallskjermmateriell er beskrevet i F/NLFs Materiellhåndbok.

Vedtak kunngjort i SUs komitéreferat eller særskilt informasjonsskriv er forøvrig å anse som gjeldende bestemmelse fra dato angitt i referat eller skriv.

\subsection{Overholdelse/Disiplinærtiltak}
Alle som utøver materielltjeneste etter disse bestemmelser plikter å kjenne de deler av bestemmelsene som angår vedkommendes oppgaver, og plikter å overholde dem på alle punkter.

\subsubsection{Påtale av brudd}
Et hvert medlem av Fallskjermseksjonen/NLF har rett og plikt til å gripe inn ved brudd på Materiellreglementets bestemmelser.

Brudd på bestemmelsene i Del 200 er rapporteringspliktige jfr Del 500.

\subsubsection{Disiplinærtiltak}
Lokalt klubbstyre v/Hovedinstruktør treffer de disiplinære tiltak som anses rimelig og nødvendig. Grove brudd skal rapporteres til SU, jfr Del 500. SU skal vurdere eventuelle tiltak.

\subsubsection{Inndraging av sertifikater og/eller status}
Brudd på bestemmelsene kan medføre inndragning eller begrensning av sertifikater eller status for en begrenset tid eller for alltid, etter avgjørelse av SU (jf pkt 201.4.2)

\section{Organisasjonen}
Materielltjenesten bygger på følgende organer og personell:

Sentrale ledd:
\begin{itemize}
	\item Seksjonsstyret
	\item Materiellsjefen
	\item Sikkerhets- og utdanningskomiteen
	\item Materiellkontrollører
	\item Materiellreparatør
\end{itemize}

Lokale ledd:
\begin{itemize}
	\item Klubbstyret m/materiellforvalter
	\item Hovedinstruktør
	\item Fallskjermpakker
	\item Eier av materiell
\end{itemize}

\subsection{Materiellsjefen (MSJ)}
Materiellsjefen er det faglige organ for materielltjenesten innen sivil fallskjermsidrett i Norge. MSJ er ansvarlig for at materielltjenesten til en hver tid er regulert av bestemmelser som krav til sikkerhet gjør nødvendig. SU skal føre kontroll med at bestemmelsene overholdes.

\subsubsection{Sammensetning}
Seksjonsstyret F/NLF utpeker MSJ.

\subsubsection{Arbeidsområde}
Materiellsjefen er F/NLFs fagmyndighet i materiellspørsmål. MSJ fører overordnet tilsyn med materiellstjenesten i F/NLF.

Materiellsjefen skal spesielt:
\begin{itemize}
	\item Utrede og avgjøre saker av betydning for materiellsikkerheten.
	\item Utgi og revidere materielltekniske bestemmelser.
	\item Utføre typegodkjenning av fallskjermmateriell.
	\item Materiellsjefen er ansvarlig for revisjoner i Materiellhåndboken.
\end{itemize}

\subsection{Materiellkontrolløren}
Materiellkontrollør utfører hovedkontroll på fallskjermer og gjennomfører eksamensprøver for fallskjermpakkere i samsvar med gjeldene bestemmelser.

Materiellkontrollør skal:
\begin{itemize}
	\item Føre nødvendige merknader på de dokumenter som angår de skjermer han kontrollerer.
	\item Utføre mindre reparasjoner. Se begrensninger anført i Materiellhåndboken.
	\item Anbefale søknader om utstedelse av fallskjermpakkersertifikater i de tilfelle hvor tilfredsstillende eksamensprøve blir avlagt under hans kontroll.
	\item Føre separat logg over alt materiell som han pakker eller utfører hovedkontroll på.
	\item Ha egen plombetang registrert av FNLF Materiellkontrollørens faglige foresatte er F/NLFs Materiellsjef.
\end{itemize}

\subsection{Materiellreperatør}
Materiellreparatør kan utføre alle former for materiellreparasjoner på de fallskjermtyper han er sertifisert og autorisert for.

Materiellreparatør skal:
\begin{itemize}
	\item Utføre reparasjoner og modifiseringer i henhold til anvisninger i Materiellhåndboken.
	\item Føre separat logg over alle hovedkontroller, pakking av reserveskjermer, modifiseringer og reparasjonsarbeid utført av ham.
\end{itemize}

Materiellreparatørens faglige foresatte er F/NLFs Materiellsjef.

\subsubsection{Spesielle krav}
Materiellreparatører skal
\begin{itemize}
	\item være godkjent av SU.
	\item føre logg over sitt arbeid tilsvarende det en materiellkontrollør er pålagt.
	\item ha egen plomberingstang registrert av F/NLF.
	\item dokumentere alle rutiner skriftlig
\end{itemize}

Norsk Fallskjermverksted skal
\begin{itemize}
	\item være godkjent av SU.
	\item ha Materiellreparatør godkjent av F/NLF som Verkstedets faglige ansvarlige
	\item dokumentere skriftlig alle rutiner, herunder krav til personell som utfører arbeid for verkstedet.
	\item føre logg over sitt arbeid tilsvarende det en materiellkontrollør er pålagt.
	\item ha egen plomberingstang registrert av F/NLF.
	\item benytte verkstedets plomberingstang ved reservepakking. Pakkjobben skal signeres for av representant for verkstedet, og verkstedets stempel skal anbringes samme sted som signaturen.
	\item Ved hovedkontroll av utstyr signere for dette av representant for verkstedet, samt at verkstedets stempel skal anbringes på samme sted som signaturen.
	\item svare for alt arbeid som er plombert, signert og stemplet for iht. det ovenstående. Den enkelte medarbeider ved verkstedet er således fritatt for de krev som gjelder for øvrig personell som omfattes av Håndbokas del 200.
\end{itemize}

\subsubsection{Begrensninger}
Materiellreparatørstatus gir ikke rett til produksjon av utstyr eller materiellkomponenter med mindre særskilt dokumentasjon om at faglige kvalifikasjoner, produksjonsutstyr, fremstillingsprosess og testing (kvalitetskontroll) er godkjent og autorisert av Sikkerhets- og utdanningskomiteen.

\subsection{Klubbstyret}
Det lokale klubbstyre har uten unntak ansvaret for at klubbenes materielltjeneste gjennomføres i samsvar med gjeldende bestemmelser.

\subsubsection{Materiellforvalter}
Materiellforvalter utpekes av det lokale klubbstyret, og har Hovedinstruktøren i lokalklubben som sin nærmeste faglige foresatte.

Materialforvalter skal:
\begin{itemize}
	\item Føre tilsyn med behandling og bruk av felleseid klubbutstyr.
	\item Lede klubbens kontroll, vedlikehold og lagring av materiell.
	\item Føre arkivkort med opplysninger av betydning for hvert enkelt felleseid fallskjermsett.
	\item Merke skjermer som ikke er luftdyktige og sørge for at disse ikke blir benyttet til praktisk hopping.
	\item Sørge for at skjermene i hans forvaring blir underkastet hovedkontroll som bestemt i reglementet.
\end{itemize}

\subsection{Hovedinstruktøren}
Hovedinstruktøren har ansvaret for materiellsikkerheten i den virksomhet som han leder, herunder opplæring i bruk og behandling av fallskjermutstyr. Til å føre tilsyn med fallskjermutstyret, har HI til disposisjon en materiellforvalter utpekt av lokalt klubbstyre

\subsection{Fallskjermpakkeren}
Sertifisert fallskjermpakker kontrollerer og pakker fallskjermer i henhold til F/NLFs pakkeinstrukser og/eller de gjeldende tekniske forskrifter som er fastsatt for angjeldende skjermtype. Fallskjermpakker er selv ansvarlig for å være oppdatert på eventuelle modifikasjonsordrer.

Fallskjermpakker skal utføre pakkerkontroll av fallskjerm under pakking og sørge for at materiell som ikke er luftdyktig tas ut av praktisk bruk og underkastes nødvendig reparasjon/vedlikehold.

\subsection{Eier av materiell}
Eier av materiell er selv ansvarlig for dette. Eier av fallskjerm betraktes som skjermens materiellforvalter. Hovedkontroll, reservepakking og eventuell annen ønsket kontroll foretas av Materiellkontrollør.

\section{Registrerings- og merkeforskrifter}
\subsection{Godkjenning og kontroll}
Alle fallskjermer skal underkastes norsk hovedkontroll og godkjennes før de tas i bruk. Kontrollen utføres av norsk Materiellkontrollør. Attestasjon for utført kontroll og registrering skal føres på Hovedkontrollkort/pakkerlogg, som skal oppbevares i lomme på seletøy/pakksekk.

\subsection{Militært fallskjermmateriell}
Militært fallskjermmateriell som er Forsvarets eiendom og som er godkjent og under kontroll av militær fagmyndighet, kan nyttes uten spesielle kontrolltiltak eller spesiell registrering. Ved behov nyttes skjermens militære registreringsnummer som referanse.

\subsection{Materiell tilhørende utenlandske hoppere}
For utenlandske deltakere i organisert fallskjermhopping følges bestemmelsene i FAI Sporting Code, section 5.

\section{Fallskjermsettet}
Et fallskjermsett deles inn i følgende hoveddeler:
\begin{itemize}
	\item Seletøy/pakksekk
	\item Hovedskjerm
	\item Reserveskjerm
	\item Automatisk åpner
\end{itemize}

Alle hoveddeler skal hver for seg være typegodkjent i Norge før det tas i bruk. Sikkerhets- og utdanningskomiteens typegodkjenningsliste finnes i Materiellhåndboken. Kombinasjoner av sele/pakksekk, hovedskjerm, reserveskjerm og de tilhørende komponenter skal være i henhold til typegodkjenning.

\section{Personlig utstyr}
Personlig utstyr består av:
\begin{itemize}
	\item Hjelm
	\item Briller
	\item Hoppdress
	\item Hansker
	\item Fottøy
	\item Evt flytevest
\end{itemize}

Krav til bruk av personlig utstyr er beskrevet i Del 100.

\section{Pakking}
Selvstendig pakking kan kun gjøres av de som har gyldig pakkersertifikat for vedkommende kategori fallskjerm, ref del 300. Pakkekontroll skal utføres etter bestemmelser i Materiellhåndboken. Pakking skal skje iht. fallskjermprodusentens pakkemanual.

Fallskjermens løftestropper skal være festet til seletøy/pakksekk når fallskjermen pakkes.

Foreligger det mistanke om at væte, urenheter etc kan ha trukket gjennom/inn i pakksekken skal denne åpnes og skjermen kontrolleres. Skjermen tørkes og renses før pakking i de tilfeller hvor dette måtte vise seg nødvendig.

Åpen ild skal ikke forekomme innenfor 50 m avstand fra fallskjermmateriell.

\subsection{Ompakkingssyklus}
Hovedfallskjermer, reserveskjermer i elevutstyr, reserveskjermer i tandemutstyr og reserveskjermer i andre felleseide fallskjermsett som har vært lagret i pakket stand i mer enn 6 måneder, er ikke luftdyktig. Reservefallskjermer i privateid sportsutstyr kan lagres i 12 måneder. For igjen å bli luftdyktig må fallskjermen luftes ved at den åpnes og ristes grundig før den pakkes på ny.

\subsection{Pakkekontroll}
\subsubsection{Vingskjermer}
For vingskjermer foretas pakkekontrollen slik:
\begin{itemize}
	\item 1. kontroll: Når kalotten er lagt ned på siden med linene og kalotten strukket ut (gjelder ikke ved pro-pakking).
	\item 2. kontroll: Når kalotten er brettet sammen, panelene trukket ut, halen lagt, slideren trukket opp.
	\item 3. kontroll: Når kalotten er brettet sammen og lagt i bagen, linene sløyfet inn og bagen lagt inn i pakksekken.
	\item 4. kontroll: Når sekken er lukket.
\end{itemize}

\subsection{Pakking av reserveskjermer}
All pakking av reserveskjermer skal utføres av Materiellkontrollør med pakkersertifikat for angjeldende reserveskjermskategori, ref Del 300. Skjermens pakkerlogg skal signeres med navn og pakkersertifikatnummer. Plombetangens registreringsbokstaver føres i pakkeloggens merknadsrubrikk. Materiellkontroll-loggen ajourføres.

\subsubsection{Plombering}
Plombering utføres med rød plomberingstråd og etter metoder som beskrevet i Materiellhåndboken.

\subsection{Pakking av nødskjermer}
Nødskjermer av kategori 4 (ref Del 300) tillates bare pakket av Materiellkontrollør som innehar sertifikat for skjermtypen.

\section{Kontroll av fallskjermutstyr}
Kontroll av fallskjermutstyr omfatter
\begin{itemize}
	\item Hovedkontroll
	\item Brukskontroll
\end{itemize}

\subsection{Hovedkontroll}
\subsubsection{Omfang}
Hovedkontroll består av omhyggelig visuell kontroll av alle komponenter utført av Materiellkontrollør, etter anvisninger gitt i Materiellhåndboken.

Fallskjermsett som har forfalt til hovedkontroll skal under ingen omstendighet nyttes for praktisk hopping før kontrollen er utført.

Hovedskjerm kan hovedkontrolleres separat.

\subsubsection{Intervaller}
Fallskjermsett skal underkastes hovedkontroll:
\begin{itemize}
	\item Før fallskjermsettet kan tas i bruk (førstegangs kontroll).
	\item Innen 6 måneder etter siste hovedkontroll for elevfallskjermsett, tandemfallskjermsett og fallskjermsett som er felleseid klubbutstyr.
	\item Innen 12 måneder etter siste hovedkontroll for annet utstyr.
	\item Etter utført større reparasjon.
	\item Etter feilfunksjonert hovedskjerm.
	\item Etter aktivert reserveskjerm (kontrolleres sammen med seletøy/pakksekk.)
	\item Når det er ønskelig av ansvarlig pakker, materiellforvalter eller eier.
\end{itemize}

Elevfallskjermsett, tandemfallskjermsett og fallskjermsett som er felleseid klubbutstyr er luftdyktig i 3 måneder etter gjennomført hovedkontroll. Deretter må brukskontroll iht 207.2 gjennomføres for at utstyret skal være luftdyktig fram til neste forfall for hovedkontroll.

\subsubsection{Kontrollresultat}
Skjerm som underkastes hovedkontroll kan bli:
\begin{itemize}
	\item Godkjent.
	\item Henvist til reparasjon/utskifting av bestemte komponenter.
	\item Kassert.
\end{itemize}

\paragraph{Godkjenning}
Hovedkontroll skal registreres i Materiellkontrollørens logg, med angivelse av resultat av kontrollen. Hovedkontrollkort utstedes bare for godkjent og luftdyktig materiell.

\paragraph{Henvisning til reparasjon}
Større reparasjoner, alt arbeid på bærende deler samt modifikasjoner av fallskjermer skal utføres av godkjent Materiellreparatør. Modifiseringer kan utføres etter retningslinjer beskrevet i Materiellhåndboken eller fra Materiellsjefen. Hovedkontrollkort og arbeidsordre (spesifikasjon av skader og feil) skal følge materiellet når det sendes til reparasjon.

\subsection{Brukskontroll}
\subsubsection{Omfang}
Brukskontroll består av omhyggelig visuell kontroll av alle komponenter på seletøy som er synlige uten at pakksekk for reserveskjermen åpnes, og av hovedskjerm. Brukskontroll utføres av Materiellkontrollør, etter anvisninger gitt i Materiellhåndboken.

Fallskjermer som har forfalt for brukskontroll skal under ingen omstendighet nyttes til praktisk hopping før kontrollen er gjennomført.

\subsubsection{Intervaller}
Elevutstyr skal underkastes brukskontroll
\begin{itemize}
	\item 3 måneder etter siste hovedkontroll
	\item Når det er ønskelig av ansvarlig pakker, materiellforvalter eller eier.
\end{itemize}

Tandemutstyr skal underkastes brukskontroll
\begin{itemize}
	\item For hvert 55.hopp, alternativt etter 3 måneder avhengig av hva som kommer først. Det skal føres brukskontrollogg som skal oppbevares hos HI og medtas på tandemseminarer.
\end{itemize}

Privat utstyr skal underkastes brukskontroll
\begin{itemize}
	\item Når det er ønskelig av ansvarlig pakker, eier.
\end{itemize}

\subsubsection{Kontrollresultat}
Skjerm som underkastes brukskontroll kan bli:
\begin{itemize}
	\item Godkjent
	\item Henvist til hovedkontroll/reparasjon/utskifting av bestemte komponenter, jfr pkt 207.1.3 med underpunkter.
\end{itemize}

\section{Bestemmelser som skal flyttes til materiellhåndboken}
\subsection{Merking}
\subsubsection{Generelt}
Hovedskjermer skal være merket med produsentens navn, produktnavn, modellnummer, serienummer og produksjonsdato. Reserveskjermer og seletøy/pakksekk skal i tillegg være merket med produsentlandets offentlige godkjenningsnorm, f eks TSO-C23, category C eller D. Merkingen for seletøy/pakksekk skal være tilgjengelig i pakket tilstand.

\subsubsection{Spesielt for felleseid klubbutstyr}
Seletøy/pakksekk skal være merket med klubbens navn og som et minimum slik at størrelse på hoved- og reserveskjerm kan forstås. Merkingen skal være synlig når skjermene er pakket.

Felleseid fallskjermsett godkjent for tandemhopping fritas for krav om merking av størrelse på hoved- og reserveskjerm.

\subsection{Kassasjon av fallskjermer}
\subsubsection{Rettigheter}
Kassasjon av fallskjermer foretas av Materiellkontrollør som er sertifisert for pakking av aktuell type skjerm. Kassasjon skjer etter kontroll pga skader, slitasje eller foreldelse. Kassasjon kan også skje etter ønske fra eier eller klubb.

\subsubsection{Utfasing}
Fallskjermer av organiske stoffer (bomull, silke) eller hvor silketråd (bomullstråd) er nyttet for søm, er forbudt.

\subsubsection{Kassasjonsgrunnlag}
\paragraph{Kassasjon av fallskjermer}
Nylonfallskjermer kasseres på grunnlag av kontrollresultater, dog skal reservefallskjermer tas ut av bruk og kasseres senest 20 år etter produksjonsdato.

\paragraph{Fallskjemer som har vært i kontakt med sjøvann (saltvann)}
Komponenter som har vært i kontakt med sjøvann (saltvann), i mer enn 48 timer, eller som ikke har blitt skyllet tilfredsstillende innen 24 timer etter berging, skal kasseres.

\subsubsection{Anvendelse av kassert materiell}
Kryss skal males diagonalt over pakksekken, og over skjermens serienummer. Bruk kontrastfarget maling.

Kasserte fallskjermer (øvingsmateriell) skal lagres separat fra luftdyktig utstyr.

\subsection{Seletøy/pakksekk}
Seletøy/pakksekk består av seletøy integrert med pakksekker for hovedskjerm og reserveskjerm, felles hovedkontrollkort for hele fallskjermsettet, og eventuelt hovedhåndtak.

Reserveskjermens løftestropper skal være integrert i seletøyet.

Seletøyet skal ha 3-rings kalottfrigjøringssystem for hovedskjermens løftestropper.

\subsubsection{Pakksekk}
Reserveskjermens pakksekk lukkes og åpnes ved hjelp av låseløkker, utløserhåndtak og kabel med pinne(r). Til lukking og åpning av hovedskjermens pakksekk kan utløserline med pinne eller pinne montert på throw-out pilots bånd benyttes i stedet for utløserhåndtak.

\subsection{Hovedskjermen}
Hovedskjermen er en vingfallskjerm, konstruert og produsert spesielt for idrettslige formål, som utløses manuelt.

Kategoriinndeling er som beskrevet i Del 300.

Hovedskjermen består av kalott med bæreliner, åpningsbrems og links.

Hovedskjerm skal være forsynt med bag eller annen godkjent anordning som sikrer at bærelinene er strukket før kalotten åpner seg. Pilotskjermen skal være godkjent av seletøyprodusenten, og festes til skjermen ved hjelp av pilotbåndet. Type bånd, lengde og festemetode er beskrevet i Materiellhåndboken.

\subsection{Reserveskjermen}
Reserveskjermen er en rund kalott eller vingfallskjerm. Kun skjermer som er produsert for anvendelse som nødskjerm (low speed type parachute) kan godkjennes. Runde kalotters synkehastighet skal ikke overskride 7,5 m/sek ved en belastning på 75 kg vekt.

Kategoriinndeling er som beskrevet i Del 300.

Reserveskjermen består av pilotskjerm med bånd, innerbag/diaper, kalott med bæreliner og links (sjakler), utløserhåndtak med kabel og pinne(r), samt pakkerlogg.

Pilotskjermen skal være godkjent av seletøyprodusenten, og festes til runde skjermers toppliner ved hjelp av pilotbåndet. Ved vingreserver skal riggprodusentens fribagsystem nyttes. Type bånd, lengde og festemetode er beskrevet i Materiellhåndboken.

Seletøy skal være utstyrt med tilkoplet LOR/RSL line slik det er beskrevet av produsenten eller av annen godkjent type for automatisk aktivering av reserveskjerm ved cut away fra utløst, ikke flyvende hovedskjerm.

Dersom typegodkjent nødåpner vedlikeholdt iht. produsentens krav er montert og i bruk på reserveskjermen kan LOR/RSL-line demonteres.

\subsection{Automatiske åpnere}
Hovedskjerm og reserveskjerm kan utstyres med automatisk åpner. Åpnere skal typegodkjennes av Sikkerhets- og utdanningskomiteen. Fra 1. april 2010 er det obligatorisk med bruk av nødåpner ved hopping i Norge. HI kan gi dispensasjon fra dette ved særskilte typer hopping som for eksempel vannhopp mfl. Dette skal ikke være av generell- og ubegrenset karakter. Det skal være dokumentert.

\subsubsection{Nødåpner elevfallskjermsett}
Elevfallskjermsett skal alltid være utstyrt med nødåpner på reserveskjermen. Nødåpneren skal være godkjent for utdanning.

\subsubsection{Nødåpner tandemfallskjermsett}
Tandemfallskjermsett skal alltid være utstyrt med nødåpner på reserveskjermen. Nødåpneren skal være godkjent for tandemhopping

\section[Annet fallskjermmateriell]{Annet fallskjermmateriell - flyttes til materiellhåndboken}
\subsection{Utløserline}
Linen skal ha en bruddstyrke på minst 1000 kg, og være 3,90 m lang om ikke annet er bestemt av produsenten eller blir fastsatt for bruk i forbindelse med bestemt flytype. Utløserlinen skal ha godkjent type ankerkrok med sikringspinne. Utløserlinen skal ha krum pinne, eller annen godkjent anordning for låsing av pakksekk. Ved utsprang fra stag skal utløserlinen være innsydd til halv bredde for å redusere vindfang. Ved bruk av HD-pilot skal det være tilkoblet godkjent pilotlomme til utløserlinen. Ved utsprang fra stag skal utløserlinen være sydd til en halv tomme for å redusere vindfang.

\subsection{Instrumenter}
Et instrumentsett består normalt av visuell høydemåler, og evt. akustisk i tillegg.

\section[Vedlikehold av fallskjermer]{Vedlikehold av fallskjermer - flyttes til materiellhåndboken}
\subsection{Rengjøring}
\subsubsection{Generelt}
Rengjøring av fallskjermer skal begrenses mest mulig og må bare utføres når det er nødvendig for å hindre feilåpning eller forringelse. Når rengjøring er nødvendig, må den utføres manuelt ved utristing og/eller børsting eller punktrensing. Rengjøringen begrenses til det tilsølte område.

\subsubsection{Rensing}
Når børsting eller utristing ikke fører frem, må fallskjermen punktrenses, dvs rensingen begrenses til det tilsølte området. Dette må gjøres noe forskjellig for de forskjellige fallskjermtyper, avhengig av hvilket stoff de er laget av og hvilken del av fallskjermen som må renses.

\paragraph{Rensing av bomull}
Bomullskomponenter punktrenses ved kraftig børsting med halvstiv børste eller ren klut, som er fuktet med et tørrensende oppløsningsmiddel (f eks Inhibisol, eller tilsvarende etter anbefaling og godkjenning av Sikkerhets- og utdanningskomiteen). Det rensede område skylles deretter med oppløsningsmidlet. Man skal forsikre seg om at oppløsningsmidlet IKKE kan skade stoffet.

\paragraph{Rensing av nylon}
Nylonkomponenter punktrenses som bomullskomponentene eller med en oppløsning av varmt vann og et syntetisk rensemiddel. Oppløsningen lages ved løse opp 30-100 gram av en nøytral syntetisk oppløsningsvæske (oppvaskmiddel) i ca 4 liter varmt vann (inntil 40 grader C). Komponentene må IKKE vris.

\subsubsection{Utskylling av saltvann}
Fallskjerm som har vært utsatt for saltvann må snarest rengjøres for å sikre seg at alle etsende saltpartikler blir fjernet. Rengjøring skjer ved utskylling i rikelig mengde ferskvann. Utskyllingen må være fullstendig og under kontroll av pakker. Kalott skal ikke vris under eller etter skyllingen. Tørkingen skal vies særlig oppmerksomhet.

\subsection{Tørking}
Fallskjermmateriell skal i minst mulig grad utsettes for direkte sollys. Tørking av våte fallskjermer skal helst skje i særskilte tårn. Temperaturen skal ikke overskride 40 grader C. Skjermen må ikke komme nær varmekilden. Sterk varmepåvirkning vil kunne ødelegge skjermen eller forårsake klebing. Tørkerommet må være godt gjennomluftet.

Fallskjermen bør henge mest mulig fritt for å sikre tørking av alle komponenter. Når høyden under taket er liten, kan bærelinene flettes slik at også selen, som er den del av skjermen som normalt tørker senest, kommer opp fra gulvet. Det er praktisk om gulvet har rustfri metallrist med rom under. Dette bidrar til luftsirkulasjon samtidig som fremmedlegemer faller vekk. Varmeelement plasseres gjerne under gulvet sammen med vifter. Om man ikke har disse muligheter, bør seletøyet legges på en krakk, slik at også dette utsettes for luftsirkulasjon.

Gjennomvåte pakksekker og seletøy tørker meget sent, og må henges til tørk snarest mulig etter bruk, for å motvirke mugg etc. Metalldeler og steder hvor stroppene er sydd sammen i flere lag, må vies spesiell kontroll før selen tas ned fra tørking. For enkelte fallskjermtyper er det nødvendig å sette metallet inn med syrefritt fett for å motvirke rust. Rust på metalldeler avsettes på de øvrige komponenter og spiser opp stoffet.

Gjennomvåte skjermer skal henges til tørk i minst 48 timer, lett fuktige skjermer trenger kortere tid, men fallskjermer må alltid være tørre før lagring. Tiden bestemmes forøvrig av skjermens forskjellige komponenter. Relativ fuktighet i tørkerommet skal ikke overskride 60\% når skjermen tas ned.

\section[Lagring]{Lagring - flyttes til materiellhåndboken}
Lagerrom for oppbevaring av felleseid klubbutstyr skal være godkjent av Materiellkontrollør.

Mer enn to skjermer skal ikke lagres oppå hverandre.

Lagerrommet må være tørt, relativ fuktighet ikke over 60\%. Temperaturen bør være 15-20 grader C. Rommet bør være utstyrt med temperatur- og fuktighetsmåler. Lagerrommet skal være rent. Olje, fett, såpe, bensin, maling, lakk, kjemikalier, skadedyr må ikke forekomme i rommet.

Fallskjermer skal i sin helhet lagres slik at de ikke blir utsatt for direkte sol, også UV-lys fra lysstoffrør vil svekke nylonstoffer.

\section[Transport]{Transport - flyttes til materiellhåndboken}
Transport av felleseid klubbutstyr skal overvåkes av pakker.

Fallskjermer må ikke transporteres sammen med kjemikalier eller annet som kan skade materiellet. Skjermer må heller ikke transporteres slik at de kan tilsøles.

Ved bruk av offentlige transportmidler må det holdes kontroll med utstyret, slik at ikke uvedkommende med hensikt eller uaktsomt kan komme til å påføre utstyret skade eller påvirke funksjonsdyktigheten.

Utstyret skal derfor emballeres før transporten i egnede sekker, kasser eller kartonger, med utvendig merking som angir innholdet.

\section[Godkjenning/vedlikehold av annet materiell]{Godkjenning/vedlikehold av annet materiell - flyttes til materiellhåndboken}
\subsection{Høydemålere}
Eier er ansvarlig for at høydemålere (visuelle og evt. akustiske) er i orden og kalibrert før bruk.

\subsection{Flytevest}
Flytevester skal være godkjent av Materiellkontrollør. Godkjenning skal skje iht følgende:
\begin{itemize}
	\item Vesten kan være oppblåsbar eller av fast materiale.
	\item Vesten skal under ingen omstendighet kunne være til fare for hopperen eller til hinder for utløsning av fallskjermen eller frigjøring fra seletøyet.
	\item Vesten skal ha en oppdrift på minimum 50 Newton.
	\item Oppblåsbare vester uten trykkpatron kan godkjennes hvis ventil for innblåsing av luft vurderes å være lett tilgjengelig under bruk.
	\item Oppblåsbare vester med trykkpatron kan godkjennes hvis aktiveringsmekanisme vurderes å være lett tilgjengelig under bruk.
	\item Flytevester som er spesielt ømfintlige for slitasje kan anvendes når de er utstyrt med overtrekk.
\end{itemize}

Felleseide flytevester skal underkastes kontroll av Materiellkontrollør når det anses nødvendig, dog minst en gang årlig.

\subsection{Nødåpnere}
Nødåpnere skal underkastes periodisk kontroll/overhaling av produsent eller annet autorisert verksted iht. krav gitt av produsent eller SU.
