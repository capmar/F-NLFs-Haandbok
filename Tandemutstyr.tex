\part{Tandemutstyr}

\section{Generelt}
Dette kapitlet beskriver bruk og kontroll av utstyr som er godkjent for bruk til tandemhopping i Norge.

Tandemutstyr krever en høy grad av kunnskap og erfaring for å kunne vedlikeholdes og kontrolleres korrekt. Utstyret består av mange ekstra spesial deler, som kan være konstruert med forskjellig formål enn for vanlig utstyr. Det er av stor viktighet at Materiellkontrollør behersker utstyrets sammensetning og virkemåte, da feilaktige vurderinger kan få store konsekvenser. Som materiellkontrollør er det viktig å kjenne sin begrensning når en behandler utstyr som en ikke er 100\% kjent med.

Utstyret er utsatt for store belastninger og slites fortere enn vanlig utstyr. Toleransene ved kontroll blir derfor mindre. Fabrikantene har satt opp lister over alminnelige serviceintervaller for forskjellige deler. Disse bør følges, og det bør medfølge en logg for hver enkelt tandemutstyr som gir informasjon om tilstand og bruk av utstyret. Fabrikantens manualer og oppdateringer bør alltid være tilgjengelig ved kontroll av tandemutstyr

En annen faktor er også at tandemutstyret blir brukt av mange forskjellige brukere, og dette i seg selv vil medføre til større slitasje.

På grunn av utstyrets spesielle beskaffenhet, med større og flere deler, bør en sjekkliste følges for hovedkontroll og ompakking av reserver.

At klubben taper penger på at utstyr ikke er oppdatert tidsnok til tandemhopping eller demo er tegn på dårlig internorganisering, og bør ikke bli en MK’s problem.

Det finnes et kinesisk ordtak for slike situasjoner: ``Bedre du bli sur enn jeg bli sur.''

\section{Typegodkjent tandemutstyr}
Følgende utstyr er godkjent for generelt bruk til tandemhopping i Norge:

\subsection{Seletøy}
\begin{table}
	\caption{Typegodkjente Tandem Seletøy}
	\begin{tabular}{ | l | l |  }
		\hline
		Produsent & Type \\
		\hline
		Parachutes de France & Atom Tandem \\
		\hline
		Parachutes de France & Galaxy Tandem \\
		\hline
		Relative Workshop & Vector Tandem TV2 \\
		\hline
		Relative Workshop & Vector Sigma Tandem System \\
		\hline
	\end{tabular}
\end{table}

\subsection{Reserver}
Kun reserver produsert for den aktuelle seletøytype er godkjent for bruk. Det vil si franske reserver i fransk utstyr, og Relative Workshops (Performance Designs) reserver i Vector.\footnote{Relative Workshop er ansvarlig markedsfører for sine tandem skjermer, selv om de produseres av Performance Designs.}

\begin{table}
	\caption{Typegodkjente Tandem Reserver}
	\begin{tabular}{ | p{2cm} | p{1cm} | p{1cm} | p{1cm} | p{1cm} | p{1cm} | p{1cm} | p{2cm} | }
		\hline
		Fabrikant & Modell & Celler & Areal kv. fot & Pakkevolum cu. in. & Anbefalt vektområde– kg. & Max vekt kg. & Kommentarer \\
		\hline
		Parachutes de France & BT-80 & 9 & 420 & & & & \\
		\hline
		Parachutes de France & Galaxy & 9 & 400 & & & 190 & Kommer fra ``gammel manual'' – 1988 \\
		\hline
		Parachutes de France & Galaxy & 9 & 370 & 750 & & 210 & \\
		\hline
		Vector / Performance Designs & PD 360 R & 9 & 360 & 908 & 105 – 156 & 204\footnotemark & \\
		\hline
		Vector / Performance Designs & PD 420 R & 9 & 420 & 908 & 124 – 181 & 204 & Lite brukt – prototyp for PD-421 \\
		\hline
		Vector / Performance Designs & PD 421 R & 9 & 421 & 1150 D 1050 M\footnotemark & 124 – 181 & 204 & \\
		\hline
		Vector / Performance Designs & PD 500 R & 9 & 500 & 1260 & 148 – 216 & 204 & \\
		\hline
	\end{tabular}
\end{table}
\footnotetext{Relative Workshop anbefaler ikke at tandemutstyr blir belastet med mer enn 204 kg. (450 lbs)}
\footnotetext{D= Dakron liner – M = Microliner. Vanligst er 600 lb. Dakron liner. Microline versjon er merket ``422''}

\subsection{Hovedskjermer}
\begin{table}
	\caption{Typegodkjente Tandem Hovedskjermer}
	\begin{tabular}{ | p{2cm} | p{1cm} | p{1cm} | p{1cm} | p{1cm} | p{1cm} | p{3cm} | }
		\hline
		Fabrikant & Modell & Celler & Areal kv. fot & Pakkevolum cu. in. & Max vekt kg. & Kommentarer \\
		\hline
		Parachutes de France & BT-80 & 9 & 420 & & 210 & Pakkes etter fabrikantens anvisninger \\
		\hline
		Parachutes de France & BT-80 P & 9 & 420 & & 210 & Pakkes etter fabrikantens anvisninger
		BT 80 P er betegnelsen på modeller av BT 80 fra 1995 \\
		\hline
		Parachutes de France & Galaxy & 9 & 400 & & 190 & \\
		\hline
		Vector / Performance Designs & EZ 384 & 9 & 384 & & 204\footnotemark & Toppduk i Zero P, Bunnduk og ribber i F111 \\
		\hline
		Vector / Performance Designs & EZ 425 & 9 & 425 & & 204 & Toppduk i Zero P, Bunnduk og ribber i F111 \\
		\hline
		Vector / Performance Designs & PD 360 & 9 & 360 & 908 & 204 & Skal ettermonteres ``Vector Double Brake System''.
		Bremsesystem skal ikke være kaskader. \\
		\hline
		Vector / Performance Designs & PD 420 & 9 & 420 & 908 & 204 & Lite brukt – prototyp for PD-421 \\
		\hline
		Vector / Performance Designs & PD 421 & 9 & 421 & 1150 D 1050 M\footnotemark & 204 & \\
		\hline
		Vector / Performance Designs & PD 500 & 9 & 500 & 1260 & 204 & Lite brukt til annet enn tunge militære operasjoner. Er for stor for vanlig tandemhopping. \\
		\hline
	\end{tabular}
\end{table}
\footnotetext{Relative Workshop anbefaler ikke at tandemutstyr blir belastet med mer enn 204 kg. (450 lbs)}
\footnotetext{D= Dakron liner – M = Microliner. Vanligst er 600 lb. Dakron liner. Microline versjon er merket ``422''}

\subsection{Nødåpnere}
Kun Cypres og Cypres 2– Tandemversjon er godkjent for bruk til tandemhopping.

Fabrikantens pålagte serviceintervaller for fabrikkontroll, og intervaller for batteribytte skal følges for nødåpnere på tandemutstyr. Det er ikke opp til eier/bruker selv å bestemme lenger serviceintervaller.
\begin{itemize}
	\item Batterier på Cypres byttes annethvert år eller hvert 500 hopp (det som kommer først), eller når displayet viser at batteribyttet er nødvendig.
	\item Serviceintervall på Cypres og Cypres 2 med innsending til produsent er hvert fjerde år ± 3 måneder i henhold til produksjonsdato.
\end{itemize}

For mer utførlig beskrivelse av funksjon, virkemåte og vedlikehold se Kapittel 8.3.

\subsection{Passasjerseletøy}
Passasjerseletøyet tilhører hovedselen, og er godkjent med den aktuelle hovedsele. Dog gjelder at Sigma passasjerseletøy kan benyttes til Vector TV2 seletøy. Det skal kontrolleres og vedlikeholdes som en del av denne, og de samme kriterier legges til grunn for vurdering av tilstand.

Vær spesielt oppmerksom på spennene for innfesting av passasjer. Sjekk fjæring og at spennene lar seg lukke ordentlig.

På Vector passasjerseletøy skal de øvre låsespenne ha en fungerende åpningssperre (gjennomgangsbolt) som forhindrer utilsiktet åpning. Dette gjelder ikke passasjer- seletøyet til Vector Sigma Tandem System.

På eldre modeller av Parachutes de France Atom Passasjerseletøy var det nødvendig med en modifisering av øvre låsespenner til en type med gjennomgående bolt. Dette er ikke lenger nødvendig på nyere passasjerseletøy, da denne låsespenna er forbedret for å hindre utilsiktede åpninger.

Quick-ejectorene som er montert for nedre innfesting skal ha tilfredsstillende lukke- og fjærmekanisme. Merk at påmontering av line for lettere å åpne ``Lever arm handle'' (arm for åpning, se Poynter Vol. 2, 4.110 s 97) kan medføre at spenna ikke er ordentlig lukket.

Husk at passasjerseletøyet får minst like mye slitasje som hovedseletøyet, og skal derfor kontrolleres like ofte og grundig som en del av hovedselen.

\section{Vedlikehold og pakking}
På grunn utstyrets beskaffenhet, og de store påkjenninger for tandemutstyr må det utvises stor nøyaktighet ved all utførelse av kontroll og pakking av tandemutstyr.

Konsekvensene for at tandemutstyr ikke er fullt operativt, eller er mangelfullt vedlikeholdt kan være enorme for tandemekvipasjen, klubben og pakker.

Som materiellkontrollør er det du som signerer for at utstyret er luftdyktig. Gjør ikke det med mindre du har spesielt god kjennskap til både tandemutstyret og bruken av det.

\subsection{Pakking}
Tandem hovedskjermer skal flatpakkes, slik som elevskjermer. ``Nose-down'' pakking er ikke en velegnet pakkemetode for tandemskjermer.

På grunn av størrelsen på skjermene, og ekstra lengder på liner, er det vanskeligere å få til gode pakkinger med ``nose-down'' pakking på tandemskjermer. Med det store sideforholdet på tandemskjermer kan en ``line-over'' feilfunksjon skje lettere ved ``nose-down'' pakking fremfor vanlig sidepakking.

\subsection{Kontroll og vedlikehold}
Det er kun to tandem systemer godkjent i Norge. Noen kontroller er spesifikke på det enkelte utstyr, og andre kontroller er felles for alt tandemutstyr.

I det følgende er felles kontroller listet opp først, deretter en sjekkliste for kontroller for det enkelte utstyr.

\subsubsection{Kontroll av tandemutstyr, generelt.}
\begin{itemize}
	\item Kontroll at indre styreliner går separat til hver sin ring.
	\item Kontroll av strikk på innerbag. Skal være ``Tandem type'' eller Tandem Tube Stoes, og være av god og jevn kvalitet. Strikk av forskjellige typer bør ikke blandes på en innerbag. Bytt ut eventuelle slitte strikk slik at alle strikk har tilnærmet lik tilstand.
	\item Kontroll av innvendig feste av plasthåndtaket for drogue. Dette har en tendens til å slite innvendig på det hulvevde båndet.
	\item Festeløkke for feste av pilotline (drogue innerline) i kalotten. Denne blir slitt i knuten der den er festet i ringen, og er ikke alltid like synlig.
	\item Generell kontroll av drogue med innfesting og deler. Droguen får stor medfart og har en relativt høy slitasjefaktor.
	\item Reservehåndtak skal være av en åpen type, slik at hånda kan få tak. Reservehåndtak med myk pute (tilsvarende kuttpute) er ikke tillatt.
	\item Ved påmontert ekstra frigjøringshåndtak for drogue for passasjer (eller hoppmester) skal den doble loopen ved droguefrigjøring kontrolleres for riktig montering og treing av kabler. Frigjøringskablene må tres gjennom hver sin låseløkke.
	\item Kontroller webbing i området rundt ringen for festing av passasjerseletøyets låsespenner. På grunn av belastningsretningen ved skjermåpning, er denne ekstra utsatt for belastning.
\end{itemize}

\subsubsection{Kontroll av passasjerseletøy}
Vær påpasselig med blant annet følgende på passasjerseletøyet:
\begin{itemize}
	\item Webbing rundt hofteinnstrammere for slitasje
	\item Fjærer i alle spenner og eventuelle låsebolter
\end{itemize}

\subsubsection{Vector}
\begin{itemize}
	\item Hovedløftestropper skal være minimum av Type VII webbing - Ikke Type VIII (alminnelige løftestropper).
	\item Kontroll av ``W'' sømmønster ved innfesting av passasjerring. Poynter Vol. 2, s 194 (nederst) . Det skal være dobbel stingrad på langs av webbingen inn mot passasjerringen.
	\begin{figure}
		%\includegraphics[width=60mm]{Strekktesting av kalottduk.pdf}
		\caption{Kontroll av W- sømmønster Vector}
	\end{figure}

	\item Kontroll av kule ved LOR for manuell trekk av reserve. Det skal være påmontert en plastkule på enden av LOR lina.
	\item Kontroll av stoppwebbing på droguens centerline finnes og er forsvarlig montert. Denne forhindrer at hovedpinnen trekkes ut før frigjøring av drogue.
	\begin{figure}
		%\includegraphics[width=60mm]{Strekktesting av kalottduk.pdf}
		\caption{Stopwebbing innerline Vector Drogue}
	\end{figure}

	\item Kontroll av webbing ved festet av ringen som passasjerseletøyet kobles til – denne har en tendens til å bli ``spist opp'' i kantene.
	\item Droguens innerline, spesielt i området 180 – 210 cm ovenfor pinnen. Dette området ligger inntil den del av kevlar båndet som går oppover ved droguefrigjøring, og får da stor friksjon og påkjenning.
	\item Centerlina kan bli lysebrun (ikke forveksles med alminnelig skitten), og den gule merketråden blir svak. Dette er tegn på at innerlina er slitt og må skiftes.
	\item Drogue centerline skal være av 3⁄4'' tubular webbing (ikke 1⁄2'').
	\item Kontroll av feste fra drogue/ring til toppduken. Sjekk sømmer og slitasje i stoff.
	\item Avstiverplater på skulderputen ved kutt og reservehåndtak skal være påmontert.
	\item Det skal være påmontert leppe på siste lukkeklaff som beskytter pinnen på hovedcontaineren (ikke velcrolukking).
	\item Kun Vector pilotskjerm uten mesh, med 6'' topplate skal brukes på tandem utstyr.
	\item Kontroll og eventuell forsterking av sikk sakk søm på innsiden av reservepakksekken. De aktuelle sømmene (både høyre og venstre side) skal være forsterket med ytterligere 2 stk sikk sakk sømmer. Se F/NLF Modordre 7902, Kapittel 10.
	\begin{figure}
		%\includegraphics[width=60mm]{Strekktesting av kalottduk.pdf}
		Lokalisering av søm (tilsvarende på venstre side)
		Opprinnelig sømmønster
		Forsterkning av sømmønster
		\caption{Forsterkning i reservecontainer, Vector}
	\end{figure}

\end{itemize}

\subsubsection{Atom / Galaxy}
Alminnelig hovedkontroll ved pakking:
\begin{itemize}
	\item Alminnelig kontroll av webbing, velcro, duk, etc som hovedkontroll av annet utstyr.
	\item Kontroll av ekstra reservehåndtak, med innvendig montering som er koblet til LOR line.
	\item Kontroll av LOR 2, kabler, håndtak, og slitedeler.
	\item Pilotskjerm – stivhet i toppen, fester, fjær, fjærstyrke ikke under 12 kg.
	\item Drogue – slitasje ved møte mellom indre og ytre bånd.
	\item Hoftespennene på passasjerseletøy kontrolleres spesielt for funksjon og slitasje.
	\item Kontroll av LOR 2 lukking på lukkeklaff reserve.
\end{itemize}

\begin{figure}
	%\includegraphics[width=60mm]{Strekktesting av kalottduk.pdf}
	\caption{Lukking av Atom tandem reserve}
\end{figure}

\subsection{Livstidsløp byttedeler tandemutstyr}
På grunn av den eksepsjonelle bruk og de belastninger og slitasje tandemutstyr er utsatt for anbefaler fabrikantene å bytte / vedlikeholde tandemutstyr spesielt etter følgende:

Norske bestemmelser er at disse anbefalinger gjelder som påbudte serviceintervall på tandemutstyr.

\begin{table}
	\caption{Livstidsløp byttedeler tandemutstyr}
	\begin{tabular}{ | l | l | }
		\hline
		Del & Serviceintervall \\
		\hline
		Hovedskjerm i 0-P & Subjektiv bedømmelse av MK \\
		\hline
		Hovedskjerm i F-111 & byttes etter 600 hopp \\
		\hline
		Drogue med bånd og line & byttes etter 600 hopp \\
		\hline
		Linesett & byttes etter 200 – 300 hopp \\
		\hline
		Centerline i drogue & byttes etter 300 hopp \\
		\hline
		Reserve & begrenses til 20 gangers bruk \\
		\hline
		Hovedkontroll av pakksekk/seletøy & minimum hvert 200 hopp \\
		\hline
	\end{tabular}
\end{table}

Etter hvert femtiende hopp skal følgende i tillegg kontrolleres:
\begin{itemize}
	\item Bytt frigjøringskabler dersom de er deformerte, eller har merker i plastikken. (Både kabler på frigjøringshåndtak og drogueline).
	\item Rens og smør alle frigjøringskabler inn med 3-i-1 olje.
	\item Kontrollér 3-rings og kabelføring for virkemåte og montering. Rens eventuelt kabelhuset med børste. Kontroller låseløkkas tilstand.
	\item Kontrollér alt av metallkomponenter.
	\item Kontrollér innerbag, og bytt strikk om nødvendig.
\end{itemize}

Kontroll av tilstand og slitasje krever at det følger med en logg på tandemutstyret, som blir oppdatert etter bruk og kontroll. Dette bør finnes sammen med alt tandemutstyr.

NB. Listene begrenser ikke bytte og alminnelig vedlikehold under bruk og inspeksjoner.

\section{Anmerkninger og modifiseringer}
\subsection{Påbudte modifiseringer og endringer}
\begin{table}
	\caption{Påbudte modifiseringer tandemutstyr}
	\begin{tabular}{ | p{2cm} | p{1cm} | p{3cm} | p{4cm} | }
		\hline
		Produsent & Dato & Utstyr & Modifikasjon \\
		\hline
		Relative Workshop & 88.11.17 &
		Vector Tandem seletøy
		Se pkt. 7.7.3.2.3 – Vector &
		Kontroll av 4 punkts ``W''-søm på seletøy.
		Forsterkningssøm på ovennevnte sømmønster
		Montering av ``Bridle Stop'' på drogue.
		Kontroll av frigjøringshåndtak
		Montering av avstivere på skulderpadding for kutt og reservehåndtak.
		Montering av ``kule'' på LOR line for manuelt trekk av reserve
		Montering av ``leppe'' på siste lukkeklaff (ikke velcro) \\
		\hline
		Relative Workshop & 97.22.05 &
		Vector Tandem - Se F/NLF Serviceordre 9702 &
		Kontroll og evt. forsterking av sikk sakk søm ved skulderparti på innsiden av reservepakk- sekken i Vector tandemrigger \\
		\hline
		Relative Workshop & & PD 421 og 360 hovedskjermer &
		Linesett byttes fra 600 punds liner til 900 punds liner på ``A'' og ``C'' liner, samt 1500 punds liner på nedre styreliner. \\
		\hline
		Generelt & & &
		Montering av reserve håndtak i ``åpen'' type, enten metall håndtak eller fabrikantens type med stiv webbing.
		Reservehåndtak av myk type (kutt–pute) er ikke tillatt. \\
		\hline
	\end{tabular}
\end{table}

\subsection{Andre anmerkninger – oppdateringer}
\subsubsection{Pilotskjerm reserve}
På Vector tandemutstyr skal Vector Pilotchute modell 2 (uten mesh, og med stor topp og sterk fjær) benyttes. Tidligere modeller av Vector reservepilotskjermer (med mesh) er ikke tillatt.

På Atom og Galaxy skal Quick 2 eller Quick 3 fra Parachutes de France benyttes.

\subsubsection{Skifte av styreliner}
PD 360 skal ha installert eget bremsesystem, med flere forgreninger til halen. Dette gir bedre åpninger, og øker flare på landinger.

Denne endringen var aktuell rundt 1989, og siden den tid er de fleste PD 360 hovedskjermer byttet ut.

\subsubsection{Korte løftestropper}
PdF introduserte korte løftestropper på tandem hovedskjerm i mai 1994. Dette ble gjort for å kunne rekke opp til slideren og trekke denne inn, og dermed få bedre ytelse på skjermen, og minske slitasje på linene.

Bremsesettingen på disse løftestroppene sitter nærmere sjaklene enn på de med vanlig lengde (10 cm), og dette medfører at bremselinene må være 13 cm kortere. Med de korte løftestroppene følger nye nedre del av bremseliner for montering.

Vær varsom med at ikke det påmonteres korte løftestropper, uten at det samtidig byttes nedre del av bremseliner.

Parachutes de France oppgir følgende mål på nedre styreliner:
\begin{table}
	\caption{Spesifikasjoner lengder bremseliner Atom}
	\begin{tabular}{ | l | l | l | }
		\hline
		& Indre bremseline & Ytre bremseline \\
		\hline
		Standard løftestropper: & 415 cm & 290 cm \\
		\hline
		Korte løftestropper: & 402 cm & 277 cm \\
		\hline
	\end{tabular}
\end{table}

\subsubsection{Innerbager}
På BT-80 hovedskjerm skal innerbag fra Parachutes de France benyttes, selv om BT- 80 er montert i Vector seletøy.

Denne har 10 maljer for feste av liner, og er sikrere i konstruksjonen ved at skjermen bedre holdes inne i bagen til alle linene har sluppet strikkene.

Andre innerbager bør ha minimum 4 lukkemaljer montert.

Det skal brukes tandem strikk av god kvalitet, Tube Stoes for tandem eller tilsvarende. Strikkenes låsefunksjon kan være avgjørende for åpningssekvensen av hovedskjermen. Bytt strikk før den ryker.

Ved pakking med vanlig strikk skal denne surres dobbelt rundt linene. Dette gjelder ikke strikken som holder lukkeklaffen lukket på innerbagen.

For ytterligere og utfyllende informasjon om tandemutstyret henvises til fabrikantenes egne tandem manualer.
