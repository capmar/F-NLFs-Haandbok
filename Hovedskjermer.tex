\part{Hovedskjermer}

\section{Produsenter og godkjenning}
\subsection{Generelt}
I alminnelighet godkjennes de forskjellige produsenter for hovedskjermer, uten at det er spesielle hindringer for de enkelte skjermtyper. Produsenten godkjennes etter spesielle kriterier, beskrevet nærmere i kapittel 1.2 i denne boka. Produsenter med alminnelig typegodkjenning for hovedskjermer er angitt i vedlegg 1.

Se forøvrig kapittel 1.2.2 Nyregistrering og typegodkjenning av nytt utstyr. For utfyllende informasjon om typegodkjenningsprosessen.

For nærmere opplysninger henvises til manualer, Poynter Vol. 2, og hver enkelt produsent.

\section{Montering og modifiseringer}
\subsection{Montering av hovedskjermer}
På hovedskjermer kan følgende links benyttes:
\begin{enumerate}
	\item Rapid-Link Maillon Rapide: Nr. 5 og 6, samt rustfrie nr. 4 eller 3.5.(Rustfrie er stemplet INOX)
	\item Parachutes de France link – ``D-type'' med bolt og venstredreid låseskrue.
	\item Liner tillates sydd direkte på løftestropper – men dette anbefales ikke da det vanskeliggjør vedlikehold og bytte av slidermaljer.
	\item Soft links fra typegodkjent produsent av hovedskjermer.
\end{enumerate}

Det anbefales å bruke sliderstoppere (``bumpers'') av gummi, webbing eller lignende, for å unngå skader på slidermaljer som vil slite unødig på linene. Ved bruk av soft links er det anbefalt å bruke soft bumpers. Det er kun bumpers produsert for soft links som kan nyttes.

Ellers henvises det til Kapittel 9 – Komponenter for mer utfyllende informasjon og begrensninger vedrørende links.

\subsection{Kontroll og montering av soft links}
Soft links skal alltid monteres i henhold til monteringsbeskrivelse fra produsenten. Soft links levert av typegodkjent produsent er tillatt brukt i Norge. Hvis det er montert soft links med ukjent opphav eller fra ikke typegodkjente produsenter (for eksempel laget av en MR/rigger), så skal disse fjernes og erstattes før skjermen tillates brukt.

Soft links må inspiseres grundig ved hver hovedkontroll. Kontroller om den er riktig montert og at den holdes enkelt på plass. Enkelte typer soft links må gjerne holdes på plass ved bruk av vokset tråd eller tilsvarende. Se også etter slitasje, spesielt rundt stoppekloss/ring som holder soft links sammen.

Som hovedregel bør det benyttes soft links fra samme produsent som hovedskjermen så langt det er mulig og formålstjenlig.

\subsection{Hovedskjermer med bruksforbud}
Det er sjelden at hovedskjermer blir ilagt bruksforbud. Grunner for dette kan være at produsenten ikke er godkjent, men at skjermene er i alminnelig bruk i andre land, eller at Norge ikke ønsker å godkjenne enkelte skjermtyper av forskjellige grunner. Det er som hovedregel produksjonsfeil på enkelte hovedskjermer som medfører bruksforbud og eventuelt inndragelse av typegodkjenning hos produsenten. Manglende kvalitetssikringssystem og oppfølging av serviceordre/modifiseringer er også grunn for å nekte tÿpegodkjenning av produsenter av hovedskjermer.

Det er til enhver tid opp til SU å kunne nekte/inndra typegodkjenning av produsenter og nedlegge bruksforbud av enkelte hovedskjermer.

Se tabell 3 i vedlegg 1 for nærmere beskrivelse av produsenter uten typegodkjenning og hovedskjermer med bruksforbud.

\subsection{Andre anmerkninger}
Foreligger det mistanke om at væte, urenheter etc. har trukket gjennom/inn i pakksekken, skal skjermen åpnes og kontrolleres. Skjermen tørkes og renses før bruk i de tilfeller hvor dette måtte vise seg nødvendig.

\section{Pakking av hovedskjermer}
\subsection{Generelt}
Fallskjerm som har vært lagret i pakket stand i mer enn 6 måneder er ikke luftdyktig. For å bli luftdyktig igjen, må fallskjermen luftes og pakkes på ny.

Foreligger det mistanke om at væte, urenheter etc. har trukket gjennom/inn i pakksekken, skal skjermen åpnes og kontrolleres. Skjermen tørkes og renses før bruk i de tilfeller hvor dette måtte vise seg nødvendig.

Selvstendig pakking kan kun gjøres av den som har gyldig pakkesertifikat for vedkommende type fallskjerm. Pakkekontroll skal utføres under pakking. Pakking og kontroll utføres etter gjeldende instruks.

Produsentens pakkemanualer skal følges når det foreligger. Løftestroppene skal være festet til selen før fallskjermen pakkes.

Se forøvrig Poynter Vol. 2, 9.6 for illustrasjoner av sidepakking, pro pakking og rullepakking.

\subsection{Pakking for riktige åpninger}
Det er mange faktorer som bidrar til åpningskarakteristikkene på en fallskjerm. Noen faktorer ligger i design og linetyper. Åpningsbelastningen som blir overført til hopperen blir for eksempel redusert i forhold til hvor mye linene strekker seg.

Spectraliner strekker seg mindre enn Dacronliner – så høyere åpningsbelastninger blir overført ved bruk av Spectraliner. Tilsvarende gjelder for liner av typen Vectran og HMA. Alle kalotter har innebygde åpningskarakteristikker, men de åpninger du erfarer vil være basert på andre faktorer som du selv kontrollerer. Noen skjermer er mer vare på disse faktorene enn andre. Når ikke disse faktorene blir kontrollert, kan de bidra til å forårsake ekstremt harde eller uregelmessige åpninger, kalottskade, feilfunksjon og til og med personskader.

Det er viktig at du forstår hvordan disse faktorene virker sammen, slik at du kan kontrollere og påvirke åpningens karakteristikkene.

Hovedfaktorene som har innvirkning på åpningskarakteristikkene er:
\begin{enumerate}
	\item Pakkemetode
	\begin{enumerate}
		\item Skjermens symmetri ved pakking
		\item Bretting / rulling av front
		\item Bretting/rulling av halen
		\item Plassering av front
		\item Slider plassering
		\item Innleggiinnerbag
	\end{enumerate}
	\item Innfesting av liner
	\item Pilotskjerm
	\item Hopperens frittfall hastighet og kroppsstilling
\end{enumerate}

\subsubsection{Pakkemetoder}
Det er anbefalt at du følger fabrikantens pakkemanualer, da det ikke er sikkert at andre fungerer like bra over tid. Rullepakking er generelt ikke anbefalt, da de folder seg usymmetrisk ut. Dette kan føre til usymmetrisk fylling av luft, med harde åpninger, dreining eller lignende. En korrekt utført PRO pakk vil ivareta nødvendig symmetri. Pass på at duken legges utover mellom linefestene (A – B linene, B-C linene og C- D linene).

Rulling av fronten vil ofte medføre saktere åpninger, men kan også medføre mer uryddige åpninger. Spesielt på høyverdige skjermer anbefales det ikke rulling av fronten fordi det kan medføre usymmetriske åpninger.

Rulling av halen utføres stort sett for å holde duken bedre på plass. Det er ofte liten direkte betydning på selve åpningsforløpet, men når duken holdes på plass vil det kunne bedre åpningssekvensen. Pass på at styreliner ikke dras rundt mot fronten av skjermen når man ruller halen. Dette kan forårsake en line-over feilfunksjon.

Jo mindre kalotten er, jo mer følsom er den for dreining ved skjev belastning.

Det er viktig at slideren plasseres riktig. Den skal dras helt opp til sliderstopperne som er sydd på stabilisatorene på skjermen. Linene bør ligge jevnt på oversiden av slideren, slik at disse ikke presser slideren ned for tidlig. Midten av slideren bør dras inn mot midten av centercella. Slideren foldes ut fullt og jevnt til alle sider, slik at den ligger klar for å fange luft.

På enkelte 0-porøsitets skjermer kan fremre halvdel av slideren trekkes ut forbi ``A'' linene for bedre å fange luft. På slidere av kollapsbar type er det veldig viktig at linene/strammeren blir strammet ut og at de dras maksimalt tilbake i kanalen i slideren. Slider med en snor med stramming må legges ned i midten av slideren slik at denne ikke kan henge seg opp i linekaskader.

Når skjermen legges i bagen må pakkingen og slideren uroes minst mulig, og den skal inn i bagen så jevnt og likt som mulig. Jo mer det stappes, jo mer uorganisert blir skjermen og sliderens posisjon. La halen ligge tett rundt linene slik at slideren blir liggende på plass. En jevn innlegging i bagen gir bedre resultat når alt skal ut igjen.

\subsubsection{Innfesting av liner}
Liner skal ideelt slippes ut av strikkene med en innfesting av gangen. Dette virker selvfølgelig, men er ikke like enkelt som det høres ut til. Når pilotskjermen trekker bagen ut av containeren, bremser den bagens hastighet nedover dramatisk. Linenes hastighet videre nedover blir bestemt av hvor hardt de er innfestet i strikkene. Ved løse strikker vil hastigheten av linene ikke bli bremset opp, og de vil fortsette videre sammen med hopperen. Dersom flere strikker er løse, kan flere linebunter slippe samtidig. Dette er ``line dump'' og kan innvirke dramatisk på videre åpningssekvens. Skjermen vil kunne slippe ut av innerbagen lenge før linene er stramme, og begynne å fylles med luft, og linene vil være helt ukontrollerte. Dette kan føre til store skader for utstyr og ikke minst hopperen som kan få en voldsom belastning.

For å unngå dette må linene være skikkelig festet til innerbagen. Linene bør sløyfes inn med 5-6 cm gjennom strikkene for å unngå at linene slipper med for lav trekkraft. En måte å kontrollere dette på er å se om de holder mellom 4 og 6 kilo når du trekker linene ut av innerbagen på bakken. (Bruk en vanlig fiskevekt på pilotbåndet). Større, tyngre skjermer trenger mer kraft på innfestingene enn mindre, og også skjermer som åpner ved større frittfall hastigheter

Kontroller strikkene og bytt dem før de ryker! De bør være av samme kvalitet, og slippe jevnt. Stram dem inn om nødvendig ved montering. Eventuelt kan dobbel strikking være en løsning hvis de ikke holder tilstrekkelig trekkraft. Ved bruk av Tube Stoes eller lignende bør disse være festet ifølge fabrikantens anbefalinger, med muligheter for innstramming. Tube Stoes anbefales ikke brukt på microliner eller andre tilsvarende liner som for eksempel Spectra, Vectran og HMA.

\subsubsection{Pilotskjerm}
Pilotskjermen har stor innvirkning på åpningen. Størrelsen, duktype, lengde på pilotbånd, er bare noen av mange faktorer som innvirker. Enkelte pilotskjermer har for mye drag (motstand) ved terminalhastighet. Dette kan lettere føre til line dump, fordi innerbagen blir bremset ned ekstremt hurtig.

Etter at linene har strukket seg vil hopperen trekke med seg skjermen, pilotskjermen og linene videre til nær terminal hastighet. (``Snatch Force''). En for stor pilotskjerm kan ha nedbremset innerbagen såvidt mye, at denne belastningen blir langt større enn nødvendig. Dette merkes av hopperen, og det slites på innfestingen av pilotbåndet på toppduken av skjermen. Skjermen får også et kraftig rykk i seg, og pakkejobben blir spredt ut hurtigere. Alt dette kan forårsake hardere åpninger, siden kalotten kommer ut uorganisert.

Dersom pilotskjermen er for liten, eller gir for lite drag ved at den er utslitt, er dette også farlig. Pilotskjermen skal være av riktig størrelse og produsere tilstrekkelig drag.

En F-111 pilotskjerm bør være 30'' til 36'' i størrelse. Strikkpilot av F-111 duk tillates ikke brukt

Pilotskjermer laget i 0-porøsitetsduk er mer følsomme for design, og to forskjellige typer i samme størrelser kan ha store innbyrdes forskjeller. Nettingstørrelse og hullet ved feste av pilotbåndet har store innvirkninger. 0-porøsitets pilotskjermer bør ligge på rundt 24'' til 30'' diameter og bør ha relativ tynnmasket mesh.

Forskjellige typer inntrekkbare pilotskjermer har forskjellige innvirkninger også. En kollapsbar pilotskjerm med strikk kan for eksempel feilfunksjonere pilotskjermen dersom duken er gammel og slitt. Strikkpilot av F-111 duk tillates derfor ikke brukt. Det er viktig å velge riktig størrelse og type pilotskjerm basert på seletøyfabrikantens og hovedskjermfabrikantens anbefalinger.

\subsubsection{Hopperens frittfall hastighet og kroppsstilling}
Alle som har trukket i marsj vet og forstår at jo høyere hastighet en har ved skjermåpning, jo hardere kan åpningene bli.

Mindre hoppdresser og mer bruk av bly har økt fallhastigheten i FS, og kan bidra til økt åpningsbelastning. Freeflying gir også høyere fallhastighet. Det er viktig å tenke på nedbremsing av hastigheten før trekket. Flat marsj og god tid til seperasjon er veldig viktige faktorer.

Dersom det hoppes på hoppfelt med stor høyde over havet, vil åpningene generelt være hardere på grunn av tynnere luft. En litt mindre pilotskjerm, og eventuelt en litt lenger pilotline kan bidra til å minske disse problemene

Også hopperens kroppsstilling kan ha stor betydning for åpningen. Spesielt på høyverdige skjermer kan usymmetrisk kroppsstilling påvirke åpningen slik at det påføres tvinn eller gir uryddig åpning på andre måter. Hoppere som klager på dreining i åpningen bør tenke på sin egen kroppsstilling. Hvis skuldrene ikke holdes vannrett vil dette kunne påføre dreining.

\subsection{Åpningshastighet}
Over tid åpner skjermen seg vanligvis saktere på grunn av økt luftgjennomstrømming. Dette kan være ubehagelig og eventuelt også føre til uønskede feilfunksjoner og ``unødvendige'' kutt. På en del 0P skjermer kan faktisk effekten være motsatt fordi luft slippes inn andre stedet enn fronten som normalt og skjermen fyller seg med luft raskere.

Det er ikke sikkert at enhver endring virker likt på forskjellige skjermer. Vær derfor varsom med for store endringer av gangen.

Følgende tiltak er blant noen som er mulige:
\begin{table}
	\caption{Endringer av åpningshastighet}
	\begin{tabular}{ | p{2cm} | p{7cm} | p{1cm} | }
		\hline
		Tiltak & Forklaring & Utføres av \\
		\hline
		Endre pakkemetode &
		Pakke slik at fronten får tak i mer ``frisk luft'' under åpningssekvensen.
		Ikke rull fronten, eller stapp den inn i midten &
		Pakker \\
		\hline
		Bytte slider &
		Bytte til mindre slider, utføres med varsomhet.
		Montering av ``Bikini'' slider har gitt gode resultater på enkelte skjermer. Denne slideren har et diagonalt kryss i slideren hvor de to trekantfelt sideveis er fjernet. &
		MK \\
		\hline
		Skjære hull i slider &
		Et hull med cirka 20 cm i diameter skjæres ut med varmekniv i midten av slideren.
		Vær varsom så ikke man skjærer i den gjenværende del av slider &
		MK \\
		\hline
		Fjerne duk i slider,
		Slisse i front &
		Et forsterkningsbånd syes på slideren cirka 5 cm bak inn fra fronten, og stoffet imellom dette og fronten fjernes med varmekniv.
		Performance Designs anbefaler dette på sine 9-cellere. &
		MR \\
		\hline
		Sy inn slider &
		Slideren kan syes inn til cirka 60\% av det tidligere arealet i skjermens fartsretning. &
		MR \\
		\hline
		Endring av bremseplassering &
		Bremsesettingen kan flyttes litt oppover (mere hastighet). Dette bør prøves ut først ved bruk av ``Daisy Chain''.
		Noen skjermer vil gjerne ha mere brems for bedre åpning (Foil).
		Når bedre bremesetting er funnet kan en ny fingertrapping monteres på stedet. &
		MK
		MR \\
		\hline
		All duk i slider &
		Duken byttes ut med netting, som gir en erstattes med mesh ``åpen'' slider. &
		MR \\
		\hline
		Slisse i fronten av &
		Spesielt på PD skjermer. En cirka 5 cm’s slider åpning lages i forkant av slider og forsterkes med bånd. Dette slipper mer luft inn i fronten av skjermen under første del av åpningssekvensen &
		MR \\
		\hline
		Innkorting av liner &
		Fremre liner kan kortes inn cirka 5 cm, eventuelt kan fremre løftestropp kortes inn det samme.
		Innkorting av linelengder og trim påvirker skjermens åpnings- og flyegenskaper, og må gjøres i samråd med fabrikkens anbefalinger. &
		MR \\
		\hline		
	\end{tabular}
\end{table}

Vær varsom med å gjøre flere av disse endringer samtidig. Prøv ut én av gangen, fra de enkleste og oppover.
