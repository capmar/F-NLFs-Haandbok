\part{Komponenter}

\section{Fallskjermduk}
Ripstop veving er meget motstandsdyktig mot å revne, og kjennes lett igjen på det rutete vevingsmønsteret. All fallskjermduk i dag er produsert med ripstop veving, og det finnes forskjellige produksjons og tilvirkningsprosesser som gir mange muligheter for spesialproduksjon av ønskede egenskaper.

Ref.: Poynter Vol. 2, 4.021, 4.034, 4.036

\subsection{Historikk}
I grove trekk er moderne fallskjermduk utviklet gjennom fire (syntetiske) steg siden silke ble benyttet: (Ref: Tabell 1 – Sammenligningstabell for duktyper)
\begin{enumerate}
	\item Ripstop nylon, MIL-C-7020. Denne typen finnes i MA-1 pilotskjerm, og i enkelte eldre reserveskjermer (T-10R, NAVY 26 fot Conical).
	\item Kalendrert (behandlet) MIL-C-7020, ripstop nylon. Luftgjennomstrømmingen i denne duken ble redusert ved en behandlingsprosess, hovedsakelig av Pioneer og Security i 60 årene. Denne typen finnes bl.a. i eldre Triconical reserver. Denne ble produsert stort sett i vektområder fra 1,1 til 2,2 oz/sq.y. (ounce / square yard).
	\item F-111 generasjonen: F-111 er et eget varemerke, ble introdusert av George Harris i 1979, og var den første kalottduk som var spesielt produsert for vingskjermer. Duken ble vevet av uspunnede fibre, og fikk derfor en lav permeabilitet (se 9.9.1.2), og ble meget populær. (Pegasus, Viking Superlite, Cloud Lite etc.) Duken benyttes fremdeles i dag på de fleste alminnelige firkantskjermer, og i de fleste firkantreserver. Av andre duker med tilsvarende egenskaper finnes: bl.a. Sport Chute, Exacta-Chute, PF 2000 og PF 2500.
	\item Belagt nylon duk. Null-porøsitetsduk ble belagt for å tette duken helt. Det ble brukt akryl, urethan eller silikon for å oppnå dette. Akryl minket rivestyrken noe og ble ikke brukt videre. Silikonbelegg kan faktisk øke bruddstyrken på duken med opp til 15\%. Alt belegg øker både vekt og pakkevolum på skjermen, og gir en glatt overflate som gjør det vanskelig å presse luften ut av skjermen under pakking.
	Ved bruk av 0-porøsitetsduk er det blitt mulig å utnytte helt andre vingeprofiler enn tidligere, og dette har resultert i at skjermene tåler høyere vingebelastning, og kan få tilsvarende eller bedre flyegenskaper enn skjermer produsert i vanlig F-111.
	0 - Porøsitetsduk blir i dag kun benyttet til hovedskjermer, men vil etterhvert sannsynligvis også finne veien til reserveskjermer. I dag finnes det en rekke merkevarer for 0-porøsitetsduk: Zero-P3, Soar-Coat, PF 3000 er blant de mest brukte.
\end{enumerate}

\subsection{Porøsitet (Porosity) - Permeabilitet (Permeability)}
Porøsiteten på nylonduk angis i måleenheten kubikkfot luft pr. minutt som strømmer gjennom en kvadratfot med duk, under trykk tilsvarende 1/2 tomme vanntrykk. Forkortet skrives dette f.eks. 100 CFM (Cubic Feet Minute).

Permeabilitet er betegnelsen som blir brukt for denne verdien. Den er vanskelig å kontrollere, og oppgis derfor med en øvre og nedre verdi. For eksempel har ubehandlet 1,1 oz ripstop 80-120 CFM.

Ved en varmvalse-behandling dras duken gjennom et sett roterende varmevalser. Fibrene klemmes flate mot hverandre og porøsiteten blir da 40 - 50 CFM for 1.1 oz ripstopnylon. Denne typen finnes i Strong, Pioneer og Security 26 fot lopo reserveskjermer. Valseprosessen svekker duken en del.

Impregnering kan sette porøsiteten ytterligere ned til 0 CFM. Det er i løpet av tiden oppdaget forskjellige ulemper ved impregnering (bl.a. klebing og forråtnelse), og derfor er dette sluttet med i de senere år. Blant annet derfor vil 20 års grensen for reserveskjermer bli opprettholdt. Nyere produksjonsprosesser har fjernet mye av disse problemene.

Jo større luftgjennomgangen er, dess mindre ytelse er det igjen i skjermen. Dette merkes spesielt ved midtcelleområdet på eldre firkantskjermer. Duk som ikke er belagt har en tendens til å åpne seg ved alminnelig bruk og porøsiteten øker spesielt etter 300 – 400 hopp. Dette merkes spesielt ved F-111 duk.

Ny F-111 duk har en permeabilitet på 0-3 CFM når den er ny. Etter noen hopp, begynner vevingen å åpne seg, og skjermen blir mer porøs etter hvert hopp. Når permeabiliteten går mot 8 CFM begynner skjermen å miste flyegenskaper, og ved 13 CFM begynner den å snivle i åpningen, synker fortere, og lander hardere.

Etter bare 50 hopp merkes forskjell på både flyegenskaper og håndtering sammenlignet med en ny tilsvarende skjerm, etter 300 hopp har kalotten vesentlig forskjellige ytelse, og etter + 500 hopp er det ikke mye liv igjen i duken. Dette merkes best etter landing, dersom skjermen kollapser fullstendig er den sannsynligvis meget porøs. Det finnes måleinstrumenter på markedet for dette, men det er ikke praktisk å sette grenser for hvilke verdier som skal aksepteres.

Vann sliter mye på kalottduk, og skal derfor unngås i størst mulig grad. Vannet ``avkalendrerer'' belegget, og åpner for luft mellom fibrene. Etter alt fra 3 – 30 vannhopp vil kalotten få store åpningsproblemer. Unngå derfor vannhopp, vask ikke kalotten nedsenket i vann, og hopp heller ikke i regn med F-111 duk.

Det er ikke mulig å ``oppgradere'' slitt duk, ved å belegge dette på nytt. Det har vært eksperimentert noe med å belegge slitt duk, men dette har ikke gitt gode resultater.

Permeabiliteten endrer seg ikke i pakket tilstand, men er avhengig av bruk og slitasje. Performance Designs har etablert et testprogram for sine reserver, der de vil sjekke permeabiliteten etter bruk, et visst antall hopp, og et visst antall pakkinger.

Det er også produsert duktyper i polyester og kevlar, men disse egner seg ikke for fallskjermer foreløpig, da de blir for tykke og tunge. Disse duktypene er populære innen produksjon av bl.a. paraglidere.

\begin{table}
	\caption{Sammenligningstabell for duktyper}
	\begin{tabular}{ | l | c | c | c | c | }
		\hline
		Merke & Permeabilitet (CFM) & Vekt (oz /sq.yard) & Bruddstyrke (lbs) & Rivestryke (lbs)  \\
		\hline
		Ripstop, ubehandlet & 80-100 & 1,1 & 42 & 5 \\
		\hline
		Ripstop, behandlet & 40-50 & 1,1 & 42 & 5 \\
		\hline
		F-111 & 0-3 & 1,12 & 45 & 5 \\
		\hline
 		Soar Coat & 0 & 1,13 & 47 & 14 \\
		\hline
	\end{tabular}
\end{table}

\subsection{Vekt av nylonduk}
Vekt av nylonduk er en av spesifikasjonene som oppgis, og det brukes vanligvis amerikanske enheter.

\begin{table}
	\caption{Konverteringstabell mål}
	\begin{tabular}{ | l | l | }
		\hline
		Engelsk & Metrisk \\
		\hline
		1 lb(s) / 16 oz (ounce) & 453,6 gram \\
		\hline
		1 oz & 28,35 gram \\
		\hline
		1 yard / 3 fot & 91.44 cm \\
		\hline
		1 fot / 12 tommer (inch) & 30,48 cm \\
		\hline
		1 inch & 2,54 cm \\
		\hline
	\end{tabular}
\end{table}

At en type nylonduk angis med vekt 1,1 oz/yd, betyr at en kvadratyard veier 1,1 oz. Se sammenligningstabell for kjente duktyper. For ytterligere konvertering av mål og verdier henvises til Appendix i Kapittel 11.

\subsection{Ultrafiolette stråler og solskinn}
De ultrafiolette strålene fra solen vil skade nylon og ødelegge en kalott på kort tid. Når skaden er skjedd kan ikke stoffet bringes tilbake til opprinnelig styrke. Sørg derfor for at skjermen ikke er eksponert for sol mer enn høyst nødvendig. Lysstoffrør avgir også UV-stråler.

UV stråler kan også trenge igjennom duk. Dette er en av grunnene til at reserveskjermer er gitt operativitet i 20 år – nylon vil tape styrke over tid og all påvirkning av UV-stråler.

Følgende viser litt om hvordan styrken på fallskjermduk svekkes over tid:

\begin{table}
	\caption{UV strålers innvirkning på fallskjermduk}
	\begin{tabular}{ | l | l | }
		\hline
		Duk, lagt utendørs i sollys & \% svekkelse fra opprinnelig styrke \\
		\hline
		en uke & 52 \% \\
		\hline
		to uker & 71 \% \\
		\hline
		tre uker & 94 \% \\
		\hline
	\end{tabular}
\end{table}

\subsection{Ripstoptape}
RIPSTOPTAPE er 1.5 oz ripstopnylon med en klebrig bakside.

Den kan benyttes til lapping av små hull og rifter, hjørnene må rundes for best feste. Tapen plasseres først på innsiden med god overlapp til skaden. Dersom duk er fjernet må all eksponert limflate bindes med talkum, og eventuelt forsterkes med en søm i kanten. Deretter festes en tilsvarende bit på utsiden. Denne bør ikke være større enn den indre biten.

Lapper med ripstoptape bør kontrolleres spesielt, da limet på noe ripstop kan svekke duken over tid.

Denne reparasjonsmetode er IKKE godkjent på reserveskjermer.

\section{Liner}
Valget av liner på en skjerm har stor viktighet for både flyegenskaper og åpningskarakteristikker. Det har derfor lenge vært eksperimentert med forskjellige linetyper, antall linefester, og forskjellige linetykkelser. Over 30 \% av luftmotstanden til skjermen kommer fra linene. Det er ikke så mye tykkelsen som fasong av linene som utgjør endringer i luftmotstanden. En noe tykkere, vinge- eller dråpeformet line kan ha mindre luftmotstand enn en tradisjonell rund line.

I de senere år har det blitt mer alminnelig med tynnere og tynnere liner for å oppnå best ytelse og glidetall på skjermen. Disse er meget tynne, er veldig sterke, og har liten elastisitet. Dette betyr videre at større deler av åpningssjokket vil bli overført til skjermen og seletøyet.

Det er stor forskjell på overføringen av åpningssjokket gjennom linene til seletøyet og hopperen, og dette vil igjen gi forskjellige åpningskarakteristikker. Jo mindre elastisk linene er, jo mer slitasje på utstyret og hopperen, og jo mer elastiske de er, jo mer kan skjermen få dårligere flyegenskaper.

Det meste av belastningen vil hopperen merke under ``snatch force'', og er tiden det tar fra linene er strukket ut av pilotskjermen, og til hopperen setter disse og skjermen i bevegelse igjen. Målinger har vist at for eksempel Spectra liner kan ha så mye som 4 ganger så mye belastning i dette området som mer elastiske liner.

\begin{figure}
	%\includegraphics[width=60mm]{Strekktesting av kalottduk.pdf}
	\caption{Liner og åpningssjokk}
\end{figure}

\subsection{Nylon Cord}
MIL C-5040: Denne linetypen kjennetegnes ved en hylse med et varierende antall kordeler (3-11) inni. Kordeler ser ut som tykk tråd.

Hylsa alene tåler 200 lbs, og hver av kordelene tåler 50 lbs. Disse ble i stor del brukt på militære skjermer, og finnes kun på T-10R av godkjent utstyr i dag. Hylsa uten kordeler er fin til å bruke som pullupcord.

Nylon Cord, Braided MIL-C-7515 (Poynter Vol. 1, 4.041): ``Braided line'' er flettede liner uten indre kordeler. Flate 650 lbs liner av denne typen finnes på National 26 fot lopo, Tri Conical, Strong 26 fot lopo.

Nylon Cord, Braided, Tubular Spliceable MIL-C-17183 (Poynter Vol. 1, 4.043 side 73).

Flettede liner uten indre kordeler kan finger-trappes.

\subsection{Dacron}
\begin{description}
	\item[Dacron cord, Braided Tubular, Spliceable,] (Poynter Vol. 1, 4.043 side 73)

	Dacron er et registrert varemerke på en spesiell type polyester fiber. Det betyr at noen fallskjermprodusenter sier at bærelinene er av Dacron mens andre sier polyester. Polyesterliner ligner nylon, men er noe mindre elastiske.

	Dagens firkantskjermer har vanligvis (dacron) liner av denne typen. Siden disse linene ikke er produsert etter noen mil. standard, varierer kvalitet og brudd- styrke. Typiske bruddstyrker kan være 400 lbs til 925 lbs.

	\item[Dacron Cord, Braided, Flat.] Flatvevd polyester line er brukt i noen få firkantskjermer. Denne typen kan ikke finger-trappes, men må syes sammen.

	Dacron liner er forstrukket ved produksjon for å opprettholde riktige linelengder. De strekker seg derfor lite ved bruk, og har større motstand mot UV stråler enn nylon.

	For å få tilmålt nøyaktige lengder, må alle nylon liner strekkes (pre-streching), og så måles og kappes. Flettede liner er strekt på forhånd, og har dessuten en annen konstruksjon som ikke krever en slik prosedyre i samme grad. Det er allikevel å anbefale å strekke dem litt før måling og kapping.
\end{description}

\subsection{Kevlar}
Bærelinene av kevlar er brukt i noen få typer firkantskjermer. Kevlarliner har en høy bruddstyrke i forhold til volum og vekt, men de har to utpregede svakheter:
\begin{enumerate}
	\item Kevlar er betydelig mindre elastisk enn polyester (Dacron) og nylon. Derfor vil en fallskjerm med kevlarliner gi kalotten, løftestroppene og seletøyet en langt høyere påkjenning under åpningssjokket siden linene ikke vil ``fjære'' så mye som liner av polyester og nylon.
	\item Kevlar tåler lite fysisk slitasje og skarpe knekk. Dette gjør at sammenføyninger og festepunkter blir kritiske og at levetiden kan bli langt kortere enn for polyester eller nylonliner.
\end{enumerate}

\subsection{Spectra (microline)}
Spectra er opprinnelig en fiber som ble introdusert i 1985 for militært bruk. Når denne fiberen ble brukt i linesett viste det seg at det oppsto en del vanskeligheter med montering og forskjellig strekk av linene. Det tok mange år før Spectra ble utviklet til en line som hadde de egenskaper som trengtes til fallskjermproduksjon. Store mengder fallskjermer måtte returneres til fabrikker på grunn av ujevnt strekk i linelengdene.

Spectra er i dag tilgjengelig og alminnelig brukt som fallskjermliner i to bruddstyrker, 550 lbs og 825 lbs. Diameteren på 550 lbs Spectra er ikke tykkere enn blyet i en blyant.

Dette har betydning for nøyaktigheten ved kontroll av liner. Et lite kutt på 0,5 mm av en line med bare 2 mm tykkelse vil gi en langt større svekkelse enn tilsvarende kutt på tykkere line!

Spectra har etterhvert fått gode egenskaper for fallskjermer, de strekkes svært lite. Senere undersøkelser har vist at Spectra liner har en slitasjeegenskap at de krymper over tids belastning, og ikke strekker seg vesentlig. Dette må tas hensyn til ved kontroll av kalotter, da spesielt mindre skjermer er avhengig av innbyrdes riktig linelengder, samt riktig tyngdepunkt, for å gi riktig håndtering og ytelse. Undersøkelser har vist at enkelte liner kan bli mellom 2,5 og 10 cm kortere, avhengig av hvor de er på kalotten, og styreliner opptil 20 cm kortere etter 800 - 1000 hopp.

Det bør derfor vises aktsomhet ved bytte av liner, slik at en ikke blander sammen Spectra og andre typer liner i samme skjerm.

\subsection{PF liner}
Parachutes de France har produsert egne linetyper, bla 320 kg Optima som er mest i bruk på skjermer etter 1992.

\subsection{Vectran}
Nyutviklet linetype fra 1997 med mål å utfylle Spectra, da Vectran skal ha bedre strekk egenskaper over tid, og ikke vil endre karakter i så stor grad som Spectra. Linene utvikles av Precision Aerodynamics, USA.

\section{Tråd}
(Poynter Vol. 2, pkt. 4.045.)

Det er viktig å bruke den tråd som er dimensjonert for forskjellig arbeid. Bomull og nylon bør ikke blandes sammen ved bruk av tråd og duk, nylonfilamentene er sterkere enn bomullen, og skjærer disse av etter en stund.

Kontroller merking av tråd før bruk. Bomullstråd kan brukes for festing av kabelføring etc. En bomullsknute vil holde bedre enn en knute i tilsvarende tråd i nylon. Bruk gjerne egen vokset nylon tråd for forskjellig håndarbeid.

\begin{table}
	\caption{Trådtyper}
	\begin{tabular}{ | l | l | l | l | }
		\hline
		Type & Bruddstyrke & Kvalitet & Anvendelse \\
		\hline
		E & 8,5 lbs / 3,8 kg & nylon & Kalotter, pakksekk rep. \\
		\hline
		3 cord & 24 lbs / 11 kg & nylon & Pakksekker \\
		\hline
		5 cord & 42 lbs / 19 kg & nylon & Seletøy \\
		\hline
		6 cord & 50 lbs / 23 kg & nylon & Seletøy \\
		\hline
		20/4 & 4,7 lbs / 2,1 kg & bomull & Brytetråd \\
		\hline
		Super Tack & 80 lbs / 37 kg & nylon & Vokset festetråd, flat \\
		\hline
	\end{tabular}
\end{table}

\section{Bånd og webbing}
I fallskjermer blir bånd (tapes) og webbing ofte brukt om hverandre, selv om bånd generelt er tynnere og letter materiale. (Poynter Vol. 2, kap. 4.060.)

Bånd brukes for det meste til forsterkninger og avslutning av søm i nylonduk (kantebånd), men også til pilotbånd og lignende.

Bånd er enten flate eller hulvevde. Det hulvevde brukes i mindre grad på sportsutstyr i dag, men er å finne som pilotline på runde reserver, og som innerline i droguen på Vector tandemutstyr. Vær forsiktig så ikke hulbånd blir hektet fast i velcro, da denne er relativt løs i vevingen. Dette vil trekke ut tråder og vevingen ødelegges.

Webbing brukes for det meste i løftestropper og seletøy. Den vanligste spesifikasjon for webbing er MIL-SPEC W-4088. Etter 1996 har Parachute Industry Association (PIA) overtatt vedlikeholdet av standarden for webbing og bånd. Disse vil etterhvert bli omtalt med nye spec identifikasjon, for eksempel PIA.W-4088, i stedet for Mil-W-4088.

De mest brukte webbing og bånd er listet opp i nedenstående tabell: Ved skjæring svies alle ender så vevingen ikke flises opp (Hot knife).

\begin{table}
	\caption{Bruksområder og spesifikasjoner bånd og webbing}
	\begin{tabular}{ | l | l | l | l | l | l | l | }
		\hline
		Gruppe & Type & MIL Spec PIA Spec & Bredde (± 1/16'') & Bruddstyrke & Merking & Anvendelse \\
		\hline
		Tubular & & W-5625 & 1/2 & 1000 lbs & Én centerline & Pilotbånd \\
		\hline
		Tubular & & W-5625 & 9/16 & 1500 lbs & Én i center, og en i hver ende, eller tre i midten & Pilotbånd \\
		\hline
		Tubular & & W-5625 & 3/4 & 2300 lbs & Én centerline & Pilotbånd – Innerline drogue \\
		\hline
		Tubular & & W-5625 & 1 & 4000 lbs & Én centerline & Staticline (utløserline) \\
		\hline
		Bånd & III & T-5038 & 1/2 & 250 lbs & Ingen & Forsterkning \\
		\hline
		Bånd & III & T-5038 & 3⁄4 & 400 lbs & Ingen & Forsterkning – Kantebånd \\
		\hline
		Bånd & III & T-5038 & 1 & 525 lbs & Ingen & Forsterkning \\
		\hline
		Bånd & IV & T-5038 & 1 & 1000 lbs & Ingen Firkantvevd & Pilotbånd Forsterkning \\
		\hline
		Webbing & VII & W-4088 & 1 23/32 & 6000 lbs & Gul i kantene & Seletøy, løftestropper \\
		\hline
		Webbing & VIII & W-4088 & 1 23/32 & 4000 lbs & Svart i midten & Løftestropper, forsterkerbånd \\
		\hline
		Webbing & XII & W-4088 & 1 23/32 & 1200 lbs & Rød i kantene & Forsterkerbånd \\
		\hline
		Webbing & XIII & W-4088 & 1 23/32 & 7000 lbs & Svart i kantene & Seletøy \\
		\hline
		Webbing & XVII & W-4088 & 1 & 2500 lbs & Ingen & ``Mini''- løftestropper
		Brystropper \\		
		\hline
	\end{tabular}
\end{table}

\subsection{Skader på bånd og webbing}
I 1993 ble det i Relative Workshops regi utført en del tester på forskjellige typer webbing for å som om det var mulig å lage retningslinjer for hvilke effekter forskjellige typer skader utgjorde, og hvor mye dette gjorde webbingen svakere. Webbingen på et seletøy er relativt utsatt, og det kommer ofte i kontakt med eksterne ting som kan forårsake skader. Ofte kan det være vanskelig å se disse skadene, og å fastslå i hvilken omfang skadene har betydning for operativiteten til utstyret.

Testen benyttet ``frisk'' webbing fra samme rull som sammenligningsgrunnlag. I det følgende gis det en kort oppsummering av resultatene av denne testen. Resultatene er et gjennomsnitt, og angir hvilken behandling som ble gjort med webbingen, og hvilken endring dette medførte i styrke. For det meste ble det brukt Type VII – 6000 lbs, og Type VIII – 4000 lbs webbing, som er mest benyttet på seletøy i dag. Kolonnen for resultat viser hvilken forskjell i bruddstyrke skadene hadde sammenlignet med uskadet materiale.

\begin{table}
	\caption{Skader på webbing}
	\begin{tabular}{ | l | l | l | }
		\hline
		Type & Behandling & Resultat \\
		\hline
		Vannbehandling & Webbing ble skyllet i ferskvann i vaskemaskin i to timer. & Lite endring – 3\% svekkelse. \\
		\hline
		Søm med skadet nål (butt). & Tre sikksakk sømmer sydd på tvers av 1'' webbing (som pilotbånd) med skadet nål & 28 \% svekkelse i sømmer i forhold til samme sømmønster med uskadet nål. \\
		\hline
		Kutt i webbing & Cirka 2,5 cm kutt med varmekniv, som skar over øverste lag med tråder & 22 \% svekkelse i forhold til uskadet materiale \\
		\hline
		Kutt i webbing & Samme lengde kutt som ovenfor, men dypere, mellom en tredel og en halvpart av webbingens tykkelse & 55 \% svekkelse i forhold til uskadet materiale \\
		\hline
		Kutt i kanter på \emph{seletøy} webbing & Cirka 11⁄2 mm kutt i kanten av webbingen & 15 \% svekkelse i forhold til uskadet materiale \\
		\hline
		Kutt i kanter på \emph{løftestropp} webbing & Cirka 11⁄2 mm kutt i kanten av webbingen & 50 \% svekkelse i forhold til uskadet materiale. \\
		\hline
		Skarpt knivkutt & Sammenligning ved skader som kan skje ved oppspretting av sømmønstre – små, grunne kutt på ytre lag av tråder & 40 \% svekkelse i forhold til uskadet materiale. \\
		\hline
		Skarpt knivkutt -- litt dypere & Litt dypere kutt som ovenfor – begge disse skader var nær usynlig å oppdage om ikke webbingen ble brettet ut med skaden & 50 \% svekkelse i forhold til uskadet materiale. \\
		\hline
		Knivkutt ved reparasjon & Små kutt som ved oppspretting av stingmønster – deretter ble ny webbing sydd over (simulering av bytte av bryststropp) & 27 \% svekkelse i forhold til uskadet materiale. \\
		\hline
		Landingsskade & Skader på tvers av webbing simulert som skrubblanding på asfalt – ofte i nærheten av benstropp spenne. & 36 \% svekkelse i forhold til uskadet materiale. \\
		\hline
		Skade på seletøy søm & Tilsvarende landingsskade, men direkte slitasje på seletøytråden, der flere sting røk & 60 \% svekkelse i forhold til uskadet materiale. \\
		\hline
		Velcro & Gjentagne slitasje med velcro for å dra opp ``nupper'', simulerer velcroslitasje på løftestropper. Etter først skadene kom ble de hurtig merkbart større. & 30\% svekkelse i forhold til uskadet materiale. \\
		\hline
		Rust & Rust fra våte metalldeler hadde fått lov til å trekke inn i seletøywebbing – cirka halve tykkelsen & 30\% svekkelse i forhold til uskadet materiale. \\
		\hline
		Sollys &
		Seletøy webbing ble plassert utendørs i ett år for å se på svekkelse fra ultrafiolett stråling.
		Resultatene viser i tillegg til alminnelig svekkelse at jo tynnere webbingen var i utgangspunktet, jo hurtigere ble den svekket, da denne har mindre beskyttelse av ytre fibre.
		Videre viste farver og skarphet i disse seg som indikatorer på slitte deler. &
		Dager Svekkelse
		40 8\%
		80 16 \%
		120 35 \%
		160 46 \%
		200 49 \%
		240 53 \%
		280 58 \%
		320 56 \%
		360 57 \% \\
		\hline
	\end{tabular}
\end{table}

Resultatet av denne testen viser blant annet at det er vanskelig og vil alltid være en subjektiv vurdering når utstyr skal tas ut av drift og erstattes eller repareres. Det finnes ingen fasitsvar, og vurderingen er mer sammensatt enn denne tabellen gir svar på.

Det testen derimot belyser er enkelte av de faktorene en Materiellkontrollør må vurdere ved kontroll av seletøyet, og forhåpentligvis hjelper med å ta en veloverveid avgjørelse om utstyrets tilstand.

\section{Metallkomponenter}
(Poynter Vol. 1, pkt. 4.100 side 96-139, Poynter Vol. 2, 4.103 / 4.104 s. 93).

De alminnelige feste og innstrammingsdeler for seletøy er delt inn i Kroker (Snaps), Ringer (Rings), og Spenner (Adapters). Kroker og ringer er etterhvert erstattet med spenner på alminnelig sportsseletøy (``step-in''), men er allikevel alminnelig på tandemutstyr og redningsskjermer.

For å hindre rustdannelse på metallkomponentene påføres de et beskyttende lag av kadmium. Denne prosessen kalles ``plating''. Brukte komponenter som kommer inn til et større fallskjermfirma brukes på nytt etter rensing og ``replating''.

Denne ``replating'' prosessen starter med rensing i syrebad, og påføring av nytt rustbeskyttende lag. Her trenger også hydrogen-ioner inn i porene på metallet, og det blir sprøtt og ubrukelig dersom det ikke bakes ut etter bestemte prosedyrer. Derfor skal metallkomponenter kun kjøpes gjennom anerkjente forhandlere.

Forkromming er også egnet som rustbeskyttende lag. Imidlertid er dette en prosess som kan utføres av en mengde småfirma og privatpersoner uten den nødvendige tekniske kunnskap, og uten garanti for at spesifikasjonene for slikt arbeid er fulgt. Derfor skal du være meget skeptisk på slike komponenter, og ALDRI GODKJENNE DEM. Det samme gjelder sveisede komponenter. Disse må aldri monteres på fallskjermutstyr.

\subsection{Kroker (Snaps)}
(Poynter Vol. 2, pkt. 4.110 side 96)

Det er viktig at de forskjellige kroker og ringer er tilpasset hverandre. Dersom ringene er for store for krokene, vil ikke disse feste ordentlig, og dersom de er for små kan de vris ut av kroken, slik at de løsner utilsiktet.

To typer er mest alminnelig:
\begin{description}
	\item[MS 22044 2500 lbs (B12 snap).] Anvendes også som feste for passasjer på tandemseletøy.
	\item[MS 22017 Kvikkrok 2500 lbs (Quick Ejector Snaps)] Anvendes også som hofte-innfesting av passasjerseletøy til tandemseletøy.
\end{description}

\subsubsection{Krok utløserline}
I Norge benyttes i alminnelighet én type:
\begin{description}
	\item[MS 70120 1750 lbs] krever låsepinne for godkjent bruk.
\end{description}

\subsection{Ringer (Rings)}
(Poynter Vol. 2, pkt 4.111 s. 99)

Her er beskrevet ringer som brukes for innfesting av seletøy. (Ringer til tre-rings systemer har eget avsnitt, 4. Drag Chute Sjakkel tilsvarende speedlinks.

Ringer for tre-rings løftestropper- 9.0).

De vanligste typer er:
\begin{description}
	\item[V-ring 2500 lbs Standard kategori.] Brukes sammen med kvikkroker for festing av benstropper.

	Finnes i en rekke varianter med og uten regulerbar spenne.
	\item[MS 22020 Triangular ring 2500 lbs.] Egner seg også for bruk med kvikkroker.
	\item[D-Ring] For feste av tandemseletøy til hovedseletøy. Forskjellige typer fra forskjellige produsenter. (Brukes ikke i forbindelse med benstropper).
\end{description}

\subsection{Spenner (Adapters)}
(Poynter Vol. 2, 4.112, s 102)

Spenner finnes både som justerbare (Friction Adapters), og faste.

De justerbare brukes i forbindelse med innfesting og tilstramming, og har en glidende bolt i midten av spenna.

De faste brukes der det ikke er nødvendig med justering for hver gangs bruk. Disse brukes for eksempel for faste justeringer av seletøy, og er vanlig på justerbart seletøy av elev og tandemtype.

Det mest vanlig å benyttejusterbare spenner (Quick Fit Adapter) for innstramming og festing av benstropper, selv om kroker og ringer også er et alternativ.

Vanlige typer er:
\begin{itemize}
	\item MS 22040 – 2500 lbs
	\item MS 22019 – 2500 lbs
	\item MS 70114 – 2500 lbs
\end{itemize}
eller tilsvarende.

\begin{figure}
	%\includegraphics[width=60mm]{Strekktesting av kalottduk.pdf}
	\caption{MS 22040}
\end{figure}

På bryststropp brukes tilsvarende en ``Lightweight'' Quick Fit Adapter,
\begin{itemize}
	\item MS 70101 – 500 lbs.
\end{itemize}

Denne finnes også i en 1'' (2,5 cm) utgave for bryststropper av Type 17 webbing.

\subsubsection{Kontroll av MS 22040}
I juni 1993 ble det sent ut en modifikasjonsordre fra RW Shop, der en del av spennene av type MS 22040 som ble montert mellom 22.4.93 og 10.6.93 var feilprodusert, ved at bolten i midten var snudd ved montering. Dette er en alminnelig kontroll ved hovedkontroll av alle typer spenner. Se forøvrig F/NLF modifikasjonsordre 9307 som beskriver dette, samt kontroll av benstropper.

\begin{figure}
	%\includegraphics[width=60mm]{Strekktesting av kalottduk.pdf}
	\caption{Feilmontering av spenne MS 22040}
\end{figure}

Andre spennetyper uten stramming er i bruk på delbare seletøy, både ved tandem og elevutstyr. Disse er vanligvis uten innstramming.

Det må vises aktsomhet ved inntreing og kontroll av webbingen i disse slik at webbingen låser seg selv og ikke kan skli i spenna. Enden på webbingen skal også ha påsydd en stoppfeste, slik at den ikke kan skli ut mellom bolten og den faste delen. Poynter Vol. 2, 4.105, s 94.

\subsection{Links / Sjakler (Connector Links)}
Sjakler og links forbinder linene fra fallskjermen til løftestroppene. På hoved- og reserveskjermer skal disse være av godkjente typer.

Løftestroppene kan trenges modifisering for at ulike typer av links skal kunne anvendes (Poynter Vol. 2, Kap. 4.113, s. 109). Dette gjelder spesielt runde reserveskjermer.

\subsubsection{Rapid Connector Links, (Maillon Rapide)}
(Poynter Vol. 2, Kap. 4.113, s .108)

Godkjente Rapid connector links er merket med produsentens navn – for eksempel slik:

Maillon Rapide no 4 \\
Made in France \\
SWL 280 kg \\
INOX

Spesifikasjoner på godkjente Maillon Rapide Links:
\begin{table}
	\caption{Spesifikasjoner på Maillon Rapide Links}
	\begin{tabular}{ | l | l | l | l | l | l | l | }
		\hline
		Størrelse & Dimensjon & Type & Begrensning & SWL\footnotemark – kg. & BL\footnotemark – kg. & Mutter mm \\
		\hline
		3,5 INOX\footnotemark & 3,5 mm gods & Rustfritt stål & Kun Hovedskjerm & 220 & 1 100 & 7 \\
		\hline
		4 INOX & 4 mm gods & Rustfritt stål & & 280 & 1 400 & 8 \\
		\hline
		5 & 5 mm gods & Standard (Karbon stål) & & 280 & 1 400 & 9 \\
		\hline
		6 & 6 mm gods & Standard (Karbon stål) & & 400 & 2 000 & 11 \\
		\hline
	\end{tabular}
\end{table}
\footnotetext{SWL = Safe Working Load}
\footnotetext{BL = Breaking Load}
\footnotetext{INOX = Rustfri type}

Det er meget viktig at låsemutteren trekkes riktig til. Først trekkes de godt til med fingrene, deretter ca 1/4 omdreining med en liten skiftenøkkel. Det skal kjennes et naturlig stoppepunkt for låsemutteren. Mutteren skal omkranse alle gjenger. Trekkes det for hardt til kan stoppkransen ryke, og linken er svekket.

Mutteren sitter ikke på midten, og den lengste, frie, enden monteres mot webbingen på løftestroppen slik at gnisninger mot denne unngås.

Maillon rapide bør videre monteres slik at mutteren er plassert innover, slik at skader på slidermaljer unngås.

Kontrollér videre at det ikke finnes skarpe metallbiter som stikker ut ved gjenger eller låsemutter.

\textbf{NB: I åpen tilstand har Maillon Rapid links nær ingen styrke.}

Det anbefales å bruke sliderstoppere (``bumpers'') av gummi, webbing eller lignende, for å unngå skader på slidermaljer, som vil slite unødig på linene. Disse skal festes betryggende så de ikke kan skli oppover linene (med tråd eller elektrikerstrips). Poynter Vol. 2, 4.155 s. 120.

\subsubsection{Andre sjakler}
\begin{description}
	\item[MS 22002 L-Sjakkel] er godkjent for reserveskjerm montert på 2 eller 4 løftestropper. (Poynter Vol. 2, 4.113, s. 105).
\end{description}

``Knoken'', (den korte delen av ``L'' stammen der skruen går igjennom), skal peke opp og frem mot kalotten for å hindre liner i å henge seg opp.

\begin{figure}
	%\includegraphics[width=60mm]{Strekktesting av kalottduk.pdf}
	\caption{L-sjakkel}
\end{figure}

Skruene må trekkes til \textbf{minimum 6 fulle omdreininger} for å få full bruddstyrke i alle retninger. Disse skal etterkontrolleres ved hver hovedkontroll.

Det må ikke monteres to halvdeler fra forskjellige produsenter.

\textbf{Separasjon av L–Sjakler:}

Det bør ikke bankes på låseskruene for å separere de enkelte sjakkeldelene, da dette kan skade gjengene. Bruk en sjakkelspreder og hammer med nylonhode som for eksempel følgende skisse:

\begin{figure}
	%\includegraphics[width=60mm]{Strekktesting av kalottduk.pdf}
	\caption{Sjakkelspreder}
\end{figure}

\subsubsection{Ikke tillatte links}
\textbf{Følgende links er ikke tillatt:}
\begin{description}
	\item[Rapid-Link fra Taiwan – Rapid connector] links merket: ``Made in Taiwan'' har ofte dårlig styrke og er upålitelige, og er ikke godkjent for bruk på fallskjermutstyr. (Tidligere ofte levert i str. 6 påmontert ParaFoil skjermer).
	\item[Speedlink] med låseplate er ikke tillatt brukt Poynter Vol. 2, s 105.
	\begin{figure}
		%\includegraphics[width=60mm]{Strekktesting av kalottduk.pdf}
		\caption{Speedlink}
	\end{figure}

	\item[Drag Chute Sjakkel] tilsvarende speedlinks.
\end{description}

\subsection{Ringer for tre-rings løftestropper}
Poynter Vol. 2, 4.114, s 110 .

Det finnes flere typer tre-rings sett produsert gjennom tidene. Den opprinnelig tre- rings besto av tre runde ringer. Etter 1981 ble den største ringen produsert med en spor (modell RW-1), for å få bedre utnyttelse av festet på seletøyet. Nå gikk hele seletøyet igjennom sporet, isteden for at ringen var sydd løst på yttersiden.

Firmaet Forgecraft i USA produserte et stort antall ringer som ikke var sluttbehandlet, og var merkbart svakere. Disse var merket RW-1-82 og RW-1-83. Disse ble tilbakekalt, og samtlige utstyr i felten med disse merkingene måtte gjennomgå strekktesting av ringer for å verifisere styrken eller at ringene måtte byttes. Dette skulle normalt være gjort på alt utstyr i dag, men en kan ikke være sikker dersom det ikke er ført inn i hovedkontrollkortet til utstyret. Finnes ikke dokumentasjon må ringene strekktestes, eventuelt byttes. Se forøvrig Kapittel 3.5 - Seletøy - Løftestropper.

Etterhvert ble det produsert mindre ringer. I en overgangsperiode fantes det en del svakheter med disse, da enkelte brukte den opprinnelige midterste ringen som utgangspunkt, noe denne ikke var dimensjonert for. I dag er mini rings vanlig merket

RW-7 på amerikansk utstyr. Disse har mindre toleranser for innbyrdes plassering på løftestroppene.

Parachutes de France har produsert ringer etter egne spesifikasjoner. Disse ringene har bedre plass innbyrdes. Her kan ringenes innbyrdes plassering på løftestroppene være annerledes enn på amerikanske.

Normalt skal ikke franske løftestropper benyttes på amerikansk utstyr og vice versa. Sjekk i alle fall montering og brukskontroll ved at ringene har metall til metall kontakt, at låseløkka er lang nok og myk nok, og at de frigjør lett.

Se videre kapittel 3.3 Frigjøringssystemer for beskrivelse av kontroll av trerings løftestropper.
