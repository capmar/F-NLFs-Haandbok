\part{Instruktørlisenser}
\setcounter{section}{399}

\section{Definisjoner}
\subsection{Instruksjon}
Med instruksjon menes her grunnopplæring og videregående opplæring i fallskjermhopping iht F/NLFs bestemmelser Del 600, samt hoppmestring av elever.

\subsection{Hoppmestring}
Med hoppmestring menes her kontroll, overoppsyn og kommando over en eller flere andre hoppere fra disse er oppført på manifest, til de har forlatt flyet. Hoppmestring medfører ansvaret for at hopperne blir korrekt inspisert, orientert, instruert, spottet og utsatt, og korrekt vurdert etter utført hopp.

\section{Generelt}
\subsection{Kompetansekrav}
Ingen kan utføre instruksjon i fallskjermhopping innen Fallskjermseksjonen/NLF uten å inneha gyldig instruktørlisens.

\subsection{Kandidater}
Sikkerhets- og Utdanningskomiteen F/NLF vurderer og tar ut instruktørkandidater til klasse 2 eller høyere, etter søknad og anbefaling fra lokalklubbens Hovedinstruktør. Personer som ønsker å bli antatt som kandidater leverer søknad gjennom lokalklubbs Hovedinstruktør innen de tidsfrister som SU fastsetter. Instruktørkandidater av klasse 3 vurderes og uttas av lokalklubbens Hovedinstruktør.

\subsection{Utstedelse}
F/NLFs instruktørlisenser utstedes av Seksjonsstyret F/NLF til medlemmer av F/NLF og lokalklubbene, etter søknad attestert og anbefalt av lokalklubbens Hovedinstruktør, som bekrefter at han har kontrollert at søkeren har gjennomført alle de ferdighets- og teoriprøver som kreves, og at han etter vurdering av søkeren anser at denne innehar den vurderingsevne og de personlige egenskaper og holdninger som gjør ham skikket for instruktørvirksomhet. Bestått eksamen fra B- eller C-kurs attesteres av Instruktør/ Eksaminator.

Alle lisenser er nummerert og Seksjonsstyret F/NLF fører register over alle utstedelser.

\subsection{Gyldighet}
Instruktørlisenser utstedes for livstid, men er kun gyldig for praktisk instruksjon og utøvelse av de øvrige rettighetene jfr pkt'ene 402.2.3 og 402.3.3, sammen med gyldig sertifikat av nødvendig klasse, jfr Del 300.

\subsection{Fornyelse}
\subsubsection{Instruktør}
For at en instruktørlisens skal fornyes, kreves utover ferdighets- og teorikrav at søkeren har gyldig fallskjermsertifikat av den klasse som lisensen krever. Ved opphold i instruksjon som følge av opphold i hoppingen, skal Hovedinstruktør kontrollere at ferdighets- og teorikravene er oppfylt før lisensen fornyes.

Dersom søkeren innehar sertifikat av nødvendig grad, men pga. medisinske eller andre årsaker ikke kan få dette fornyet, kan han likevel få fornyet sin instruktørlisens etter nærmere vurdering fra Hovedinstruktør.

Instruktør kan ikke fornye egen lisens.

For AFF- og Tandem-Instruktør gjelder spesielle praktiske vedlikeholdskrav for at denne skal beholde sin operative status.

\subsubsection{Instruktør/eksaminator}
For Instruktør/Eksaminator gjelder ingen fornyelseskrav. SU vil benytte vedkommendes kompetanse ved behov.

\subsection{Inndragning/disiplinærtiltak}
Dersom det finnes nødvendig av hensyn til sikkerheten eller fallskjermidrettens anseelse, kan Sikkerhets- og utdanningskomiteen F/NLF inndra lisenser for kortere eller lengre tid eller for bestandig.

\subsection{Utenlandske lisenser}
Instruktør med gyldig utenlandsk instruktørlisens kan etter søknad fra Hovedinstruktør til SU gis tidsbegrensede rettigheter ut fra følgende:
\begin{itemize}
	\item Static-line Instructor kan gis rettighet som Instruktør 3
	\item Tandem Instructor kan gis rettighet som Instruktør Tandem
	\item AFF Instructor kan gis rettighet som Instruktør 3/AFF
\end{itemize}

\subsubsection{Kriterier for vurdering av utenlandske instruktører}
For at utenlandske instruktører kan gis rettigheter som instruktør i F/NLFs tilsluttede klubber må søker skriftlig dokumentere at:
\begin{itemize}
	\item gjennomgått utdannelse ikke er mindre omfattende enn tilsvarende F/NLF- eller USPA-utdanning
	\item erfaringskrav for deltagelse i instruktørutdannelse ikke er lavere enn de laveste av F/NLFs og USPAs krav
	\item vedlikeholdskrav ikke er lettere enn de letteste av F/NLFs og USPAs krav
\end{itemize}

SU skal vurdere utenlandsk instruktørs erfaring, ferdigheter og holdninger og kan gi avslag på søknad selv om formelle kriterier er oppfylt.

\subsection{Revisjon}
Del 400 revideres av SU, som også utgir supplerende bestemmelser.

\section{Klasser og betegnelser}
Følgende klasser og betegnelser benyttes for personell som utfører instruksjon i Fallskjermseksjonen/NLF:
\begin{itemize}
	\item Instruktør 3 (I-3)
	\item Instruktør 3/AFF (I-3/AFF)
	\item Instruktør 2 (I-2)
	\item Instruktør 2/AFF (I-2/AFF)
	\item Instruktør 1 (I-1)
	\item Instruktør Tandem
	\item Instruktør/Eksaminator (I/E)
\end{itemize}

\subsection{Instruktør 3 (I-3)}
Instruktør 3 er en fallskjermhopper som innehar Instruktørlisens 3, og som er godkjent av Hovedinstruktør til å hoppmestre elever på line, fritt fall og AFF-elever fra nivå 8, samt til å attestere alle hopplogger, dog ikke for spesielle progresjonshopp som krever godkjenning av instruktør av høyere klasse. Instruktør 3 kan hoppmestre elever som utfører linehopp, etter at utsjekk for hoppmestring av lineelever, punkt 402.1.5 er utført.

\subsubsection{Erfaringskrav}
Inneha norsk C-sertifikat. Ha gjort tjeneste som HFL ved hopping med elever minst 5 ganger. Ha vært hjelpeinstruktør på minst 3 grunnkurs del 1 og minst 2 grunnkurs del 2.

\subsubsection{Krav til teorikunnskaper}
Ha gjennomført F/NLFs Hoppmesterkurs og avlagt skriftlig eksamen med tilfredsstillende resultat.

\subsubsection{Ferdighetskrav}
Ha avlagt praktisk prøve (Hoppmesterutsjekk) med godkjent resultat.

Dette omfatter 4-6 løft med komplett hoppmestring av elever, herunder vurdering, instruksjon og inspeksjon av elev, briefing av flyger, spotting, utsetting og etterfølgende debrief av elev. Prøven avlegges overfor egen lokalklubbs Hovedinstruktør, eller den Instruktør 1 denne bemyndiger. Kandidaten vurderes etter vurderingsskjema utarbeidet særskilt for F/NLFs Hoppmesterkurs.

\subsubsection{Rettigheter og oppgaver}
Hoppmestre hoppere på alle trinn unntatt AFF nivå 1-7, tandemhopp, natthopp, oksygenhopp og oppvisningshopp. Kan hoppmestre elever som utfører linehopp, etter utsjekk for hoppmestring av lineelever er utført, se pkt 402.1.5.

Attestere alle hopplogger, dog ikke for progresjonshopp som krever attestasjon fra Instruktør av høyere klasse.

\subsubsection{Utsjekk for hoppmestring av lineelever}
For å hoppmestre elever som utfører linehopp skal det avlegges en praktisk prøve med godkjent resultat. Dette omfatter 4-6 løft med minst en lineelev på hvert løft. Prøven avlegges overfor egen lokalklubbs Hovedinstruktør, eller den Instruktør 1 denne bemyndiger. Utsjekken kan utføres samtidig med praktisk prøve til Instruktør 3, punkt 402.1.3. Godkjent utsjekk attesteres i hopperens loggbok.

\subsection{Instruktør 2 (I-2)}
Instruktør 2 er en fallskjermhopper som innehar Instruktørlisens 2, og som er godkjent av Seksjonsstyret F/NLF til å instruere elever og viderekomne hoppere på grunnkurs og progresjon under ledelse av en Instruktør 1, etter program og retningslinjer gitt av SU, ref Del 600.

Instruktør 2 har ikke lov til å instruere elever på AFF-progresjon nivå 1-7 uten samtidig å inneha status som AFF-instruktør.

\subsubsection{Erfaringskrav}
Ha deltatt i til sammen 2 års praktisk hoppvirksomhet, herav minst 12 mndr som Instruktør 3, og derigjennom tilegnet seg allsidig erfaring i hoppmestring av elever, samt erfaring med hoppfeltorganisasjon (HFL- tjeneste). Ha vært hjelpeinstruktør på minst 2 komplette grunnkurs (både del 1 og del 2), etter utstedt Instruktørlisens 3.

\subsubsection{Krav til teorikunnskaper}
Ha avlagt teoriprøver vedrørende vann-, natt-, FS- og CF-hopp, samt skjermteori.

Gjennomført F/NLFs B-kurs med tilfredsstillende resultat.

\subsubsection{Rettigheter og oppgaver}
Etter fastsatt program, ref. Del 600, og retningslinjer gitt av Instruktør 1 instruere og hoppmestre hoppere på alle trinn unntatt AFF nivå 1-7, tandemhopp, vannhopp, natthopp, oksygenhopp og oppvisningshopp. Kan hoppmestre elever som utfører linehopp, etter at utsjekk for hoppmestring av lineelever er utført iht pkt 402.1.5. Kontrollere, attestere og godkjenne fornyelse av alle typer sertifikater. Rekognosere og anbefale godkjenning av hoppfelt for alle typer hopping. Være HL for alle typer hopping unntatt natthopp, vannhopp og oksygenhopp. Være HFL for alle typer hopping.

\subsection{Instruktør 1 (I-1)}
Instruktør 1 (I-1) er en fallskjermhopper som innehar Instruktørlisens 1 og som er godkjent av Seksjonsstyret F/NLF til selvstendig å legge opp, instruere og kontrollere alle former for opplæring og videregående trening av hoppere etter program og retningslinjer gitt av SU, ref Del 600.

\subsubsection{Erfaringskrav}
Ha deltatt i minst 3 års praktisk hoppvirksomhet, herav minst 12 mndr som I-2 og derigjennom ha tilegnet seg allsidig erfaring i grunnleggende og videregående utdannelse og kontroll av elever på alle trinn, samt erfaring som Hoppleder. Som Instruktør 2 ha vært instruktør ved minst 3 komplette grunnkurs frem til elevenes FF-status, dvs både grunnkursets del 1 og del 2. Inneha norsk D-sertifikat og Demosertifikat.

\subsubsection{Krav til teorikunnskaper}
Ha avlagt alle prøver til norsk D-sertifikat. Gjennomført teorikurs i flymedisin med lavtrykkskammerprøve ved Flymedisinsk Institutt. Gjennomført F/NLFs C-kurs med tilfredsstillende resultat.

\subsubsection{Rettigheter og oppgaver}
Etter fastlagte programmer iht Del 600 og retningslinjer gitt av lokal Hovedinstruktør, selvstendig legge opp, lede og gjennomføre fallskjermkurs og videregående utdannelse av hoppere frem til og med D-sertifikat. Instruere i vannhopp, natthopp og oksygenhopp, samt all annen alminnelig hopping. Instruere i CF dersom dette er gjennomført selv. Undervise, kontrollere og autorisere fallskjermpakkere på alle typer sportsskjermer som han selv er sertifisert for. Være eksaminator og sensor ved prøver av alle typer og grader, unntatt F/NLFs B-kurs, C-kurs, MK-kurs, AFF-kurs og Tandemkurs. Være HL for alle typer hopping. Rekognosere og anbefale godkjenning av hoppfelt for alle typer hopping. Som HL godkjenne hoppfelt etter gjeldende bestemmelser.

\subsection{Instruktør tandem}
Instruktør Tandem er en fallskjermhopper som innehar Instruktørlisens Tandem, og som er godkjent av Seksjonsstyret F/NLF til å instruere og hoppmestre tandemelever under ledelse av Hovedinstruktør, etter program og retningslinjer gitt av SU, ref del 600.

\subsubsection{Erfaringskrav}
Inneha instruktørlisens 2 eller høyere. Ha vært aktiv hoppmester under hopping med elever.

Minimum 100 hopp siste år før utsjekk.

Under kontroll og oppfølging fra egen Hovedinstruktør ha gjennomført 20 utviklingshopp i løpet av 12 måneder, derav minst 4 hopp i løpet av de første 3 månedene etter gjennomført Tandemhoppmesterkurs.

\subsubsection{Ferdighetskrav}
D-sertifikat og særlig vurderte personlige egenskaper og dyktighet. Anbefaling fra Hovedinstruktør i lokalklubb. SU kan avslå søknader om Tandemutdannelse uten nærmere begrunnelse.

\subsubsection{Krav til teorikunnskaper}
Ha gjennomført F/NLFs Tandeminstruktørkurs, ref del 600.

\subsubsection{Medisinske krav}
Tandeminstruktør-kandidat skal ha gjennomgått standard legeundersøkelse i løpet av de siste 3 måneder før instruktørkurset. Likeens skal legeattest vedlegges ved fornyelse av tandeminstruktør- lisensen fom fylte 50 år.

Gyldig Legeattest klasse 1 og 2, iht. Luftfartstilsynet krav, vil kunne erstatte F/NLFs legekontrollskjema.

\subsubsection{Rettigheter og oppgaver}
I regi av klubb som har OT-1, og derigjennom er gitt generell tillatelse til å gjennomføre slik hopping, og etter retningslinjer gitt av denne klubbens styre og Hovedinstruktør, selvstendig gjennomføre opplæring av, og hopping med tandemelev, ref pkt 617.

\subsubsection{Gyldighet}
Gyldig D-sertifikat.

Hvis det er et opphold etter sist utførte tandemhopp på mer enn 90 dager, skal Tandeminstruktøren utføre minst ett hopp med en erfaren hopper før elever uten hopperfaring kan tas med.

Hvis oppholdet er på mer enn 180 dager, skal det utføres minst ett solohopp og ett hopp med erfaren hopper. Begge utsprangene skal utføres fra minst 10500 fot med stabilt avsprang og drouge ute innen 10 sekunder. Laveste trekkhøyde for hovedskjerm er 5 000 fot. Hoppene skal gjennomføres med stabil heading, men skal i tillegg inneholde minst 2 stk 360 graders svinger (OH, OV). Landing innenfor 50 meter fra planlagt landingspunkt i begge tilfeller.

\subsubsection{Spesielle vedlikeholdskrav}
Fornyelse av Instruktør Tandem lisens krever 50 tandemhopp siste kalenderår, alternativt minimum 100 hopp siste kalenderår, hvorav minst 25 tandemutsprang.

Tandeminstruktør som ikke oppfyller dette kravet skal dersom han ønsker fortsatt fornyelse rette søknad om dette til SU. Som hovedregel vil denne måtte gjøre oppfriskningshopp, samt minst ett hopp med tandemeksaminator, før Tandem Instruktør lisens kan fornyes.

Tandeminstruktør som ikke har hoppet tandem i løpet av siste 730 dager må følge nytt tandem instruktørkurs før fornyelse av lisens.

\subsubsection{Hopping med funksjonshemmede}
Krav til instruktøren som for tandem demosertifikat, med følgende tilleggskrav:
\begin{itemize}
	\item 50 tandemhopp siste 6 mnd
	\item Søknad til SU i hvert enkelt tilfelle
\end{itemize}

\subsection{Instruktør 3/AFF (I-3/AFF)}
Fom 2006 er det krav til I-2 for å gå på F/NLF sitt AFF instruktørkurs. I fremtiden er det kun hoppere som har fått godkjent utdannelse i utlandet som vil kunne bli ny Instruktør 3/AFF. Instruktør 3/AFF er en fallskjerm- hopper som innehar Instruktørlisens 3/AFF, og som er godkjent av SU til å hoppmestre line elever, fritt fall elever og AFF elever på alle progresjonstrinn, samt til å attestere alle hopplogger, dog ikke for spesielle progresjonshopp som krever godkjenning av instruktør av høyere klasse.

\subsubsection{Erfaringskrav}
Inneha norsk D-sertifikat og samt være I-3.

\subsubsection{Krav til teorikunnskaper og ferdigheter}
Ha gjennomført AFF instruktørkurs arrangert av USPA og avlagt USPA’s skriftlige eksamen med tilfredsstillende resultat.

Ha gjennomført AFF instruktørkurs arrangert USPA med godkjent resultat, samt minimum ha 6 timer FF-tid akkumulert.

\subsubsection{Krav for konvertering}
\begin{itemize}
	\item Inneha norsk D lisens og være norsk I-3.
	\item Overfor AFF eksaminator utpekt av SU
	\begin{itemize}
		\item Bestå teoriprøve, 80 \% riktig score, multiple choice
		\item Bestå evaluering hoppteknikk, ila maksimum seks hopp
		\item Bestå to hopp nivå 3 som hhv primær og sekundær instruktør - Bestå ett hopp nivå 5
		\item Bestå to hopp ferdighetstest
	\end{itemize}
\end{itemize}

Utgifter til utsjekk dekkes av kandidaten etter satser bestemt av F/NLF. Ved ikke bestått hoppteknisk evaluering kan kandidaten tidligst gå opp til ny hoppteknisk vurdering etter to måneder.

\subsubsection{Rettigheter og oppgaver}
Hoppmestre elever på alle nivå i AFF progresjonen.

Attestere alle hopplogger, dog ikke for progresjonshopp som krever attestasjon fra Instruktør av høyere klasse.

\subsubsection{Spesielle vedlikeholdskrav}
Fornyelse av Instruktørlisens 3/AFF krever 50 AFF hopp siste kalenderår, alternativt minimum 100 hopp siste kalenderår, hvorav minst 25 AFF hopp.

Instruktør som ikke fyller disse kravene underkastes ny vurdering av SU, og må gjennomføre ny utsjekk med AFF-eksaminator godkjent av SU før status igjen kan gjøres gyldig.

\subsection{Instruktør 2/AFF}
Instruktør 2/AFF er en fallskjermhopper som innehar Instruktørlisens 2/AFF, og som er godkjent av Seksjonsstyret F/NLF til å instruere elever på grunnkurs del I, del II og AFF under ledelse av Hovedinstruktør og etter program og retningslinjer gitt av Seksjonsstyret F/NLF v/SU, ref Del 600.

\subsubsection{Erfaringskrav}
Ha deltatt i til sammen minimum 3 års praktisk hoppvirksomhet, og derigjennom tilegnet seg allsidig erfaring i hoppmestring og instruksjon, samt erfaring med hoppfeltorganisasjon (HFL-tjeneste). Ha vært hjelpeinstruktør på minst 2 komplette grunnkurs AFF etter utstedt Instruktør 2 lisens.

Instruktør 2 som består F/NLFs eller USPAs AFF instruktør kurs vil få utstedt Instruktørlisens 2/AFF.

\subsubsection{Ferdighetskrav}
Som for Instruktør 2, samt minimum 6 timer FF-tid akkumulert.

\subsubsection{Krav til teorikunnskaper}
Ha avlagt alle teoriprøver til norsk D-sertifikat. Ha avlagt teoriprøver vedrørende vann-, natt-, FS- og CF-hopp, samt skjermteori. Gjennomført F/NLFs B-kurs med tilfredsstillende resultat.

\subsubsection{Spesielle vedlikeholdskrav}
Fornyelse av Instruktørlisens 2/AFF krever 50 AFF hopp siste kalenderår, alternativt minimum 100 hopp siste kalenderår, hvorav minst 25 AFF hopp.

Instruktør som ikke fyller disse kravene underkastes ny vurdering av SU, og må gjennomføre ny utsjekk med AFF-eksaminator godkjent av SU før status igjen kan gjøres gyldig.

\subsection{Instruktør/eksaminator (I/E)}
Instruktør/Eksaminator (I/E) utnevnes av SU.

\subsubsection{Rettigheter og oppgaver}
I/E kan på oppdrag fra SU ha ansvar for, være eksaminator/foreleser og sensor på F/NLFs sentrale kurs så som B-kurs, C-kurs, MK-kurs, AFF- kurs og Tandemkurs.

\section{Dispensasjonsbestemmelser}
Ved spesielle behov, så som ved nystartede lokalklubber og ved kursvirksomhet rundt om i landet, kan det unntaksvis dispenseres fra enkelte av de bestemmelser som her er satt. Slike dispensasjoner kan kun gis av Sikkerhets- og utdanningskomiteen.

Alle dispensasjoner som gis av Sikkerhets- og utdanningskomiteen skal være tidsbegrenset, og søkeren gis et bestemt tidsrom der han forutsettes å tilegne seg de nødvendige ferdigheter, slik at han kan oppnå formell Instruktørlisens.
