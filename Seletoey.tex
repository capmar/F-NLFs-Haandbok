\part{Seletøy}
%\setcounter{section}{0}

\section{Generelt}
Dette kapitlet beskriver hvilke seletøy systemer som er godkjent for bruk i Norge, med eventuelle henvisninger til modifiseringer.

Det skal videre gi nærmere informasjon om enkeltkomponenter, tjene til å klargjøre hvilke regler som gjelder for bruk av disse.

Det er viktig å se seletøy som en helhet, der det er nødvendig med kjennskap til identifikasjon av enkeltdeler og kombinasjoner av disse. Feil bruk og sammensetninger kan virke mot sin hensikt, og i enkelte tilfelle være svært farlig.

\subsection{Merking av seletøy}
Godkjente seletøy skal ha påsydd merking fra produsent som minimum skal gi informasjon om produsentens serienummer og produksjonsdato.

I tillegg skal TSO godkjenning fremkomme av merkingen.

Dette blir ofte montert på hovedselen på en av reserveløftestroppene, og kan i tillegg være montert under reserveklaffen, eller på ryggstykket, der det er plass til hovedkontrollkortet.

\subsection{Typegodkjente seletøy}
Enkelte av seletøyene har tidligere vært godkjent, men vil utfases da disse enten ikke fortsatt er i produksjon, eller at nye utgaver ikke er godkjente. Disse seletøy er merket med *. Det henvises til Poynter Vol. 1 og Vol. 2, under kapitlene for de enkelte produsenter (kapittel 5) der det gis ytterligere opplysninger om nødvendige oppdateringer og modifikasjoner.

Tandem- og elevutstyr er spesielt merket (``T'' eller ``E''), for utfyllende informasjon om disse henvises det til egne kapitler i Materiellhåndboka, Kapittel 6 – Elevutstyr, og Kapittel 7 – Tandemutstyr.

Nedenfor er det angitt hvilke modifikasjoner som tidligere skal være ivaretatt der utstyret fremdeles kan være i bruk i dag. Stikkord til hvilken modifikasjon som gjelder med referanse til Poynter er ført opp under kolonnen Modifikasjon / Kommentarer.

\section{Påbudtemodifiseringer}
\subsection{Generelt}
I listen over typegodkjente seletøy er det henvisninger til Poynters manualer for tidligere eller spesielle modifiseringer. Dette er referansen for hvert enkelt seletøy til nødvendige, tidligere oppdateringer. Hovedkontrollerer du et eldre utstyr er det din plikt å kontrollere med Poynter og modifikasjonslisten bakerst i boka om det finnes eldre modifikasjonsordre på utstyret.

Over tid kan slike lett bli glemt, og dersom det ikke er anført i hovedkontrollkortet, eller modifiseringen er synlig er ikke ustyret luftdyktig før det er oppdatert.

Det understrekes viktigheten ev utførelse av egenkontroll. Man kan ikke uten videre være sikker på at feil og modifikasjoner er ivaretatt av tidligere kontroller.

I det følgende omhandles generelle modifiseringer som gjelder alle typer seletøy.

\subsection{Vector III}
Vector III ble designet med et nyutviklet konsept for tildekking av løftestropper, for å forhindre at disse løsner f.eks. under ``free-flying''. Feil lukking, ved at både hoved- og reserveløftestroppene ble plassert sammen under en lukkeklaff gjorde det trangt, og disse kunne føre til alvorlige feilfunksjoner, bl.a. at hovedskjermens løftestropper ikke frigjøres under skjermåpning eller ved gjennomføring av nødprosedyre. Konseptet er slik at det er svært fort gjort å gjøre feil dersom man ikke er gjort spesielt oppmerksom på denne muligheten.

Produsenten erkjente problemet, og leverte etter en tid nye Vector III med en presisering av forholdet trykket på en merkelapp som er til de aktuelle skulderklaffene, jfr skisse nedenfor.

I tillegg henvises det spesielt til disse punktene på forsiden av (Vector II-manual med tillegg for Vector III).

Dersom ikke en merkelapp med angivelse av løftestroppenes plassering er montert fra fabrikanten skal dette gjøres av MK før utstyret igjen er luftdyktig.

\begin{figure}
	%\includegraphics[width=60mm]{Strekktesting av kalottduk.pdf}
	\caption{Merking og lukking av Vector 3}
\end{figure}

En merkelapp utført i Tyvek-papir som angir korrekt plassering av hovedskjermens løftestropper, identisk med merkelapper produsert The Relative Workshop, skal syes fast lett synlig på løftestroppenes dekklaff ``A'',

Merkelapper kan fåes tilsendt henvendelse til F/NLFs sekretariat. Merkingen kan sys på med E-tråd, for hånd eller med maskin, av Materiellkontrollør.

Enkelte rigger kan ha et bånd sydd fast til klaff ``A'' fra hoveddelen på skulderklaffene, for å hindre løftestropper i å bli pakket under denne klaffen. Dette båndet skal fjernes, slik at reserveskjermens løftestropper kan plasseres under klaff ``A'', slik pakkemanualen tilsier. Bånd sydd fast til skulderklaff ``A'' kan fjernes av Materiellkontrollør ved forsiktig bruk av egnet sprettekniv for syarbeid.

For korrekt lukking av Vector III skal reserveløftestropppene legges under klaff ``A'', mens hovedløftestroppene skal plasseres over klaff ``A''.

\subsection{Trerings RW-1-82 eller RW 1-83}
Gjelder alle seletøy hvor den største ringen (den som er festet til seletøyet) er merket med:
\begin{itemize}
\item RW-1-82 eller RW-1-83
\end{itemize}

Grunnet en produksjons defekt i en del av de ringene med nevnte merker kan dette resultere i deformering under åpningssjokket og umuligjør cut away. Disse er ikke luftdyktig, og selv om det er lenge siden dukker disse ringene opp med jevne mellomrom. Det er også vanskelig med riktig dokumentasjon om disse ringene har vært testet, da utstyret kan ha hatt flere eiere i mellomtiden:

I utgangspunkt er disse ringene underkjent inntil ett av følgende er utført:
\begin{enumerate}
\item Ringene er byttet.
\item Begge ringene er testet som beskrevet nedenfor og dette er ført inn i riggens hovedkontrollkort samt i MK'ens logg av materiellkontrollør med: ``Strekktestet OK''. Dersom bare én av ringene ikke består strekktest skal begge byttes.
\end{enumerate}

Strekktest, gjennomført av materiell-kontrollør med tilgjengelig apparatur for strekk- testing (pull-test) med en belastning på 2500 lbs (1 136 kilo).

Dersom ringen ikke deformeres, er den tilstrekkelig herdet.

Dersom den bare deformeres ubetydelig, SKAL begge ringer byttes! Dersom du er i tvil skal du kontakte forhandler.

Ved godkjennelse av ringen skal dette anmerkes i hovedkontrollkort for seletøy, og ringen skal merkes (plomberes) med tynn ståltråd (ikke plombetråd) i ringens nedre del, i slissen for webbingen (ikke inne i selve ringen).

Denne modifikasjonen har etter tid vist seg å være vanskelig å kontrollere. Etter over 10 år er det ikke sikkert at testen fremdeles er å finne i hovedkontrollkortet.

\subsection{Reserveloops}
Reserveloops skal fingertrappes og eventuelt syes, slik at det ikke kan dannes en løkke inne i pakksekken. Noen seletøy bruker andre metoder for innstramming av reserveløkka (Racer - Reflex), og trenger da ikke syes. Fribagens utforming vil også være avgjørende om duk kan feste seg i låseløkka eller ei. ``Molar-bager'' har gjennomgående malje der låseløkka ikke kommer i kontakt med kalottduken.

Vanligvis benyttes Type II sleeving som loopmateriale, men også microline er anvendelig.

Ved bruk av dobbel loop for dobbel LOR skal det nyttes loops av ``tynn'' line (microline eller tilsvarende).

Når seletøyet er utstyrt med Cypres nødåpner skal AirTecs silikonbehandlede loop benyttes.

Bytt loops til reserven ved det minste tegn på slitasje og det bør som minimum utføres ved hver hovedkontroll. Dette er en god og billig forsikring.

\section{Frigjøringssystemer}
\subsection{Generelt}
En mengde ulike frigjøringssystemer har vært konstruert og anvendt gjennom tidene. I dag er kun et system godkjent for bruk i Norge – Tre rings systemet (Three Ring Circus) oppfunnet av Bill Booth, Relative Workshop i 1982.

For nærmere beskrivelse av virkemåten til trerings systemet henvises til Poynter Vol. 1, kapittel 4.116, side 122.

Flere seletøy produsenter lager sine egen ringer for trering frigjøringssystem. Merk at det kan være forskjellige spesifikasjoner på trerings, slik at løftstroppene fra de forskjellige produsentene vil være litt forskjellig med hensyn til låseløkke og avstand mellom ringene.

Tre-rings finnes i to størrelser: Stor og ``mini''. Virkemåten er den samme, men mini ringer har mindre toleranser for ringenes innbyrdes plassering som kan medføre endrede belastninger på systemet. Ref: Poynter Vol. 1, side 124 for riktig innbyrdes plassering av ringene.

\subsection{Vedlikehold og inspeksjon av Tre-rings kvikklås}
Trerings systemet må vedlikeholdes og inspiseres jevnlig for å være fullt operativt. Vær oppmerksom på følgende:
\begin{enumerate}
\item Tre-rings systemet bør kontrolleres før hvert hopp.
\item Sjekk virkemåten på systemet én gang i måneden. Trekk cutkabelen helt ut, og frigjør løftestroppene.
\item Vri nedre del av løftestroppen ved ringen frem og tilbake så den er myk og fleksibel. Dette er spesielt viktig på løftestroppene på elevutstyr, som lett blir stivere.
\item Låseløkken må være tynn og myk for lett å kunne slippe gjennom malja. Myk opp denne tilsvarende. Løkka skal ikke ha løse tråder, se etter slitasje. Løkkas lengde skal være så lang at ringene ligger parallelle ved lett belastning (sett fra siden).
\item Maljen på løftestroppene skal være jevn og fri for skader. Etter en stunds bruk kan enkelte fibre stikke ut fra maljefestet på oversiden av malja. Dette er ikke skadelig i seg selv, men må passes på slik at ikke malja løsner helt. Kontakt reparatør hvis du er i tvil.
\item Ringene skal ha metall til metall kontakt, og være parallelle når systemet har lett belastning. Dersom ringen som er festet til seletøyet har kontakt med webbingen i stedet for ringen på løftestroppen blir betraktelig mer av åpningsbelastningen overført til den minste ringen og låseløkka.
\begin{figure}
	%\includegraphics[width=60mm]{Strekktesting av kalottduk.pdf}
	\caption{Trerings sett fra siden}
\end{figure}

\item Rengjør frigjøringskabelen jevnlig med tørkepapir. Kablene kan settes inn med et tynt lag 3-i-1 olje (eller tilsvarende tynn olje) eller silicon. Kontroller at kabelendene er jevne, og tettet av plastikken – ingen løse wiredeler må stikke ut!
\item Kabelføringen for kutt-kabelen må være lang nok til at det ikke er den som opptar belastningen ved åpning. Det skal være noe bevegelse i kabelføringene når det er belastning på systemet. Sjekk malja eller hullet der låseløkka passerer igjennom på kabelføringen – den må være jevn og fri for skader. Kableføringene kan være festet eller ikke ved bryststroppen – avhengig av seletøyprodusent.
\item Sjekk borrelås på både kutthåndtaket og på webbingen.
\item Etter en stunds bruk kan deformeringer forekomme i ringene. Rotér derfor ringene en kvart omdreining jevnlig for å spre belastningen.
\item Parachutes de France stempler alle ringene sine med ``PF''. På den minste ringen kan dette forårsake slitasje på webbingen som går gjennom ringen. Kontroller at det ikke er skarpe kanter i forbindelse med stemplingen.
\item Vær også oppmerksom på fuktighet i alle deler av treringssystemet. Ved kulde i høyden kan deler fryse og stivne, og dermed hindre riktig funksjon.
\item På enkelte løftestropper kan det forekomme at tverrbåndet (confluence wrap) som fester låseloopen er festet for nærme malja, eller er festet med en ``X'' der den minste ringen kan sette seg fast i, og forsinke eller forhindre frigjøringDette båndet skal derfor være festet i nedkant med en tversgående søm.
\begin{figure}
	%\includegraphics[width=60mm]{Strekktesting av kalottduk.pdf}
	\caption{Confluence wrap tre rings løftestropper}
\end{figure}

\end{enumerate}

\subsection{Godkjennelse av tre-rings.}
Det kan forekomme at forskjellige fabrikanter har produsert tre-rings løftestropper med forskjellige standarder. En trerings på elevseletøyet fra Parachutes de France vil være litt annerledes bygd enn en trerings fra for eksempel Relative Workshop. Videre kan låseløkka for kuttkabelen være lagd i forksjellige materiale.

Trerings godkjennes ved en brukskontroll, der en kontrollerer at ringene er nær parallelle sett fra siden ved lett belastning, de skal ha metall til metall kontakt, webbingen og låseløkka skal være fleksibel, og viktigst: at de frigjør uten problemer ved en funksjonskontroll. Ref: Poynter Vol. 2, 4.114, s. 112.

Spesiell oppmerksomhet må vises når det foretas kontroll av ``mini'' tre-rings. Siden hele tre-rings systemet er basert på et vektarmprinsipp, betyr dette at toleransene blir mindre og belastningene større jo mindre ``armen'' er. Feil avstand mellom ringene kan påføre så stor belastning på låseløkka at denne trekker kuttkabelen igjennom malja på løftestroppen, med total blokkering for cut-away.

Ved lett belastning skal den minste ringens nederste del så vidt overlappe den største ringens øverste del.
\begin{figure}
	%\includegraphics[width=60mm]{Strekktesting av kalottduk.pdf}
	\caption{Kontroll av ringavstand}
\end{figure}

På store ringer er den største belastningen på låseløkka målt til 7 kilo. På mini-rings kan tilsvarende belastning være opp til 5 ganger så høy! Dette er mer enn nok til å trekke gjennom kuttkabelen.

Mindre skjermer med nullporøsitetsduk og uelastiske liner påfører mer belastning på seletøyet ved åpning, og toleransene ved bruk blir derfor mindre.

\subsection{Kutthåndtak}
Kutthåndtak skal kontrolleres for ujevnheter i plastbelegget, og at ikke enden er oppfliset og at wire korder stikker ut. Dette kan forårssake låsing av kabelen ved cut- away.

Merk at det er nødvendig med justering av kutt kablenes lengde når LOR er montert. Kabelen på den siden LOR er montert skal være 2-3 cm lenger enn den siden LOR ikke er montert. Dette skal gi en bedre mulighet til at den løse løftestroppen er frigjort før LOR kabelen trekker reserven. (Dette er ikke nødvendig ved LOR 2).

\subsubsection{Justering av kutthåndtak}
For å få til korrekt lengde på kutt kablene er det viktig at det er belastning i seletøyet. Det er derfor en oppgave for to personer:
\begin{enumerate}
\item Seletøyet monteres på en person som trekker i løftestroppene for å gi motstand i systemet.
\item Person nummer to trekker kabelendene opp for å ta ut slakk i kablene
\item Videre trekker han sidelengs i kabelføringene for kutthåndtaket for å fjerne slakk.
\item Deretter måles 12,5 cm fra 3 ring låseløkka på den siden det ikke er montert LOR, og 15 cm på den siden LOR er montert, og det settes et merke på kablene. Dette betraktes som minimum hvis ikke annet er spesifisert fra produsent.
\item Trekk sakte i kutthåndtaket, og kontroller at det korteste merket når låseløkka før det lengste.
\item Kablenejusteresmedengodavbitertang,ogvarmesoppforsiktigmedenlighter eller lignende. Plasten blir myk og formes med fingrene over wire kordeler:
\begin{figure}
	%\includegraphics[width=60mm]{Strekktesting av kalottduk.pdf}
	\caption{Justering av ende på kutt kabler}
\end{figure}

\end{enumerate}

\section{Reservehåndtak}
Reservehåndtak skal være stemplet eller merket av produsent, ellers er de ikke luftdyktige. Flere produsenter benytter merking av annen type en stempling f. Eks. PdF og flere typer soft håndtak som har påsydd merking.

Eksempler på innstempling:
\begin{itemize}
\item (W) – Wasley
\item pf eller cs – Parachutes de France
\item dj – DJ Associates
\item lf – Lite Flite Inc.
\item se – Strong Enterprises
\item naa – North American Aerodynamics.
\end{itemize}

Stempel/merking fra produsent er eneste mulighet for sporbarhet til produsent (og andre håndtak) skulle det oppstå problemer i fremtiden. Stempel viser også at produsent har godkjent håndtak ved produksjon – og at det er strekktestet. Reservehåndtak (pinner og endestykke) skal strekktestes til 140 kilo i 3 sekunder ved produksjon. Ref. Poynter Vol. 2, 4.100.

Relative Workshop har siden 1992 levert reservehåndtak for tandemutstyr uten metallhåndtak, men med en avstivet webbingløkke der kabelen er festet til. Dette er godkjent for bruk selv om det ikke er stemplet

Etter at freeflying ble en mer aktiv del av sporten ble det av sikkerhetsmessige grunner fra produsentene produsert reservehåndtak av kuttputetype. Dette ble gjort for å forhindre utilsiktet reservetrekk ved kontakt med andres ben og lignende, og er alminnelig på seletøy med ringer. Dette kan medføre vanskelighet med å avgjøre om reservehåndtaket er festet godt nok inne i puten. I fremtiden vil det også være vanskelig å se om håndtaket virkelig er produsert av en seletøysprodusent eller av andre uten tilstrekkelig kunnskap og verktøy. Disse håndtak skal merkes for gjenkjenning av produsenten, slik at de kan gjenkjennes ved senere kontroller.

\subsection{Reservehåndtak uten typegodkjenning}
Følgende reservehåndtak er forbudt brukt:
\begin{itemize}
\item Alle typer plasthåndtak.
\item Håndtak av type ``Blast Handle'' eller ``Lollypop'' Poynter Vol. 1, 6.15.1.1 er forbudt i bruk.
\item Håndtak med tinnloddet kule.
\item Håndtak med oval kule.
\item Håndtak med låsepinne som er klemt ned på midten – i motsetning til langs hele kontaktflaten til wiren.
\item Håndtak uten merking av produsent eller årstall
\end{itemize}

Dersom du er i tvil kan du sende håndtak til produsent for strekktesting og påstempling. Kasserte håndtak skal destrueres, og bør sendes SU for bruk i opplæringsøyemed.

\section{Løftestropper}
\subsection{Løftestropper for hovedskjermer}
Løftestropper leveres normalt av seletøyprodusenten, men det kan finnes flere typer i vanlig bruk. Sjekk uansett at virkemåte i forhold til trerings er tilfredsstillende (Frigjøringssystemer, kapittel 3.3.3).

Løftestroppene lages vanligvis i følgende typer:
\begin{description}
\item[Brede] i Type 8, 4.000 punds webbing (bredde cirka 47mm.),
\item[Smale] i Type 17, 2.500 punds webbing (bredde cirka 25 mm). (``Mini''-risers)
\item[Reverserte], smale, produsert blant annet av Parachutes de France (Atom) og Rigging Innovations (Flexon). Disse har i motsetning til de vanlige ikke hull med malje for låseløkka, og er derfor sterkere. Løftestroppene er konstruert i Type 17, 2.500 punds webbing.
\end{description}

Brede løftestropper kan være påmontert både vanlige og mini-trerings. Smale løftestropper finnes kun med mini-trerings.

I 1992 oppsto det enkelte hendelser der de smale trerings løftestroppene røk ved åpningsbelastning. Etter dette begynte produsentene å forsterke disse løftestroppene i nedre del. Dette kjennetegnes ved et kontrastfarvet bånd som er sydd inne i nedre del (buen) der løftestroppene får mest belastning. (Se kapittel 3.3.6, LOR – RSL angående bruk av mini-løftestropper og LOR).

\subsection{Løftestropper og LOR}
\begin{itemize}
\item LOR tillates ikke brukt sammen med smale løftestropper som ikke er forsterket (se ovenfor).
\item LOR tillates brukt sammen med smale, reverserte løftestropper uten gjennomgående malje.
\item Fallskjermseletøy som ikke har montert nødåpner skal ha påmontert LOR som skal være påkoblet under hopping.
\end{itemize}

Se forøvrig kapittel 3.3.6, LOR – RSL.

\subsection{Løftestropper for reserveskjermer}
Runde reserver kan være montert på 2 eller 4 løftestropper (eldre seletøy). Se forøvrig Kapittel 4 for reserveskjermer for videre spesifikasjon av hvilke typer som kan monteres på 2 eller 4 løftestropper.

Dersom reserve er montert på 2 løftestropper skal disse være montert på løftestroppene som er en gjennomgående del av seletøyet.

Alle firkantreserver skal monteres på 4 løftestropper.

\section{LOR – RSL}
LOR og RSL er to betegnelser som generelt går ut på det samme: et aktiviseringssystem for reserveskjermen.

Systemet er et tilleggsystem for åpning av reservecontaineren i de tilfelle hvor hopperen frigjør seg fra hovedskjermen (vanlig cut-away).

RSL er en forkortelse Reserve Static Lanyard, og LOR er forkortelse for det samme på fransk. Siden LOR ble introdusert i Norge via det franske elevutstyret i 1985 vil videre omtale av alle produsenters systemer bli omtalt som LOR.

Materielldelen del 200 i Håndboken regulerer bruken av LOR i Norge. Når LOR er montert på utstyret skal den også være i bruk. Unntak for denne bestemmelse er gitt i F/NLF’s Håndbok del 200, og omfatter følgende tilfeller:
\begin{itemize}
\item Dersom nødåpner er montert og i bruk (aktivert)
\end{itemize}

LOR har eksistert i forskjellige former siden rundt 1970. Det har lenge vært enighet om at LOR er et sikkert tillegg for elever som i en stresset situasjon ikke har husket / rukket å trekke reservehåndtaket. Ulykkesstatistikken viste imidlertid over tid at det et urovekkende antall erfarne hoppere som også var gått i bakken uten å ha trukket reservehåndtaket. Etter hvert har bruken av LOR fått bedre innpass til bruk også for erfarne hoppere, selv om det er mange argumenter for at LOR også i spesielle tilfelle kan føre til ulykker. I løpet av årene 1991 – 1996 er det ikke registret at LOR har ført til noen merkbar økning av ulykker, eller vært direkte årsak til dette.

LOR har hatt mange navn gjennom tidene (Last Chance Cord; Jesus Cord) – det vanligste er kanskje Stevens Cutway System etter Perry Stevens som konstruerte det første gjennomtenkte systemet for produksjon i 1963.

\subsection{Beskrivelse}
Generelt er LOR en kobling fra én eller begge løftestropper på hovedskjermen til enten reservehåndtaket / reservewiren, eller direkte til pinnene som holder reservepakksekken lukket.

Ved en feilfunksjon, og en vellykket kuttprosedyre (begge løftestroppene er frigjort), vil pinnen(e) som holder reservepakksekken lukket bli trukket ut når løftestroppene forsvinner fra hopperen.

LOR er derfor avhengig av en korrekt utført kuttprosedyre for å ha noen virkning. Det må allikevel ikke herske tvil om at reservehåndtaket alltid skal trekkes av hopperen etter en kutt.

LOR kan være påkoblet enten den ene eller begge løftestroppene. Det vanligste når den er koblet til én løftestropp er at den er koblet til den høyre. Felles for alle systemer er at det skal være påmontert en form for styring (ring, klaff eller lignende) slik at trekkretningen for reservepinnen blir med så lite motstand som mulig. På Campus elevutstyr ivaretas denne funksjonen av velcroen som lukker øverste reserveklaff. Denne må derfor alltid være i god stand og feste godt.

Alle LOR skal ha påmontert en form for frigjøringsanordning som kobler ut LOR lina. Dette er en sikkerhetsanordning som bare skal brukes i nødsfall der det ikke er ønskelig at reservecontaineren åpnes når hovedskjermen kuttes, for eksempel ved vannlanding, landing i sterk vind, eller popping av skjerm i fly.

I enkelte tilfelle kan den også frigjøres før cutaway, som ved innfiltring i hovedskjermer med annen hopper. Det er viktig at frigjøringsanordningen og innfestingen av denne kontrolleres ved jevne mellomrom.

Som materiellkontrollør er det viktig med kjennskap til de forskjellige systemers konstruksjon og virkemåte. Etter hvert som det blir flere forskjellige systemer i bruk, og i tillegg ettermontering på eldre utstyr, vil det være mange løsninger som kan se identiske ut, men ha forskjellig virkemåte. Det må derfor opparbeides kunnskap om det enkelte system for at korrekt vedlikehold og kontroll kan finne sted.

\subsection{De vanligste typer}
\subsubsection{Én pins}
Den ene enden av LOR lina er festet i en av løftestroppene, og i den andre enden av lina er reservepinnen festet. Vanligvis er lina festet via borrelås på seletøyet, og går igjennom en form for styrering som bestemmer trekkretningen på pinnen.

Denne typen LOR finnes på bl.a. Vector seletøy.

En pins LOR trekker ut reservepinnen når løftestroppen der LOR lina er festet forlater hopperen (uavhengig av den andre løftestroppens posisjon).

\subsubsection{To pins}
Det er to LOR liner, som er påmontert i hver sin løftestropp.

Pinnene skal gå til hver sin låseløkke (NB:dobbel låseløkke) i reservecontaineren.

Denne typen finnes på det meste av Fransk utstyr. (CAMPUS 2, Atom og Atom Tandem).

På dette systemet holdes reservepakksekken lukket til den siste løftestroppen forlater hopperen. Dette systemet skal forhindre tidlig pakksekkåpning ved ujevn frigjøring av løftestroppene. (Poynter Vol. 2, s 258.)

Ved bruk av dobbel låseløkke skal disse være sammensydd og av ``tynn'' type, Optima, Spectra eller lignende.

\begin{figure}
	%\includegraphics[width=60mm]{Strekktesting av kalottduk.pdf}
	\caption{LOR 2}
\end{figure}

Begge disse systemer har det til felles at reservepinnene sitter på LOR lina (linene) og at wiren på reservehåndtaket slutter i en form for en ring som festes rundt pinnen(e).

\subsubsection{Trekk av kabel eller pinne}
I enden på lina er det festet en ring i stedet for en pinne. Reservekabelen og pinnen tres igjennom denne før containeren lukkes, og ved kutt trekker lina i reservewiren.

Denne typen finnes bl.a. på Javelin og Talon. (Poynter Vol. 2, s 258)

En annen form for denne er systemet på på Campus 1 elevutstyr, der det er en liten løkke på enden av LOR lina som festes der reservepinnen er festet til reservewiren.

Dette systemet virker likt én pins systemet ved reservepinnen dras ut når ene løftestroppen forlater hopperen.

\begin{figure}
	%\includegraphics[width=60mm]{Strekktesting av kalottduk.pdf}
	\caption{LOR for trekk av kabel}
\end{figure}

\subsubsection{Racer / Pop Top}
Siden Racer med sin Pop-Top løsning på reserven har pinnene montert på ryggen, har Jump Shack konstruert et system der en line er festet mellom begge løftestroppene. Denne lina er montert under kabelføringen for reservehåndtaket. Kabelføringen er todelt, omtrent på midten, og festet i et strikkfeste.

\begin{figure}
	%\includegraphics[width=60mm]{Strekktesting av kalottduk.pdf}
	\caption{LOR for Racer}
\end{figure}

Ved en kutt vil lina trekke i kabelføringen, som vi gi seg der den er delt, og LOR lina vil videre trekke i reservewiren og frigjøre pinnene.

Lengden på lina gjør at begge løftestroppene må være fri fra tre- ringsen for at lina blir stram nok til å trekke wiren. (Poynter Vol. 2, s 258.) Merk at dette systemet krever spesiell kompetanse ved pakking og montering.

Ved reservepakking av Racer med LOR bør ikke hovedskjermen være tilkoblet seletøyet med LOR lina under pakkingen. Dette for å unngå at LOR lina blir pakket inn under klaffene som lukker reservecontaineren, og dermed fører til låsing ved en cutaway. LOR lina kan monteres løst rundt kableføringen og reservewiren etter at reservecontaineren er lukket. Deretter skjules resten av LOR–lina under klaffer og toppen av Pop–topen. Bruk Racer manualens tegninger for korrekt utførelse.

For videre henvisninger med illustrasjoner på forskjellige LOR/RSL typer henvises til Poynter Vol. 2, Pkt 6.23 side 257-258, samt produsentenes manualer.

\subsection{Montering}
Endring og påsying av LOR skal foretas av produsent eller materiellreparatør. Materiellkontrollør kan demontere og montere LOR under pakking og kontroll.

Det er viktig at komponenter er tilpasset utstyret og virkemåte. Kontroller med fabrikantens manualer.

\subsection{Kontroll og vedlikehold}
Følgende må sjekkes og tilpasses ved de forskjellige LOR system:
\begin{enumerate}
\item LOR er påmontert riktig, og at lina går via nødvendige føringsringer eller tilsvarende. Båndet er festet godt på velcro. På Campus 1 må velcroen øverst på ytre reserveklaff være av god kvalitet, det er denne som er styring for LOR båndet, og den blir fort slitt siden reserveklaffen åpnes for inspeksjon relativt ofte.
\item Ved enkel LOR, (kun en LOR line til ring eller pinne), skal kutthåndtakets kabellengde tilpasses slik at den kabelen som går til den løftestroppen der LOR er montert er cirka 2 cm lengre enn den andre kabelen. Dette skal avhjelpe at løftestroppen der LOR er montert forlater hopperen sist. Bruk god avbitertang, og forsegl wiren med plastikken ved hjelp av lighter.
\item Ved dobbel LOR er det spesielt viktig å passe på at ringen fra reservehåndtaket er festet rundt begge pinnene, ellers vil åpning av reservepakksekken med reservehåndtaket bli satt fullstendig ut av drift. Dette medfører dermed totalforsager ved alminnelig reservetrekk uten cutaway!
\item Ved dobbel LOR skal det benyttes dobbel PdF låseløkke i tynn mikroline (Optima, Spectra eller lignende). Utløserpinnene skal monteres i hver sin løkke.
\item Frigjøringkroken er montert og fungerer. Denne kroken er vanligvis av en såkalt ``svenske-krok'', (Bronze Fixed Bail Snap Shackle ref ParaGear varenr. H13440) som er sterk under belastning, men krever liten kraft for frigjøring. Det skal være et 5 cm langt synlig bånd festet i frigjøringsringen.
\item Pass på at tegninger på dekklaffen for reserven (der disse finnes) er oppdaterte og synlige. Sjekk anvisningen og monteringen nøye etter arbeidet.
\item Se forøvrig pkt. 3.3.5.2, Løftestropper og LOR.
\end{enumerate}

\subsection{Skyhook fra RW-Shop}
Skyhook er designet og utviklet for å trekke reserveskjermen med bag ut av containeren og gir en hurtigere åpning av reserveskjermen. Skyhook er festet til hovedløftestroppen og reservepakksekkens pilotskjermbånd. Ved frigjøring av hovedskjermen vil denne virke som ”pilotskjerm” for reserven, noe som gir hurtigere åpning. Skyhook er utviklet av Bill Booth hos Relative Workshop.

For bruk og pakking henvises det til fabrikantens manual eller webside.

\section{Links}
Links / sjakler anses som en del av fallskjermen snarere enn seletøyet, og leveres vanligvis med denne fra fallskjermprodusenten.

For ytterligere informasjon om metalldeler henvises til Kapittel 9.5, Metallkomponenter. I dette kapitlet vil det være mer utførlig godkjente og ikke godkjente links og bruken av disse.

For montering av hovedskjermer se kapittel 5 – Hovedskjermer.

Merk ellers at:

Links på hoved- og reserveskjermer skal være av godkjente typer.

Løftestroppene kan trenge modifisering for at ulike typer av links skal kunne anvendes. Se Poynter Vol. 1, kapittel 4.113. Dette gjelder spesielt runde reserveskjermer.

\section{Pilotskjermer / fribag}
\subsection{Montert på hovedskjermer}
Pilotskjermer til hovedskjermer krever generelt ettersyn, og må påregnes å byttes etter cirka 3-400 hopp. Duken i pilotskjermer i F-111 duk blir slitt og porøs, og mister en stor del av sin effekt etter hvert.

Størrelsen og porøsitet på pilotskjermen kan ha til dels dramatiske virkninger på åpningssekvensen på hovedskjermen. En skjerm med for stor dragkraft kan blant annet skade topp duken på hovedskjermen, og medføre til uryddige åpninger (``linedump''). Det er derfor viktig at pilotskjerm anpasses resten av utstyret.

\begin{itemize}
\item En vanlig F-111 pilotskjerm er cirka 34 – 36 tommer i diameter.
\item En 0-porøsitets pilotskjerm bør ikke være større enn cirka 30 tommer i diameter, og ikke mindre enn cirka 24 tommer.
\item En for svak pilotskjerm kan medføre totalforsager, og feil åpningssekvens.
\end{itemize}

\subsection{Kollapsbare pilotskjermer}
For å få best ytelse på skjermer, og samtidig minske slitasje på liner ved linksene, er det utviklet flere typer ``inntrekkbare'' pilotskjermer, for å ikke bremse farten på skjermen. Disse finnes hovedsakelig i to typer:

\subsubsection{Elastisk}
Her er en strikk er påmontert pilotbåndet og festet i toppen av pilotskjermen. Skjermen pakkes i ``sammentrukket'' tilstand, og lufthastigheten ved trekk strekker ut strikken slik at piloten åpner. Når hastigheten avtar trekker strikken toppen av pilotskjermen inn og kollapser den.

Dette systemet er sårbart for fallhastigheten ved trekk, og at strikken kan bli montert for stram. Dette kan føre til en streamer på pilotskjermen. Det bør helst ikke være en knute inne i piloten som avslutter festet av strikken.

Dersom duken i piloten blir for porøs, kan dette også føre til at pilotskjermen ikke fungerer tilfredsstillende.

NB: Vær varsom med å bytte til ny strikk på en gammel pilotskjerm.

Pilotskjermen av denne typen skal leveres med 0-porøsitetsduk. F-111 duk kan etter kort tids bruk bli for porøs til at strikken trekkes ut.

\begin{figure}
	%\includegraphics[width=60mm]{Strekktesting av kalottduk.pdf}
	Lengden på strikken skal være tilstrekkelig til at festet ligger inne i pilotskjermen.

	I ovenstående tilfelle er strikken for stram og kan forhindre eller forsinke korrekt åpning av pilotskjermen.	
	\caption{Kontroll av elastisk pilotskjerm}
\end{figure}

\subsubsection{Kill-line}
Dette er et mer avansert system, der pilotskjerm og pilotbånd er konstruert i ett. Piloten blir trukket inn av en line som går inne i et hulrom på pilotbåndet etter at skjermen har forlatt innerbagen.

Piloten ``spennes'' ved pakking (pakkes i ``åpen'' tilstand), og virkningen av ``kill-line'' kommer først etter at hovedskjermen har forlatt innerbagen. Da trekkes toppen av piloten ned og inn i midten av pilotskjermen.

Dette systemet er sårbart ved at innerlina må strammes hver gang ved pakking. Dersom dette blir glemt / utelatt vil ikke pilotskjermen åpne. Dette systemet har også en større slitasje ved at det består av flere bevegelige deler.

Vær også varsom på at over tid kan pilotbåndet og innerlina få forskjellige lengder. Dette skyldes forskjellig strekk i materialene, der sømmene i nylon båndet (hylsa) gir seg over tid, og blir lenger. Dette vil medføre at innerlina relativt sett blir kortere, og gir mindre trekkraft til piloten.

Ved kontroll av Kill Line må det påses at lengden på innerlina og pilotbåndet er riktig tilpasset hverandre. Sjekk også slitasjepunkt rett på innsiden av innerbagen, der innerlina passerer ut av pilotbåndet.

\begin{figure}
	%\includegraphics[width=60mm]{Strekktesting av kalottduk.pdf}
	\caption{Prinsippmodell Kill Line}
\end{figure}

Uansett hvilket av disse system som velges, må det vises stor aktsomhet ved montering bruk og pakking. Kjenn utstyret og virkemåte 100 \% før det tas i bruk.

\subsection{Pilotskjermer montert i reservecontainere}
\begin{itemize}
\item Kun pilotskjermer som er godkjent av seletøysfabrikanten tillates benyttet. Disse er merket med produsentens datapanel. Bytting av komponenter som for eksempel pilotskjerm fra fabrikantens standard bryter godkjennelsen på utstyret. Seletøy er kun testet og godkjent med fabrikantens egne komponenter.
\item Ved koniske eller dobbelkoniske fjærer i pilotskjermer skal stoffet trekkes ut av fjæren ved pakking. (Vector, Javelin, Quick piloter Atom og lignende.)
\item Ved sylindriske (like brede) fjærer i pilotskjermer skal stoffet legges mellom fjær- ringene ved pakking. I motsatt fall kan fjærene hekte seg sammen og blokkere pilotskjermfunksjonen. Disse finnes på blant annet på Campus og Centaurus seletøy.
\end{itemize}

\textbf{NB:} I tilfellet med sylindriske fjærer er det avvik mellom Poynters manual og Parachutes de France vedrørende plassering duken på pilotskjermen. Poynter angir at det ikke skal legges duk imellom spiralene på sylindrisk fjærer, mens produsenten angir at duken skal legges imellom. For vårt vedkommende følger vi retningslinjer fra Parachutes de France.

Se Poynter Vol. 2, 6.12, og illustrasjon av fjærer på side 246.

\subsubsection{Pilotskjerm}
Seletøyfabrikantens pilotskjerm skal benyttes.

\subsection{Fribag}
Seletøyprodusentens fribag med pilotskjerm skal nyttes ved all pakking av firkantreserver. Unntak ved Hobbit fra Strong Enterprises, som er tillatt brukt med diaper og ``direct pilotchute attatchment''.

Pass på at fribagen er i god stand, og at den er riktig størrelse i forhold til størrelsen på reservecontaineren og reservens pakkevolum.

Safetystow må kontrolleres for skader, brudd og nødvendig elastisitet.

Pilotbåndet til fribagen skal være minst like langt som reservens liner + kalott for å kunne fungere etter hensikten.
