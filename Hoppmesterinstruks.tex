\part{Hoppmesterinstruks}
\setcounter{section}{507}

\section{Hoppmesterinstruks}
\subsection{Hoppmester (HM)}
Hoppmester utpekes av HFL, og er den person som har ansvaret for, og kommando over hoppere fra disse er oppført på manifest, til de har forlatt flyet.

Hoppmestring av elever medfører ansvar for at elevene blir korrekt inspisert, orientert, instruert, spottet og utsatt, og korrekt vurdert etter utført hopp.

Tandeminstruktør, og AFF instruktør på nivå 1-7, er ansvarlig for at de elever de instruerer, følger de anvisninger som blir gitt av HM.

\subsection{Kvalifikasjoner}
Som tilstrekkelig kvalifikasjon for HM regnes:
\begin{itemize}
	\item Når alle ombordværende hoppere har A-sertifikat eller høyere:

	Hoppere med B-sertifikat. Med denne sammensetningen, kan HM forlate flyet før selvstendige hoppere, men da kun når det skjer på samme run som de øvrige.

	\item Når det blant de ombordværende hoppere er FF elever: Instruktør 3
	\item Når det blant de ombordværende hoppere er lineelever: Instruktør 3 med godkjent utsjekk for hoppmestring av lineelever, punkt 402.1.5.
	\item For AFF-elever på nivå 1-7: Instruktør 3-AFF
	\item Dersom utsprang innebærer natthopp, vannhopp eller oksygenhopp:

	Instruktørlisens klasse 1
	\item Dersom utsprang innebærer oppvisningshopp:

	Tilfredsstille krav som angitt under F/NLFs bestemmelser Del 300
\end{itemize}

\subsection{Ansvar}
HM er ansvarlig for hopperne i sitt løft, og har ansvaret overfor Flyger for at dennes anvisninger blir fulgt.

HM har ansvaret for at de utsprang som utføres under hans kommando skjer i overensstemmelse med F/NLFs bestemmelser Del 100.

Der jordtegn anvendes som signalsystem er HM ansvarlig for at utsprang ikke foretas dersom jordtegn er fjernet eller slik plassert at det må forstås at utsprang ikke skal finne sted.

Er jordtegn eneste kommunikasjon bakke/fly skal HM påse at utsprang ikke gjennomføres uten at jordtegn er synlige fra flyet.

Anvendes annet kommunikasjonssystem enn jordtegn, har HM ansvaret for at ordrer fra bakken følges.

\subsection{Myndighet}
HM har kommandomyndighet over alle hoppere i løftet fra de er manifestført til utsprang er foretatt og alle hoppere plikter å følge hans anvisninger og kommandoer, uansett om det er hoppere med høyere status med i løftet.

HM har myndighet til å delegere oppgaver til andre hoppere dersom han ser behov for dette, men kun dersom hopperene er kvalifisert til de oppgavene som blir gitt.

HM har rådgivende myndighet overfor Flyger, og kan kun beordre utsprang med Flygers samtykke.

\subsection{Plikter}
Det påligger HM å holde HFL orientert om alle forhold av betydning for hoppingens gjennomføring.

HM skal holde seg orientert om løftenes sammensetning, antatt tidspunkt for takeoff, samt holde seg selv og hopperne i løftet klar i god tid.

HM følger HFLs direktiver mht løftenes sammensetning og rekkefølge, samt eventuelle særlige instrukser gitt av lokalklubb, samt eventuelle instrukser gjeldende for det hoppfelt som benyttes.

HM har følgende spesielle plikter:

\subsubsection{Kontroll av fly, briefing av flyger}
HM skal forsikre seg om at det finnes egnet kniv lett tilgjengelig i flyet.

HM skal orientere Flyger om utsprangshøyde over terrenget, om innflygingsretning, driverkast, åpningspunkt, nedthrottling, korreksjoner etc.

Avtale klare signaler med Flyger.

\subsubsection{Kontroll av hopperne}
HM skal følge de anvisninger HL gir for å forsikre seg om at hoppernes kvalifikasjoner etter elevbevis, sertifikat, lisens og hopplogg er overensstemmende med den hopping som skal utføres og det utstyr som anvendes, og skal ikke pålegge/tillate hoppere oppgaver de ikke er kvalifisert for i henhold til progresjonsprogram og/eller rettigheter.

HM skal forsikre seg om at hver enkelt elev i løftet inspiseres før innlasting i flyet. Hoppere med A-sertifikat eller høyere er selv ansvarlig for egen inspisering, og kan kontrollere hverandre. Tandem- og AFF-instruktører skal selv inspisere de elever de instruerer på angjeldende hopp.

HM kontrollerer først om hopperens seletøy er riktig tilpasset. Deretter skjer inspeksjonen i detalj, først fremsiden, deretter baksiden, inspeksjonen foretas ovenfra og nedover.

Inspeksjonspunktene i etterfølgende kontrollrutine gjelder for standard elevutstyr, og tilpasses for kontroll av selvstendige hoppere med eget utstyr.

Inspeksjonspunktene foran:
\begin{enumerate}
	\item Hjelm, hardt skall, fastspent, hakestropp i orden.
	\item Briller, festet så de ikke kan blåse av.
	\item Hansker, skinn eller kunstskinn, ikke for tykke. Hanskene skal ikke redusere gripeevnen.
	\item Bekledning, hensiktsmessig i forhold til årstid, ferdighetsnivå og hopptype.
	\item Løftestropper, feste og ruting av utløserline for reserven (LOR), 3- ringslåser, utløserkabler, kabelhus og cut-awaypute korrekt montert i god velcro .
	\item Bryststropp korrekt tredd, låst og innsløyfet.
	\item Utløserhåndtak for reserve, av metall, støtt i lommen, ikke blokkert, god velcro. Kabelføringen hel, kulen hel og forsvarlig festet, kabelen i orden. God velcro i håndtakslommen.
	\item Nødåpner, påslåttt, kontroll av kalibrering. Dersom første hopp for dagen skal Cypres nødåpner først slås av, og deretter på før hopping.
	\item Hovedskjermens utløserhåndtak, korrekt montert.
	\item Høydemåler montert slik at den ikke er til hinder for skjermaktivering, cutaway eller kan forveksles med andre komponenter under håndgrepene for trekk, og på en måte som hindrer høydemåleren i å forskyver seg sideveis eller komme ut av stilling (synsfelt). Innstilling kontrolleres.
	\item Eventuelle sidestrammere riktig justert og foldet inn.
	\item Benstropper, ikke vridd eller krysset, korrekt tredd, riktig justert og innsløyfet.
	\item Egnet fottøy som støtter anklene. Hemper for lissene skal tildekkes med tape eller tilsvarende.
	\item Eventuell flytevest, og kniv. Hoppere med elevutstyr skal bære eventuell flytevest av fast materiale under ytterklær slik at denne ikke kan hindre aktivering av hoved- eller reserveskjerm.
	\item Hovedkontrollkort for fallskjermsettet, bruks- og hovedkontrollert iht krav satt i Del 200.
\end{enumerate}

Inspeksjonspunktene bak:
\begin{enumerate}	
	\item Løftestropper, uten vridning, riktig rutet og sikret under dekklaffer med god lukking. Ingen synlige styreliner eller -håndtak.
	\item Wire og pinne for reservehåndtak, evt. kabel fra nødåpner, LOR-line korrekt montert iht produsentens manual. Kontroll av loop og pinne. Plombering overensstemmende med pakkelogg. Reservecontainers dekklaff lukkes.
	\item Eventuell utløserline, kontroll av ankerkrok (åpne/lukke/åpne), sikringspinne tilstede. Kontroll av linen for skader, pinne korrekt montert, loop i orden, linen korrekt rutet og innsløyfet. Gode strikker. Eventuell utløserline med pilotlomme, pilotlomme korrekt lukket, pilotbånd korrekt rutet og pilotlomme korrekt påstrikket.
	\item Eventuell utløserkabel for hovedskjerm, kontroll av kabelføring, pinne og loop.
	\item Dekk-klaffen lukkes.
\end{enumerate}

\subsubsection{Orientering til hopperne}
HM skal orientere hopperne om hoppfelt, fly, vind, flyretning, forhold i fly, ordre, tegn, signaler. I tillegg for elever hver manns FF-tid, oppgave i FF, forhold vedrørende styring, landing og forhold etter landing. Kontrollere innlasting, evt. øve utsprang dersom noen ikke har hoppet fra flytypen før.

Ved hopping med elever, øve nødprosedyrer, forhold ved landing i trær, vann og ledninger, samt øve spesielt for det forestående hoppet, alt etter elevenes progresjon, inntil tilfredsstillende ferdighet er nådd. Tandem- og AFF- instruktører skal selv forestå dette for de elever de skal instruere under gjeldene hopp.

Besørge fastkroking og sikring av eventuell utløserline.

\subsubsection{Vind}
Fastslå middelvinden iht. Del 100, dersom dette ikke er gjort av tidligere løft, eller dersom det har skjedd en større endring i vindforholdene. Ved konkurranse, følge Konkurransereglementet.

\subsubsection{I flyet}
Kontrollere at hopperene sitter fastspent, og har på hjelm opp til 1000 ft. Hjelm påmontert kamera skal sitte på hodet eller spennes fast.

Bestemme flyretning, dirigere Flyger, motta klarsignal fra Flyger, bruke utsprangskommandoer iht. Del 600.

For elever, kontrollere hopperens utstyr, særlig utløserlinen, mens eleven klatrer ut. Signalisere nedthrottling til Flyger, gi hopperen hoppordre, følge med i hopperens fall til skjermen er fullt åpen, stue vekk utløserline, merke seg hoppets utførelse og punkter for tilbakemelding.

NB! Så snart alle lineelever har hoppet og før andre hoppere foretar utsprang, skal samtlige utløserliner krokes løs og stues vekk.

\subsubsection{Kontroll av landingspunkt}
Merke seg hopperens landingspunkt og ta hensyn til dette ved senere utsetting av hoppere.

\subsection{Andre forhold}
HM sørger for at det utstyr som er anvendt i hans løft blir lagt på anvist sted. Sammen med HFL gir han tilbakemelding av hoppene til de enkelte hoppere. Beskrivelse av hoppets utførelse som kan ha interesse for senere HM'er for vurdering av hopperen skal innføres i hopperens loggbok. Det benyttes nøktern terminologi. HM signerer hopplogg med navn og sertifikatnummer.

AFF-instruktører skal selv forestå dette for de elever de skal instruere under gjeldene hopp.

Ved bedømmelse ``Farlig/ukontrollert hopping'' følges bestemmelsene i Del 600.
