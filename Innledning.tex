\part{Innledning}

\section{Forord}
Materiellhåndboka (MK-boka) utgis av Fallskjermseksjonen i Norges Luftsportsforbund (F/NLF) ved Sikkerhets og utdanningskomiteen (SU)

MK-boka er en veiledning og et hjelpemiddel for F/NLF's Materiellkontrollører, samt oppslagsverk for blant annet materiellforvaltere, brukergrupper (for eksempel tandemhoppmestere) og andre materiellinteresserte. Denne Materiellhåndboka erstatter tidligere MK Bok fra 1997 , og bestemmelser i den nye versjonen er å anse som generelle materiellbestemmelser for F/NLF.

MK-boka omhandler fallskjermkomponenter som er typegodkjent for bruk i Norge. De anvisninger og modifikasjoner som er gjengitt her, er påbudt i Norge.

MK-boka er forøvrig et norsk supplement og refererer til fallskjermprodusentenes egen dokumentasjon, som f. eks. pakkemanualer og anvisninger.

Ved pakking og/eller rigging av fallskjermutstyr, skal alltid fallskjermprodusentenes håndbøker være tilgjengelig.

En materiellkontrollør plikter således å holde oppdatert følgende dokumenter i sitt bibliotek :
\begin{itemize}
	\item F/NLF's Håndbok.
	\item Materiellhåndboka (MK-boka).
	\item The Parachute Manual (``Poynter's Manual''), 1. og 2. utgave av Dan Poynter.
	\item Fabrikantens dokumentasjon for de typer fallskjermkomponenter som Materiellkontrolløren har ansvaret for.
\end{itemize}

\subsection{Revisjon - tillegg - nye utgaver}
Endringer til MK-boka blir etter behov automatisk sendt ut til alle materiellkontrollører og andre som abonnerer særskilt.

Alle sider er nummerert med kapittel- og sidenummer samt utgavenummer og dato for utgivelsen. Det vil først i boka være en liste som angir hvilke sider som er gyldige.

Endringer og/eller tillegg til MK-boka vil komme ut i form av nye sider, som enten erstatter de gamle fullstendig, eller som kommer i tillegg til eksisterende sider. Disse sidene vil være merket med samme utgavenummer, men de vil ha et tillegg som angir at siden er endret og med dato for denne.:

Eksempel: ``Versjon 1.1 - Gyldig fra 1.6.97''
``Versjon 1.1 - Endring 1.1.98''

I tillegg vil det være angitt med en loddrett strek i margen hvor endringen eller tillegget er utført. Det vil også følge med en ny liste over gyldige sider i MK-boka.

En ny utgave vil inkludere og erstatte alle endringer og tillegg som er utført i foregående utgaver.

Endringer av Materiellhåndboka, samt siste versjon vil også være tilgjengelig på Internet:

\textbf{www.nlf.no/fallskjerm}

Rettelser legges fortløpende inn i boka, og skjema for rettelser oppdateres i Avsnitt 1.1.4, Rettelser og tillegg.

Redaksjonen vil takke for kommentarer som har kommet inn.

\subsection{Kommentarer og korreksjoner til MK-boka}
Det oppfordres til alle som har forslag eller kommentarer og rettelser til MK-boka om å benytte skjema og adresse oppgitt i del 1.4 i dette kapittel.

\section{Organisasjon og bestemmelser}
\subsection{Ansvar- og arbeidsområder}
I henhold til Bestemmelser for Sivil Luftfart (BSL) D 4-2, kapittel 2.1.b) er vår organisasjon ansvarlig for utarbeidelse av bestemmelser som Materiellreglementet (Håndboka del 200), og krav til fallskjermer (denne instruks).

Materiellsjefen (MSJ) er det faglige organ for materielltjenesten innen sivil fallskjermsidrett i Norge. MSJ er ansvarlig for at materielltjenesten til en hver tid er regulert av bestemmelser som krav til sikkerhet gjør nødvendig. SU skal føre kontroll med at bestemmelsene overholdes.

MSJ oppnevnes av Styret F/NLF og utøver i samarbeid med Sikkerhets- og utdanningskomiteen (SU) overordnet kontroll med all materielltjeneste innen F/NLF. MSJ utarbeider årlig målsettinger og en arbeidsplan som godkjennes av Styret F/NLF.

MSJ typegodkjenner alt fallskjermmateriell.

SU er Seksjonsstyrets rådgivende og utførende organ for å ivareta den operative materiellsikkerheten.

\subsection{Nyregistrering og typegodkjenning av utstyr}
Alt fallskjermutstyr som nyttes til sportshopping skal derfor være typegodkjent for bruk i Norge. Dette er en prosess som foregår etter flere vurderinger av det enkelte utstyret og produsenten.

Selv om utstyr kan være i alminnelig bruk i andre land betyr ikke dette at de automatisk er godkjent i Norge. Norge er et lite land i mange forhold. For å kunne opprettholde et høyt kunnskapsnivå hos materiellansvarlige er det derfor viktig at vi har et system der det er mulig å bli kjent med forskjellige typer fallskjermer, både under praktisk bruk og kontroll, uten at det blir for mange forskjellige ting å ta hensyn til.

\subsubsection{Kriterier}
Det er flere kriterier som må oppfylles for at fallskjermutstyr skal bli godkjent for bruk i Norge, de fleste går direkte på produsenten. Noen viktige momenter er:
\begin{description}
	\item[Kjennskap til produsent] Her er det avgjørende hva denne produsenten allerede har av erfaring med produksjon og i hvilken grad utstyret er i bruk. Et tillitsforhold mellom produsent og bruker kan være avgjørende.
	\item[Kvalitetssikring, med spesiell vekt på sporbarhet] Produsentens kvalitetssikringssystem skal ha vært tilgjengelig for inspeksjon og vurdering. Alternativ til dette er en direkte kontroll av fabrikken ved fabrikkbesøk.

	Sporbarhet med materialer og arkivering av disse opplysningene er en vesentlig del av en kvalitetssikringsprosess hos en fallskjermprodusent.

	\item[Godkjenningsordninger – TSO eller tilsvarende] Produsenten skal ha godkjent sitt produksjonssystem, og de enkelte produktene (gjelder reserver og seletøy) gjennom en godkjenningsnorm, for eksempel TSO – C23 som er en amerikansk godkjenningsstandard som fastsetter minimumskrav for ytelse, design og konstruksjon av fallskjermutstyr. Versjonen som gjelder nå er versjon d og benevnes med TSO-C23d.

	I Europa har man tilsvarende godkjenningssystem som kalles JARTSO. (JAR står for Joint Aviation Requirements) Her gjelder også revisjon d, så benevnelsen her er JARTSO-D23d

	Annet kvalitetssikringsdokumentasjon er for eksempel ISO 9001.

	\item[Erfaring i markedet] Produsentens erfaring og grader av suksess i markedet basert på praktisk bruk.
	\item[Kvalifisert personell] Det er viktig å ha kjennskap til hvilke personer som står bak produsentens ledelse, hvem som er produksjonsansvarlig, og hvem som er kvalitetsans- varlig.

	Opplæring og trening av produksjonsarbeidere er også et viktig element.

	\item[Produksjonsmetoder] En skal ha kjennskap til produksjonsprosessen hos produsenten.

	\item[Erfaring fra bruk av tilsvarende skjermer] Vanligvis blir nye skjermer og seletøy testet ut over lenger tid i produsentlandet. Det er viktig for sluttbruker at utstyret har gjennomgått den viktigste del av ``testen'', daglig og vanlig bruk blant vanlige fallskjermhoppere. Det kan være mange erfaringer spart på å la andre teste utstyret før vi tar det i bruk.
\end{description}

Ut fra en totalvurdering av ovenstående kan SU velge å godkjenne eller nekte godkjennelse av produsentens enkeltkomponenter.

\subsection{Typegodkjenningsprosessen}
I Norge kan SU typegodkjenne utstyr ut fra følgende kriterier:

\subsubsection{Produsent}
\textbf{Godkjennelse av produsent} foretas etter vurdering ifølge ovenstående kriterier. En totalvurdering av produsenten vil danne grunnlag for en videre godkjenning av dennes utstyr.

Det legges spesielt vekt på produsentens kvalitetssikrings system, samt dokumentasjon og administrative rutiner i forhold til sporbarhet.

Dokumentasjon på ovenstående må kunne forevises SU.

\subsubsection{Hovedskjermer:}
\textbf{Produsenten godkjennes.}

En produsent kan ha flere typer hovedskjermer, og dersom en produsent blir godkjent blir skjermene også godkjente, også når det kommer nye på markedet.

En begrensning her vil være muligheten for SU å nekte enkelte av skjermene dersom spesielle årsaker skulle tilsi det.

\subsubsection{Reserveskjermer:}
\textbf{Produsenten og den enkelte reserveskjerm godkjennes.}

Den enkelte type reserveskjerm skal godkjennes før bruk. Minimum kriterium for godkjenning er at skjermen er godkjent gjennom Amerikansk TSO eller tilsvarende Europeisk godkjenningsordning (JAR).

Den skal videre ha vært i bruk i utlandet i ett år før den kan tas i alminnelig bruk i Norge.

Dersom typegodkjenningen gjelder en videreutvikling eller størrelse endring av en serie med skjermer (for eksempel en annen størrelse av en skjerm i en serie med reserveskjermer), kan typegodkjenning gis så snart reserven er TSO / tilsvarende godkjent.

\subsubsection{Seletøy}
\textbf{Produsenten og det enkelte seletøy godkjennes.}

Den enkelte type seletøy skal godkjennes før bruk.

Minimum kriterium for godkjenning er at seletøyet er godkjent gjennom Amerikansk TSO eller tilsvarende Europeisk godkjenningsordning (JAR).

Seletøyet skal fremlegges for SU, for kontroll.

Den skal videre ha vært i bruk i utlandet i ett år før den kan tas i alminnelig bruk i Norge.

Seletøy skal angi hvilke skjermer det er bygget for, med modellnummer eller identifikasjon av størrelse.

Dersom typegodkjenningen gjelder en videreutvikling eller endring av størrelse kan typegodkjenning gis så snart seletøyet er TSO / tilsvarende godkjent.

Enkeltdeler til seletøy godkjennes normalt med seletøyet.

\subsubsection{Automatåpnere}
\textbf{Produsenten og den enkelte automatåpner godkjennes.}

Elevversjoner godkjennes separat.

\textbf{NB: SU kan i et hvert tilfelle, og uten begrensning eller begrunnelse nekte typegodkjenning på enhver produsent eller enkeltdeler, uavhengig av tidligere vedtak.}

Dersom du er interessert i å søke om typegodkjenning av nytt utstyr må tilfredsstillende dokumentasjon på produsentens produksjons- og kvalitetssikrings prosedyrer foreligge som søknad til Materiellsjefen/SU, sammen med manualer for utstyret.

Søknadsskjema finnes på neste side.

\subsection{Søknadsblankett for typegodkjenning}

\section{Materiellkontrollørtjenesten}
Fallskjermhopping og det utstyret som er i bruk idag, er dominert av de ideer og de teknikker som anvendes i USA og Frankrike. En har de senere år sett at fallskjermindustrien har spredt seg til flere land både i Europa og andre deler av verden. Dessuten går utviklingen på markedet så raskt, at det er veldig tidkrevende å være ajour til en hver tid. En viktig forutsetning for dette vil alltid være å lese utenlandsk fallskjermlitteratur, bøker, tidsskrifter og utstyrkataloger.

Det å reise til større samlinger både innen vårt miljø og til utlandet, vil utvikle og modne hoppere på langt kortere tid enn i egen lokalklubb.

En viktig forutsetning for en fallskjermfagmann er å ha et våkent hode. Det å være åpen for nye impulser, men samtidig være konservativ når nytt utstyr lanseres, er viktige egenskaper å utvikle. Nye typer av utstyr kan bygge på prinsipper som har framtiden for seg, men er dessverre ofte belastet med en del barnesykdommer. Ha det for øyet når du hovedkontrollerer nytt utstyr, selv om det er typegodkjent av F/NLF.

\subsection{Den geniale ide – Den menneskelige faktor}
Følgende var en del av innledningen til Materiellkontrollørboka fra Finn Ove Gåsøy i 1986. Dette er tatt med her da det er like aktuelt i dag, og også forteller en del om det vi anser som utvikling innen fallskjermsporten:

Fra tid til annen får de fleste av oss geniale idé tror vi: ``Hvorfor har ingen tenkt på dette før?'' eller ``Nå har jeg vært flink''.

\textbf{Motforestillinger.}

Den ``geniale idéen'' har helt sikkert vært vurdert av andre hoppere før. Spørsmålet blir derfor: Dersom det er tilfelle hvorfor ble så idéen forkastet av disse hopperne?

Ofte har idéen blitt nøye utprøvd, og gitt negative erfaringer som vi ikke kjenner til!

\textbf{Faren ved logisk resonnement:}

Du har kontrollert produktet etter nøye vurdering. Pkt. 1 til 6 alle OK, medfører at ideen er OK. Faren ved dette er at du kanskje ikke vet om eventuelle pkt. 7, 8 og 9, eller at du ikke har forstått betydningen av dem på grunn av manglende erfaring. Altfor fascinert av egne tanker og idéer?

Nytt ``moteutstyr'' kan ha komponenter som ved første øyekast ser enkelt ut å kopiere og å montere på sitt eget utstyr. La det med en gang være sagt: ikke alle typer lar seg bygge om. Vær klar over at ofte har fabrikantene flere års testprogram bak sine produkter.

Hva vet du om korrekt lengde, type og fremføring av pilotbånd på et handdeploy system. Eller størrelse og form på pilotskjerm, pilotlomme, type låseløkker og størrelse på maljer som disse skal trekkes gjennom? Det er i tillegg en hel del andre detaljer som også kommer i betraktning.

Produsentene av slike systemer er ennå ikke ferdig med utviklingsarbeidet, men de sitter inne med en mengde erfaring på systemet. ``Har du det?'' Det er et faktum at hoppere som får ``smarte'' ideer, bidrar til å komplisere utstyret sitt ved slike modifiseringer. På grunn av manglende innsikt og erfaring gjøres feil som kunne vært unngått dersom andre erfarne hoppere, evt. produsentene av slikt utstyr hadde vært spurt til råds på forhånd.

Tenk nøye gjennom hvilke forutsetninger som gjelder for sikker bruk av utstyret.

Prøv å forstå produsentenes resonnement bak konklusjon av utstyr. Kanskje er idéen god, men at detaljer nedsetter funksjonssikkerheten. Hvilke feil kan oppstå? Hvor lett kan disse feilene oppdages ved inspeksjon? Hvordan kan de forhindres? Tenk Murphy's lov. Derfor skal alle systemer/prosedyrer konstrueres med tanke på at de skal brukes av idioter.

Utviklingen i dag, med masse moteutstyr er betenkelig. Utviklingen har vist at det mer og mer går i retning av ``100 \% utstyr'', hvor det må utføres perfekte hopp hver gang for å oppnå tilfredsstillende grad av funksjonssikkerhet. Derfor skal du være konservativ og kritisk i ditt valg og anbefalinger av utstyr til andre hoppere. Spesielt gjelder dette utstyr som krever regelmessig utskifting av komponenter (symaskinarbeid).

\subsection{Utstyr – opplæring – sikkerhet – holdninger}
\textbf{Ikke alltid direkte sammenheng mellom popularitet og virkelig forbedring.}

Lag en bedre musefelle...., konstruer en del av et fallskjermutstyr som er nytt, forskjellig og som kan fungere raskere eller er lettere enn andre eksisterende sett. Deretter vil det sannsynligvis bli solgt en mengde av dette. Dessverre er det ikke alltid en direkte sammenheng mellom popularitet og virkelig forbedring, spesielt ikke når forbedringer blir målt med sikkerhet, pålitelighet og ``motstand mot menneskelig feil''... den menneskelige faktor.

Utstyr endrer seg hurtig over tid. I løpet av 10 år har trekksystemet endret seg minst et dusin ganger til ``alminnelig'' BOC i dag. Dette er resultater av høyt utviklede ideer fra kunnskapsrike og oppfinnsomme hoder og har forårsaket gjennomgripende forandringer i tenkemåte og faglig dyktighet hos hopperne med hensyn til det utstyr vi bruker.

Den utviklingsprosess som fant sted fra B-4 overskuddsutstyr til dagens rigger har resultert i noe som kan sammenlignes med Alvin Tofflers ``Fremtidssjokk''. Med andre ord, en mengde hoppere på alle erfaringsnivå og med forskjellig legning for sporten bruker i dag utstyr uten tilstrekkelig kunnskap om:
\begin{itemize}
	\item Hvordan utstyret virkelig fungerer
	\item Hvordan det kan oppstå feilfunksjoner
	\item Hvordan takle feilfunksjoner på bestemte deler av utstyret.
	\item Hvor ømtålig utstyret og deler av det er overfor feil sammensetting og ukyndig bruk.
\end{itemize}

Det siste er det viktigste. I mange tilfeller går hoppernes likegyldighet og uvitenhet så langt at de ikke engang vet å stille de riktige spørsmål eller å forestille seg problemer.

Sagt på en annen måte: Endel utstyr blir utviklet og solgt uten tilstrekkelig hensyn til hvordan det kan brukes feil.

\textbf{En prosess nyskapninger går igjennom før de å presenteres på markedet.}

Før vi tar for oss flere eksempler, tar vi en titt på den prosess nyskapninger trolig går gjennom før de lanseres på markedet:
\begin{itemize}
	\item Oppfinneren får en bedre idé på grunnlag av ting han selv ikke er fornøyd med, eller som han ser rundt seg.
	\item Han leker med idéen ved å konstruere, lage proto-typer eller forandre det utstyr han allerede har.
	\item Han spør seg selv: ``Vil det fungere?'' ``Vil det holde?'' ``Hva skjer om det ikke virker eller ikke holder?'' ``Er det virkelig bedre, lettere, raskere, lettere å bruke eller er det bare annerledes?''
	\item Hvis svarene på hans spørsmål er optimistiske, lager han en prototyp og hopper det. Han tester utstyret ved bruk av ``rimelige'' sikkerhetsforanstaltninger. Utprøvingen foretar han selv eller får hjelp av en godt erfaren venn eller ansatte til prøvehopping, ``fordi disse vet hvordan utstyret skal utvikles.''
	\item Hvis prøveresultatene ser lovende ut, blir utstyret markedsført og holdt øye med, for å se om uforutsette problemer dukker opp, og om hopperne liker systemet.
	\item Mange måneder senere vil kanskje et sikkerhetsutvalg eller en interessert hopper og forfatter lage en artikkel om vedkommende materiell med beskrivelse av forsøksresultatene og hvordan utstyret skal brukes.
\end{itemize}

Denne prosessen, som i detalj er forskjellig i hvert enkelt tilfelle, har en rekke positive og konstruktive sider. Med visse unntak vil vi få utstyr som er:
\begin{itemize}
	\item Sikrere
	\item Penere
	\item Aerodynamisk ``renere''
	\item Komfortabelt
	\item Lettere i vekt
	\item Slitesterkt
	\item Mindre følsomt overfor feilfunksjoner
	\item Billigere (av og til)
\end{itemize}

Men det går ikke alltid etter planen. Hvis utstyr kan brukes feil, settes galt sammen, vil det før eller senere bli gjort.

Det opptrer feil i forbindelse med nytt utstyr. Folk skader seg eller går i bakken. Verst av alt, folk med liten eller ingen kunnskap innenfor de mangesidige forhold i forbindelse med utstyrsfremstilling, eller spesielt, folk uten kjennskap til vitenskapen om den menneskelig oppførsel, forsøker å etterligne nytt utstyr for å spare tid eller penger.

Når en utstyrsprodusent ser en feilfunksjon på utstyr han selv har laget og undersøker årsaken til feilen, hører man ofte: ``du har rigget utstyret feil, det kan ikke gjøres noe med det''. Dermed mister han poenget. Produsenten innrømmer at han lager utstyr testet under ideelle forhold, materiell som fungerer bra uten vind, kulde eller lurvete pakking for å rekke neste løft. Hvis utstyr kan brukes feil, settes galt sammen, ``vil dette før eller siden bli gjort.'' (Jamfør Murphys lover). Dette ligger i menneskets natur.

Folk blir drept fordi de trekker i feil håndtak, i feil rekkefølge, eller rett og slett ikke vet nok om den feilfunksjonen de har over seg.

Folk har trukket reserven fordi de ikke kan finne kula.

Fallskjermhoppere har tredd bryststroppen gjennom utløserhåndtaket. Dette er fort gjort når håndtaket er plassert like ved stroppen.

\textbf{Testhopperen er ikke representativ for den jevne hopper.}

Testing av utstyr hos produsenten eller hos andre erfarne folk som fullt ut forstår systemet er ikke tilstrekkelig da disse ikke utgjør en representativ brukermasse. Det må bestemmes en svakhet overfor menneskelig feil når utstyret blir brukt av mindre erfarne og lite kritiske hoppere som ikke forstår systemet.

Det er ikke noen bedre unnskyldning å velte ansvaret over på en hopper som har gjort en feil, ved bruken av sitt utstyr, enn det er å tilskrive en flyger som med sitt fly under 2. verdenskrig lander med hjulene oppe når flaps og understellshåndtak er fullstendig like og sitter side om side. Henholdsvis fallskjerm og fly virket som tilsiktet, men mulighetene for menneskelig feilhandling var store.

Den menneskelige faktor må tas i betraktning allerede når utstyret utformes på tegnebrettet og må følges opp gjennom utviklingsprosessen, alt fra sjakler til det ferdige produkt.

\textbf{Hvordan forhindre at utstyr blir feil brukt?}

I Norge må alt nytt utstyr typegodkjennes av Sikkerhets og Utdanningskomiteen.

Produsentene kan bedre forholdene gjennom utviklings-prosessen ved å stille seg spørsmålene: ``Hvor sannsynlig er det at en overambisiøs, dårlig opplyst hopper med liten erfaring og under press vil bruks dette utstyret feil? Hvor mye trøbbel vil slike feil skape?'' ``Hvordan bør utstyret utformes for å minimalisere risikoen?''

Brukeren av utstyret kan unngå problemer ved å spørre seg: - Hvilke feil er det ``mulig å gjøre med dette utstyret?''

\begin{itemize}
	\item ``Hvordan kan jeg sjekke dette før det oppstår alvorlige situasjoner?''
	\item ``Hva skjer om jeg likevel bruker det feil?''
	\item ``Er jeg så godt kjent med utstyret mitt at det ikke forstyrrer meg i hva jeg ønsker å utføre i lufta?''
\end{itemize}

Tid som brukes til å tenke igjennom mulige utstyrs-problemer før det skjer, er vel anvendt. Hvis du kan forestille deg en del uvanlige situasjoner, tenke ut korrekte handlingsmønstre for disse og deretter lære deg dem, er sjansene for at du instinktivt vil følge disse prosedyrer, om situasjonen oppstår, store og da med små krav til oppmerksomhet. å bringe dette frem fra underbevisstheten går automatisk. Du har spart mye tid. Det kan vise seg å være en god investering. Svært få cutaway i terminalhastighet ville være vellykte om hopperne hver gang måtte finne ut hva de skulle foreta seg.

Gjennom kommunikasjon formes rykter og gruppemotforestillinger. Om du gjør en feil med utstyret ditt, om det svikter eller du ser en mulig feil med det: innrøm det, dokumenter det. Færre rykter ute av kontroll! Gi beskjed til produsenten og din organisasjon. Hjelp andre med å unngå problemet, husk at det kanskje ikke er et tåpelig tilfelle eller en tilfeldig hendelse. Du kan kanskje også delvis ha blitt hengt ut for det. Etter at andre har vært borti det samme problemet er det lettere å snakke om det.

\textbf{Nye løsninger utvikler sporten, men det må ikke skje på bekostning av sikkerheten.}

Så la oss fortsatt finne frem til nye løsninger, forandre og forbedre utstyret vi bruker. De som utfører dette bidrar i stor grad til å heve sportens nivå: vi er dem stor takk skyldig. Men utviklingen er ikke så likefrem som det kan synes. Det må ikke foregå på bekostning av sikkerheten.
\begin{description}
	\item[MURPHY'S LOV] If anything can go wrong, it will....... And at the worst possible moment.
	\item[IDIOTREGELEN] Prosedyrer og utstyr skal konstrueres med tanke på at idioter skal bruke den. Kun enkle systemer hvor mulige feilkilder er eliminert kan anbefales.
	\item[VÆR NYSGJERRIG.] Ha en utilfredsstilt nysgjerrighet til å skaffe deg mer kunnskap og erfaring.
\end{description}

\section{Rettelser og tillegg}
Forslag til endringer, tillegg og rettelser til Materiellhåndboken sendes

Materiellhåndboka \\
v/ F/NLF \\
Postboks 383 - Sentrum \\
0102 OSLO \\

Fax: 23 01 04 51 \\
e-mail fnlf@nak.no

Endringer tillegg og rettelser til Materiellhåndboken vil av hensyn til rask bekjentgjøring bli meddelt gjennom SU referater og eventuelt Fritt Fall. For fortløpende registrering av tillegg og endringer ajourføres nedenstående oversikt etter hvert som nye eller reviderte punkter blir kunngjort.

Formell oppdatering vil bli utført ved løpende utgivelse av løse blad. Nye og endrede punkter vil i disse være merket slik at de lettere skal kunne identifiseres.

Utgitte modifikasjon– og serviceordre som blir utstedt mellom oppdateringer av Materiellhåndboka plasseres fortløpende i Kapittel 10.2 – Modifikasjon og serviceordre fra F/NLF .
