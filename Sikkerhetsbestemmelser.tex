\part{Sikkerhetsbestemmelser}
\setcounter{section}{99}

\section{Definisjoner}
Forkortelser, ord og uttrykk i disse bestemmelser forstås som følger:

\subsection{Forkortelser}
\begin{description}
	\item[AFF]	Accelerated Free Fall - grunnkurs med fritt fall fra 1. hopp (normalt fra 10-13 000 fot).
	\item[AGL]	Altitude above Ground Level - virkelig høyde over bakken.
	\item[AIC]	Aeronautical Information Cirriculum - informasjonssirkulære utgitt av Avinor og Luftfartstilsynet.
	\item[AIP]	Aeronautical Information Publication - informasjonssystem utgitt av Avinor til abonnenter.
	\item[BSL]	Bestemmelser for Sivil Luftfart.
	\item[CF]	Canopy Formation – kalottformasjonshopping (erstatter CRW).
	\item[CTR]	ConTRol zone - kontrollsone, normalt med radius 8 nm fra en flyplass kontrolltårn, med høyde fra bakken opp til en nærmere angitt høyde. Andre utstrekninger kan forekomme. Alle CTR er angitt i AIP Norge.
	\item[DK]	F/NLFs DommerKomite.
	\item[F/NLF]	Fallskjermseksjonen/Norges Luftsportsforbund.
	\item[FL]	Flight Level - flygenivå angitt i 100-fots inndeling, bestemt ved standard atmosfæretrykk (1013.25 hp (hektopascal) = 1 ATA). FL 65 = 6500 fot, FL 100 = 10 000 fot osv.
	\item[FMI]	FlyMedisinsk Institutt.
	\item[FS]	Formation Skydiving - fritt fall formasjonshopping.
	\item[HFL]	HoppFeltLeder.
	\item[HI]	HovedInstruktør.
	\item[HL]	HoppLeder.
	\item[HM]	HoppMester.
	\item[I]	Instruktør (alle kategorier instruktører).
	\item[I-1]	Instruktør 1.
	\item[I-2]	Instruktør 2.
	\item[I-3]	Instruktør 3 / Hoppmester.
	\item[I-2 AFF]	AFF-Instruktør 2.
	\item[I-3 AFF]	AFF-Instruktør 3 / AFF-Hoppmester.
	\item[I-E]	Instruktør/Eksaminator.
	\item[I-T]	Instruktør Tandem
	\item[MK]	MateriellKontrollør.
	\item[MR]	MateriellReparatør.
	\item[MSJ]	F/NLFs MateriellSJef
	\item[MSL]	altitude above Mean Sea Level - høyde over midlere havflatenivå.
	\item[NAK/NLF]	Norges Luftsportsforbund.
	\item[NIF]	Norges IdrettsForbund.
	\item[nm]	nautisk mil = 1852 m.
	\item[NOTAM]	NOtice To AirMen - publikasjon utgitt av Avinor, NOTAM kl 1 og kl 2: distribueres via Avinors internettjeneste, Air Information Service Norway – Internet Pilot Planning Centre (www.avinor.no).
	\item[OT]	OperasjonsTillatelse iht Del 500.
	\item[OT-1]	OperasjonsTillatelse klasse 1.
	\item[OT-2]	OperasjonsTillatelse klasse 2.
	\item[SU]	F/NLFs Sikkerhets- og utdanningskomité.
	\item[TMA]	TerMinal Area - terminalområde rundt en kontrollert flyplass, med utstrekning og høydebegrensning som angitt i AIP Norge.
	\item[VFS]	Vertical Formation Skydiving – fritt fall formasjonshopping hvor kroppen er vertikalt orientert, dvs ”head up” eller ”head down”.
	\item[XT]	eXperiment Tillatelse iht Del 500. 100.2 UTSTYR
\end{description}

\subsection{Utstyr}
\subsubsection{Fallskjerm}
Sammenleggbar ikke-avstivet innretning av duk og liner, som kan foldes ut under fritt fall, og som bremser en persons eller gjenstands frie fall. Fallskjermer inndeles i kategorier som spesifisert i Del 300.

\subsubsection{Hovedskjerm}
Fallskjerm som er festet til seletøyet med 3-rings frigjøringssystem, som skal forsøkes brukt under alle fallskjermhopp.

\paragraph{Lineutløst skjerm}
En hovedskjerm som utløses ved hjelp av en line festet til luftfartøyet og som aktiviserer skjermen uten medvirkning fra hopperens side.

\subsubsection{Reserveskjerm}
Fallskjerm som ikke kan frigjøres fra seletøyet under bruk, som kun skal anvendes i nødsituasjon.

\subsubsection{Fallskjermsett}
En komplett rigg (hovedskjerm, reserveskjerm, pakksekk, seletøy og evt nødåpner) hvor begge skjermer er ryggmontert.

\subsection{Fallskjermhopp}
Alle planlagte utsprang fra luftfartøy i den hensikt å anvende fallskjerm under hele eller deler av nedstigningen.

\subsubsection{Treningshopp}
Fallskjermhopp utført av selvstendig hopper under ordinær hoppvirksomhet.

\subsubsection{Utdanningshopp}
Fallskjermhopp utført i samsvar med særskilt opplæringsprogram som inngår i Del 600.

\subsubsection{Tandemhopp}
Fallskjermhopp med to hoppere i samme spesialkonstruerte fallskjermsett, hvor den ene er tandeminstruktør og den andre er tandemelev.

\subsubsection{Konkurransehopp}
Fallskjermhopp utført i samsvar med F/NLFs konkurransehåndbok.

\subsubsection{Oppvisningshopp}
Fallskjermhopp utført for publikum. Hopping på hoppfelt godkjent for alminnelig treningshopping med tilstedeværelse av publikum anses ikke som fallskjermoppvisning etter disse bestemmelsene.

\subsubsection{Vannhopp}
Fallskjermhopp der det er planlagt å lande i vann dypere enn 1 meter.

\subsubsection{Natthopp}
Fallskjermhopp som utføres i tidsrommet mellom borgerlig skumring og borgerlig demring (solen 6 grader under horisonten) som angitt i almanakken for gjeldende dag (Civil Twilight).

\subsubsection{Oksygenhopp}
Fallskjermhopp fra høyder over 15 000 fot MSL.

\subsubsection{Praktisk hoppvirksomhet}
All virksomhet som inkluderer fallskjermhopping, dvs trenings-, utdannings- og konkurransehopping, samt alle spesielle hopptyper.

\subsubsection{Alminnelig hopping}
Hopping omfattende treningshopp, utdanningshopp og konkurransehopp.

\subsubsection{Spesielle hopptyper}
Tandemhopp, oppvisningshopp, vannhopp, natthopp og oksygenhopp.

Slep etter fly og hopping med to hovedskjermer regnes også som spesielle hopptyper. Disse hopptypene er ikke tillatt utført uten gyldig eksperimenttillatelse for denne typen hopping.

\subsection{Lisens}
Som lisens regnes:

\subsubsection{Fallskjermlisens utstedt av F/NLF}
Fallskjermhoppere som er norske statsborgere og bosatt i Norge skal ha gyldige norske lisenser utstedt av NAK/NLF for å delta i hopping i regi av F/NLF.

\subsubsection{FAI-lisens}
Gyldig utenlandsk fallskjermlisens godkjent i henhold til FAI Sporting Code Section 5. Innehaveren gis rettigheter i henhold til F/NLFs bestemmelser Del 300 etter FAIs til enhver tid gjeldende krav til ``International Certificates of Proficiency''.

\subsubsection{Annen utenlandsk dokumentasjon}
Gyldig fallskjermlisens eller -instruktørlisens utstedt av utenlandsk nasjonal offentlig myndighet eller National Air Sport Control (NAC).

\subsubsection{Utenlandske hopperes rettigheter}
Hoppere med FAI-lisens eller annen utenlandsk dokumentasjon gis rettigheter i henhold til F/NLFs Bestemmelser Del 300.

Hoppere med utenlandsk instruktørlisens kan gis begrensede rettigheter som beskrevet i Del 400.

Norske hoppere med dublert statsborgerskap eller som har fast bostedsadresse i utlandet, og kan dokumentere dette med flyttebevis fra Folkeregisteret, gis rettigheter etter FAI-lisens eller annen utenlandsk dokumentasjon i henhold til F/NLFs Bestemmelser Del 300.

\subsection{Hopperstatus}
\subsubsection{Elev}
Innehaver av gyldig norsk elevbevis i henhold til F/NLFs Bestemmelser Del 300.

\paragraph{Tandemelev}
Person som oppfyller krav til tandemelev iht pkt 103.5.2.

\subsubsection{Selvsetndig hopper}
Innehaver av gyldig norsk A-lisens eller høyere i henhold til F/NLFs Bestemmelser Del 300, eller tilsvarende FAI lisens.

\subsubsection{Instruktør}
Innehaver av gyldig norsk instruktørlisens i henhold til F/NLFs Bestemmelser Del 400.

\subsection{Luftfartøy}
Ethvert apparat som kan holdes oppe i atmosfæren ved reaksjoner fra luften, dog ikke ved reaksjoner av luft mot jordoverflaten.

\subsection{Høyde}
\subsubsection{Åpningshøyde}
Vertikal avstand til bakken på åpningspunktet. Ved kupert terreng regnes høyden over høyeste punkt innen 500 m radius fra det beregnede åpningspunkt. Unntatt fra dette er høydeangivelser der betegnelsen MSL (Mean Sea Level = midlere havflatenivå) er anvendt. I så fall forstås høyden som høyde over havflatenivå.

\subsubsection{Utsprangshøyde}
Den høyde der fallskjermhopperen forlater luftfartøyet.

\subsubsection{Trekkhøyde}
Den høyde der skjermen aktiveres.

\subsection{Vind}
\subsubsection{Bakkevind}
Vindhastigheten målt 3-7 m over bakkenivå innenfor hoppfeltet. Styrken måles som sterkeste vindkast innenfor en 10 minutters periode.

\subsubsection{Middelvind}
Vindhastigheten som gjennomsnittet av luftstrømmene fra hoppfeltets nivå opp til 2 000 fots høyde. Middelvinden angis som driverens avdrift i distanse.

\subsubsection{Høydevind}
Vindhastigheten over 2 000 fots høyde. Normalt fastsettes høydevinden på grunnlag av sonderapporter fra meteorologiske stasjoner, eller ved observasjoner under flyging.

\subsection{Hoppfelt}
Et område som omfatter så vel landingsområde som dets omgivelser i den utstrekning det må ventes at hoppere vil kunne komme til å lande der.

\subsubsection{Landingsområde}
Den del av hoppfeltet der hopperne forutsettes å lande.

\subsubsection{Åpningspunkt}
Det punktet i terrenget som det er planlagt at hopperen skal befinne seg over ved skjermåpning.

\subsection{Måleenheter}
Under gjennomføring av fallskjermhopping skal alltid følgende måleenheter anvendes:

\subsubsection{Høydeangivelser}
FOT. Ved eventuelle omregninger settes 1 fot lik 0.3 m og 1 m lik 3.3 fot.

\subsubsection{Distanseangivelser}
METER.

\subsubsection{Vindstyrke}
KNOTS. Ved omregning anvendes 1 knot lik 0.5 m/sek og 1 m/sek lik 2 knots.

\section{Generelt}
\subsection{Hjemmel}
Disse bestemmelser er utarbeidet og blir revidert under ledelse av F/NLF v/SU, med hjemmel i Bestemmelser for Sivil Luftfart (BSL) D 4-2, pkt 2.1 og 3.2.

\subsection{Gyldigetsområde}
Disse bestemmelser gjelder for alle fallskjermhopp som utføres innen F/NLF med tilsluttede klubber.

Unntak:

Bestemmelsene kan fravikes dersom det er nødvendig som følge av at:
\begin{itemize}
	\item Hoppene utføres som nødutsprang på grunn av situasjon oppstått med luftfartøyet.
	\item Hoppene utføres i forbindelse med offentlig organisert hjelpe- og redningsarbeide.
	\item Hoppene utføres som eksperiment- eller forsøksvirksomhet etter eksperimenttillatelse (XT) iht Del 500.
\end{itemize}

\subsection{Revisjon}
Disse bestemmelser revideres under ledelse av SU.

Vedtak kunngjort i SUs komitéreferater er forøvrig å anse som gjeldende bestemmelser fra dato angitt i referatet.

\subsection{Operasjonstillatelse}
Praktisk hoppvirksomhet og teoretisk utdanning kan kun organiseres og gjennomføres av klubb, organisasjon eller person som innehar operasjonstillatelse fra F/NLF i henhold til Del 500.

\subsection{Organisering}
All hopping som utføres i henhold til disse bestemmelser, skal være organisert og ledet i henhold til F/NLFs bestemmelser Del 500.

\subsection{Materiell}
Alt materiell som anvendes i forbindelse med hopping etter disse bestemmelser, skal samsvare med F/NLFs bestemmelser Del 200 og F/NLFs Materiellhåndbok.

\subsection{Overholdelse/Disiplinærtiltak}
Alle som deltar i gjennomføringen av fallskjermhopp etter disse bestemmelser plikter å kjenne de deler av bestemmelsene som angår vedkommendes oppgaver, og plikter å overholde dem på alle punkter.

I tillegg skal de offentlige bestemmelser som til enhver tid måtte gjelde vedrørende fallskjermhopping overholdes.

Brudd på bestemmelsene kan medføre muntlig eller skriftlig advarsel, inndragning av lisenser for kortere eller lengre tid eller for alltid, etter avgjørelse av Sikkerhets- og utdanningskomiteen F/NLF eller organer utpekt av denne.

\subsection{Midlertidige reaksjoner grounding/bortvisning}
\subsubsection{Brudd på gjeldende bestemmelser}
Dersom en hopper gjennom brudd på gjeldene bestemmelser utsetter seg selv eller en annen person for uhell eller ulykke, eller dersom uhell eller ulykke av samme årsak nært hadde skjedd, skal hopperen ilegges midlertidig hoppforbud, inntil vedkommendes skikkethet er vurdert av egen Hovedinstruktør. Hovedinstruktør fatter vedtak om eventuelt forlenget hoppforbud. Avgjørelsen kan ankes inn for Sikkerhets- og utdanningskomiteen.

Dersom det er behov for permanent hoppforbud fremmes dette for avgjørelse i SU.

\subsubsection{Midlertidige reaksjoner fra hoppleder}
Enhver fallskjermhopper, uansett status, lisens og lisens som forsettlig eller uaktsomt handler i strid med disse bestemmelser, den operative ledelses vurdering mhp sikkerhet, offentlige vedtekter og/eller lokale bestemmelser på hoppfeltet, kan gis midlertidig hoppforbud og/eller bortvises fra feltet av ansvarlig Hoppleder, ref Del 500 og 600.

\section{Alminnelige bestemmelser}
Disse alminnelige bestemmelser gjelder for alle fallskjermhopp under alle forhold, unntatt der det er gitt egne bestemmelser for spesielle typer hopp.

\subsection{Aldersgrense}
Ingen tillates å gjennomføre fallskjermhopp eller påbegynne fallskjermutdannelse før vedkommende er fylt 16 år. Personer som ikke er fylt 18 år skal ha skriftlig tillatelse fra foresatte.

\subsection{Lisenser/beviser}
Ingen tillates å gjennomføre fallskjermhopp uten å ha gyldig elevbevis eller lisens, iht. bestemmelsene i Del 300.

\subsection{Fysisk/psykisk tilstand}
Ingen skal gjennomføre eller tillate gjennomført fallskjermhopp av noen som åpenbart er påvirket av alkohol (ikke edru) eller annet berusende eller bedøvende middel eller på grunn av sykdom, legemidler, tretthet eller lignende årsak er uskikket til å hoppe på en trygg måte. Den som deltar i fallskjermhopping, må ikke ha en alkoholkonsentrasjon i blodet som overstiger 0,2 promille.

\subsection{Utstyr}
\subsubsection{Fallskjermer}
Enhver som utfører fallskjermhopp skal være utstyrt med fallskjermsett, godkjent og kontrollert i henhold til krav satt i Del 200.

\paragraph{Vingbelastning}
Forholdet mellom hopperens vekt ved utsprang fra luftfartøy og størrelsen på hovedskjerm skal ikke overstige:

\begin{itemize}
	\item 0,95 hhv 1,0 lb/sqft for elever
\end{itemize}

Førstegangshoppere og uerfarne elever skal ha en lavere vingbelastning enn 0,95. Erfarne elever som hopper 0P/hybrid kan overstige 0,95 men aldri 1.0 i vingebelastning. Vekt på utstyret og bekledning som skal legges til kroppsvekten skal være 15 kg for hopping med elevutstyr og 12 kg for hopping med sportsutstyr.

\begin{itemize}
	\item 1,1 lb/sqft for hoppere med mindre enn 200 hopp
	\item 1,3 lb/sqft for hoppere med mindre enn 350 hopp
	\item 1,6 lb/sqft for hoppere med mindre enn 500 hopp
\end{itemize}

\subsubsection{Hjelm}
Hoppere skal alltid være utstyrt med hjelm.

Fallskjermhjelm skal beskytte hodet. Under hopping kan følgende hjelmer nyttes:
\begin{itemize}
	\item Hjelmer spesielt produsert for fallskjermhopping.
	\item Sports- og motorsykkelhjelmer etter godkjenning av Instruktør 1.
\end{itemize}

Elever skal benytte hjelm med hardt skall.

Hjelm med påmontert videokamera og/eller fotografiapparat skal godkjennes for bruk av HI.

\subsubsection{Påkledning}
Hopper skal benytte egnet bekledning i samsvar med hopptype, kvalifikasjoner, flytype som benyttes og temperaturforhold.

\paragraph{Påkledning elever}
Elever skal ha heldekkende hoppdress/bekledning (type kjeledress) av størrelse og egenskaper som ikke medfører fare for ustabilitet eller kan være til hinder for trekk, skjermåpning eller -utvikling.

\subsubsection{Hansker}
Elever skal benytte hansker godkjent av Instruktør. Strikkede og glatte vanter godkjennes ikke.

\subsubsection{Fottøy}
Fottøy for elever og tandemelever skal støtte opp mot ankelleddet. For elever skal eventuelle hemper for lissene, som kan medføre opphengning av duk eller bæreliner, dekkes med tape eller tilsvarende.

\subsubsection{Høydemåler}
Hopper skal under alle hopp være utstyrt med visuell høydemåler. Visuell høydemåler skal være produsert spesielt for fallskjermhopping. Akustisk høydevarsler kan supplere, men ikke erstatte høydemåler. Høydemåler skal være montert slik at den er fullt leselig under hele hoppet, og slik at den ikke er til hinder for skjermens aktivering eller gjennomføring av nødprosedyre.

Planlagt vannhopp med utsprang under 5 000 fot kan gjennomføres uten høydemåler.

\subsubsection{Nødåpner}
Nødåpner er påbudt ved hopping i Norge fra 1. april 2010. HI kan gi særskilt tillatelse til ikke å benytte nødåpner ved visse typer hopping som for eksempel vannhopp mfl. Dette skal ikke være av generell- og ubegrenset karakter. Det skal være dokumentert

\subsubsection{Modifikasjoner og ekstrautstyr}
Det er ikke tillatt å anvende spesielt modifisert utstyr eller tilleggsutstyr for det formål å styre eller bremse i fritt fall, dersom utstyret kan medføre risiko for ustabilitet, hindre skjermutløsning eller skjermens funksjon.

Enhver form for vinger med spilearrangement er forbudt.

Gjenstander som er konstruert og produsert for fritt fall, som vingedress, brett eller lignende, tillates bare når det kan dokumenteres bestått godkjent opplæring etter SUs bestemmelser.

\subsection{Utsprangs- og trekkhøyder}
\subsubsection{Trekkhøyde ved utsprang over 2 500 fot}
Trekk skal utføres slik at hopper ved normalt åpningsforløp vil oppnå flygende hovedskjerm 2 000 fot AGL. Minste tillatte vertikal avstand til bakken under fritt fall er 2500 fot AGL. Se også pkt 100.7.1

\subsubsection{Under 2500 fot}
Ved utsprangshøyder 2 500 fot eller lavere, skal trekket utføres innen 2 sek etter utspranget.

\subsubsection{Laveste høyde for elever / elevutstyr}
For hopp på progresjonsplan etter Grunnkurs del II er laveste tillatte utsprangshøyde 3 500 fot, og laveste trekkhøyde 3 000 fot. Laveste tillatte utsprangshøyde for linehopp er 3 000 fot.

For hopp på progresjonsplanen etter Grunnkurs AFF, nivå 1 til 7 er laveste tillatte utsprangshøyde 9 000 fot.

Denne bestemmelsen gjelder alle tilfeller der elevfallskjermsett med nødåpner type FXC 12000 benyttes, uansett hopperens erfaringsnivå og lisens, og uavhengig av om nødåpner er på/av.

\subsubsection{Laveste utsprangshøyde for selvstendige hoppere}
Laveste tillatte utsprangshøyde er 1 500 fot. Er flyhastigheten ved utspranget lavere enn 60 knots, er laveste utsprangshøyde 1 800 fot.

\subsubsection{Høydebegrensning}
Alminnelig fallskjermhopping skal ikke utføres fra større høyder enn 15 000 fot MSL. Fallskjermhopp fra større høyder enn 15 000 fot MSL skal utføres iht bestemmelsene om oksygenhopp.

Ved utsprang mellom 13 000 fot og 15 000 fot MSL er maksimalt tillatt eksponeringstid over 13 000 fot er 10 minutter.

\subsection{Vindgrenser}
\subsubsection{Generell grense}
Fallskjermhopp skal ikke utføres dersom bakkevinden overstiger 22 knots.

\subsubsection{Grense for elever}
For elever er bakkevindsgrensen 14 knots. Elever skal ikke tillates å hoppe i større middelvindstyrke enn 25 knots (1 250 m driverdistanse).

Særlig erfarne elever kan av Hoppleder tillates å hoppe ved bakkevind inntil 18 knots og middelvindstyrke over 25 knots.

\subsubsection{Vindgrense for hoppere med rund reserve}
For hoppere utstyrt med rund reserve er bakkevindgrensen 18 knots.

\subsection{Kontroll av vindforhold}
Middelvindens retning og styrke skal alltid fastslås på en av følgende måter:

\subsubsection{Driver}
Med vinddriver bestående av en krepp-papirremse på 25 x 625 cm, med en vekt på 30 gram i den ene enden, sluppet fra den planlagte åpningshøyden. Eventuelt kan annen tilsvarende innretning benyttes. Driver eller annen innretning skal ha en synkehastighet på 20 fot pr sek, (100 sek fra 2 000 fot).

\subsubsection{Prøvehopp}
Ved prøvehopp utført av en hopper med B-lisens eller høyere, eller med tilsvarende erfaring i styring og landing. Denne metode skal kun anvendes når det med rimelig sikkerhet kan antas at hopperen vil lande innenfor hoppfeltet.

\subsubsection{Værtjeneste}
Ved opplysninger fra værtjeneste eller flygerinformasjonsenhet.

\subsubsection{Hoppflyets GPS-system}
Ved opplysninger fra hoppflyets GPS.

\subsubsection{Overførte opplysninger}
Ved å anvende opplysninger fra en av de forannevnte kilder vedrørende et sted i nærheten, og overføre disse til hoppfeltet, dersom det med rimelig sikkerhet kan antas at forholdene begge steder er like.

\subsection{Hoppfelt for alminnelig hopping}
Hoppfelt som skal anvendes for alminnelig fallskjermhopping skal tilfredsstille følgende krav:

\subsubsection{Omfang}
Hoppfeltet skal ha en diameter på minst 200 m, der det ikke skal finnes vesentlige faremomenter for hopperne.

Trær, skog, gjerder ol regnes ikke som vesentlige faremomenter, dog bør ikke større skogfelter forekomme.

\paragraph{Hoppfelt for hoppere med A-sert. og høyere status}
Treningshoppfelt for hoppere med A-lisens og høyere skal ha en diameter på minst 150 m, med et landingsområde som angitt i pkt. 102.8.2. Hoppfeltets beskaffenhet skal være som pkt. 102.8.1.

\paragraph{Hoppfelt for D-hoppere}
Hoppfelt med begrenset hopping utført av D-hoppere godkjennes av HI og samme krav til landingsområde som for oppvisningshopp gjelder.

\subsubsection{Landingsområde}
Innenfor hoppfeltet, i det området hopperne forutsettes å lande, skal det være et landingsområde med minst 50 m diameter der marken skal være tilnærmet horisontal med jevn overflate, uten grøfter, diker, ledninger, trær, større busker, stolper eller staur. Master for vindpølse og vindmåler regnes ikke som hindringer.

Marken bør være gress, jord, sand eller singel. Fast dekke (asfalt, betong etc) kan tillates innenfor landingsområdet i mindre utstrekning.

\subsubsection{Vann}
Redningsmateriell som ved vannhopp skal forefinnes ved hopping dersom vann dypere enn 1 m finnes innenfor 1 000 m fra beregnet landingspunkt.

Redningsvest godkjent iht Del 200 skal benyttes dersom åpent vann, innsjø eller sjø dypere enn 1 m finnes innenfor:

\paragraph{Elever, hoppere med A-lisens og hoppere med rund reserveskjerm}
1 000 m fra beregnet landingspunkt. Elever tillates ikke å benytte oppblåsbar redningsvest uten trykkpatron.

\paragraph{Hoppere med B- til og med D-lisens og for tandemhopping}
250 m fra beregnet landingspunkt.

\paragraph{Hoppere med demolisens}
250 m fra beregnet landingspunkt, og korteste utstrekning på vannet (fra bredd til bredd) innenfor området er lengre enn 100 meter.

\subsection{Signalsystem fra bakke til fly}
Ved all fallskjermhopping skal det være et signalsystem mellom bakketjenesten og luftfartøyet. Dette kan skje ved en av følgende metoder:

\subsubsection{Jordtegn}
Jordtegn lagt ut på en slik måte at det klart kan sees fra flyet. Midlertidig hoppforbud markeres ved at tegnet legges som en linje. Hoppforbud markeres ved at hele tegnet tas inn.

\subsubsection{Radio}
Ved 2-veis radioforbindelse med flyet.

\subsubsection{Telefon/radio}
Ved telefonforbindelse direkte fra hoppfeltet til den flygekontrollenhet som flyet har forbindelse med.

\subsection{Minimum sikkerhetsutstyr}
På hoppfeltet ved alminnelig fallskjermhopping og under tandemhopp, vannhopp, natthopp og oksygenhopp skal følgende minimum sikkerhetsutstyr være tilstede:
\begin{itemize}
	\item Båre for transport av skadede
	\item Telefonforbindelse
	\item Førstehjelpsutstyr som omfatter støttebandasjer, isposer, armbind, tape, plaster.
	\item Vindindikator (vindpølse, kreppapir, flagg eller annen innretning som viser vind og styrke).
\end{itemize}

\subsection{Fritt fall formasjonshopping (FS/VFS)}
Ved alle hopp der to eller flere hoppere har til hensikt å oppnå kontakt eller fly ``non contact'' i fritt fall, gjelder følgende bestemmelser:

\subsubsection{Høydebegrensning}
Det er ikke tillatt å ha kontakt mellom hoppere i fritt fall under 3 500 fot. Ved separasjon skal hopperne fjerne seg fra hverandre tilstrekkelig til å unngå kollisjon under eller etter skjermåpning.

\subsubsection{Vikeplikt}
Under alle manøvrer i fritt fall har overliggende hopper vikeplikt, og det påligger overliggende hopper å påse at han ikke er i en posisjon som kan medføre fare i forhold til underliggende hoppere.
\begin{itemize}
	\item Før trekk skal hopperne gi tegn ved å vifte med begge hender (wave off) i ca. 2-3 sek., slik at overliggende hoppere skal kunne fjerne seg.
\end{itemize}

\subsubsection{Kvalifikasjoner}
Formasjonshopping tillates ikke utført av hoppere med elevbevis. Unntatt fra dette er opplæring i formasjonshopping slik denne er beskrevet i del 600 med vedlegg. For hoppere med A-lisens gjelder begrensninger anført i del 300.

\subsubsection{Opplæring}
Følgende bestemmelser gjelder for opplæring i formasjonshopping:
\begin{itemize}
	\item Teoretisk innføring i sikkerhetsmessige forhold og orientering om elementære grunnregler for formasjonshopping skal gjennomgås før første formasjonshopp.
	\item Praktisk opplæring skal gjennomføres sammen med instruktør.
\end{itemize}

Godkjenning for gjennomført teoretisk og praktisk utsjekk skal være attestert for i hopperens loggbok.

\subsubsection{Planlegging}
All formasjonshopping skal være planlagt på forhånd.

\subsection{Kalott formasjonshopping (CF)}
Ved alle hopp der to eller flere hoppere har til hensikt å oppnå kontakt eller fly ``non contact'' etter skjermåpning, skal følgende bestemmelser etterfølges:

\subsubsection{Høydebegrensning}
Det er ikke tillatt å forsøke å oppnå kontakt (docking) mellom to hoppere i skjerm under 1 500 fot. Ved kalottformasjoner med flere enn to hoppere, er minstehøyden 2 500 fot.

Under opplæring i CF er minstehøyden uansett 2 500 fot.

\subsubsection{Utstyr}
Alle former knuter i lineforgreiningene (linekaskadene)er forbudt ved CF- hopping.
\begin{itemize}
	\item Kniv er påbudt i forbindelse med CF-hopping.
\end{itemize}

\subsubsection{Begrensninger}
CF, eller forsøk på CF, er forbudt i alle tilfeller hvor den vertikale og/eller den horisontale sikt er redusert på grunn av skyer, tåke ol.

Natt-CF og CF med tandemekvipasjer er forbudt.

\subsubsection{Opplæring}
Følgende bestemmelser gjelder for opplæring i CF:
\begin{itemize}
	\item Minimum 150 hopp
	\item gjennomgått F/NLFs teorikurs
	\item avlagt prøve overfor Instruktør 1 (Ref Del 600).
\end{itemize}

Godkjenning for gjennomgått teorikurs skal være attestert for i hopperens loggbok.

\subsubsection{Planlegging}
Alle CF-hopp skal være planlagt på forhånd.

\subsection{Spesielle hopptyper}
Fallskjermhopping som regnes som spesielle hopp skal utføres i samsvar med punkt 103.

\section{Spesielle bestemmelser}
\subsection{Bestemmelser for oksygenhopp}
\subsubsection{Over 15 000 fot MSL}
Utsprang fra høyder over 15 000 fot betinger bruk av oksygenutstyr og personell, bestemmelser og opplegg skal godkjennes av SU i hvert enkelt tilfelle.

\subsection{Bestemmelser for natthopp}
Ved all fallskjermhopping som utføres mellom borgerlig skumring og borgerlig demring som angitt i almanakken, skal følgende regler etterfølges:

\subsubsection{Utstyr}
Hopperne skal være utstyrt med følgende tilleggsutstyr:
\begin{itemize}
	\item Markeringslys festet til hopperens ben som avgir nok lys til at hopperen kan sees av andre hoppere i luften. Lyset skal være montert slik at den lyser i en sfæresektor på minst 210 grader. Kjemisk lys kan benyttes.
	\item Instrumentbelysning med tilstrekkelig lysstyrke til at høydemåler kan leses tydelig i fritt fall.
	\item En lykt som hopperen skal kunne lyse opp skjermen tilstrekkelig til at hopperen kan kontrollere kalotten etter åpning.
	\item Dersom hopperen skal anvende briller, skal disse ha klart glass. Fargede briller er ikke tillatt ved natthopp.
\end{itemize}

\subsubsection{Merking av landingsområde}
Landingsområde skal ved natthopp være markert med lys på en slik måte at det kan skilles fra eventuelle andre lys i nærheten. Hoppforbud angis ved at disse lys slukkes.

\subsubsection{Rapportering}
Etter landing skal hopperne uten unødig opphold rapportere til HFL. Hoppere som lander langt fra landingsområdet skal straks signalisere til HFL for å angi sin posisjon og tilstand.

\subsubsection{Kvalifikasjoner}
Natthopp tillates bare utført av hoppere med C-lisens eller høyere. Hoppere skal før de tillates å utføre natthopp ha gjennomgått teorikurs og avlagt prøve overfor Instruktør 1.

Godkjenning for gjennomført teorikurs skal være attestert for i hopperens loggbok.

\subsection{Bestemmelser for vannhopp}
For alle fallskjermhopp der det er planlagt landing i vann dypere enn 1 m skal følgende bestemmelser etterfølges:

\subsubsection{Utstyr}
Hopperne skal være utstyrt med følgende tilleggsutstyr:
\begin{itemize}
	\item Flytevest godkjent iht. Del 200.
	\item Ved anvendelse av drakter/flyteplagg med oppdrift større eller lik 50 Newton kan flytevest sløyfes. Faste og/eller oppblåsbare flytemidler kun festet til armer eller ben kan supplere, men ikke erstatte, flytevest.
	\item Kniv festet slik at den ikke forsvinner med frigjort seletøy.
\end{itemize}

Det er ikke tillatt å hoppe med blyvest, blyinnlegg i seletøy eller andre ekstra vekter.

\subsubsection{Redningsmateriell}
Ved landingsområdet skal det finnes egnet båt med en bemanning på minst 2, hvorav minst en skal være fallskjermkyndig. Båten(e) skal ha motor og være på vannet med bemanning så lenge hoppingen pågår.

Kniv skal være tilgjengelig i båten.

Personell med førstehjelpskurs som omfatter gjenopplivning skal være tilstede i redningsbåten(e).

Ved planlagt vannhopp skal det være en båt pr hopper som skal lande i vannet samtidig. Det er ikke tillatt å droppe flere hoppere før de foregående alle er plukket opp. Hver båt skal være utrustet iht dette punktet.

\subsubsection{Utdannings-vannhopp}
Vannhopp i utdanningsøyemed tillates ikke utført med fallskjermutstyr utstyrt med:
\begin{itemize}
	\item Nødåpner
	\item RSL/LOR-bånd som ikke kan frikobles.
\end{itemize}

\subsubsection{Kvalifikasjoner}
Vannhopp tillates bare utført av hoppere med minst B-lisens. Hoppere skal, før de tillates å utføre vannhopp, ha gjennomgått teorikurs og avlagt prøve overfor Instruktør 1.

Godkjenning for gjennomført teorikurs skal være attestert for i hopperens loggbok.

\subsection{Bestemmelser for oppvisningshopp}
Ved alle fallskjermhopp som utføres i forbindelse med fallskjermoppvisning, skal følgende bestemmelser etterfølges:

\subsubsection{Demolisens}
I tillegg til C- eller D-lisens skal alle deltagende hoppere ha gyldig demolisens av klasse som kvalifiserer til det oppdrag som skal gjennomføres.

\subsubsection{Utstyr}
Ved oppvisning skal hopperne være utstyrt med firkant reserveskjerm (Kategori 3).

Følgende tilleggsutstyr kan anvendes:
\begin{itemize}
	\item Røykbokser, såfremt disse ikke er av type som brenner i stykker under bruken. Boksen skal festes på en slik måte at den ikke kan løsne utilsiktet.
	\item Flagg med vekt eller lignende gjenstander hengende under hopperen. Hopperen er ansvarlig for at flagg med flaggline er egnet til formålet og forsvarlig festet. Hopperen er videre ansvarlig for at hengende gjenstand ikke treffer personer eller gjenstander ved landing. Hengende utstyr skal festes slik at det raskt skal kunne frigjøres med en hånd.
	\item Flagg med vekt skal plasseres innenfor hoppdressen eller i fastspent pose. Plasseringen skal være slik at den ikke hindrer fallskjermutstyrets funksjon.
\end{itemize}

\subsubsection{Slipp av gjenstander}
Det er strengt forbudt å slippe ned gjenstander fra luften, unntatt i nødsituasjon.

Ved bruk av gjenstander til å holde i hånden, er det ikke tillatt å feste disse til hånd eller arm.

\subsubsection{Hoppfelt}
Hoppfelt for oppvisningshopp skal være av en slik beskaffenhet at hopperne med rimelig sikkerhet kan lande der uten å volde skade på personer eller eiendom.

Kravene til landingsområdets omfang og beskaffenhet fastsettes i hvert enkelt tilfelle, under hensyntagen til hopperens kvalifikasjoner, fallskjermtype som anvendes, topografiske forhold, samt vindforholdene. Landingsområde for tandemekvipasje skal ha en utstrekning tilnærmet en fotballbane.

Det skal finnes alternative landingsområder som kan anvendes av hopperne dersom de ikke kan lande på det angitte mål. Krav til alternative landingsområder fastsettes i hvert enkelt tilfelle under samme hensyn som ovenfor. Dypt vann kan godtas som alternativt landingsområde dersom bestemmelser for vannhopp følges.

\paragraph{Minimum sikkerhetsutstyr}
Ved oppvisningshopping skal følgende minimum sikkerhetsutstyr være tilstede:
\begin{itemize}
	\item Båre for transport av skadede
	\item Telefonforbindelse
	\item Førstehjelpsutstyr som omfatter minimum støttebandasjer, isposer, armbind, tape og plaster.
	\item Vindindikator eller jordtegn
\end{itemize}

\subsubsection{Avstand til publikum}
Avstand mellom landingsområde og publikum skal være større enn 15 meter horisontalt og 25 fot vertikalt. Jfr BSL D 4-3.

\subsection{Bestemmelser for tandemhopping}
Ved alle tandemhopp skal følgende bestemmelser etterfølges:

\subsubsection{Krav til instruktør}
Instruktør skal inneha gyldig lisens som Instruktør Tandem, jfr. Del 400.

\subsubsection{Krav til elev}
Eleven skal ha underskrevet Egenerklæring om helsetilstand i henhold til Del 300, og ha gjennomgått bakketreningskurs i henhold til Del 600.

For øvrig gjelder bestemmelsene i pkt 102.1 og 102.3.

Som et minimumskrav skal eleven være i stand til å bevege seg til flyet ved egen hjelp. SU kan dispensere fra dette etter søknad (jfr Del 300). Begrunnelse på avslag gis ikke.

Tandeminstruktøren skal i god tid forsikre seg om at Tandemeleven vil gjennomføre tandemhoppet av eget ønske.

Elev skal ha opprettet medlemskap i NLF.

\subsubsection{Krav til kameramann ved tandemfilming}
Kameramenn som skal filme tandemhopp, skal ha gyldig tandem video-lisens.

\subsubsection{Utsprangs- og trekkhøyde}
For tandemhopping er laveste tillatte utsprangshøyde 7500 fot, og laveste trekkhøyde 5000 fot.

\subsection{Bestemmelser for bretthopping}
Det vises til kompendium i bretthopping.

\subsection{Bestemmelser for hopping med vingedress}
\subsubsection{Definisjon}
Hopp med vingdress er et hopp utført med drakt som har vinger mellom armer og ben i den hensikt å oppnå en vingeprofil.

\subsubsection{Erfaringskrav}
B lisens eller høyere

Godkjenning av lokal HI eller den instruktør HI bemyndiger

Minimum 200 hopp siste 18 måneder

Følge kursopplegg i henhold til opplæringskompendiet med instruktør godkjent av lokal HI

Minimum 5 godkjente utsjekkshopp i henhold til progresjonsplan

For hoppere med over 500 hopp kan kravet om 200 hopp siste 18 måneder fravikes etter godkjenning fra lokal HI

På minimum de 5 første hoppene skal man benytte dresser beregnet for nybegynnere. Som en hovedregel skal dette være dresser hvor man kan nå styreliner uten å måtte ”kutte” vingene. Vingedress instruktør kan i samråd med HI øke dette kravet dersom det er påkrevd

\subsubsection{Materiell}
Følgende krav stilles til materiell/utstyret:

Det skal anvendes akustisk høydevarsler egnet for vingedresshopping

Drakten skal være utstyrt med ett ”kutt system” for armvingene eller man skal kunne nå styrelinene uten kutt.

Det er ikke tillatt å benytte strikkpilot (pilot med strikk som gjør at piloten kollapser under en viss hastighet)

Pilotskjerm kan kun være plassert i BOC lomme eller i lomme på dressen laget for dette formålet av produsenten

På de 25 første hoppene er det ikke tillat å hoppe med full elliptiske og/eller ekstremt høyverdige skjermer. I tvilstilfeller avgjør HI

\subsubsection{Progresjonsplan}
Hopp nr 1: Marsjehopp

Gjennomføres med Instruktør, men uten vingedress. Eleven skal fly som om vedkommende hopper med vingedress. Instruktøren hopper med og observerer dummy trekk, samt generelle kunnskaper og evne til å gjøre ett sikkert hopp.

Hopp Nr 2: Vingedress hopp

Gjennomføres med Instruktør. Eleven gjennomfører ett hopp med fokus på å trekke skjerm. Instruktøren observerer elevens flygeferdigheter og evne til å gjøre ett sikkert hopp.

Hopp Nr 3-5: Tilvenningshopp

Gjennomføres under veiledning av Instruktør eller med Instruktør. Eleven skal trene på generelle flygeferdigheter. Disse hoppene kan utføres uten at instruktøren hopper sammen med eleven.

\subsubsection{Krav til instruktør}
Følgende krav stilles til instruktøren:
\begin{itemize}
	\item Utsjekk og lokal opplæring utføres under ledelse av Instruktør utpekt av HI. Dersom Instruktøren ikke er erfaren vingedresshopper assisteres utsjekkene av erfaren vingedresshopper også godkjent av HI
	\item Utsjekken dokumenteres i hopperens Loggbok når progresjonsplanen er gjennomført og alle hoppene er godkjent.
\end{itemize}

\subsection{Bestemmelser for hopping med tube}
Det vises til SU bulletin \# 1/2011.
