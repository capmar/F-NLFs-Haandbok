\part{Fallskjermlisenser}
\setcounter{section}{299}

\section{Definisjoner}
\subsection{Fritt fall}
Med fritt fall menes her et hopp der hopperen selv utløser sin skjerm etter at han har forlatt luftfartøyet.

\subsection{Godkjent hopp}
Godkjent hopp menes et hopp som er utført som sportshopp (planlagt hopp) og hvor hopperen hatt kontroll med hele hoppet fra han forlater flyet og til han står på bakken. Dersom en hopper planlegger en frittfall manøver, og mislykkes i denne, kan hoppet likevel godkjennes dersom det forøvrig har vært under hopperens kontroll. Årsaker til underkjenning av hopp er blant annet for stort avvik fra planlagt frittfalltid, ustabilitet, spinn ol (se også Del 613).

\subsection{Hoppmestring}
Med hoppmestring menes her kontroll, overoppsyn og kommando over en eller flere andre hoppere fra disse møter til innlasting, til de har forlatt flyet. Hoppmestring av elever medfører ansvar for at elevene blir korrekt inspisert, orientert, instruert, spottet og utsatt samt korrekt vurdert etter utført hopp.

\subsection{Instruksjon}
Med instruksjon menes her grunnopplæring og videregående opplæring i fallskjermhopping iht F/NLFs bestemmelser Del 600, samt hoppmestring av elever.

\section{Generelt}
\subsection{Utstedelse}
F/NLFs fallskjermlisenser er kompetansebevis som utstedes av Sikkerhets- og utdanningskomiteen F/NLF til medlemmer av F/NLF og lokalklubbene etter søknad påført attestasjon om at kravene er oppfylt. Etter utstedelse av lisens plikter søkeren i gyldighetsperioden til enhver tid å underkaste seg doping- eller rusmiddelkontroll, herunder utvidet blodprøve.

Søknader om utstedelse av elevbevis, fallskjermlisens eller lisens sendes NLFs sekretariat, påført attestasjon av Instruktør 1 eller høyere. Instruktøren er ansvarlig for at søknaden samsvarer med gjeldene vedlikeholdskrav. Søknader om fornyelse sendes NLFs sekretariat, påført attestasjon av Instruktør 2 eller høyere.

Fallskjermpakkerlisens utstedes etter anbefaling fra Instruktør 1 eller Materiellkontrollør, med bekreftelse om tilfredsstillende gjennomført eksamensprøve.

Alle lisenser er nummerert, og NLF fører register over alle utstedelser og fornyelser.

\subsection{Gyldighet}
Lisensene utstedes for livstid, men er kun gyldig for deltakelse i praktisk hopping i den perioden de er utstedt eller fornyet for, og ved samtidig gyldig og betalt medlemskap i F/NLF. Kriterier for medlemskap, straffeforføyninger og gjenopptak er gitt i NIFs og NLFs lover.

Lisens anses gyldig ved postlegging av skjema for utstedelse eller fornyelse, dersom dette er bevitnet av instruktør 2 eller høyere.

Fallskjermpakkerlisens er automatisk gyldig dersom innehaveren også har gyldig fallskjermlisens av klasse A eller høyere. Ellers gjelder vedlikeholdsbestemmelsene i pkt 304.8.

\subsection{Rettigheter}
F/NLFs fallskjermlisenser gjelder som attestasjon for innehaverens praktiske ferdigheter, og gir alt etter sin klasse innehaveren bestemte rettigheter ved deltakelse i praktisk hopping. Ferdighets- og erfaringskravene til norske lisenser vil følge de internasjonale lisenskrav fastlagt av Federation Aeronautique Internationale (FAI). De norske lisenser A-, B-, C- og D- er derfor gyldige International Certificates of Proficiency, innen rammen av FAI Sporting Code.

\subsection{Begrensninger}
Rett til å drive instruksjon følger av bestemmelsene om fallskjermlisenser (Del 300) samt bestemmelsene om instruktørlisenser (Del 400).

\subsection{Fornyelse}
Fornyelse av lisenser utføres av NLFs sekretariat etter søknad med attestasjon av Instruktør 2 eller høyere for at fornyelseskravene er oppfylt. Fornyelse skjer for et kalenderår. Instruktør kan ikke fornye egne lisenser.

\subsection{Inndragning/disiplinærtiltak}
Dersom det av hensyn til sikkerheten anses nødvendig, kan Sikkerhets- og utdanningskomiteen F/NLF eller organer utpekt av denne, inndra lisenser for kortere eller lengre tid, eller for bestandig. Se pkt. 302.1.1 og 302.2.1 samt 101.7 og 506.7.3.5 Dersom det av hensyn til sportens anseelse anses nødvendig, kan styret F/NLF gå inn å frata hopperen sitt medlemskap for kortere eller lengre tid.

Dersom innehaveren av en lisens åpenbart viser at han/hun ikke innehar den kompetanse som kreves for hver lisens kan lisenser inndras på samme måte.

\section{Medisinske krav}
\subsection{Førstegangsutstedelse}
Ved førstegangs utstedelse av elevbevis, skal søkeren fremlegge legeattest basert på F/NLFs legekontrollskjema. Legeattesten må ikke være eldre enn 3 måneder gammel ved fremleggelsen. Ved tvil om søkerens medisinske skikkethet, skal F/NLFs medisinske konsulent kontaktes.

Gyldig Legeattest klasse 1 og 2, iht. Luftfartstilsynet krav, vil kunne erstatte F/NLFs legekontrollskjema. Kopi av attesten skal vedlegges søknad om første gangs utstedelse.

\subsubsection{Tidligere misbruk av rusmidler}
Dersom søkeren tidligere har misbrukt narkotika, eller dersom vedkommende har misbrukt andre rusmidler på en måte som har krevd behandling, kreves en dokumentert rusfri periode på 2-5 år før lisenser kan utstedes. Varigheten av den rusfrie perioden avhenger av misbrukets omfang og karakter, og av en vurdering av vedkommendes risiko for tilbakefall.

\subsection{Fornyelse}
Ved fornyelse av elevbevis, eller utvidelse eller fornyelse av fallskjermlisens, skal det fremlegges egenhendig undertegnet erklæring om at søkeren etter førstegangs legeattest ikke har endret legems- eller helsetilstand, fortsatt er fysisk normal og ikke har vært til behandling for: hjertefeil eller hjertesykdom, lengre sykdom med varig mèn, sukkersyke, epilepsi, besvimelsesanfall eller kramper, nervøse lidelser, nyresykdom, høyt eller lavt blodtrykk.

Pkt. 302.1 og 302.2 gjelder ikke for fallskjermpakkerlisens.

\subsubsection{Gjenutstedelse etter inndragning}
Dersom lisens er inndratt av sikkerhetsmessige årsaker relatert til hopperens fysiske eller psykiske skikkethet (jfr. Del 100 og pkt 301.6), kreves fremlagt ny legeattest iht 302.1 med underpunkter. Legeattest skal i slike tilfeller vurderes av F/NLFs medisinske konsulent.

\subsection{Konvertering fra annen fallskjermutdanning}
Ved førstegangs utstedelse av fallskjermlisens, der dette tas ut på bakgrunn av militær eller utenlandsk hopputdanning, skal søkeren fremlegge legekontrollskjema som angitt i pkt 302.1.

\subsection{Tandemelever}
Før deltakelse som elev ved Tandemhopp skal eleven ha fylt ut og undertegnet egenerklæring om helsetilstand som beskrevet i pkt 302.2.

\section{Lisensklasser}
Det utstedes følgende typer lisenser:
\begin{itemize}
	\item Fallskjermpakkerlisens.
	\item Elevbevis
	\item A-lisens
	\item B-lisens
	\item C-lisens
	\item D-lisens
	\item Demo lisens
	\item Video lisens
\end{itemize}

\subsection{Generelt krav til erfaring og ferdigheter}
Hopp utført i militær regi teller på linje med hopp utført i sivil regi, dersom hoppene er tilfredsstillende dokumentert.

Hopp utført for et lavere lisens, teller også for høyere lisens.

Som kvalifisering regnes utførte fallskjermhopp fra hvilket som helst type luftfartøy.

Utførte hopp av høyere vanskelighetsgrad kvalifiserer for hopp av lavere vanskelighetsgrad (et hopp på 30 sek FF kvalifiserer også for et på 20 sek FF).

Hopp utført i utlandet teller på linje med hopp utført i Norge, dersom hoppene er tilfredsstillende dokumentert.

Sertifisering/autorisering som militær pakker på bestemt type fallskjerm godtas som kvalifisering for utstedelse av fallskjermpakkerlisens dersom den er tilfredsstillende dokumentert, men søkeren må avlegge eksamen i pakking og teori overfor Materiellkontrollør på vanlig måte.

Fabrikk-kurs i vedlikehold og pakking av fallskjermer skal godkjennes av SU etter vurdering i hvert enkelt tilfelle, før utstedelse av fallskjermpakkerlisens.

Fallskjermpakkerutdannelse gjennomgått i utlandet kan godtas dersom den er tilfredsstillende dokumentert, men søkeren må avlegge norsk eksamensprøve i pakking og teori på vanlig måte.

\section{Fallskjermpakkerlisens}
Grunnopplæring som fallskjermpakker og spesialopplæring som pakker på bestemt type fallskjerm.

Pakkelisenset deles inn i 3 hovedgrupper: Hovedskjermer (Mains), Reserver (Reserves) og Nødskjermer (Emergency).

\subsection{Hovedskjermer (Mains)}
\subsubsection{Vingskjermer}
\paragraph{Square sport canopy/slider (kategori 1)}
Omfatter alle typer firkantskjermer med slider og deploymentsystem med bag, diaper eller fripakking.

\subsection{Reserver (Reserves)}
\subsubsection{Runde}
\paragraph{Reserve, circular tandem (kategori 2)}
Omfatter alle rundkalottreserver montert i piggyback pakksekk med låseløkker med eller uten diaper.

\subsubsection{Vingskjermer}
\paragraph{Square reserve canopy (kategori 3)}
Omfatter alle typegodkjente reserveskjermer av vingskjermtypen (ram air canopies).

\subsection{Nødskjermer (Emergency)}
\subsubsection{Emergency parachutes (kategori 4)}
Omfatter alle standard nødskjermtyper som er til bruk som redningskjermer i fly (ikke sportshopping), B4, NB6, Security 150, Pioneer Thin Pack, EFA Emergency, Strong Para Cushion ol.

\subsection{Utdanning/eksaminasjon}
Materiellkontrollører som signerer for ovenstående kategorier pakkekvalifikasjoner skal selv inneha pakkelisens for typene, og plikter å påse at kandidaten innehar de nødvendige kunnskaper som kreves for pakking av samtlige typer og varianter som hver enkelt kategori omfatter ved grunnopplæring som fallskjermpakker og spesialopplæring som pakker på en bestemt kategori fallskjerm. (Unntatt herfra er utdannelse etter Del 600). Som minstekrav skal utdannelsen følge program fastsatt av Sikkerhets- og utdanningskomiteen F/NLF og omfatte:
\begin{itemize}
	\item Materiellreglementets bestemmelser.
	\item Skjermteori.
	\item Tilfredsstillende pakking av den aktuelle kategori, med teoretisk og praktisk eksamensprøve.
\end{itemize}

\subsection{Sertifisering - utvidelse}
Lisensen kan utvides til også å omfatte andre kategorier fallskjermer ved at kandidaten eksamenspakker en skjerm av den aktuelle kategori på en tilfredsstillende måte. Pakkingen skal skje under kontroll av Materiellkontrollør og godkjennes av denne.

Opplæring i pakking av flere kategorier av skjermtyper kan skje fortløpende på samme kurs.

Ved eventuell godkjenning av militære kurs, og kurs ved fabrikk eller i utlandet, refereres til pkt 303.1.

Fallskjermpakkerlisens for andre kategorier enn kategori 1 vil bare bli utstedt til Materiellkontrollører.

\subsection{Grunnutdanning}
Grunnutdannelse som fallskjermpakker skal ledes av instruktør 1. Personell med B-lisens eller høyere kan anvendes som hjelpeinstruktører. Ved sertifisering skal fallskjermpakker utstyres med et komplett sett F/NLF bestemmelser (Håndbok) som sitt personlige eksemplar, samt pakkeinstruks (Manual) på angjeldende kategori fallskjerm.

Innehaver av pakkerlisens betegnes ``Fallskjermpakker''.

\subsection{Videreutdanning}
Fallskjermpakkerlisens kan utvides til to høyere klasser:
\begin{itemize}
	\item Materiellkontrollør.
	\item Materiellreparatør.
\end{itemize}

\subsubsection{Materiellkontrollør}
Ved utvidelse til Materiellkontrollør fordres minimum 3 års praktisk pakkererfaring fra aktiv hoppvirksomhet (pakkeren behøver ikke hoppe selv), 1 års opplæring under veiledning av Materiellkontrollør og bestått F/NLFs Materiellkontrollørkurs.

Materiellkontrollør skal som et minimum inneha pakkerlisens for kategori 1, 2 og 3 fallskjermer.

Innehaver betegnes ``Materiellkontrollør''.

\paragraph{Rettigheter og begrensninger}
Fallskjermpakkerlisens utvidet til Materiellkontrollør gir rett til selvstendig pakking av reserveskjerm (kategori 2 og 3) montert i piggyback pakksekk.

Fallskjermpakkerlisens utvidet til Materiellkontrollør, kan utvides til også å omfatte kategori 4, ref pkt 304.3, og gir da rett til selvstendig pakking av nødskjerm.

Fallskjermpakkerlisens utvidet til ``Materiellkontrollør'' gir innehaveren rett til å foreta hovedkontroll av de fallskjermtyper han har pakkerlisens for.

Han har også rett til å utføre mindre reparasjoner, dog ikke reparasjoner som er kritisk for åpningssikkerheten eller som kan endre utstyrets funksjon.

Materiellkontrollør skal ha plomberingsutstyr med pregemerke samt loggbok med oversikt over utført arbeid.

\subsubsection{Materiellreparatør}
Ved utvidelse til Materiellreparatør skal søkeren ha minst ett års erfaring som Materiellkontrollør under praktisk hopping (han behøver ikke hoppe selv) og ha gjennomført kurs i fallskjermreparasjon, med avsluttende eksamen. Slikt kurs skal være godkjent av Sikkerhets- og utdanningskomiteen. Kurset og eksamensprøven skal ledes av en offentlig eller F/NLF godkjent eller militært autorisert fallskjermreparatør.

Innehaveren betegnes ``Materiellreparatør''.

\paragraph{Rettigheter og begrensninger}
Fallskjermpakkerlisens utvidet til ``Materiellreparatør'' gir innehaveren samme rett som ``Materiellkontrollør'' og dessuten rett til å foreta alle former for reparasjoner og modifikasjoner på de fallskjermtyper han har lisens for.

Materiellreparatørstatus gir ikke rett til produksjon av utstyr eller materiellkomponenter med mindre særskilt dokumentasjon om at faglige kvalifikasjoner, produksjonsutstyr, fremstillingsprosess og testing er i samsvar med regler som angitt i ``The Parachute Manual Vol I.'' kan fremlegges.

\subsection{Gyldighets- og fornyelseskrav}
Dersom fallskjermpakkeren ikke har gyldig fallskjermlisens A eller høyere, må han ha attestasjon fra Materiellkontrollør på at han har pakket minst 5 skjermer av den aktuelle typen i løpet av siste kalenderår. Om dette ikke er tilfellet, kan lisensen gjøres gyldig gjennom minst en godkjent pakking av den aktuelle type under kontroll av Materiellkontrollør eller Instruktør 1.

\subsubsection{Tilleggskrav for materiell}
For fallskjermpakkerlisens utvidet til Materiellkontrollør gjelder følgende tillegg:
\begin{itemize}
	\item Ha hovedkontrollert minst fem komplette fallskjermsett i løpet av siste 365 dager.
\end{itemize}

\section{Elevbevis line}
\subsection{Lineutløst fallskjerm – elevbevis line}
\subsubsection{Kvalifikasjonskrav}
Gjennomgått grunnkursets del 1 etter program fastsatt av Sikkerhets- og utdanningskomiteen F/NLF, (Del 600) under ledelse av Instruktør 1 eller 2. Avlagt avsluttende prøve overfor Hovedinstruktør eller annen instruktør 1 som denne bemyndiger, med bekreftelse om at eleven anses fysisk, psykisk og ferdighetsmessig i stand til å gjennomføre fallskjermutsprang med lineutløst fallskjerm i henhold til progresjonsplan for lineelever.

\subsubsection{Rettigheter og begrensninger}
Elevbevis Line gir rett til å delta i praktisk hoppvirksomhet etter F/NLFs Bestemmelser Del 500 under kontroll av Hoppleder og med Hoppmester i flyet.

Innehaveren betegnes ``Line-elev''.

\subsubsection{Gyldighet og fornyelse}
Elevbevis Line er gyldig ut året det er utstedt i, og for gjennomføring av de fallskjermhopp som i henhold til progresjonsplanen skal gjennomføres med lineutløst fallskjerm.

Lineelev må gjøre neste hopp innen 1 måned. Hvis dette kravet ikke er oppfylt, men innehaveren har hoppet i løpet av siste 3 måneder, gjelder følgende avhengig av oppholdets lengde:

Opphold 1 til 3 måneder: Siste godkjente hopp må repeteres.

Opphold 3 til 12 måneder: Praktisk prøve grunnkurs del I avlegges overfor instruktør-2. To påfølgende godkjente hopp av siste godkjente hopptype må repeteres.

Opphold 12 til 24 måneder: Praktisk prøve grunnkurs del I avlegges overfor instruktør-1. Hele progresjonsplan repeteres.

Lineelev som har lengre opphold enn 3 måneder mer enn en gang, må følge nytt komplett grunnkurs og repetere hele progresjonsplan.

\section{Elevbevis AFF}
\subsection{Kvalifikasjonskrav}
Gjennomgått grunnkurs AFF etter program fastsatt av Sikkerhets- og utdanningskomiteen F/NLF, (del 600), under ledelse av Instruktør 1 eller 2 med særskilt godkjenning som AFF- instruktør. Avlagt avsluttende prøve overfor Hovedinstruktør eller Instruktør 1, med bekreftelse om at denne anser eleven fysisk, psykisk og ferdighetsmessig i stand til å gjennomføre fallskjermutsprang med fritt fall i henhold til progresjonsplan for AFF-elever, ref Del 600.

\subsection{Rettigheter og begrensninger}
Elevbevis AFF gir rett til å delta i praktisk hoppvirksomhet etter F/NLFs Bestemmelser Del 500 under kontroll av Hoppleder og med Hoppmester i flyet. Hopp iht. progresjonsplanens nivå 1-7 skal gjennomføres sammen med godkjent AFF-Instruktør, ref Del 600.

Innehaveren betegnes ``AFF-elev''.

\subsection{Gyldighet og fornyelse}
Elevbevis AFF er gyldig ut året det er utstedt i, for gjennomføring av de i progresjonsplanen fastsatte hoppene med typegodkjent elevskjermsett.

AFF-elev som har lengre opphold mellom hoppene enn 3 måneder mer enn en gang, må følge nytt komplett grunnkurs og repetere hele progresjonsplanen.

\subsubsection{Krav til progresjon nivå 1 til nivå 3}
AFF-elev som befinner seg på nivå 1-3 må gjøre neste hopp innen 1 mnd. Dersom det går mer enn en måned mellom hoppene, gjelder følgende avhengig av oppholdets lengde:

Opphold 1 til 3 måneder: Siste godkjente nivå må repeteres.

Opphold 3 til 12 måneder: Praktisk prøve grunnkurs AFF avlegges overfor instruktør-2/AFF. Siste godkjente nivå må repeteres.

Opphold 12 til 24 måneder: Praktisk prøve grunnkurs AFF avlegges overfor instruktør-1. Hele progresjonsplan repeteres.

\subsubsection{Krav til progresjon nivå 4 til nivå 7}
AFF-elev som befinner seg på nivå 4-7 må gjøre neste hopp innen 1 mnd. Dersom det går mer enn en måned mellom hoppene, gjelder følgende avhengig av oppholdets lengde:

Opphold 1 til 3 måneder: Siste godkjente nivå må repeteres.

Opphold 3 til 12 måneder: Praktisk prøve grunnkurs AFF avlegges overfor instruktør-2/AFF. Nivå 3 og siste godkjente nivå må repeteres før eleven tillates å fortsette.

Opphold 12 til 24 måneder: Praktisk prøve grunnkurs AFF avlegges overfor instruktør-1. Progresjonsplan fra og med nivå 3 repeteres.

\subsubsection{Krav til progresjon nivå 8 og hand deployed utsjekk}
AFF-elev som befinner seg på nivå 8 eller på hand deployed utsjekk, må gjøre neste nivå i progresjonen innen 3 mnd. Dersom det går mer enn 3 måneder før nytt hopp gjennomføres, gjelder følgende avhengig av oppholdets lengde:

Opphold 3 til 12 måneder: Nivå 5 må repeteres før eleven tillates å fortsette.

Opphold 12 til 24 måneder: Praktisk prøve grunnkurs AFF avlegges overfor instruktør-1. Nivå 3, nivå 5 og nivå 7 må repeteres før eleven tillates å fortsette.

\section{Elevbevis fritt fall (FF-elevbevis)}
\subsection{Kvalifikasjons- og ferdighetskrav}
Gjennomgått progresjonsplan for lineutløst hopp iht. Del 600. Gjennomgått grunnkursets del 2 etter program fastsatt av Sikkerhets- og utdanningskomiteen F/NLF (del 600) under ledelse av Instruktør 1 eller 2. Avlagt avsluttende teoretisk og praktisk prøve overfor Instruktør 1, og med anbefaling fra denne om utstedelse av FF-elevbevis.

Til hoppere som kan dokumentere militært fritt fall kurs og oppfylte vedlikeholdskrav, utstedes FF-elevbevis etter innføring i sivilt elevskjermsett og avlagt prøve til pakkerlisens og FF- elevbevis.

\subsection{Rettigheter og begrensninger}
FF-elevbevis gir samme rettigheter som elevbevis line, idet innehaveren fortsatt kun må delta i praktisk hopping under ledelse og kontroll av Hoppleder og med Hoppmester i flyet.

FF-elevbevis er kun gyldig for gjennomføring av de i progresjonsplanen, (ref Del 600) angitte hoppene utført med typegodkjent elevskjermsett.

Ved tilbakeføring til FF-elevbevis er det tillatt, dersom dette har automatåpner, å benytte eget, men ikke lånt, privat utstyr.

Elever som gis FU eller som på grunn av stabilitetsproblemer og/eller får sitt hopp underkjent på grunn av for lang FF-tid i forhold til beordret, på flere hopp etter hverandre, tilbakeføres i progresjon.

Innehaveren betegnes ``Fritt fall elev''.

\subsection{Gyldighet og fornyelse}
FF-elevbevis kan fornyes for et kalenderår om gangen dersom innehaveren i løpet av siste kalenderåret har gjennomført minst 20 fallskjermhopp (hopp med Elevbevis line og lineutløst fallskjerm inkludert).

FF-elev som har lengre opphold mellom hoppene enn 3 måneder mer enn en gang, må følge nytt komplett grunnkurs del I og repetere hele progresjonsplan fra UL.

\subsubsection{Krav til progresjon FF-elever under 15 sek FF}
FF-elev på progresjon under 15 sek FF må gjøre neste hopp innen 1 mnd. Dersom det går mer enn en måned mellom hoppene, gjelder følgende avhengig av oppholdets lengde:

Opphold 1 til 3 måneder: Siste godkjente hopp må repeteres.

Opphold 3 til 12 måneder: Praktisk prøve grunnkurs del I og del II avlegges overfor instruktør- 2. Ett hopp UL/T og siste godkjente FF-hopp må repeteres før eleven tillates å fortsette.

Opphold 12 til 24 måneder: Praktisk prøve grunnkurs del I og del II avlegges overfor instruktør-1. To hopp UL/T, to hopp 5 sek FF og to hopp siste godkjente FF må repeteres før eleven tillates å fortsette.

\subsubsection{Krav til progresjon FF-elever fra og med 15 sek FF}
FF-elev på progresjon over 15 sek FF må gjøre neste hopp innen 3 mnd. Dersom det går mer enn 3 mnd mellom hoppene, gjelder følgende avhengig av oppholdets lengde:

Opphold 3 til 12 måneder: Siste godkjente FF-hopp må repeteres før eleven tillates å fortsette.

Opphold 12 til 24 måneder: Praktisk prøve grunnkurs del I og del II avlegges overfor instruktør-1. Ett hopp UL/T, ett hopp 5 sek FF, ett hopp 10 sek FF og siste godkjente hopp FF må repeteres før eleven tillates å fortsette.

\section{A-lisens}
\subsection{Kvalifikasjons- og ferdighetskrav}
\subsubsection{Med bakgrunn i grunnkurs, line}
Fullført progresjonsplan iht Del 600. Minst 20 godkjente fritt fall hopp det siste kalenderåret. For elever som ikke har hatt rimelig tid til å hoppe antallet hopp som kreves første kalenderår, kan HI utstede A lisens innenfor rammen av 365 dager fra utstedelse av elevbeviset. Minst 10 hopp med landing innenfor 50 meter. Utsjekk firkantreserve, utsjekk høyverdig firkantskjerm. Deltatt på A-sertfikatkurs under ledelse av instruktør 2. Inneha pakkelisens for hovedskjerm kategori 1. Avlagt teoriprøve med tilfredsstillende resultat overfor Instruktør 1, i Håndbokas Del 100, Del 200 og Del 500.

\subsubsection{Med bakgrunn i grunnkurs, AFF}
Fullført progresjonsplan iht Del 600. Minst 20 godkjente fritt fall hopp siste kalenderår. Forøvrig krav som med bakgrunn i grunnkurs line.

\subsubsection{Med bakgrunn i militært fritt fall kurs}
For utstedelse av A-lisens kreves minimum 20 godkjente fritt fall hopp siste kalenderår, samt:

\begin{itemize}
	\item Driverkast og 3 godkjente spottinger.
	\item 3 hopp FS-utsjekk, godkjent.
	\item 3 hopp hand-deployed utsjekk, godkjent.
	\item Deltatt på A-sertfikatkurs under ledelse av instruktør 2. Avlagt teoriprøve til A-lisens overfor Instruktør 1.
\end{itemize}

\subsection{Rettigheter og begrensninger}
A-lisens gir innehaveren rett til å gjennomføre solo fritt fall under ledelse og kontroll av Hoppleder, og til å hoppmestre seg selv på solo fritt fall hopp. A-lisens gir hopperen rett til å hoppe med typegodkjent privateid fallskjermsett (fallskjermsett som ikke er elevskjermsett).

Innehaver av A-lisens kan kun delta i formasjonshopping (FS, VFS) under veiledning av instruktør. Formasjonene må ikke bestå av mer enn totalt tre hoppere. Dersom det hoppes 3'er skal minst en av hopperne ha B sert eller høyere.

Innehaveren betegnes ``Solo frittfall hopper / parachutist''.

\subsection{Fornyelseskrav}
\subsubsection{Generelt krav}
Gjennomført minimum 20 fallskjermhopp siste kalenderår.

HI kan unntaksvis og etter individuell vurdering fornye A-lisens med færre enn 20 hopp siste kalenderår. Dersom dette gjøres skal HI vedlegge en særskilt vurdering når fornyelsespapirene sendes til F/NLF.

\section{B-lisens}
\subsection{Kvalifikasjons- og ferdighetskrav}
Inneha A-lisens. Minst 50 godkjente fritt fall hopp, hvorav minimum 20 siste kalenderår. Mer enn 30 minutter akkumulert frittfall tid.

Fullført progresjonshopp for A-lisens omfattende skjermkontroll og sikkerhet under formasjonshopping (FS og VFS).

Avlagt skriftlig prøve overfor Instruktør 1 omfattende Håndbokens del 100, 200, 300, 400 og 500.

\subsection{Rettigheter og begrensninger}
B-lisens gir innehaveren rett til å delta i praktisk hopping under ledelse og kontroll av Hoppleder, og til å hoppmestre seg selv. Det gir rett til å delta i konkurranser, nasjonalt og internasjonalt. B-lisens gir også rett til å attestere hopplogg for andre hoppere med A-lisens eller høyere, dog ikke for de spesielle progresjonshopp som skal kontrolleres av Instruktør.

B-lisens gir hopperen rett til å fungere som Hoppfeltleder, med plikter og rettigheter iht Del 500.

Innehaveren betegnes ``Hoppfeltleder / freefall parachutist''.

\subsection{Fornyelseskrav}
Som for A-lisens.

\section{C-lisens}
\subsection{Kvalifikasjons- og ferdighetskrav}
Inneha B-lisens. Fylt 18 år. 200 godkjente fritt fall hopp, hvorav 40 det siste kalenderår. Minst 50 formasjonshopp (FS og/eller VFS) hvor minst 10 må ha bestått av hhv fire (FS) eller tre (VFS) eller flere deltagere. Mer enn 1 time akkumulert fritt fall tid.

Ha deltatt i minst 1 års praktisk (aktiv) hoppvirksomhet som selvstendig hopper. Ha gjort tjeneste som HFL minst 5 ganger. Bestått skriftlig prøve overfor Instruktør 1 omfattende Håndbokens del 100, 200, 300, 400 og 500. Av lokal Hovedinstruktør være funnet skikket til å fungere som Hoppleder ved treningshopping for hoppere med minimum A-lisens, jfr. Pkt. 310.2.

\subsection{Rettigheter og begrensninger}
C-lisens gir samme rettigheter som B-lisens, samt rett til å attestere alle hopplogger, dog ikke for progresjonshopp som krever attestasjon av Instruktør.

C-lisens gir rett til å fungere som Hoppleder ved treningshopping for hoppere med minimum A-lisens.

Innehaveren betegnes ``Hoppmester / experienced parachutist''.

\subsection{Fornyelseskrav}
Fornyelse av C-lisens krever minimum 40 hopp siste kalenderår. Om hopperen har gjennomført mer enn 20 fallskjermhopp siste kalenderår kan B-lisens fornyes. C-lisens kan fornyes når hopperen har oppnådd 40 hopp siste kalenderår.

Om hopperen har gjennomført mindre enn 20 hopp siste 365 dager vil fornyelseskrav som for A og B lisens gjelde, ref pkt 308.3.

\section{D-lisens}
\subsection{Kvalifikasjons- og ferdighetskrav}
Inneha C-lisens og Instruktør 3-lisens.

Minst 500 godkjente fritt fall hopp, hvorav 40 siste kalenderår. Mer enn 3 timer akkumulert fritt fall tid. Avlagt skriftlig prøve overfor Instruktør 1 omfattende Håndbokens del 100, 200, 300, 400, 500 og 600, med vedlegg for alle kapitler. Ha teoretisk kjennskap til CF-hopping. Prøvene skal være avlagt overfor Instruktør 1.

\subsection{Rettigheter og begrensninger}
D-lisens gir samme rettigheter som C-lisenset.

D-lisens gir hopperen rett til å fungere som Hoppleder ved alminnelig hopping. Innehaveren betegnes ``Hoppleder / senior parachutist''.

\subsection{Fornyelseskrav}
Som for C-lisens, ref pkt 310.3.

\section{Gjenutstedelse av A-, B-, C- eller D- lisens når det er hoppet de siste 10 år}
Fallskjermhopper som tidligere har innehatt A-, B-, C- eller D-lisens, men som ikke kan få dette fornyet på grunn av manglende antall vedlikeholdshopp, kan, etter søknad attestert av Instruktør 1, få utstedt elevbevis, dersom han har hoppet i løpet av de siste 10 år.

Det er tillatt å benytte eget privateid sportsutstyr. Det er påbudt med nødåpner. Ved progresjonshopp AFF nivå 1 til nivå 7 gjelder krav om sekundært utløserhåndtak for hovedskjerm. Etter vurdering av hovedinstruktør kan de hopp som ikke er på nivå 1-7 gjennomføres med elevskjermsett med BOC-plassert hand deployed pilotskjerm.

HI kan, for de som hadde B-, C- eller D lisens på det tidspunkt de stoppet å hoppe, utstede A eller B lisens selv om minimumskravet til antall hopp (20 stk det siste kalenderåret) ikke er oppnådd. Da skal HI vedlegge en redegjørelse for sin vurdering.

\subsection{Gjenutstedelse av tidligere A-, B-, C- og D-lisens ved opphold kortere enn 3 år}
Når det gjelder A-, C- og D-lisens skal minimumskravet for antall hopp siste kalenderår være gjennomført før lisenset kan gjenutstedes. Dersom det ila oppholdet er innført strengere lisenskrav enn det hopperen hadde da han/hun sluttet, skal de nye kravene oppfylles før det aktuelle lisenset gjenutstedes.

\begin{table}
	\caption{Tilbakeføring til progresjon etter grunnkurs del 2:}
	\begin{tabular}{ | p{2cm} | p{4cm} | p{4cm} | }
		\hline
		Erfaring / tid siden siste hopp & Mindre enn 2 år & Mer enn 2 år \\
		\hline
		A-lisens & Repetere 5 sek FF, 10 sek FF, 30 sek FF og deretter tre hopp FS-utsjekk. Maks vingbelastning 0,95. Oppsamling til 20 hopp siste år. & Praktisk prøve GK I og II for instruktør 1. Repetere ett hopp 5 sek FF, 7 sek FF, 10 sek FF, tre 30 sek FF og tre hopp FS- utsjekk. Maks vingbelastning 0,95. Opp- samling til 20 hopp siste år. \\
		\hline
		B – C lisens \textless 200 hopp & To solo FF og tre hopp FS-utsjekk. Maks vingbelastning 1,1 Oppsamling til min 20 hopp siste år. & Repetere ett hopp 5 sek FF, 10 sek FF, tre 30 sek FF og tre hopp FS-utsjekk. Maks vingbelastning 1,1 Oppsamling til 20 hopp siste år. \\
		\hline
		B-D lisens \textgreater 200 hopp & Fem solo FF. Maks vingbelastning 1,3. Oppsamling til 20 hopp siste år. & 10 solo FF. Maks vingbelastning 1,3. Oppsamling til 20 hopp siste år. \\
		\hline
	\end{tabular}
\end{table}

\begin{table}
	\caption{Tilbakeføring til progresjon etter grunnkurs AFF:}
	\begin{tabular}{ | p{2cm} | p{4cm} | p{4cm} | }
		\hline
		Erfaring / tid siden siste hopp & Mindre enn 2 år & Mer enn 2 år \\
		\hline
		A-lisens & Repetere nivå 3, nivå 5 og fire valgfrie hopp nivå 8. Maks vingbelastning 0,95. Oppsamling til 20 hopp siste år. & Praktisk prøve GK AFF for instruktør 1. Repetere nivå 3, nivå 5, nivå 7 og alle hopp nivå 8. Maks vingbelastning 0,95. Oppsamling til 20 hopp siste år. \\
		\hline
		B – C lisens \textless 200 hopp & To solo FF og tre hopp FS-utsjekk. Maks vingbelastning 1,1 Oppsamling til 20 hopp siste år. & Repetere nivå 3, nivå 7 og fire valgfrie hopp nivå 8. Maks vingbelastning 1,1 Oppsamling til 20 hopp siste år. \\
		\hline
		B-D \textgreater 200 hopp & Fem solo FF. Maks vingbelastning 1,3. Oppsamling til 20 hopp siste år. & 10 solo FF. Maks vingbelastning 1,3. Oppsamling til 20 hopp siste år. \\
		\hline
	\end{tabular}
\end{table}

\subsection{Gjenutstedelse av tidligere A-, B-, C- og D-lisens ved opphold lengre enn 3 år}
A-lisens kan gjenutstedes etter at de foreskrevne hopp på elevbevis, jfr tabell i pkt 312.3.1 og 312.3.2 er gjennomført og teoriprøve er bestått.

B-lisens kan gjenutstedes etter at de foreskrevne hopp på elevbevis, jfr tabell i pkt 312.3.1 og 312.3.2 er gjennomført og teoriprøve er bestått.

C-lisens kan gjenutstedes etter at ny avsluttende teoretisk prøve er avlagt og de beskrevne hopp i tabell i pkt 312.3.1 og 312.3.2 er gjennomført, og hopperen igjen har min 40 hopp det siste året.

For å få gjenutstedt D-lisens skal vedkommende i tillegg til det som er anført i tabellene i pkt 312.3.1 og 312.3.2, ha deltatt som hjelpeinstruktør under grunnkurs (Del 1 og Del 2 eller AFF), og ha gjennomført ny hoppmester utsjekk i henhold til Del 400, med godkjent resultat. Gjeldende krav til antall hopp for gjenutstedelse av D-lisens er minimum 40 hopp det siste året.

Dersom det ila oppholdet er innført strengere lisenskrav enn det hopperen hadde da han/ hun sluttet, skal de nye kravene oppfylles før det aktuelle lisenset gjenutstedes.

HI kan, for de som hadde B-, C- eller D lisens på det tidspunkt de stoppet å hoppe, utstede A eller B lisens selv om minimumskravet til antall hopp (20 stk det siste året) ikke er oppnådd. Da skal HI vedlegge en redegjørelse for sin vurdering.

\begin{table}
	\caption{Tilbakeføring til progresjon etter grunnkurs del 2:}
	\begin{tabular}{ | p{2cm} | p{8cm} | }
		\hline
		Erfaring / tid siden siste hopp & Mer enn 3 år \\
		\hline
		A-lisens & Praktisk prøve GK I og GK II for I-1. Starte på nytt på progresjonsplan etter GK II. Maks vingbelastning 0,95. Oppsamling til 20 hopp siste år. \\
		\hline
		B - C lisens \textless 200 hopp & Praktisk prøve GK I og II for I-1. Starte på progresjonsplan på 5 sek FF. Maks vingbelastning 0,95. Oppsamling til 20 hopp siste år. \\
		\hline
		B-D \textgreater 200 hopp & Praktisk prøve GK I og II for I-1. 20 solo FF. Maks vingbelastning 1,1. \\
		\hline
	\end{tabular}
\end{table}

\begin{table}
	\caption{Tilbakeføring til progresjon etter grunnkurs AFF:}
	\begin{tabular}{ | p{2cm} | p{8cm} | }
		\hline
		Erfaring / tid siden siste hopp & Mer enn 3 år \\
		\hline
		A-lisens & Praktisk prøve GK AFF for I-1. Starte på nytt på progresjonsplan AFF. Maks vingbelastning 0,95. Oppsamling til 20 hopp siste år. \\
		\hline
		B - C lisens \textless 200 hopp & Praktisk prøve GK AFF for I-1. Starte på nivå 3 AFF. Maks vingbelastning 0,95. Oppsamling til 20 hopp siste år. \\
		\hline
		B-D \textgreater 200 hopp & Praktisk prøve GK AFF for I-1. 20 solo FF. Maks vingbelastning 1,1. \\
		\hline
	\end{tabular}
\end{table}

\subsection{Ikke hoppet siste 10 år}
Fallskjermhopper som tidligere har innehatt lisens, men som ikke kan få dette fornyet på grunn av manglende antall vedlikeholdshopp, og som ikke har hoppet på 10 år eller mer, må gjennomgå nytt komplett grunnkurs og få fornyet elevbevis, og videre følge progresjonsplanen for dette.

\section{Videolisens}
\subsection{Videolisens tandem}
\subsubsection{Kvalifikasjonskrav}
C-lisens. Utført minimum 50 hopp som fotograf for FS. Før kursstart skal resultatet av 20 av hoppene/ferdighetene presenteres på DVD. Ved kursstart skal nye kandidater presentere resultatet av 20 av disse fotohoppene for vurdering av ferdighetene til å filme på magen i minimum 1:1 i front slik tandem vil være filmet. For bedre å få inn dette med kompetanse, ferdigheter og kunnskap skal kurset holdes av en erfaren VT hopper sammen med en I-1 som har eller har vært instruktør tandem.

\subsubsection{Rettigheter}
Fotografere tandemhopp fra fritt fall.

\subsubsection{Vedlikeholdskrav}
Inneha gyldig C- eller D-lisens.

Dersom kravet til hopping ikke er oppfylt skal innehaveren på nytt vurderes av lokal Hovedinstruktør. Attestasjon for dette skal påføres hopperens loggbok.

\section{Demolisenser}
Demolisenser utstedes i tre klasser.

\subsection{Demolisens II}
\subsubsection{Kvalifikasjonskrav}
C-lisens og særlig vurderte personlige egenskaper og dyktighet. Minimum 300 hopp. Vurdering foretas av lokal Hovedinstruktør. Særlig utsjekk på hopp med utstyr etter Del 100.

\subsubsection{Rettigheter}
Delta i oppvisningshopp ledet av innehaver av demolisens klasse I.

\subsubsection{Vedlikeholdskrav}
Inneha gyldig C- eller D-lisens.

Dersom kravet til hopping ikke er oppfylt skal innehaveren på nytt vurderes av lokal Hovedinstruktør, jfr pkt 0. Attestasjon for dette skal påføres hopperens loggbok.

\subsection{Demolisens I}
\subsubsection{Kvalifikasjonskrav}
D-lisens samt å ha deltatt i minst 10 oppvisninger. Inneha demolisens II. Særlig vurderte personlige egenskaper og dyktighet. Vurdering foretas av lokal Hovedinstruktør.

\subsubsection{Rettigheter}
Lede gjennomføringen av en fallskjermoppvisning.

\subsubsection{Ansvar}
Leder av en fallskjermoppvisning plikter å forvisse seg om at deltakerne er kvalifisert for oppgaven, samt at de enkelte er innforstått med hvilke oppgaver de skal utføre.

\subsubsection{Vedlikeholdskrav}
Inneha gyldig D-sertfikat.

\subsection{Demolisens tandem}
\subsubsection{Kvalifikasjonskrav}
Godkjent Tandeminstruktør. Inneha Demolisens I. Ha vært godkjent Tandeminstruktør i minst 1 år. Ha gjennomført minst 200 Tandemhopp. Inneha særlige personlige egenskaper og dyktighet. Anbefaling gis av lokal Hovedinstruktør. Sikkerhets- og utdanningskommiteen vurderer og godkjenner. Avslag kan gis uten begrunnelse.

\subsubsection{Rettigheter}
Skal lede oppvisninger der Tandemhopp inngår som en del av oppvisningen.

\subsubsection{Ansvar}
Som pkt 314.2.3 (Demo-I). Innehaver av Demo-T har et særlig ansvar for at Tandemeleven er fysisk og psykisk skikket til å gjennomføre en slik oppvisning, jfr. pkt 315.1.1. Skal selv godkjenne landingsområde for Tandem, enten ved visuell kontroll eller ved hjelp av foto/økonomisk kartverk.

\subsubsection{Vedlikeholdskrav}
Gyldig D-lisens. Gyldig demolisens klasse I. Ha gjennomført minst 50 tandemhopp siste 365 dager

\section{Tandemelev}
\subsubsection{Kvalifikasjonskrav}
Gjennomgått program for Tandemelever, jfr. Del 600, under ledelse av godkjent Tandeminstruktør. Av Tandeminstruktør, være vurdert fysisk og psykisk skikket til å gjennomføre tandemhopp.

\subsubsection{Rettigheter og begrensninger}
Gjennomført og attestert Tandemelevkurs gir rett til å utføre tandemhopp koblet til Tandeminstruktør.

Blir ikke tandemhoppet gjennomført samme dag, skal nytt kurs gjennomføres. Ved hvert utsprang skal det gjennomføres nytt program for tandemelever.

\subsubsection{Tandemelever med funskjonshemming eller sykdom}
Krav som for funksjonsfriske (jfr Del 100). Dersom eleven ikke kan gå til flyet selv eller ikke kan signere på egenerklæring, kommer følgende tilleggskrav:
\begin{itemize}
	\item Elev må gjennomgå en legeundersøkelse, der legen konkluderer med at vedkommende anses som skikket for tandemhopp.
	\item Tandeminstruktøren må oppfylle kravene til hopping med funksjonshemmede som beskrevet i Del 400.
	\item Søknad rettes til SU for hvert enkelt tilfelle (jfr Del 400)
\end{itemize}
