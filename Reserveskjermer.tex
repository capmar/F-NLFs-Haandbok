\part{Reserveskjermer}

\section{Generelt}
Dette kapitlet beskriver hvilke reserveskjermer som er godkjent for bruk i Norge, med eventuelle henvisninger til modifiseringer.

Det henvises til Poynter 1 og 2, referanse til produsenter samt oppdateringslisten på eldre modifikasjoner for videre informasjon om oppdateringer og instrukser for eldre utstyr, og spesielt informasjon om runde reserver.

\subsection{Typegodkjenning av reserveskjermer}
Typegodkjenning av reserveskjermer gjøres etter flere vurderinger av det enkelte utstyret. Selv om enkelt utstyr kan være i alminnelig bruk i andre land betyr ikke dette at de automatisk er godkjent i Norge. Norge er et lite land i mange forhold. For å kunne opprettholde et høyt kunnskapsnivå hos materiellansvarlige er det derfor viktig at vi har et system der det er mulig å bli kjent med forskjellige typer reserveskjermer, både under praktisk bruk og kontroll, uten at det blir for mange forskjellige ting å ta hensyn til. De godkjente reservetypene tilfredsstiller de aller fleste behov for størrelse, pakkevolum og ytelse.

Det er flere kriterier som må oppfylles for at reserveskjermer skal bli godkjente for bruk i Norge, de fleste går direkte på produsenten. Noen viktige momenter er:
\begin{itemize}
	\item Kjennskap til produsent
	\item Erfaring i markedet
	\item Kvalitetssikring, med spesiell vekt på sporbarhet
	\item Kvalifisert personell
	\item Produksjonsmetoder
	\item Godkjenningsordninger – TSO etc
	\item Erfaring fra bruk av tilsvarende skjermer i utlandet
\end{itemize}

Dersom du er interessert i å søke om typegodkjenning av en ny reserveskjerm må tilfredsstillende dokumentasjon på produsentens produksjons- og kvalitetssikrings prosedyrer foreligge som vedlegg til en søknad til SU.

De fleste produsenter av reserveskjermer har en serie skjermer som stort sett er skåret over samme lest. Disse finnes i forskjellige størrelser tilpasset beregnede vektområder. Vanligvis blir nye medlemmer i familien godkjent uten egen ny godkjenningsprosedyre, men de skal allikevel formelt godkjennes av SU.

Dersom produsenten lanserer en helt ny type reserveskjerm som er basert på enten ny teknologi eller produksjonsprosess skal disse først godkjennes etter en totalevaluering.

Det er kun skjermer som er på listen over godkjente skjermer, ref 4 som tillates brukt i Norge.

F/NLF’s Håndbok del 200 tillater bruk av reserveskjermer i inntil 20 år etter produksjonsdato.

\subsection{Diaper}
Diaper er en innretning som monteres ved / på periferibåndet på runde reserver. Den har til hensikt å styre åpningssekvensen slik at duken og alle linene er stramme, før det slippes luft inn i kalotten. Det finnes forskjellige typer, men alle har stort sett samme virkemåte. Det må vises aktsomhet ved pakking av de forskjellige typene.

I hovedsak finnes det to typer: ``Full stow diaper'' der alle liner snøres inn i strikk på diaperen. Når reserveskjermen løftes ut av reservecontaineren løftes alle linene ut samtidig, og de slipper strikkene etter hvert som hopperen faller videre fra reserven.

Disse er vanlige på Phantom og Preserve reserver.

\begin{figure}
	%\includegraphics[width=60mm]{Strekktesting av kalottduk.pdf}
	\caption{Pakking med full stow diaper}
\end{figure}

I tillegg finnes en annen variant, der bare den ene av linebuntene lukker diaperen, og holder den lukket til linene er strukket, og linene er strikket inn i reservecontaineren. Den andre linebunten blir pakket med en slakk som tas inn i første innstrikking i reservecontaineren. Denne varianten er blitt kalt for ``two- stow diaper''.

\begin{figure}
	%\includegraphics[width=60mm]{Strekktesting av kalottduk.pdf}
	\caption{Lukking med two-stow diaper}
\end{figure}

Dersom diaper av denne typen er påmontert reserven SKAL maljeavstanden være minst 10 cm. Dette for å forhindre at løkkene i strikkene kan komme i kontakt med hverandre under åpningssekvensen. Dette var en vanlig modifikasjon på Strong diaper som opprinnelig ble levert med cirka 5 centimeters mellomrom mellom maljene.

\begin{figure}
	%\includegraphics[width=60mm]{Strekktesting av kalottduk.pdf}
	\caption{Two-stow diaper}
\end{figure}

Diapere kan ikke tas av reservene dersom de er opprinnelig testet og godkjent med diaper. Kun fabrikanten kan autorisere fjerning av diaper.

Det finnes mange forskjellige diapere, sjekk produsentens manual og Poynter grundig dersom du er i tvil. Kontakt SU dersom du er i tvil. Pakk aldri en reserve dersom du er i tvil!

Feilaktig pakking med diaper kan forårsake \textbf{full låsing av reserven}!

For tegninger og referanser til forskjellige Diaper typer henvises til Poynter Vol. 2, s. 239. (Merk at bildet til høyre på side 239 viser en Diaper med for liten maljeavstand!)

\subsection{Pilotbånd}
36'' (91 cm) langt når det er ferdigmontert dersom ikke fabrikanten av vedkommende reserveskjerm/pakksekktype har bestemt andre mål.

Min 1/2'' rundvevd nylonbånd (400 kg), Type IV 1'' square weave (1000 lbs), eller tilsvarende type. Minste tillatte bruddstyrke for pilotbånd til reserver er 275 kg.

Festemetode: Sydde løkker eller pålestikk. Dersom pålestikk benyttes SKAL knuten sikres med et par sting tråd (håndsydd).

\begin{figure}
	%\includegraphics[width=60mm]{Strekktesting av kalottduk.pdf}
	\caption{Festeløkke pilotbånd}
\end{figure}

\subsection{Påbudte modifiseringer}
I kolonnen for ``Merknader'' i lista over typegodkjente runde reserver er det henvist til eventuelle modifiseringer og oppdateringer som er utgitt tidligere.

Ytterligere informasjon om disse endringer er å finne i Poynters Manual, etter henvisning i typegodkjenningslista.

Siden disse reservene begynner å bli gamle og utdaterte blir de ikke videre omtalt her.

Er du i tvil ved hovedkontroll av eldre utstyr må du ta kontakt med andre med mer erfaring fra dette utstyret, eventuelt direkte med SU.

\subsection{Acid Mesh}
I 1986 dukket det opp et problem med enkelte reserver, der det ble funnet svak duk på en reserve. Videre undersøkelser viste at det muligens var en flammehemmende syre som nettingen til styreåpningene ble impregnert med (denne nettingen ble også produsert som myggnetting for telt!), som kunne være årsaken til svekkelsen i duken.

I de påfølgende årene ble det iverksatt forskjellige tiltak for å kontrollere omfanget av dette problemet på alle reservetyper. Ph testing av netting og strekktesting av duken ble en standard prosedyre som følges fremdeles i dag. En mengde reserver ble tilbakekalt til fabrikanter, og en del reserver ble dratt i stykker ved strekktesting.

Etter disse årene er ikke problemet gått videre, og selv om testing fremdeles foregår ser det ut til at problemet har løst seg selv, også ved at få nye runde reserver har funnet veien ut til markedet.

For utfyllende og interessant informasjon om dette emnet anbefales lesing i Poynter Manual Vol. 2, – 4.032 side 71.

\section{Montering av reserveskjermer}
Alt monteringsarbeid utføres av F/NLF materiell-kontrollør etter fabrikantens anvisninger og F/NLF’s tilleggsbestemmelser – se videre Kapittel 2 – Pakking og vedlikehold.

\subsection{Montering av firkant reserveskjermer}
Pakking utføres i henhold til fabrikantens pakkemanualer. En beskrivelse av forskjellige pakkemetoder for firkant reserver finnes i Poynter Vol. 2, kap. 9.6, s 369.

\subsection{Montering av runde reserveskjermer}
Pakking utføres i henhold til fabrikantens pakkemanualer. En beskrivelse av generell pakking av runde reserver finnes i Poynter Vol. 1, kap. 9.5.

Alle runde reserver gir best ytelse for styring og synkehastighet dersom de er montert på 4 løftestropper. Alle runde reserver kan monteres på 4 løftestropper, mens noen bare er tillatt montert på 2 løftestropper. Når det gjelder montering av runde reserver må vi kunne henvise til Poynter.

\subsection{Sjakler / links på reserver}
For utfyllende informasjon om forskjellige sjakler og links henvises til Kapittel 9, Komponenter. Der vil det også bli diskutert begrensninger og anmerkninger vedrørende de forskjellige typer links.

Soft links (se kapittel 3 seletøy) som er produsert for bruk på reserveskjermer av godkjent produsent kan benyttes på firkantreserver i Norge.

På reserveskjermer montert på 2 løftestropper (kun runde) er følgende typer godkjent:
\begin{description}
	\item[L-sjakkel] MS 22002 eller tilsvarende.
\end{description}

På reserveskjermer montert på 4 løftestropper (runde og firkant) er følgende typer godkjent:
\begin{description}
	\item[L-sjakkel] MS 22002 eller tilsvarende.
	\item[Rapid-Link] Maillon Rapide: Nr. 5 og 6, samt rustfrie nr. 4. (Rustfrie er stemplet INOX).
	\item[PF-link] ``D'' formet link fra Parachutes de France med venstre dreid låseskrue for bolten. Egen ``tykkere'' utgave for tandem.
\end{description}

\subsubsection{Maillon Rapid links}
Dersom rustfri \textbf{Maillon Rapid-Link nr. 4} benyttes skal enten:
\begin{itemize}
	\item Sliderstopper monteres og festes tilfredsstillende. Dette sikrer blant annet at belastning kommer i riktig retning på Rapid-Link.
	\item Mutter sikres med Loc-Tite, Truelock eller tilsvarende.
	\item Mutter merkes med neglelakk eller lignende slik at man kan se om den har beveget seg.
	\item Brytetråd med 1,5 kg bruddstyrke monteres rundt linene slik at det unngås at de sklir ned på langsiden til linksen og korrekt belastning er sikret.
\end{itemize}

Maillon Rapide links er stemplet med SWL (Safe Working Load) og BL (Breaking Load).

Det er meget viktig at mutteren er skrudd ordentlig fast til gjengene. Korrekt stramming er å skru fast med fingrene pluss cirka en kvart omdreining med skiftenøkkel. Trekk ikke mutteren over gjengene – det vil svekke den. Alle gjenger skal være dekket av låsemutteren.

\textbf{NB: I åpen tilstand har Maillon Rapid links nær ingen styrke.}

Når sliderstoppere (bumpers) benyttes på reserveskjerm skal de festes slik at de ikke kan skli oppover linene (med tråd eller elektrikerstrips). Eks Poynter Vol. 2, 4.155.

Mutteren sitter ikke på midten, og den lengste, frie, enden monteres mot webbingen på løftestroppen slik at gnisninger mot denne unngås.

Kontroller videre at det ikke finnes skarpe metallbiter som stikker ut ved gjenger eller låsemutter.

\subsubsection{L–sjakler}
Når L–sjakler benyttes, skal ``knoken'' med låseskrue vende ut til siden og fremover mot kalotten. Dette vil redusere risiko for at bæreliner legger seg over knoken, og forårsaker feilfunksjoner under skjermåpning.

Dersom det er mer plass på sjaklen enn det er liner fra reserven, skal linene spres jevnt utover stolpen på sjakelen, og skilles med en line som flettes rundt stolpen og mellom linene.

Låseskruene trekkes godt til, med minimum 6 omdreininger på låseskruene. Se forøvrig pkt, 4.113 side 111 i Poynter Vol. 1, samt Kapittel 9- komponenter.

\textbf{VIKTIG: SPEED–LINK eller DRAG CHUTE SJAKLER SKAL IKKE BRUKES.}

Disse kan av utseende forveksles med L-sjakler. De har en festeplate med låseskruer på samme side. Se Poynter Vol. 2, s 105.
