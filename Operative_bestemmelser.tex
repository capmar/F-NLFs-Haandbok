\part{Operative bestemmerlser}
\setcounter{section}{500}

\section{Generelt}
Del 500 av F/NLF Bestemmelser fastlegger forutsetningene for utøvelse av utdanning og praktisk hoppvirksomhet, og foreskriver hvordan organisering og drift skal skje.

Praktisk hoppvirksomhet er å forstå som all virksomhet som inkluderer utsprang fra luftfartøy i luften - elevhopping, treningshopping, konkurransehopping og alle spesielle hopptyper som oppvisnings-, natt-, vann- og oksygenhopping, etc.

All utdanning og praktisk hoppvirksomhet skal skje etter denne F/NLF håndbok som er godkjent som sikkerhetssystem etter Luftfartstilsynets BSL- D 4.2 som regulerer all sivil fallskjermhopping i Norge.

Del 500 beskriver dessuten det ansvar og den myndighet de enkelte ledd i den operative organisasjonen har, og gir instrukser for det operative personellet (se vedlegg 1-4).

Ethvert medlem som befinner seg på hoppfeltet, plikter å rette seg etter de anvisninger som gis av ansvarlig personell. Ethvert klubbmedlem har rett og plikt til å gripe inn ved brudd på bestemmelsene, ref Del 100 og del 200.

\section{Organisasjon}
All utdanning og praktisk hoppvirksomhet innen Fallskjermseksjonen/NLF baseres på følgende ledd:
\begin{itemize}
	\item Sikkerhets- og utdanningskomiteen (SU)
	\item Materiellsjefen (MSJ)
	\item Hovedinstruktør (HI)
	\item Instruktør (I)
	\item Hoppleder (HL)
	\item Hoppfeltleder (HFL)
	\item Hoppmester (HM)
	\item Flyger
	\item Klubbmedlemmer
\end{itemize}

Gjennomføring av enhver form for praktisk hoppvirksomhet skal være organisert med følgende funksjoner:
\begin{itemize}
	\item Overordnet ledelse av en Hovedinstruktør (HI).
	\item Direkte ledelse av en Hoppleder (HL).
	\item Bakketjeneste ledet av en Hoppfeltleder (HFL).
	\item Flygingen ledet av Flyger.
	\item Hopperne i flyet ledet av en Hoppmester (HM).
\end{itemize}

\subsection{Sikkerhets- og Utdanningskomiteen (SU)}
SU oppnevnes av Styret F/NLF og utøver overordnet kontroll med all utdanning og praktisk hoppvirksomhet innen F/NLF. SU utarbeider årlig målsettinger og en arbeidsplan som godkjennes av Styret F/NLF.

Med hjemmel i Bestemmelser for Sivil Luftfart (BSL) D 4-2 §§ 2 og 4, utsteder, iverksetter og håndhever SU, gjennom F/NLFs Håndbok Del 000- 100, 300- 600, bestemmelser for F/NLFs praktiske hoppvirksomhet og utdanning.

SU godkjenner klubbers oppnevnelse av Hovedinstruktør, og utsteder/inndrar fallskjermklubbers og andre organisasjonsenheters Operasjonstillatelser.

SU ivaretar NLFs kontakt med luftfartsmyndighetene i fallskjermrelaterte saker. Ved alvorlige ulykker oppnevnes undersøkelseskommisjon av Styret F/NLF.

\subsection{Materiellsjefen (MSJ)}
MSJ oppnevnes av Styret F/NLF og utøver i samarbeid med SU overordnet kontroll med all materielltjeneste innen F/NLF. MSJ utarbeider årlig målsettinger og en arbeidsplan som godkjennes av Styret F/NLF.

Med hjemmel i Bestemmelser for Sivil Luftfart (BSL) D 4-2 § 4 (2) b, utsteder, iverksetter og håndhever SU gjennom MSJ og F/NLFs Håndbok Del 200 og Materiellhåndboka, bestemmelser for F/NLFs materielltjeneste.

MSJ typegodkjenner alt fallskjermmateriell.

\subsection{Hovedinstruktør (HI)}
Hovedinstruktør er den Instruktør som er oppnevnt av styret i klubb eller annen organisasjonsenhet til å lede all utdanning og praktisk hoppvirksomhet. Oppnevnelsen skal godkjennes av SU. Klubbens operasjonstillatelse er avhengig av Hovedinstruktørens lisensklasse. Dersom spesielle grunner tilsier det, kan SU dispensere fra kravet om HI av klasse I-1 for klubber/enheter som ønsker OT-1.

HI har ansvaret for at all utdannelse og praktisk hopping skjer etter en plan godkjent av klubbstyret, og i henhold til de til enhver tid gjeldende bestemmelser. Planen skal inneholde delmål og resultatmål, samt planer for hvordan klubbens operative drift skal koordineres, gjennomføres og kontrolleres. Planen skal som et minimum inneholde målsettinger for hovedområdene hoppfeltdrift, utdanning, materielltjeneste, kommunikasjon og kvalitetskontroll.

HI organiserer og har ansvar for all praktisk hopping og utdanning som klubben har driftstillatelse for gjennom fordeling av oppgaver til sin operative stab av instruktører, hoppledere, hoppmestere og hoppfeltledere, med tilfredsstillende kvalifikasjoner.

HI godkjenner hoppfelt for alminnelig hopping i henhold til klubbens operasjonstillatelse og innhenter grunneiers tillatelse, varsler politi og lufttrafikktjeneste og ivaretar koordinering mot relevant myndighet. Godkjenning av hoppfelt for alminnelig hopping til varig bruk varsles SU som gjennom kontakt med Luftfartstilsynet påser at hoppfeltet opptas i AIP Norge, ENR 5.5-4 Fallskjermhopping faste steder.

Hovedinstruktøren er faglig ansvarlig for sikkerhet og materiell. Han er faglig bindeledd mellom lokalklubb og SU.

Hovedinstruktøren har ansvaret for klubbens operative arkiv, manifester, kursprotokoller, NOTAM, korrespondanse med SU og F/NLF. Han er ansvarlig for at rutinemessige rapporter som vedrører den operative drift blir utarbeidet og innsendt i rett tid, samt at hendelsesrapporter blir utarbeidet, kommentert og rapportert til SU v/ F/NLFs sekretariat uten unødig opphold.

\subsection{Instruktør (I)}
Instruktør ivaretar og utfører de oppgaver og funksjoner som etter disse bestemmelser krever instruktørkvalifikasjoner, med de rettigheter og begrensninger som gjelder for den enkelte lisensklasse.

\subsection{Hoppleder (HL)}
HL er den person som er overlatt ansvaret for gjennomføring av praktisk hoppvirksomhet på ETT STED i et BESTEMT TIDSROM. HL skal være utpekt før hopping startes av lokal HI. HL skal følge Hopplederinstruksen, se vedlegg 1 til del 500.

\subsection{Hoppfeltleder (HFL)}
HFL utpekes av HL, er underlagt denne, og har til oppgave å administrere, organisere og overvåke hoppfeltet, og påse at hoppingen gjennomføres etter gjeldende bestemmelser. HFL skal følge Hoppfeltlederinstruks, se vedlegg 2 til del 500.

\subsection{Hoppmester (HM)}
HM er den person som har ansvaret for, og kommandoen over hoppere under innlasting i flyet og ved utsprang. HM skal følge Hoppmesterinstruksen, se vedlegg 3 til del 500.

\subsection{Fly/flyger}
Fartøysjef på ikke kommersielt luftfartøy som anvendes for fallskjermutsprang er en del av den operative hopporganisasjonen. Han skal være kjent med F/NLFs bestemmelser Del 500 og skal spesielt følge bestemmelsene i Flygerinstruksen, se vedlegg 4 til del 500. Fartøysjef på kommersielle luftfartøy forutsettes å ha egne prosedyrer for fallskjermutsprang i sine operative prosedyrer, godkjent av Luftfartstilsynet. Jfr. BSL D 4-2 § 9.

Bruk av helikopter for fallskjermutsprang må kun skje gjennom kommersielle flyoperatører. Jfr. BSL D 4-2 § 8.

\subsection{Klubbmedlemmer}
Ethvert klubbmedlem som deltar i praktisk hoppvirksomhet skal kjenne instruksene i Del 500 med vedlegg, og rette seg etter disse.

\subsection{Kombinasjon av funksjoner}
HFL tillates ikke å hoppe, og skal under hopping til enhver tid være tilstede på hoppfeltet. Funksjonen som HL kan kombineres med funksjon som HFL, HM eller Flyger.

HL kan dersom aktiviteten krever det utpeke flere HFL’er, som skal virke sammen. HL har ansvar for oppgavefordelingen mellom disse.

\section{Operasjonstillatelser}
Utdanning og praktisk hoppvirksomhet kan bare organiseres og gjennomføres med tillatelse fra SU eller den SU har bemyndiget. Slik tillatelse kan utstedes til:
\begin{itemize}
	\item Fallskjermklubber tilsluttet NLF.
	\item Driftsorganisasjoner (riks- og regionalsentra eller samarbeidsorganisasjoner for grupper av klubber) godkjent av F/NLF.
	\item Prosjektorganisasjoner etablert av F/NLF.
	\item I særlige tilfeller, enkeltpersoner.
\end{itemize}

Tillatelse utstedes i følgende former:
\begin{itemize}	
	\item Operasjonstillatelse (OT), som utstedes i to klasser, avhengig av klubbens/organisasjonens forutsetninger, behov, kvalifikasjoner og omfang, eller
	\item Eksperimenttillatelse (XT), som gjelder spesiell forsøksvirksomhet og prosjekter.
\end{itemize}

\subsection{Operasjonstillatelse klasse 1 (OT 1)}
Organisasjon som innehar OT 1 kan organisere og gjennomføre alle typer fallskjermutdanning og praktisk hoppvirksomhet innenfor et bestemt geografisk område.

OT 1 utstedes normalt bare til etablerte fallskjermklubber eller godkjente driftsorganisasjoner, under forutsetning av at følgende betingelser kan dokumenteres oppfylt:
\begin{itemize}	
	\item For klubber: Er opptatt i NLF.
	\item For driftsorganisasjoner: Er godkjent av F/NLF.
	\item Har Hovedinstruktør klasse I/1 godkjent av SU.
	\item Har Materiellkontrollør godkjent av F/NLF.
	\item Disponerer tilstrekkelig utdanningsmateriell.
	\item Disponerer hoppfelt iht Del 100
	\item Har tilstrekkelig med operativt personell.
\end{itemize}

\subsection{Operasjonstillatelse klasse 2 (OT 2)}
Organisasjon som innehar OT 2 kan organisere og gjennomføre fallskjermhopping for hoppere med A-sertifikat og høyere, innenfor et bestemt geografisk område, men ikke teoretisk eller praktisk utdanning i fallskjermhopping som definert i Del 600, eller spesielle hopptyper som definert i Del 100.

OT 2 utstedes til klubber eller organisasjoner som kan dokumentere at de tilfredsstiller følgende krav:
\begin{itemize}
	\item For klubber: Er opptatt både i NLF.
	\item For driftsorganisasjoner: Er godkjent av F/NLF.
	\item Har Hovedinstruktør klasse I/2 godkjent av SU.
	\item Disponerer hoppfelt iht Del 100.
\end{itemize}

\subsection{Eksperimenttillatelse (XT)}
Organisasjon som innehar XT kan organisere og gjennomføre hopping i forsøks- og utviklingsøyemed, inklusive testing av fallskjermutstyr.

XT utstedes til organisasjoner eller enkeltpersoner som gjennomfører programmer etter oppdrag fra, eller godkjenning av, SU. Følgende minstekrav gjelder:
\begin{itemize}
	\item Ansvarshavende skal være godkjent av SU spesifikt for programmet XT utstedes for.
	\item Dersom XT utstedes til en organisasjon, skal denne enten være en fallskjermklubb med OT 1 eller 2, eller en organisasjonsenhet godkjent av Styret F/NLF.
	\item Planer og programmer for virksomheten skal være utarbeidet i detalj på forhånd. Endringer av godkjente opplegg skal godkjennes av SU.
\end{itemize}

\subsection{Saksbehandling}
Søknad om Operasjonstillatelse og Eksperimenttillatelse sendes SU sammen med dokumentasjon for at kravene ovenfor tilfredsstilles.

Gjelder søknaden OT, oversendes den til Styret F/NLF som fastsetter det geografiske området for OT.

Vurderingen av dokumentasjonen foretas av SU. Som dokumentasjon for disponering av hoppfelt kreves normalt grunneierens og politiets skriftlige tillatelse. For hoppfelt på Luftforsvarets stasjoner dokumenteres avtale med Stasjonssjefen.

Behov for nødvendig utdanningsmateriell vurderes av SU i forhold til det planlagte utdanningsvolumet samt klubbens medlemstall.

Behovet for operativt personell vurderes av SU som det antall som er nødvendig til dekning av funksjonene HL, HFL og HM for de hopptyper OT gjelder, sett i forhold til klubbens medlemstall og planlagte aktivitetsnivå.

\subsection{Varighet}
Operasjonstillatelsen forutsetter kvartalsvis rapportering av aktivitet. Manglende kvartalsrapportering innen fristens utløp medfører at OT automatisk mister sin gyldighet.

OT forutsetter at Hovedinstruktør har utarbeidet en skriftlig plan for sitt virke, jfr. punkt 502.3 For at OT skal ha fortsatt gyldighet for et nytt kalenderår, skal HI hvert år revidere sin plan og sende en kopi av denne til SU innen 1. februar det året den har gyldighet for. Manglende innsending av HIs plan medfører at OT automatisk mister sin gyldighet.

XT utstedes normalt for ett kalenderår av gangen, om ikke spesielle forhold tilsier annet.

SU kan når som helst inndra enhver tillatelse midlertidig eller permanent om det anses nødvendig av hensyn til sikkerheten. Inndragning av OT for mer enn ett år eller permanent, kan ankes til Styret F/NLF, som avgjør saken med endelig virkning.

Manglende representasjon ved årlig HI-seminar kan medfører inndraging av OT.

\section{Klarering av aktivitet - NOTAM}
\subsection{Generelt}
Fallskjermutsprang som helt eller delvis skal finne sted innenfor områder med obligatorisk radiosamband med flygekontrolltjenestens enheter (CTR/TMA), koordineres i god tid forut for tidspunktet for hoppingen med den enhet av Flygekontrolltjenesten som har ansvaret for luftrommet.

Avinors AIS/NOTAMkontor er sertifisert etter ISO standarden, og Norges Luftsportforbund er godkjent som én av en få pålitelige kilder.

\subsection{NOTAM}
Det skal søkes om NOTAM dersom hoppingen skjer utenfor faste hoppfelt og er av større omfang og/eller foregå om natten. Større omfang er justert av ICAO og defineres som mer enn 10 hopp pr dropp og /eller mer enn 10 løft totalt og/eller mer enn 100 hopp totalt og/eller aktivitet ut over 24 timer.

\subsubsection{NOTAM}
Innsendes av F/NLFs sekretariat etter anmodning fra lokalklubbens Hovedinstruktør.

NOTAM-anmodning til F/NLF skal alltid inneholde følgende data:
\begin{enumerate}
	\item Navn på den klubb som organiserer hoppingen.
	\item Korrekt stedsnavn på hoppfeltet som skal benyttes.
	\item Hoppfeltets posisjon i grader og hele minutter nord og øst Greenwich. NB! Militære kartreferanser. NGOs betegnelser eller Oslo meridiansystem skal ikke anvendes.
	\item Nøyaktig tidsangivelse for hoppingen, med fullstendig dato og klokkeslett for start og avslutning i lokal tid.
	\item Største utsprangshøyde i fot AMSL (Above Mean Sea Level).
	\item Hoppingens omfang, med antall hopp.
	\item Det skal oppgis fullt navn og fallskjermsertifikatnummer på den enkeltperson som er HL for den hopping som NOTAM-søknaden gjelder for.
	\item Dersom det ønskes større radius for NOTAM enn 2 nautiske mil, må dette angis spesielt.
	\item Dato og tidspunkt da Luftromseier ble kontaktet/varslet (Den luftromskontrollen som har ansvaret i området)
	\item Referanseperson hos luftromseier (enhet og person).
\end{enumerate}

\section{Handlingsinstruks ved ulykke}
SÆRTRYKK/VEDLEGG TIL DEL 500, ref pkt 504.6 og 505.5.4

Dette særtrykk skal, i utfylt stand, alltid være tilgjengelig på hoppfeltet, og kjent for alt det operative personellet.

\subsection{Prioriterte tiltak}
\begin{enumerate}
	\item Varsle 113
	\item Start førstehjelp
	\begin{itemize}
		\item Frie luftveier, stans synlige blødninger, hold pasienten varm
		\item Bevisstløse som puster bør legges i sideleie - pass da på å bevege hodet, nakke og rygg som en enhet
		\item Hvis personen ikke puster normalt eller slutter å puste, start HLR
	\end{itemize}
	\item Tilkall politi: tlfnr:....................
	\item Varsle Flygekontrollenhet: tlfnr:....................
	\item Varsle F/NLFs sentrale organisasjon v/ Fagsjef: tlfnr:....................

	evt. Leder F/NLF: tlfnr:....................

	evt. Leder SU: tlfnr:...................

	som vil sørge for at sentrale tiltak iverksettes.
	\item Varsle egen Hovedinstruktør: tlfnr:....................
\end{enumerate}

NB! Det er politiets ansvar å underrette forulykkedes pårørende.

Fagsjef og/eller Leder SU vil så langt det er mulig reise til ulykkesstedet for å bistå.

\subsection{Forhold til presse/media}
Fagsjef bistår ift pressen, ta kontakt for nærmere avklaring, se også Handlingsplan ved ulykke. Som norm forsøk å avstå fra intervju på ulykkesdagen. Henvis ev til F/NLF sentralt. Bruk om nødvendig denne standarduttalelse:
\begin{itemize}
	\item ``Ulykken vil bli etterforsket av politiet og av kommisjon fra Norges Luftsportsforbund. Vi avstår fra å kommentere årsaksforholdet inntil kommisjonens rapport foreligger.''
\end{itemize}

NB! Oppgi IKKE forulykkedes navn eller andre personalia. Det er politiets ansvar.

Pass på at andre på feltet ikke kommer med egne uttalelser som er i strid med de ovenstående retningslinjer.

Ved særlig pågående pressefolk:

Gjør disse oppmerksom på ``vær varsom''-paragrafen som er trykket på deres eget pressekort. Der står det blant annet:
\begin{itemize}
	\item ``Vis særlig hensyn overfor personer som ikke kan ventes å være klar over bivirkningen av sine uttalelser. Misbruk ikke andres følelser, uvitenhet eller sviktende dømmekraft.''
\end{itemize}

\subsection{Bistand til politiet}
Utover det som er nødvendig for å gi førstehjelp og assistere lege, skal ikke noe flyttes før politiet er ankommet.

Bistå politiet med bevissikring. Følg deres anvisninger. Merk av. Ta fotografier om mulig. Lever ett eksemplar av F/NLFs Håndbok til politiet med forklaring om at dette er F/NLFs operasjonsgrunnlag i henhold til Bestemmelser for Sivil Luftfart (BSL) D 4-2.

\subsection{Rapport}
Hendelsesrapport skal skrives og sendes til F/NLF innen 48 timer.

\subsection{Rapporteringssystemet}
Hendelsesrapporteringssystemet er elektronisk og Hovedinstruktør i klubbene har alle brukernavn og passord. Når det er oppstått en rapporteringspliktig hendelse(se vedlegg 1 til del 500 pkt 506.7.3), skal HL fylle ut en hendelsesrapport som sendes HI for vurdering og kommentarer. HI sender inn denne via websiden til Fagsjef F/NLF innen 48 timer. Dersom det er behov for ytterligere undersøkelser kan det avtales innsending av en midlertidig rapport.

Fagsjef tar ut hendelsesrapporter jevnlig ila året og sender disse til SU sine medlemmer. SU vurderer fortløpende om det er behov for å gjøre endringer innen sikkerhetssystemet Ved årets slutt tar Statistikkansvarlig og presenterer statistikken for Fagseminaret og basert på funn og trender samt diskusjon vurderer SU tiltak der dette anses nødvendig.

For detaljerte bestemmelser vises det til HB pkt 506.7.2