\part{Kontroll og Vedlikehold}

\section{Generelt}
Forutsetningene for å kunne avgjøre om en fallskjerm er luftdyktig, er at man har erfaring i å vurdere fallskjermmaterialets tilstand og kvalitet. Det er nødvendig å vite hvilke styrkeforhold som fallskjermen er satt sammen etter og de sikkerhetsmarginer som gjelder.

En god hjelp her, vil være å studere \emph{Poynter Vol. 2, kapittel 9} (side 351) ``Parachute Inspection and Packing'' meget nøye \cite{poynter2}. Deretter bør man praktisere dette sammen med andre, erfarne materiellkontrollører.

\subsection{Hovedkontroll}
\label{sec:kapittel8}
Hovedkontroll består av omhyggelig visuell kontroll av hele utstyret med alle komponenter utført av materiellkontrollør.

Det skal brukes et skjema med sjekklister for gjennomføring av hovedkontroll (et eksempel er vedlagt dette kapittel). Se forøvrig \emph{Poynter Vol. 2, kapittel 9} for utfyllende informasjon.

Fallskjermutstyr skal underkastes hovedkontroll i henhold til bestemmelser i F/NLFs håndbok del 200.

Det vil lette arbeidet, og gi en sikrere gjennomføring dersom en legger opp en hovedkontroll ved systematikk og orden. Det beste er å ha et skjema som leder en gjennom kontrollen, slik at ikke kontroll av enkeltdeler og komponenter overses.

Hovedkontroll skal gjennomføres i følgende tilfelle:
\begin{enumerate}
	\item Før skjermen kan tas i bruk (førstegangskontroll).
	\item For \emph{elevutstyr} – senest 6 måneder etter siste hovedkontroll.
	Denne hovedkontrollen skal også inkludere funksjonsprøving av FXC i trykkammer. (\ref{sec:kapittel8})
	\item For \emph{tandemutstyr} – senest 6 måneder etter siste hovedkontroll.
	\item For \emph{privateid utstyr} – senest 12 måneder etter siste hovedkontroll.
	\item Når det finnes ønskelig av ansvarlig pakker, materialforvalter eller eier.
	\item Etter utført større reparasjon.
	\item Ved utskifting av hovedkalott kan denne hovedkontrolleres separat.
\end{enumerate}

Fallskjermer som er forfalt for hovedkontroll skal under ingen omstendighet nyttes for praktisk hopping før kontrollen er utført. Hovedkontroll skal innføres i skjermens hovedkontrollkort med angivelse av resultat av kontrollen. Hovedkontrollkort skal medfølge skjermen ved reparasjon og kontroller.

Skjerm som underkastes hovedkontroll kan bli:
\begin{enumerate}
	\item Godkjent.
	\item Henvist til reparasjon / utskifting av bestemte komponenter.
	\item Kassert
\end{enumerate}

\subsection{Brukskontroll}
\label{sec:Reservehandtak}
Brukskontroll skal gjennomføres i henhold til bestemmelser i F/NLFs håndbok del 200, og omfatter alle kontroller spesifisert for hovedkontroll, med unntak av de kontroller som krever at reservepakksekken åpnes.

Brukskontroll er beregnet for en kontinuerlig visuell kontroll av de deler av utstyr som er mest utsatt for slitasje, som hovedskjerm, liner, og deler av seletøy.

Dersom en er i tvil om utstyrets tilstand bør full hovedkontroll gjennomføres.

\subsection{Ompakksyklus}
F/NLF’s håndbok del 200 angir syklus for ompakk av fallskjermer.

En fallskjerm er luftdyktig i inntil 6 måneder siden den sist ble pakket. Dette gjelder både for hovedskjermer og reserver.

Etter 6 måneder skal skjermen åpnes, ristes og luftes og pakkes igjen før den er luftdyktig.

\subsection{Kassasjon}
Alle reservefallskjermer skal kasseres etter bestemmelser i del 200 i F/NLF’s Håndbok.

Fallskjermer av organiske stoffer (bomull, silke) og/eller kalotter hvor bomullstråd er nyttet for søm, er ikke tillatt brukt

For fallskjermer som har vært i kontakt med sjøvann (saltvann) gjelder spesielle regler:
\begin{itemize}
	\item Komponenter som har vært i kontakt med sjøvann (saltvann), i mer enn 48 timer eller som ikke er blitt skyllet tilfredsstillende innen 24 timer etter berging, skal kasseres.
\end{itemize}

\section{Gjennomføring av hovedkontroll og brukskontroll}
Hovedkontroll er i det følgende beskrevet på en generell måte. Vær klar over variasjoner på forskjellige typer utstyr.

Det er forsøkt å legge opp hovedkontrollen slik at en kan følge en naturlig gang i prosessen, ved at ting som hører sammen kontrolleres samtidig eller etter hverandre.

Det er i det følgende tatt hensyn til at delene som tilhører seletøyet (pilot, pilotbånd, innerbag og løftestropper) kontrolleres sammen med seletøyet, selv om de er festet til fallskjermen. På tilsvarende måte er links (sjakler) tatt med under kontroll av skjermene.

\subsection{Seletøy / pakksekk}
\label{sec:Seletoy_og_LOR}
\begin{description}
	\item[Pakksekken] inspiseres for kutt, slitasje, frynser og sømmer som er gått opp. Låseløkkene inspiseres for slitasje. Velcro, maljer og kantebånd kontrolleres.

	\item[Låseløkker] av for grov line kan gi unødig hard trekkraft, spesielt i kombinasjon med handdeploy system der den bøyde låsepinnen kan bli sittende så hardt fast at pilotskjermen ikke klarer å åpne pakksekken.
	Låseløkker byttes ved tegn på slitasje. Låseløkker skal være fingertrappet eller sydd sammen for å hindre at det dannes en åpen løkke inne i containeren.

	\item[Seletøyet] inspiseres for kutt, slitasje, frynser, løse– eller brutte sømmer eller sting.

	\item[Handdeployed lomme] kontrolleres for skader, velcro, og nødvendig elastisitet.

	\item[Metalldeler] inspiseres for rust, unormal slitasje, skarpe kanter og deformiteter / sprekker. Fjæringen i kvikkroker, karabinkroker og lomme for reservehåndtak prøves. Rustavsetning på webbing kan svekke webbingen med så mye som 50\%!

	\item[Løftestropper], sjakler og bæresnorer inspiseres ved å følge dem opp mot kalotten. Forbindelsen til selen kontrolleres (3-rings). Ringene skal ha metall til metall kontakt ved lett belastning. Løftestroppene inspiseres for kutt, frynser, flekker, løse eller brutte sømmer eller sting.. Bremsefeste med ringer kontrolleres. Malje og låseløkke for kutt kabel kontrolleres for slitasje. Webbingen rundt ringene kan med fordel mykes opp.

	\item[Reserveløftestropper med bremsesetting] kontrolleres.

	\item[Reservehåndtak] inspiseres. Pinner inspiseres særlig nøye. Det kontrolleres at kabelens kule er i orden og festet på godkjent måte. Se forøvrig \ref{sec:Reservehandtak}.

	\item[Kutthåndtak] trekkes helt ut og inspiseres for sår i plastikken. Kablene kan med fordel tørkes av og settes inn med et tynt lag 3-1 olje eller silicon. Kontroller kablenes innbyrdes lengde (se \ref{sec:Seletoy_og_LOR}). Kontroller at endene er brent av og ingen metalldeler fra wire stikker ut.

	\item[Kabelføringer for reserve og kutthåndtak] kontrolleres for riktig innfesting, skader i maljer etc.

	\item[Hand deploy pilot] og senterline kontrolleres for sømmer og skader. Kontroller rundt festet for kule/hackey. Eventuell inntrekkbar / kollapsbar pilotskjerm skal fungere og ikke ha skader, eller være slitt.

	\item[Pilotbånd] kontrolleres for innfesting og skader. Kontroller pinne på pilotbånd for skader og deformiteter. Kontroller velcro. Kontroller innbyrdes lengder på pilotbånd og senterline ved kill-line piloter, og spesielt for friksjonsslitasje.

	\item[Innerbagen] inspiseres for mulige skader. Særlig viktig er festet for pilotskjermen og bunnmaljen. Det nødvendige antall strikk for innsløyfingen skal være på plass og i god tilstand. Slitt velcro byttes, eller splittbag kan syes igjen.

	\item[Pilotskjermen] og pilotbåndet på reserven

	\item[Fribagen til reserven] kontrolleres sammen med pilotskjermen. Kontroller for slitasje og kutt etc på bånd og duk, kontroller videre spesielt safety stow, maljefeste og velcro.

	\item[Utløserline] kontrolleres for kutt, slitasje, brutte sømmer og sting. Ankerkroken med låsepinne prøves og inspiseres inspiseres for frynser og rakning. Fjæren prøves og det kontrolleres at den er rett. for rust og deformiteter. Pinne inspiseres for feste, rust og deformiteter.
\end{description}

\subsection{Hovedkontroll av firkantskjermer}
Her er beskrevet generelle regler for hvordan man foretar en hovedkontroll av firkantskjermer. Antall celler, antall liner, o.l., kan variere fra kalott til kalott. Her henvises det til fabrikantens dokumentasjon. For ytterligere opplysninger henvises til \emph{Poynter Vol. 2, 9.3.19, s 359}.

Inspeksjon av vingskjermer gjøres lettest når kalottens hale er hengt høyt, slik at celleåpningene (nesen) henger fritt. På denne måten er det lett å komme inn i cellene for å kontrollere disse, samt sjekke linene og linekontinuitet. Alternativt må duken inspiseres del for del.

\subsubsection{Inspeksjon av underduken}

\begin{figure}
	%\includegraphics{Inspeksjon av underduk.pdf}
	\caption{Inspeksjon av underduk}
\end{figure}

Spre ut kalotten med underduken opp og halen mot sekk/seletøyet.

Plasser deg selv på den ene siden. Kontroller stabilisatorpanelet og endecellen.

Kontroller linefester, sømmer, forsterkerbånd og fallskjermduk for skader og (mistenkelige) flekker.

Trekk neste celle til deg og gjør likedan.
Forsett slik, hver enkelt celle inntil du har kommet til stabilisatorpanelet på den andre siden.

\subsubsection{Inspeksjon av toppduken}
Snu seletøyet rundt, med pakksekken ned, og toppen mot kalotten.

\begin{figure}
	\caption{Inspeksjon av toppduk}
	%\includegraphics{Inspeksjon av toppduk.pdf}
\end{figure}

Legg alle cellene oppå hverandre med venstre endecelle på toppen.

Ta tak i hele kalotten, på midten av sømmen, i toppen av cellene og brett dem over mot sekk/seletøyet oppå linene.

Kontrollér stabilisatorpanelet og endecellene.

Trekk neste celle mot deg mens du kontrollerer den. Fortsett slik til du kommer til stabilisatorpanelet på den andre siden.

\subsubsection{Inspeksjon av ribber og celler}
Legg sekk/seletøyet med seletøyet opp (pakksekken ned) og toppen mot kalotten.

\begin{figure}
	%\includegraphics{Inspeksjon av ribber og celler.pdf}
	\caption{Inspeksjon av ribber og celler}
\end{figure}

Ta opp venstre endecelle i celleåpningen (fronten). Gå så opp på en stol. Løft opp celleåpningen til skulderhøyde og spre cellen slik at du kan komme inn i åpningen. Dra celleåpningen over hodet og dra deg helt inn til halen og ut igjen mens du inspiserer. Kontrollér alle cellene på denne måten.

Vær på jakt etter eventuelle rifter, brannsår, forskyvning av tråder, brutte sting i sømmer samt mistenkelige flekker.

Et kryssventileringshull som har revnet kan få kalotten til å dreie.

Vær særlig nøye med å kontrollere linefestene nær fronten av kalotten. Kontrollér alle sømmer og forsterkerbånd. Foreta en linesjekk ved å holde i toppen av høyre endecelle og kontroller at høyre ``A''–line løper fritt gjennom slideren til yttersiden av sjakkelen på fremre høyre løftestropp.
Pass på at løftestroppene ikke er vridd.

Ta så tak i toppen av neste celle og kontroller at denne ``A''–linen løper fritt gjennom slideren til neste posisjon, innover på sjakkelen.

Fortsett på denne måten langs fronten av kalotten og kontroller at linene løper fritt og at antallet er likt på de korresponderende sjaklene.

\subsubsection{Inspeksjon av liner og sjakler}

\begin{figure}
	%\includegraphics{Linekontinuitet.pdf}
	\caption{Linekontinuitet}
\end{figure}

Lokaliser ``C''–linen til venstre endecelle og kontroller at den løper fritt gjennom slideren til ytre venstre side av sjakkelen, på venstre bakre løftestropp.

Lokaliser ``C''–linen til høyre endecelle og kontroller at den løper fritt gjennom slideren og er festet til ytre høyre side av sjakkelen på høyre bakre løftestropp.

Kontroller alle liner, gjennomføringer, fingertrappinger og sømmer. Se etter brutte fibre og/eller slitasje. Kontroller at linelengder er like over hele kalotten.

Kontroller at sliderstoppere er montert og er festet tilfredsstillende på ytterlinene slik at ikke slideren kan bli presset opp over duken under åpningssekvensen.

Kontroller styrelinene spesielt, for slitasje ved fingertrappingene. (Feste for bremser). Ta ut tvinn fra styrelinene, og kontroller innbyrdes lengde.

Kontroller at ``Rapid Links'' – sjaklene er riktig montert, og tilstrammet. Beskyttere for slidermaljer (bumpers) anbefales påmontert. Kontroller at disse ikke er uslitt, og at de er festet på en betryggende måte så de ikke kan skli oppover linene.

Dersom du skrur mutteren på med for stor kraft, kan du sprenge istykker mutteren. Se derfor godt etter om du finner sprekker i den. Hopperen bør sjekke at sjaklene ikke har løsnet og eventuelt stramme dem igjen.

Kontroller om slideren har skader i duk eller i maljene, og at den er montert i riktig retning. Skarpe kanter på maljene vil raskt kunne skjære gjennom og ødelegge bærelinene.

Små grader i maljene kan til nød poleres glatte med fin stålull.

\subsection{Reserveskjermen}
\begin{description}
	\item[Firkantreserver] kontrolleres på samme måte som hovedskjermer ( 2.2.2.2, Hovedkontroll av firkantskjermer).

	\item[Rund reserve] strekkes ut og inspiseres felt for felt. Duken inspiseres for kutt, frynser og brannskader, flekker, brudd, hull, løse eller brutte sømmer eller sting, svake punkter eller fremmedlegemer.

	Vevingen skal ikke prøves ved riving eller sterkt strekk, da dette kan svekke stoffet eller øke skadens omfang.

	\item[Linene] inspiseres ved å følge og se etter kutt, brudd, svekkelser, brannskader, løse sømmer brutte sømmer og sting. Liner kontrolleres samtidig for linekontinuitet og riktig linelengder.
\end{description}

\subsubsection{Strekktest}
Alle reserver skal strekktestes ved første gangs kontroll. Testen foretas som stikkprøvekontroll i to retninger. Dersom reserven består av stoff fra forskjellige duk typer (farver – nyanser) bør strekktest foretas på de forskjellige områder.

Strekktest kan foretas av Materiellkontrollør ved ompakk eller hovedkontroll etter vurdering.

Bruk riktig utstyr: Tenger for strekktesting – Aerostar P/N 51406M og fjærvekt med skala til minst 25 kilo. Skalaen må være kalibrert så den viser nøyaktig.

Sett merke på kalottstoffet 7,5 cm fra hverandre, klemmene festes i stoffet som vist i følgende figur.

\begin{figure}
	%\includegraphics[width=60mm]{Strekktesting av kalottduk.pdf}
	\caption{Strekktesting av kalottduk}
\end{figure}

\begin{itemize}
	\item F-111 stoff strekkes til 18 kilo (40 lbs) i 3 sekunder
	(eks: Phantom 22 – 24 – 26, Pioneer K-serien)
	\item LoPo stoff strekkes til 16 kilo (35 lbs) i 3 sekunder
	(eks: PISA 26’, Strong 26 LoPo, Pioneer 26 LoPo) Dersom stoffet har tegn på svekkelse er materiellet kassert.
\end{itemize}

\subsection{Nødåpner}
Nødåpnere kontrolleres i henhold til instrukser i kapittel 8 – Nødåpnere og Instrumenter.
Husk at dato for eventuell funksjonstest og batteribytte skal innføres i hovedkontrollkortet.

\section{Vedlikehold og rengjøring}
Se \emph{Poynter Vol. 2, kapittel 9.4 og 7.92} for utfyllende informasjon om vedlikehold og rengjøring.

\subsection{Generelt}
Rengjøring av fallskjermer skal begrenses mest mulig og må bare utføres når det er nødvendig for å hindre feilåpning eller forringelse.

Når rengjøring er nødvendig, må den utføres manuelt ved utristing og/eller børsting eller punktrensing.

Rengjøringen begrenses til det tilsølte området.

\subsection{Rensing}
Når børsting eller utristing ikke fører frem, må fallskjermen punktrenses, dvs. rensingen begrenses til det tilsølte området. Dette må gjøres noe forskjellig for de forskjellige fallskjermtyper, avhengig av hvilket stoff de er laget av og hvilken del av fallskjermen som må renses.

Bomullskomponenter, for eksempel innerbager, punktrenses ved kraftig børsting med halvstiv børste eller ren klut, som er fuktet med det tørrensende oppløsningsmiddel INHIBISOL. Det rensede området skylles deretter med oppløsningsmiddelet. Man skal forsikre seg om at det oppløsningsmiddel som brukes IKKE kan skade stoffet.

Nylonkomponenter punktrenses som bomullskomponentene eller med en oppløsning av varmt vann og Zalo eller lignende. Oppløsningen lages ved å løse opp 30 – 100 gram i ca. 4 liter varmt vann (inntil 40 grader C). Komponentene skal ikke vris, hverken under eller etter skyllingen.

Rensing av seletøy kan gjøres i et badekar eller lignende med lett såpevann, der det får ligge en stund. Bruk mye vann, og skift vann flere ganger ved skylling. Det er en fordel å smøre metalldelene inn med vaselin først.

\subsection{Vasking}
Generelt bør vasking av fallskjermer unngås, da dette vil slite i duken, og øke luftgjennomstrømningen. Dersom allikevel skjermen må skylles av, bruk rikelig mengder rent vann, for eksempel i et badekar, og vær forsiktig med å vri eller utøve trykk på duken.

Seletøy kan vaskes i lunkent vann med en svak såpeblanding, dersom ikke børsting gir tilfredsstillende resultat. Det kan ligge noen timer, til smuss og skitt har løsnet. For sterk såpekonsentrasjon kan fjerne belegget på nylonduken. Bruk minst tre - fire skyllinger i rent vann, og beveg seletøyet forsiktig opp og ned i vannet uten å vri på enkeltdeler. Borrelås bør blendes av før vasking.

\subsection{Tørking}
Tørking av fuktig fallskjerm foregår best opphengt hvor temperaturen ikke skal overskride 40 grader C. Skjermen må ikke komme nær varmekilde. Sterk varme- påvirkning vil kunne ødelegge skjermen eller forårsake klebing. Tørkerommet må være godt gjennomluftet. Skjermene må ikke utsettes for direkte sol. Lys fra lysstoffrør kan også være skadelig over tid.

Rund reserve henges etter toppløkken. Fallskjermen bør henge mest mulig fritt for å sikre tørking av alle komponenter. Når høyden under taket er liten, kan bæresnorene flettes slik at også selen kommer opp fra gulvet.

Gjennomvåte pakksekker og seletøy som tørker meget sent, og må henges til tørk og få god luftsirkulasjon så snart som mulig etter bruk, for å motvirke mugg, jordslag, rust etc. Metalldeler og steder hvor stroppene er sydd sammen i flere lag, må spesielt kontrolleres før selen tas ned fra tørking. Det kan være nødvendig å sette metall lett inn med vaselin for å motvirke rust. Rust og metalldeler avsettes på de øvrige komponenter og spiser opp og svekker stoffet. Snu gjerne seletøyet med jevne mellomrom slik at det tørker fortere. Lett bruk med hårtørrer kan hjelpe tørring rundt metalldeler.

Gjennomvåte skjermer skal henges til tørk i minst 48 timer. Når fuktigheten er liten, er et mindre antall timer nok. Fallskjermer må alltid være tørre før lagring. Tiden bestemmes forøvrig av skjermens forskjellige komponenter. Fuktighetsgraden i tørkerommet skal ikke overskride 60\% når skjermen tas ned.

\subsection{Lagring}
Lager for fallskjerm skal være godkjent av materiellkontrollør.

Lagerrom for fallskjermer må være tørt med en fuktighetsgrad under 60\%. Temperaturen bør være 15 – 20 \ensuremath{^\circ} C. Rommet bør være utstyrt med temperatur– og fuktighetsmåler. Lagerrommet skal være rent. Olje, fett, såpe, bensin, maling, lakk, kjemikalier, rotter og mus må ikke forekomme på rommet.

Fallskjermer skal i sin helhet lagres slik at de ikke blir utsatt for sol. Mer enn to skjermer skal ikke lagres oppå hverandre.

Fallskjerm som har vært i pakket stand i mer enn 6 måneder er ikke luftdyktig. For å bli luftdyktig igjen, må fallskjermen luftes og pakkes på nytt. Håndboka F/NLF gir retningslinjer for påbudt ompakksyklus.

\subsection{Transport}
Transport av fallskjermer skal overvåkes av pakker. Fallskjermer må ikke transporteres sammen med kjemikalier eller andre ting som kan skade materiellet. Skjermer må heller ikke transporteres slik at de tilsøles. Ved bruk av offentlig transportmidler må en særlig være oppmerksom på ytre påvirkning i form av ukyndig håndtering, tilsøling mv. Skjermen bør derfor emballeres for forsendelse i egnede sekker, kasser eller kartonger, med utvendig merking som angir innholdet.


\section{Instrukser for Materiellkontrollører}
\subsection{Utstyr og hjelpemidler for pakking}
Materiellkontrollør bør være utstyrt med godt utvalg av nødvendig håndverktøy. Verktøyet skal være tilpasset bruk, og helst ikke brukes til andre formål enn til fallskjermmateriell.

\textbf{Generelt verktøy}
\begin{itemize}
	\item Plomberingsutstyr
	Tang, Tråd og 10 mm plomber
	\item Målebånd
	\item Saks
	\item Hammer med nylonhode
	\item Assorterte skrutrekkere
	\item Assorterte nåler (med rund spiss)
	\item Fastnøkler i 7, 8, 9 og 11 mm, eventuelt skiftenøkkel
	\item God avbitertang
	\item God nebbtang
	\item Lighter
	\item Diverse pullupcords (opptrekksnorer) tilpasset looper.
	Også dobbelløkke for lukking av fribager for firkantreserver.
	\item T-håndtak
	For lukking Pop-Top
	\item Hjelpepinner
	Merket med langt kontrastfarvet bånd
	\item Packing Paddle
	I tre eller polert stål
	\item ``Line Group Card''
	For festing av linebunter ved demontering av skjermer
	\item Poynter Volum 1 og 2
	\item F/NLF MK-Bok
\end{itemize}

\textbf{Reservedeler:}
\begin{itemize}
	\item Ripstop
	\item Diverse Liner og Tråd
	For enkle feltreparasjoner
	\item Supertack tråd
	\item Ferdigsydde looper
	\item Vanlige Skiver og Cypres skiver
	\item Strikk / Tube stoes
	\item Safety Stow
\end{itemize}

\textbf{Spesielt verktøy}
\begin{itemize}
	\item Line Separator
	For pakking av runde reserver
	\item Strekkplate
	For jevn pakking av linebunter
	\item Pakkevekter
	Nylon poser eller bånd, fylt med blyhagl
	\item Strekktesteutstyr
	Spesialtenger
	\item Fiskevekt
	For strekktesting og kontroll av reservetrekk
	\item Velcrobeskytter
	Avblending av velcro på fribager
	\item Utstyr for Fingertrapping
	For enkle feltreparasjoner, (fingertrapping nål eller tynn dobbel ståltråd).
	\item Links separator
	For L-links - runde reserver
\end{itemize}

Kraftoverføringsutstyr for lukking av pakksekker er anbefalt brukt av enkelte utstyrsprodusenter. Slikt utstyr må benyttes med varsomhet og skal ikke være et redskap som dekker over dårlig pakketeknikk og feil lengde på containerens lukkeloop. Ved bruk av slikt utstyr må en være sikker på at loopen har riktig lengde og at produsentens pakketeknikk benyttes.

\subsection{Plombering}
Samtlige reserver skal plomberes etter F/NLF's forskrifter for at materiellet skal regnes som luftdyktig. Plombering skal foregå etter følgende instruks.

\begin{enumerate}
	\item Klipp av en passende lengde brytetråd, og før én ende under pinnen (eller begge pinnene) som låser containeren. Dette gjelder enten pinnen sitter på enden av en wire, eller enden av et LOR bånd.

	\item Knyt to-tre halvstikk på oversiden av pinnen, eventuelt i ringen på én av LOR pinnene. Klipp av tråden så endene er like lange.

	\textbf{NB: Ved plombering av LOR 2 skal bryetråden festes i kun en av pinnene.}

	\item Tre plomben inn på den dobbelte brytetråden gjennom ett av de små hullene.

	\item Gi litt slakk på tråden, og knut to-tre halvstikk på den andre siden av plomben. Gi nok slakk her til at tråden ikke ryker når du klemmer til med plombetangen.

	\item Tre den dobbelte brytetråden tilbake gjennom det store hullet, og ut gjennom det andre mindre hullet i plomben.

	\item Klem til med plombetangen, og klipp av den løse enden av tråden.
\end{enumerate}

Plombering på denne måten sikrer at plomben blir festet med dobbel tråd et stykke fra maljen. Det er viktig at plomben festes i den delen som blir med pinnen ved trekk, og at den ikke forblir en løs enhet som blir igjen i området rundt malja etter at pinnen er trukket og tråden er røket.

Kun den ytre (siste) låsepinne plomberes, slik at tråden ryker ved åpning av pakksekken (uttrekk av en eller flere låsepinner).

\textbf{NB: Det er viktig at det er kun enkel brytetråd fra knuten og rundt loopen.}

Nødvendig utstyr:
\begin{itemize}
	\item Bly plomber
	\item Plomberingstråd, rødfarget, med bruddstyrke 1,5 til 3,0 kg.
	\item Plombetang preget med pakkeseglkode
\end{itemize}

\subsection{Trekkbelastning på reserveskjermer}
Trekkbelastningen på utløserkabelen for reserveskjermer skal aldri overstige 22 lbs (10 kg), uten plomberingstråd.

Trekkbelastningen vil som regel variere med:
\begin{enumerate}
	\item Pakkemetode
	\item Kabelhus og avstiverplate
	\item Pakkevolum reserve/pakksekk
	\item Låseløkkelengden
	\item Type pilotskjermfjær
\end{enumerate}

Trekkbelastningen bør måles ved hver inspeksjon eller hovedkontroll – bruk en fjærvekt. Seletøyet må være korrekt påselet for å gi riktig måling.

Trekkbelastningen med plomberingstråd kan øke nødvendig trekkraft med 2-3 kilo.

\subsection{Hovedkontrollkort}
Et hovedkontrollkort skal fylles ut ved utstyrets førstegangskontroll og skal følge seletøyet så lenge dette er i bruk. Det skal gi opplysninger om når og av hvem utstyret er hovedkontrollert og reserven pakket. Utstyr uten hovedkontrollkort er ikke luftdyktig til ny hovedkontroll har funnet sted.

Hovedkontrollkortet skal minst inneholde følgende opplysninger:
\begin{enumerate}
	\item Betegnelse på de enkelte komponenter. (Hovedskjerm, reserve, seletøy og eventuelt nødåpner.)
	\item Produsent.
	\item Serienummer.
	\item Produksjonsdato.
	\item Dato for utført hovedkontroll og ompakk av reserve.
	\item Dato for funksjonstest, eventuelt kontroll/batteribytte av nødåpner.
	\item Angivelse av formell luftdyktighet.
	\item Angivelse av Materiellkontrollør(er) som har utført hovedkontroll og pakking av reserve, med pakkersertifikatnummer og forseglingskode.
	\item Utførte modifikasjoner og serviceordrer.
	\item Eventuelt informasjon om eier.
\end{enumerate}

Reparasjoner og modifikasjoner skal føres inn i hovedkontrollkortet. Disse opplysningene tilhører utstyrets historie. Utelates dette, kan utstyret måtte utsettes for nye kontroller for å være sikker på at tidligere modifikasjoner virkelig har funnet sted.

\subsection{Arbeidslogg}
Materiellkontrollør skal føre en fortløpende logg som dokumenterer alt arbeid han foretar seg i egen pakkelogg.

Det skal anføres utstyrstyper, eier, dato for utført arbeid, og eventuelle modifikasjoner som er utført.

\subsection{Skadeskjemaer}
Benytt tegning av kalott ved skadevurdering. Dette sendes med kalott til reparasjon og skal være ferdig utfylt med beskrivelse av skaden.

Bemerkninger på utført hovedkontroll skal også anføres i kontrollskjema / skadeskjema.

\subsection{Verksted}
Godkjent fallskjermverksted i Norge:

Sky Design AS Tønsberg, mailadresse; verksted@skydesign.no.

\section{Kontrollskjema for hovedkontroll}
Her følger et eksempel på et kontrollskjema for hovedkontroll, som kan nyttes som huske- og kontrolliste for gjennomføring av hovedkontroll.

Skjemaet er bygget opp slik at du kan følge en naturlig gang i arbeidet og få med alle momenter. Det er satt av plass til kommentarer dersom arbeidet kun omfatter reserveompakk, og generelle anmerkninger.

I tillegg er ekstrautstyr i forbindelse med tandemutstyr tatt med som egen sjekkliste.

Husk at en sjekkliste er bare så bra som brukeren vil ha den til. Det er blant annet ikke en god fremgangsmåte å foreta hele hovedkontrollen, for deretter å krysse av på sjekklista.

Husk at også verktøy, manualer, hovedkontrollkort etc inngår som en del av sjekkliste for gjennomføring av hovedkontroll.

Skjemaet kan fritt kopieres for eget bruk.
