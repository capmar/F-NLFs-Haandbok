\part{Elevutstyr}

\section{Generelt}
Den 1.5.1985 ble alt elevutstyr i Norge konvertert fra militært overskuddsmateriell (B4, frontmontert reserve – TU / LL– T-10-R osv) til moderne utstyr med ryggmontert reserve og firkant hovedskjermer.

Utstyret ble opprinnelig levert med MAGIC hovedskjermer, og etter et par år ble DRAKKAR også tatt i bruk som elevskjermer. Etter 1990 ble det vanlig med andre typer hovedskjermer, først på prøve, og deretter som generelt godkjent elevutstyr.

Dette kapitlet beskriver bruk og kontroll av elevutstyr som er godkjent for bruk til elevhopping i Norge. Listen nedenfor beskriver utstyr som er alminnelig tillatt for bruk i Norge, med henvisninger til modifikasjoner og oppdateringer som er skjedd underveis.

Elevutstyr har en atskillig større slitfaktor enn privateid utstyr. Så lenge enkeltpersoner ikke er direkte ansvarlig for utstyret, kan dette medføre at utstyret ikke får den kontroll og vedlikehold som er nødvendig for at det skal holde med den slitasjen det blir påført. Et fornuftig vedlikeholdsprogram for elevutstyret i klubben kan spare mange penger ved at utstyret må repareres oftere enn nødvendig, og ikke er operativt når det trengs.

Som Materiellkontrollør for elevutstyr er det derfor viktig å være spesielt aktsom og kritisk til kontroller og oppdateringer, og å forsikre deg om at elevutstyret alltid er oppdatert og sikkert. Når ansvaret fordeles på flere i klubben, kan det lett oppstå situasjoner der en tror at ``andre'' sikkert har kontrollert utstyret.

\section{Typegodkjent elevutstyr}
Følgende utstyr er alminnelig typegodkjent for bruk som elevutstyr:

\subsection{Hovedskjermer}
Hovedskjermer fra godkjente produsenter kan nyttes som elevskjermer med visse begrensninger.

For alle elevskjermer gjelder følgende generelle regler for bruk:
\begin{enumerate}
	\item Skjermen skal være 7- eller 9-celler, vingskjermer med areal på 210 kvadratfot eller mer.
	\item Skjermer som er markedsført av produsenten som elevskjermer kan brukes.
	\item Om hovedskjermene ikke er spesifikt markedsført som elevskjerm av produsent skal skjermen være produsert i F-111 eller 0P/Hybrid, ha rektangulær grunnflate, og lik cellehøyde i alle cellene.
	\item Vektbegrensning for bruker avgjør tillatt skjermtype for hopper (se6.2.1.1, Vektbegrensninger hovedskjermer)
	\item Skjermens skal kunne pakkes med ``sidepack'' som pakkemetode
	\item Styreliner skal være så lange at skjermen ikke kan steiles ut av hoppere som er innenfor lovlig vektområde for skjermen
	\item Det skal ikke vært montert stallsperre på styrelinene.
\end{enumerate}

\subsubsection{Vektbegrensninger hovedskjermer}
Fra mai 2003 ble det innført nye regler for maks vektbegrensning for hovedskjermer for elever. Endringer er etter dette gjort i mars 2007.

I tillegg til å forholde seg til en maks belastning for de enkelte skjermer må en også ta hensyn til nedre vektgrense.

\begin{itemize}
	\item Vektområdet utregnes ved å ta hopperens vekt uten utstyr, legge til 15 kg, og dele på skjermens størrelse i kvadratfot.
\end{itemize}

\begin{displaymath}
	\frac{(\text{Hoppers vekt (kg)} + 15 \mathrm{kg}) \times 2.2}{\text{Skjermens størrelse (kvadratfot)}} = \text{Vektbelastning uttrykt i lb/sqft}
\end{displaymath}

Eksempel: En hopper uten utstyr veier 80 kilo. Med en skjerm på 280 kvadratfot vil hopperen få en vingebelastning på 95*2,2/280 = 0,75

De aktuelle vektområdene er som følger:

\begin{table}
	\caption{Tillatt vingebelastning elevskjermer}
	\begin{center}
	\begin{tabular}{ | l | l | }
		\hline
		Type & Vingebelastning \\
		\hline
		7 celler \& 9 celler &  0,55 – 0,95(1,0)\footnotemark \\
		\hline
	\end{tabular}
\end{center}
\end{table}
\footnotetext{1,0 gjelder erfarne elever med 0P/hyblid}

I de påfølgende illustrasjoner er det satt opp det aktuelle vektområdet som skjermen tillates flydd med.

Merk at det er hopperens totale vekt i det han går inn i flyet som er avgjørende for vingebelastningen. Et komplett elevutstyr og bekledning veier cirka 15 kg.

\begin{figure}
	%\includegraphics[width=60mm]{Strekktesting av kalottduk.pdf}
	\caption{Tillatt vektbelastning elevskjermer}
\end{figure}

Det er kjent at samme hovedskjerm med lik vingbelastning i stor og liten størrelse ikke har lik karakteristikk. En bør derfor være påholden med å tillate uerfarne elever å bruke de minste elevskjermene lastet opp til maksimum. Det anbefales at elevene i hvert fall mot slutten av sin progresjon får introduksjon til og erfaring med hopping med vingbelastning i øvre tillatte sjikt, for å lære hvordan en skjerm forandrer egenskaper ved økt vingbelastning.

Håndboka åpner også for å hoppe med opp til og med 1,0 i vingebelastning for erfarne og dyktige elever når disse benytter hybrid eller 0P skjermer.

\subsubsection{Eksempel på typegodkjente hovedskjermer, jfr. pkt. 6.2.1}
\begin{table}
	\caption{Eksempel på typegodkjente hovedskjermer for elevbruk}
	\begin{tabular}{ | p{4cm} | p{2cm} | p{2cm} | p{2cm} | p{2cm} | }
		\hline
		Fabrikant & Modell & Celler & Areal kv. fot & Pakkevolum cu. in. \\
		\hline
		Aerodyne International & Solo & 9 & 230 250 270 & \\
		\hline
		Icarus Canopies & Student ZP7 & 7 & 229 249 269 289 & \\
		\hline
		Parachutes de France & Magic & 5 & 250 & \\
		\hline
		Parachutes de France & Prima & 9 & 230 260 290 & \\
		\hline
		Performance Designs & Navigator & 9 & 220 240 260 280 300 & \\
		\hline
	\end{tabular}
\end{table}

I prinsippet blir flere skjermer enn de listet i Tabell 2 - Eksempel på typegodkjente hovedskjermer for elevbruk godkjent, så lenge de har rektangulær planform, lik cellehøyde og ellers tilfredsstiller punkt 6.2.1. Merk også at alle hovedskjermer fra godkjente produsenter som markedsføres til elevbruk tillates brukt.

\subsection{Seletøy}
For alle elevseletøy gjelder følgende generelle regler for bruk:
\begin{enumerate}
	\item Produsenten skal være godkjent seletøysprodusent.
	\item Seletøy til bruk for elever godkjennes spesielt av SU, med henblikk på standardisering.
	\item Seletøyet skal ha standard brede løftestropper, standard store 3-rings eller rustfritt 3-rings utløsersystem og BOC/DRCP-type utløsersystem for hovedskjerm.
	\item Seletøyet skal fra produsenten være ferdig oppsatt for
	\begin{enumerate}
		\item AFF-hopping,
		\item nødåpner godkjent for elevhopping, og
		\item RSL
	\end{enumerate}
	\item Produsenten skal levere ``static'' line med seletøyet.
	\item Produsenten skal på forespørsel kunne levere treningsseletøy
\end{enumerate}

Da elevutstyr fra ulike produsenter er svært likt for de detaljer som er av kritisk betydning for sikker operasjon (pakkeprosedyre, håndtaksplassering og utløsningssystem), har SU vurdert at standardiseringskonseptet ikke gir grunnlag for å kreve at samtlige klubber skal benytte det samme seletøysystemet. Flere års vellykket bruk av både Campus I og Campus II bygger opp under dette.

Fra juli 2000 er følgende elevutstyr godkjent:

\begin{table}
	\caption{Typegodkjente elevseletøy}
	\begin{tabular}{ | p{3cm} | p{3cm} | p{4cm} | }
		\hline
		Produsent & Type & Modifikasjon / Komm. \\
		\hline
		Aerodyne & Icon Student & Godkjent se SU ref 4/06 \\
		\hline
		Parachutes de France & Campus 1 & Se mod liste i kapittel 11 \\
		\hline
		Parachutes de France & Campus 2 & \\
		\hline
		Parachutes de France & Atom Atom Evolution & Elevversjon Elevversjon \\
		\hline
		Sun Path & Javelin Javelin Odyssey & Elevversjon Elevversjon \\
		\hline
		Strong Enterprises & Quasar Trainer & \\
		\hline
		Rigging Innovations & Telesis & Elevversjon \\
		\hline
	\end{tabular}
\end{table}

Ved innføring av nytt elevutstyr skal det ikke være mer enn to typer seletøy i bruk i en og samme klubb. PdF Campus (I og II) regnes her som én type. Klubber som innfører nytt elevseletøy kan i en innføringsperiode på inntil 2 år operere med Campus seletøy og nytt seletøy. En og samme klubb kan altså i en overgangsperiode utdanne elever på både Campus elevrigg og nytt elevseletøy. Denne overgangsperioden kan ikke overstige 2 år.

Elever som utdannes på Campus elevrigg kan omskoleres til ny elevrigg, men ikke omvendt eller tilbake igjen.

\subsection{Reserver}
Reserveskjermer til elevbruk må være spesielt godkjent av SU. Både runde- og vingskjermer tillates brukt. Hvis en klubb benytter både runde- og vingreserver i sitt elevutstyr, skal ingen elever tillates å veksle mellom de to ulike typene reservefallskjermer. Dette for å redusere sannsynlighet for feil respons på nødsituasjoner som for eksempel to skjermer ute.

\subsubsection{Runde reserver}
Det vil ikke bli godkjent flere runde reserveskjermer til elevbruk enn de som følger av tabell 4 – Typegodkjente runde reserver for elevutstyr.

\begin{table}
	\caption{Typegodkjente runde reserver for elevutstyr}
	\begin{tabular}{ | p{4cm} | p{2cm} | p{4cm} | }
		\hline
		Fabrikant & Modell & Merknader \\
		\hline
		Parachutes de France
		PISA
		Godalming, England &
		PISA PF 26' &
		Forbud mot diaper på PISA elev skjermer.
		Produsert på lisens. \\
		\hline
		Strong Enterprises Inc. & Strong Lo Po Lite 26' & Tillates brukt med diaper. \\
		\hline
		Strong Enterprises Inc. &
		Strong Lo Po 26' /
		Mid Po 26' &
		Bæreliner serienummer 3000 -
		4000 Desember 1977 - August
		1978 (LoPo 26.)
		Tillates brukt med diaper. \\
		\hline
		National &
		Phantom 26'
		Phantom 28' &
		Kevlar forsterket
		Diaper modifisert
		(Se kapittel 4 for reserver) \\
		\hline
	\end{tabular}
\end{table}

\subsubsection{Vingreserver}
Vingreserver typegodkjent, jfr. kap. 4, kan godkjennes av SU for elevbruk etter følgende kriterier:
\begin{enumerate}
	\item Produsenten av reserven skal være godkjent.
	\item Reserveskjermen skal være typegodkjent
	\item Kun reserveskjermer på 235 kvadratfot eller større kan godkjennes til elevbruk.
\end{enumerate}

\begin{table}
	\caption{Typegodkjente vingreserver for elevutstyr}
	\begin{tabular}{ | p{4cm} | p{2cm} | p{4cm} | }
		\hline
		Fabrikant & Modell & Merknader \\
		\hline
		Aerodyne International & Smart 250 & \\
		\hline
		Parachutes de France & Techno 240 & Serviceordre 0305 \\
		\hline
		Performance Designs & PD235R PD253R PD281R & \\
		\hline
		Paratec & Speed 250 & \\
		\hline
		PISA & Tempo 250 & \\
		\hline
		Precision Aerodynamics & Raven DASH-M 249 Raven DASH-M 282 & \\
		\hline
		Strong Enterprises Inc. & Stellar 240 & \\
		\hline
	\end{tabular}
\end{table}

\subsection{Automatåpnere}
Følgende automatåpnere er godkjent for elevutstyr og elevhopping:
\begin{itemize}
	\item FXC Corporation FXC 12000 oppdatert til ``J'' standard.
	\item Airtec Student Cypres og Student Cypres 2
\end{itemize}

Elevutstyr skal alltid være utstyrt med automatåpner.

\subsubsection{FXC}
``J'' standard kjennetegnes synlig ved at det er montert tre gull-farvede filtere i Betjeningsdelen, ellers er dette også påført lapp på hoveddelen. Se forøvrig Kapittel 8 – Nødåpnere.

FXC skal underkastes funksjonstest i trykkammer innen 7 måneder siden siste funskjonstest, ellers er de ikke operative. (Ref: Kapittel 8.1.5). For utførlig beskrivelse av funksjon, virkemåte og montering se Kapittel 8.1.

Dato for siste funksjonstest skal noteres og oppdateres på hovedkontrollkortet.

\subsubsection{Cypres elevversjon}
Bruks– og virkemåte for Cypres elevversjon er den samme som for Cypres Expert versjon. Innstilling og start foretas en gang om dagen, og ingen videre innstillinger er nødvendig etter at Cypres er slått på.

Se videre i Kapittel 8.3 og 8.4 for ytterligere informasjon om Cypres.

Cypres elevversjon er fastsatt fra fabrikanten til å fyre i området fra 750 – 1000 fot over bakkenivå (avhengig av vertikal hastighet), ved en vertikal hastighet som overstiger cirka 50 km/t.

\section{Anmerkninger og modifiseringer}
\subsection{Modifiseringer av elevutstyr}
Følgende modifiseringer av elevutstyr er gitt ut fra produsent eller forbund. De er delt inn i en generell del som går på alt elevutstyr samt tilbehør, og en spesifikk del som går på elevutstyr fra hver enkelt produsent. Dette er angitt i kolonnen Produsent / Myndighet.

De fleste av modifiseringene er nødvendige og obligatoriske. Unntakene er listet opp separat i kolonnen for beskrivelse. Denne kolonnen angir en kort beskrivelse av bakgrunn og omfang av modifikasjonen.

Pass på at du bruker riktig dokumentasjon når du kontrollerer elevutstyr. Tidligere utgitte modifiseringsordre er angitt i kapittel 11 i denne boka.

Se forøvrig kapittel 11, som inneholder informasjon om generelle modifikasjoner på fransk utstyr.

Listen er sortert på den enkelte produsent, eller hvilken type utstyr modifikasjonen omfatter, og deretter stigende dato for utstedelse. Det er tiltenkt at dette skal kunne lette arbeidet ved å søke mot informasjon om det utstyr en trenger opplysninger om.

\begin{table}
	\caption{Modifikasjoner og serviceordrer elevutstyr}
	\begin{tabular}{ | p{1cm} | p{1cm} | p{1cm} | p{1cm} | p{3cm} | p{3cm} | }
		\hline
		Produsent/Myndighet & Produkt & Dato & SO & Type & Bakgrunn ( Kontroll) \\
		\hline
		FXC & 12000 & 94.01.12 & & \textbf{FXC-12000} oppdatering & Oppdatering til ``J'' standard. \\
		\hline
		FXC & 12000 & 94.05.20 & 9402 & \textbf{FXC} – funksjonstest ved \textbf{``Plastposetest''} utgår & Utgår på grunn av innvendige skader på FXC, og innføring av funksjonstest. \\
		\hline
		FXC & 12000 & 95.05.20 & 9503 & Kontrollsyklus \textbf{FXC} – trykkammertest & Innføring av 7 måneders syklus for funksjonstest av FXC \\
		\hline
		Generelt & Utløserliner & 88.01.01 & & \textbf{Utløserliner} & Brettes til 1⁄2 bredde (cirka 20 mm) og syes sammen ved bruk av hengende exit. \\
		\hline
		Generelt & Campus & 93.08.29 & 9308 & Kontroll av \textbf{hovedhåndtak Campus} & Slitasje i plastikken rundt wire ved låseløkka \\
		\hline
		Generelt & Hoved-skjermer & 95.05.20 & 9501 & Styreliner elevskjermer, kontroll &
		Alle elevskjermer måles for lengder av styreliner og testhoppes for kontroll av stall egenskaper \\
		\hline
		Generelt & Seletøy/pakksekk & 03.07.18 & 0303 & Test av reservepinner & Svake pinner som knakk under pakking \\
		\hline
		PdF & Magic 5 & 87.11.18 & & \textbf{Magic elevskjerm} – modifisering for raskere åpninger, innkorting av bremseliner & Den indre lina av de øvre styrelinene (3 stk) flyttes en halv celle mot midten, og kortes inn 15 cm. \\
		\hline
		PdF & Campus & 91.01.21 & 910101 & Treghet ved \textbf{kutthåndtak}, olje er igjen i kabelføring og tiltrekker seg skitt & Rensing av kabelføringer med trikloretylen – bruk av pussestokk eller lignende. \\
		\hline
		PdF & Magic 5 & 91.10.03 & & \textbf{Styreliner på Magic} elevskjermer & Fjerning av stallsperre, og montering av lengre styreliner. \\
		\hline
		PdF & Campus & 91.10.03 & & Montering av \textbf{LOR 2} på reserve pakksekk & Oppgradering av elevutstyr fra LOR 1 til LOR 2. Ordningen er frivillig. \\
		\hline
		PdF & Campus & 91.10.03 & & Montering av \textbf{karabinlås} på \textbf{LOR} line & Muliggjør frigjøring for å unngå utløst reserveskjerm ved popping av hovedskjerm eller vedvannlanding. \\
		\hline
		PdF & Campus & 92.11.17 & 92004 & Kontroll av \textbf{tre- rings og kuttkabel, minirisers.} & Dårlig frigjøring – hardt trekk av kabel. 9 kg max trekk av kutthåndtak Alminnelig kontroll ved hovedkontroll. \\
		\hline
		PdF & Campus & 92.12.14 & 92006 & \textbf{Reservelooper}, byttes til tynne. Gjelder for Jaguar, Campus 1 og 2, Galaxy og Atom & Låsing av reservecontainer. Montering av dobbel låseløkke (1.5 mm microline) \\
		\hline
		PdF & Campus & 93.10.18 & 9310 & Beskyttelsesklaff for reserveklaff \textbf{Campus 1} & Montering av klaff for å beskytte reserveklaffen mot liner etc., tilsvarende Campus 2 \\
		\hline
		PdF & Atom Evo. & 00.04.28 & 0001 & Styringsklaff i hovedpakksekk & Montering av styringsklaff for å beskytte hoveddekklaff. \\
		\hline
		PdF & Techno 240 & 03.10.17 & 0305 & Bytte slider, nedre styreliner og endrett pakk & Opprettholde ytelsesmaksimum etter SO 0302 \\
		\hline
		RWS & Vector & 98.10.09 & 9804 & Kontroll av kabelføring kutthåndtak & Løse kabelsko klemt fast med feil prosedyre \\
		\hline
		Sun Path & Javelin & 00.04.29 & 0002 & Kontroll av malje for låseløkke hovedpakksekk & Skjerme malje fra liner på hovedskjerm \\
		\hline
		Sun Path & Javelin Javelin Odyssey & 03.07.29 & 0304 & Kontroll av webbing i justerbare seletøy & Høy slitasje ved innfesting av justeringsspenne \\
		\hline
	\end{tabular}
\end{table}

\subsection{Andre anmerkninger – oppdateringer}
\subsubsection{Reservedekklaff – L.O.R.}
På Campus er det viktig å kontrollere velcrobiten som festes på utsiden av siste reservedekklaff. Denne fungerer som styring for L.O.R. båndet, og har til hensikt at reservepinnen får et rett trekk. Det er derfor viktig å kontrollere tilstanden til velcroen, og kontrollere at denne sitter godt og fungerer som tiltenkt.

For ytterligere opplysninger om L.O.R. og montering henvises til Kapittel 3.6.

På innersiden av reservedekklaffen bør tegning av riktig lukking med reservekabel, L.O.R. bånd og FXC kabel være synlig og oppdatert.

\begin{enumerate}
	\item Reservekabel
	\item Styrering for reservekabel
	\item Øye for FXC kabel
	\item Reservepinne
	\item Ring (eller løkke) for LOR line
	\item Låseløkke
	\item LOR bånd
	\item Kabelføring for reservekabel
	\item Kabelføring for FXC kabel
	\item FXC kabel
\end{enumerate}

\begin{figure}
	%\includegraphics[width=60mm]{Strekktesting av kalottduk.pdf}
	\caption{Reservelukking med LOR 1}
\end{figure}

\begin{figure}
	%\includegraphics[width=60mm]{Strekktesting av kalottduk.pdf}
	\caption{Reservelukking med LOR 2}
\end{figure}

\subsubsection{Bremsesystem}
Det opprinnelige leverte bremsesystem fra Parachutes de France, med plaststyrehåndtak og pinne bør byttes til alminnelig bremsesystem der styrehåndtak låser bremselina.

Dette systemet hadde en del svakheter, der blant annet styrehåndtak ikke alltid holdt bremselinene ordentlig på plass under skjermåpning. Systemet hadde også en annen svakhet, ved en ekstra løkke som først ble tredd igjennom bremselina, deretter stryeringen, for til slutt ble låst fast av pinnen. I enkelte tilfelle betydde dette at bremselina måtte trekkes ut i full lengde for at bremsene var frigjorte. Ved forlengning av nedre del av styrelinene er det ikke sikkert at eleven hadde lange nok armer til å frigjøre bremsene helt.

Endringen medfører at det monteres ny velcro og en mindre styrering for bremselina, samt nye styrehåndtak.

\subsubsection{Skifte av styreliner}
Den mest vanlige slitasjen på styreliner skjer på nedre del, enten ved gnisning av slider, eller slitasje i bremsesetting. Disse finnes prefabrikkert og kan byttes enkelt av en MK. Merk at på både Magic og Drakkar er disse produsert over tid med forskjellige lengder på styrelinene, så det er viktig å angi hvilken lengde den opprinnelige styrelinen hadde.

Nødvendige mål er avstand fra kaskaden fra øvre forgreninger og ned til bremsesetting, samt avstand fra bremsesetting til styrehåndtak.

Lengden av den frie delen av styrelinene (nedenfor bremseløkka) tilpasses skjermen slik at skjermen ikke lar seg steile ut ved fullt pådrag. Tilpasning avgjøres lokalt av HI.

Linene monteres enkelt ved å tre de fingertrappede løkkene gjennom den frie delen på styrelinen for låsing.

\begin{figure}
	%\includegraphics[width=60mm]{Strekktesting av kalottduk.pdf}
	\caption{Montering av styreliner}
\end{figure}

\begin{figure}
	%\includegraphics[width=60mm]{Strekktesting av kalottduk.pdf}
	\caption{Lengdeanvisning for styreliner}
\end{figure}

\subsubsection{Handdeployed utløsersystem}
Når elevutstyr skal brukes med handdeployed pilotskjerm (for handdeployed utsjekk) skal det være påmontert elastisk lomme for pilotskjermen på bunn av hovedskjermens pakksekk (BOC). Den opprinnelige lommen i cordura på Campus er for liten og trang for dette formålet. Evt. påmontert lomme på beinstropp (ROL) skal ikke nyttes.

NB: Vær oppmerksom på innsløyfingen av pilotbåndet ved bruk av handdeployed på Campus. Pilotbåndet skal tres opp fra innerbagen på høyre side av siste lukkeklaff, for festing av pinne og videre festing på velcro.

Dersom pilotbåndet trekkes opp fra venstre side av siste lukkeklaff, over denne og ned på høyre side, kan borrelåsen som holder klaffene lukket føre til totalforsager på hovedcontainer!

\begin{figure}
	%\includegraphics[width=60mm]{Strekktesting av kalottduk.pdf}
	\caption{Treing av pilotline ved handdeployed utløsersystem / Campus}
\end{figure}

Utstyr med påmontert handdeployed pilotskjerm krever egen merking for å unngå forveksling med fallskjermsett med DRCP.

\subsubsection{AFF}
Fra 01.01.05 skal elevfallskjermsett som nyttes til AFF-hopping være spesielt modifisert med sekundært utløserhåndtak til hovedskjerm. Dette medfører at det er påmontert et sekundært håndtak for aktivering av hovedskjerm på venstre side, tilgjengelig for sekundærhoppmester.

Det er to vanlige systemer for dette:
\begin{enumerate}
	\item Sekundærhoppmesteren har et eget håndtak som frigjør selve låseloopen til hovedcontaineren, vanligvis underfra.
	\item Kabelen til hovedhåndtaket til seletøyet tres via en styrering som er montert på et håndtak til sekundærhoppmester, og at denne da trekker kabelen i det vanlige reservehåndtaket.
\end{enumerate}

Andre funksjoner på utstyret endres ikke ved at utstyret er utstyrt med AFF modifisering. Ved omkonfiguerring av utløsersytem fra DRCP til handdeployed throw-out, må sekundærhåndtaket også monteres av. Dette for å eliminere sannsynligheten for hestesko feilfunksjon. Hvis det benyttes annet feste for låseløkke ved throw-out pilot, vil kabel til sekundærhåndtaket kunne forårsake pakksekklåsing hvis det ikke fjernes ved omkonfigurering.

\begin{figure}
	%\includegraphics[width=60mm]{Strekktesting av kalottduk.pdf}
	\caption{AFF montering for Campus og Atom Evolution}
\end{figure}

\subsubsection{Links PdF}
Dersom Parachutes de France med venstre–dreide låseskruer benyttes, kontroller for korrekt montering og slitasje, eventuell bytt om til Maillon Rapide nr. 5.

\subsection{Tilleggsutstyr}
\subsubsection{Utløserliner}
Utløserline skal være produsert av minimum 1000 lbs webbing, og skal være 390 cm lang med godkjent festekrok og låsepinne. Utløserlina skal ikke være noe særlig bredere enn 20 mm ved elevhopping med hengende exit.

Utløserliner med påmontert kabel i stedet for låsepinne er tillatt. Kablene er lenger enn pinnene, og skal lettere forhindre utilsiktede åpninger på staget. En har erfart at visse typer kabler kan forårsake betydelig slitasje på låseløkke.

\subsubsection{Håndtak}
Vær varsom på slitasje av elevhåndtakene. Disse utsettes for en del slitasje, spesielt ved pakking ved at plasten gradvis blir slitt på det punktet der låseløkka er. Dette kan skyldes alminnelig slitasje samt at pullupcord trekkes gjennom løkka/kabel med belastning, og sliter av plasten. Blir plasten borte kan dette medføre hardt trekk eller eventuelt låsing, eller, ved alvorligere tilfelle, at kabelen blir trukket ned igjennom maljene med fullstendig låsing av pakksekk som resultat.
