\part{Flygerinstruks}
\setcounter{section}{508}

\section{Flygerinstruks}
\subsection{Generelt}
Denne instruks fastlegger hvilke instrukser Flygeren skal rette seg etter når det gjelder den hopptekniske delen av flygingen, og disposisjon av flyet og ansvarsforholdene ombord.

Denne instruks gjelder bare så langt som det andre bestemmelser Flygeren lovmessig er pålagt ikke bestemmer annet.

\subsection{Autorisasjon av flygingen}
HL avgjør hvorvidt flyging med hoppere skal finne sted. Har HL gitt Flyger beskjed om at flyging med hoppere ikke skal finne sted, er Flygeren ansvarlig for at dette overholdes.

Dersom spesielle forhold gjør at Flygeren ikke ønsker å gjennomføre flyging med hoppere, skal han straks underrette HL om dette og angi årsaken. Flygerens avgjørelse er endelig.

\subsection{Innlasting}
Flyger bestemmer antall hoppere som kan tas ombord og hoppernes plassering i flyet. HFL avgjør hvilke personer som skal delta i de enkelte løft. Ved innlasting bestemmer HM hoppernes innbyrdes plassering i flyet.

Dersom spesielle forhold gjør at Flygeren ikke ønsker den manifesterte last av hensyn til vekt- eller plassforhold, underretter han straks HFL med angivelse om hvilke endringer som må gjennomføres.

\subsection{Gjennomføring av flyging}
HM angir ønsket innflygingsretning, utsprangshøyde, hopprekkefølge og utsprangspunkt. Flyger er ansvarlig for overholdelse av beordret utsprangshøyde og for korrekt flyhastighet ved dropp.

Dersom det oppstår forhold som gjør at Flygeren ikke ønsker å følge de anvisninger som er gitt av HM, underretter han straks HM som avgjør hvorvidt utsprang skal finne sted eller ikke.

Flyger følger HMs kurskorreksjoner under innflyging for utsprang.

\subsection{Utsprang}
Er flygingen under kontroll av lufttrafikk-kontrollenhet, er Flyger ansvarlig for klarering for utsprang. Flyger skal gi HM tegn om at utsprang er klarert. Det er Flygers ansvar at utspranget er klarert når han gir HM tegn om dette.

All kommunikasjon mellom Flyger og hoppere skjer gjennom HM. Ved en eventuell nødsituasjon ombord er HM ansvarlig for at Flygers anvisning blir fulgt.

\subsection{Andre forhold}
Flyger skal være kjent med disse bestemmelser (Del 500).

Flyger skal være kjent med de signaler som er avtalt iht Hoppfeltleder instruks, og bekrefte at han er innforstått med å følge disse.

Særtrykk av disse bestemmelser (Del 500) skal alltid være tilgjengelig under hopping.
